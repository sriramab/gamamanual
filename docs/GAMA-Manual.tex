\documentclass[]{book}
\usepackage{lmodern}
\usepackage{amssymb,amsmath}
\usepackage{ifxetex,ifluatex}
\usepackage{fixltx2e} % provides \textsubscript
\ifnum 0\ifxetex 1\fi\ifluatex 1\fi=0 % if pdftex
  \usepackage[T1]{fontenc}
  \usepackage[utf8]{inputenc}
\else % if luatex or xelatex
  \ifxetex
    \usepackage{mathspec}
  \else
    \usepackage{fontspec}
  \fi
  \defaultfontfeatures{Ligatures=TeX,Scale=MatchLowercase}
\fi
% use upquote if available, for straight quotes in verbatim environments
\IfFileExists{upquote.sty}{\usepackage{upquote}}{}
% use microtype if available
\IfFileExists{microtype.sty}{%
\usepackage{microtype}
\UseMicrotypeSet[protrusion]{basicmath} % disable protrusion for tt fonts
}{}
\usepackage[margin=1in]{geometry}
\usepackage{hyperref}
\hypersetup{unicode=true,
            pdftitle={GAMA Manual},
            pdfauthor={Srirama Bhamidipati},
            pdfborder={0 0 0},
            breaklinks=true}
\urlstyle{same}  % don't use monospace font for urls
\usepackage{natbib}
\bibliographystyle{plainnat}
\usepackage{longtable,booktabs}
\usepackage{graphicx,grffile}
\makeatletter
\def\maxwidth{\ifdim\Gin@nat@width>\linewidth\linewidth\else\Gin@nat@width\fi}
\def\maxheight{\ifdim\Gin@nat@height>\textheight\textheight\else\Gin@nat@height\fi}
\makeatother
% Scale images if necessary, so that they will not overflow the page
% margins by default, and it is still possible to overwrite the defaults
% using explicit options in \includegraphics[width, height, ...]{}
\setkeys{Gin}{width=\maxwidth,height=\maxheight,keepaspectratio}
\IfFileExists{parskip.sty}{%
\usepackage{parskip}
}{% else
\setlength{\parindent}{0pt}
\setlength{\parskip}{6pt plus 2pt minus 1pt}
}
\setlength{\emergencystretch}{3em}  % prevent overfull lines
\providecommand{\tightlist}{%
  \setlength{\itemsep}{0pt}\setlength{\parskip}{0pt}}
\setcounter{secnumdepth}{5}
% Redefines (sub)paragraphs to behave more like sections
\ifx\paragraph\undefined\else
\let\oldparagraph\paragraph
\renewcommand{\paragraph}[1]{\oldparagraph{#1}\mbox{}}
\fi
\ifx\subparagraph\undefined\else
\let\oldsubparagraph\subparagraph
\renewcommand{\subparagraph}[1]{\oldsubparagraph{#1}\mbox{}}
\fi

%%% Use protect on footnotes to avoid problems with footnotes in titles
\let\rmarkdownfootnote\footnote%
\def\footnote{\protect\rmarkdownfootnote}

%%% Change title format to be more compact
\usepackage{titling}

% Create subtitle command for use in maketitle
\newcommand{\subtitle}[1]{
  \posttitle{
    \begin{center}\large#1\end{center}
    }
}

\setlength{\droptitle}{-2em}

  \title{GAMA Manual}
    \pretitle{\vspace{\droptitle}\centering\huge}
  \posttitle{\par}
    \author{Srirama Bhamidipati}
    \preauthor{\centering\large\emph}
  \postauthor{\par}
      \predate{\centering\large\emph}
  \postdate{\par}
    \date{2018-07-14}

\usepackage{makeidx}
\makeindex

\usepackage{amsthm}
\newtheorem{theorem}{Theorem}[chapter]
\newtheorem{lemma}{Lemma}[chapter]
\theoremstyle{definition}
\newtheorem{definition}{Definition}[chapter]
\newtheorem{corollary}{Corollary}[chapter]
\newtheorem{proposition}{Proposition}[chapter]
\theoremstyle{definition}
\newtheorem{example}{Example}[chapter]
\theoremstyle{definition}
\newtheorem{exercise}{Exercise}[chapter]
\theoremstyle{remark}
\newtheorem*{remark}{Remark}
\newtheorem*{solution}{Solution}
\begin{document}
\maketitle

{
\setcounter{tocdepth}{1}
\tableofcontents
}
\chapter*{Notice}\label{notice}
\addcontentsline{toc}{chapter}{Notice}

The content of this manual is from gama-platform.org website. I have
only modified and edited very small portions of the content to give it a
book format.

Warning : The links do not work yet.

\chapter{Java version}\label{java-version}

Due to changes in the libraries used by GAMA 1.7 and 1.8, this version
now \textbf{requires JDK/JVM 1.8} to run. Please note that GAMA
\textbf{has not been tested with JDK 1.9 and 1.10}.

\section{Changes between 1.6.1 and 1.7/1.8 that can influence the
dynamics of
models}\label{changes-between-1.6.1-and-1.71.8-that-can-influence-the-dynamics-of-models}

\begin{itemize}
\tightlist
\item
  Initialization order between the initialization of variables and the
  execution of the \texttt{init} block in grids init -\textgreater{}
  vars in 1.6.1 / vars -\textgreater{} init in 1.7
\item
  Initialization order of agents -\textgreater{} now, the init block of
  the agents are not executed at the end of the global init, but during
  it. put a sample model to explain the order of creation and its
  differences
\item
  Initialization of vars to their default value map ? list ?
\item
  Systematic casting and verification of types give examples
\item
  header of CSV files: be careful, in GAMA 1.7, if the first line is
  detected as a header, it is not read when the file is casted as a
  matrix (so the first row of the matrix is not the header, but the
  first line of data) gives examples
\item
  the step of batch experiments is now executed after all repetitions of
  simulations are done (not after each one). They can however be still
  accessed using the attributes \texttt{simulations} (see Batch.gaml in
  Models Library)
\item
  signal and diffuse have been merged into a single statement
\item
  facets do not accept a space between their identifier and the
  \texttt{:} anymore.
\item
  simplification of equation/solve statements and deprecation of old
  facets
\item
  in FIPA skill, \texttt{content}is replaced everywhere with
  \texttt{contents}
\item
  in FIPA skill, \texttt{receivers} is replaced everywhere with
  \texttt{to}
\item
  in FIPA skill, \texttt{messages} is replaced by \texttt{mailbox}
\item
  The pseudo-attribute \texttt{user\_location} has been removed (not
  deprecated, unfortunately) and replaced by the ``unit''
  \texttt{\#user\_location}.
\item
  The actions called by an \texttt{event} layer do not need anymore to
  define \texttt{point} and \texttt{list\textless{}agent\textgreater{}}
  arguments to receive the mouse location and the list of agents
  selected. Instead, they can now use \texttt{\#user\_location} and they
  have to compute the selected agents by themselves (using an arbitrary
  function).
\item
  The random number generators now better handle seeding (larger range),
  but it can change the series of values previously obtained from a
  given seed in 1.6.1
\item
  all models now have a starting\_date and a current\_date. They then
  dont begin at an hypothetical ``zero'' date, but at the epoch date
  defined by ISO 8601 (1970/1/1). It should not change models that dont
  rely on dates, except that:
\item
  the \#year (and its nicknames \#y, \#years) and \#month (and its
  nickname \#month) do not longer have a default value (of resp. 30 days
  and 360 days). Instead, they are always evaluated against the
  current\_date of the model. If no starting\_date is defined, the
  values of \#month and \#year will then depend on the sequence of
  months and year since epoch day.
\item
  \texttt{as\_time}, \texttt{as\_system\_time}, \texttt{as\_date} and
  \texttt{as\_system\_date} have been removed
\end{itemize}

\chapter{Enhancements in 1.7/1.8}\label{enhancements-in-1.71.8}

\section{Simulations}\label{simulations}

\begin{itemize}
\tightlist
\item
  simulations can now be run in parallel withing an experiment (with
  their outputs, displays, etc.)
\item
  batch experiments inherit from this possibility and can now run their
  repetitions in parallel too.
\item
  concurrency between agents is now possible and can be controlled on a
  species/grid/ask basis (from multi-threaded concurrency to complete
  parallelism within a species/grid or between the targets of an
  \texttt{ask} statement)
\end{itemize}

\section{Language}\label{language}

\begin{itemize}
\tightlist
\item
  \texttt{gama} : a new immutable agent that can be invoked to change
  preferences or access to platform-only properties (like
  \texttt{machine-time})
\item
  \texttt{abort}: a new behavior (like \texttt{reflex} or \texttt{init})
  that is executed once when the agent is about to die
\item
  \texttt{try} and \texttt{catch} statements now provide a robust way to
  catch errors happening in the simulations.
\item
  \texttt{super} (instead of \texttt{self}) and \texttt{invoke} (instead
  of \texttt{do}) can now be used to call an action defined in a parent
  species.
\item
  \texttt{date} : new data type that offers the possibility to use a
  real calendar, to define a \texttt{starting\_date} and to query a
  \texttt{current\_date} from a simulation, to parse dates from date
  files or to output them in custom formats. Dates can be added,
  subtracted, compared. Various new operators (\texttt{minus\_months},
  etc.) allow for a fine manipulation of their data. Time units
  (\texttt{\#sec}, \texttt{\#s}, \texttt{\#mn}, \texttt{\#minute},
  \texttt{\#h}, \texttt{\#hour}, \texttt{\#day}, etc.) can be used in
  conjunction with them. Interval of dates (date1 to date2) can be
  created and used as a basis for loops, etc. Various simple operators
  allow for defining conditions based on the current\_date
  (after(date1), before(date2), since(date1), etc.).
\item
  \texttt{font} type allows to define fonts more precisely in
  \texttt{draw} statements
\item
  BDI control architecture for agents
\item
  file management, new operators, new statements, new skills(?), new
  built-in variables, files can now download their contents from the web
  by using standard http: https: addresses instead of file paths.
\item
  The \texttt{save} can now directly manipulate files and \ldots{} save
  them. So something like
  \texttt{save\ shape\_file("bb.shp",\ my\_agents\ collect\ each.shape);}
  is possible. In addition, a new facet \texttt{attributes} allows to
  define complex attributes to be saved.
\item
  \texttt{assert} has a simpler syntax and can be used in any behaviour
  to raise an error if a condition is not met.
\item
  \texttt{test} is a new type of experiments
  (\texttt{experiment\ aaa\ type:\ test\ ...}), equivalent to a
  \texttt{batch} with an exhaustive search method, which automatically
  displays the status of tests found in the model.
\item
  new operators (\texttt{sum\_of}, \texttt{product\_of}, etc.)
\item
  casting of files works
\item
  co-modeling (importation of micro-models that can be managed within a
  macro-model)
\item
  populations of agents can now be easily exported to CSV files using
  the \texttt{save} statement
\item
  Simple \texttt{messaging} skill between agents\\
\item
  Terminal commands can now be issued from within GAMA using the
  \texttt{console} operator
\item
  New \texttt{status} statement allows to change the text of the status.
\item
  light statement in 3D display provides the possibility to custom your
  lights (point lights, direction lights, spot lights)
\item
  Displays can now inherit from other displays (facets \texttt{parent}
  and \texttt{virtual} to describe abstract displays)
\item
  \texttt{on\_change:} facet for attributes/parameters allows to define
  a sequence of statements to run whenever the value changes.
\item
  \texttt{species} and \texttt{experiment} now support the
  \texttt{virtual} boolean facet (virtual species can not be
  instantiated, and virtual experiments do not show up).
\item
  \texttt{experiment} now supports the \texttt{auto\_run} boolean facet
  (to run automatically when launched)
\end{itemize}

\section{Data importation}\label{data-importation}

\begin{itemize}
\tightlist
\item
  draw of complex shapes through obj file
\item
  new types fo files are taken into account: geotiff and dxf
\item
  viewers for common files
\item
  navigator: better overview of model files and their support files,
  addition of plugin models
\end{itemize}

\section{Editor}\label{editor}

\begin{itemize}
\tightlist
\item
  doc on built-in elements, templates, shortcuts to common tasks,
  hyperlinks to files used
\item
  improvement in time, gathering of infos/todos
\item
  warnings can be removed from model files
\end{itemize}

\section{Headless}\label{headless}

\begin{itemize}
\tightlist
\item
  A new option \texttt{-validate\ path/to/dir} allows to run a complete
  validation of all the models in the directory
\end{itemize}

\section{Models library:}\label{models-library}

\begin{itemize}
\tightlist
\item
  New models (make a list)
\end{itemize}

\section{Preferences}\label{preferences}

\begin{itemize}
\tightlist
\item
  For performances and bug fixes in displays
\item
  For charts defaults
\end{itemize}

\section{Simulation displays}\label{simulation-displays}

\begin{itemize}
\tightlist
\item
  OpenGL displays should be up to 3 times faster in rendering
\item
  fullscreen mode for displays (ESC key)
\item
  CTRL+O for overlay and CTRL+L for layers side controls
\item
  cleaner OpenGL displays (less garbage, better drawing of lines,
  rotation helper, sticky ROI, etc.)
\item
  possibility to use a new OpenGl pipeline and to define keystoning
  parameters (for projections)
\item
  faster java2D displays (esp. on zoom)
\item
  better user interaction (mouse move, hover, key listener)
\item
  a whole new set of charts
\item
  getting values when moving the mouse on charts
\item
  possibility to declare \texttt{permanent\ layout:} +
  \texttt{\#splitted}, \texttt{\#horizontal}, \texttt{\#vertical},
  \texttt{\#stacked} in the \texttt{output} section to automatically
  layout the display view.
\item
  Outputs can now be managed from the ``Views'' menu. Closed outputs can
  be reopened.
\item
  Changing simulation names is reflected in their display titles (and it
  can be dynamic)
\item
  OpenGL displays now handle rotations of 2D and 3D shapes, combinations
  of textures and colours, and keystoning
\end{itemize}

\section{Error view}\label{error-view}

\begin{itemize}
\tightlist
\item
  Much faster (up to 100x !) display of errors
\item
  Contextual menu to copy the text of errors to clipboard or open the
  editor on it
\end{itemize}

\section{Validation}\label{validation}

\begin{itemize}
\tightlist
\item
  Faster validation of multi-file models (x2 approx.)
\item
  Much less memory used compared to 1.6.1 (/10 approx.)
\item
  No more ``false positive'' errors
\end{itemize}

\section{Console}\label{console}

\begin{itemize}
\tightlist
\item
  Interactive console allows to directly interact with agents
  (experiments, simulations and any agent) and get a direct feedback on
  the impact of code execution using a new interpreter integrated with
  the console. Available in the modeling perspective (to interact with
  the new \texttt{gama} agent) as well as the simulation perspective (to
  interact with the current \texttt{experiment} agent).
\item
  Console now accepts colored text output
\end{itemize}

\section{Monitor view}\label{monitor-view}

\begin{itemize}
\tightlist
\item
  monitors can have colors
\item
  monitors now have contextual menus depending on the value displayed
  (save as CSV, inspect, browse\ldots{})
\end{itemize}

\section{GAMA-wide online help on the
language}\label{gama-wide-online-help-on-the-language}

\begin{itemize}
\tightlist
\item
  A global search engine is now available in the top-right corner of the
  GAMA window to look for GAML idioms
\end{itemize}

\section{Serialization}\label{serialization}

\begin{itemize}
\tightlist
\item
  Serialize simulations and replay them (to come)
\item
  Serialization and deserialization of agents between simulations (to
  come)
\end{itemize}

\section{Allow TCP, UDP and MQQT communications between agents in
different simulations (to
come)}\label{allow-tcp-udp-and-mqqt-communications-between-agents-in-different-simulations-to-come}

\chapter{Operators (A to A)}\label{operators-a-to-a}

\begin{center}\rule{0.5\linewidth}{\linethickness}\end{center}

\textbf{This file is automatically generated from java files. Do Not
Edit It.}

\begin{center}\rule{0.5\linewidth}{\linethickness}\end{center}

\section{Definition}\label{definition}

Operators in the GAML language are used to compose complex expressions.
An operator performs a function on one, two, or n operands (which are
other expressions and thus may be themselves composed of operators) and
returns the result of this function.

Most of them use a classical prefixed functional syntax (i.e.
\texttt{operator\_name(operand1,\ operand2,\ operand3)}, see below),
with the exception of arithmetic (e.g. \texttt{+}, \texttt{/}), logical
(\texttt{and}, \texttt{or}), comparison (e.g. \texttt{\textgreater{}},
\texttt{\textless{}}), access (\texttt{.}, \texttt{{[}..{]}}) and pair
(\texttt{::}) operators, which require an infixed notation (i.e.
\texttt{operand1\ operator\_symbol\ operand1}).

The ternary functional if-else operator, \texttt{?\ :}, uses a special
infixed syntax composed with two symbols (e.g.
\texttt{operand1\ ?\ operand2\ :\ operand3}). Two unary operators
(\texttt{-} and \texttt{!}) use a traditional prefixed syntax that does
not require parentheses unless the operand is itself a complex
expression (e.g. \texttt{-\ 10}, \texttt{!\ (operand1\ or\ operand2)}).

Finally, special constructor operators (\texttt{\{...\}} for
constructing points, \texttt{{[}...{]}} for constructing lists and maps)
will require their operands to be placed between their two symbols (e.g.
\texttt{\{1,2,3\}}, \texttt{{[}operand1,\ operand2,\ ...,\ operandn{]}}
or \texttt{{[}key1::value1,\ key2::value2...\ keyn::valuen{]}}).

With the exception of these special cases above, the following rules
apply to the syntax of operators: * if they only have one operand, the
functional prefixed syntax is mandatory (e.g.
\texttt{operator\_name(operand1)}) * if they have two arguments, either
the functional prefixed syntax (e.g.
\texttt{operator\_name(operand1,\ operand2)}) or the infixed syntax
(e.g. \texttt{operand1\ operator\_name\ operand2}) can be used. * if
they have more than two arguments, either the functional prefixed syntax
(e.g. \texttt{operator\_name(operand1,\ operand2,\ ...,\ operand)}) or a
special infixed syntax with the first operand on the left-hand side of
the operator name (e.g.
\texttt{operand1\ operator\_name(operand2,\ ...,\ operand)}) can be
used.

All of these alternative syntaxes are completely equivalent.

Operators in GAML are purely functional, i.e.~they are guaranteed to not
have any side effects on their operands. For instance, the
\texttt{shuffle} operator, which randomizes the positions of elements in
a list, does not modify its list operand but returns a new shuffled
list.

\section{\texorpdfstring{}{ }}\label{section}

\section{Priority between operators}\label{priority-between-operators}

The priority of operators determines, in the case of complex expressions
composed of several operators, which one(s) will be evaluated first.

GAML follows in general the traditional priorities attributed to
arithmetic, boolean, comparison operators, with some twists. Namely: *
the constructor operators, like \texttt{::}, used to compose pairs of
operands, have the lowest priority of all operators (e.g.
\texttt{a\ \textgreater{}\ b\ ::\ b\ \textgreater{}\ c} will return a
pair of boolean values, which means that the two comparisons are
evaluated before the operator applies. Similarly,
\texttt{{[}a\ \textgreater{}\ 10,\ b\ \textgreater{}\ 5{]}} will return
a list of boolean values. * it is followed by the \texttt{?:} operator,
the functional if-else (e.g.
\texttt{a\ \textgreater{}\ b\ ?\ a\ +\ 10\ :\ a\ -\ 10} will return the
result of the if-else). * next are the logical operators, \texttt{and}
and \texttt{or} (e.g.
\texttt{a\ \textgreater{}\ b\ or\ b\ \textgreater{}\ c} will return the
value of the test) * next are the comparison operators (i.e.
\texttt{\textgreater{}}, \texttt{\textless{}}, \texttt{\textless{}=},
\texttt{\textgreater{}=}, \texttt{=}, \texttt{!=}) * next the arithmetic
operators in their logical order (multiplicative operators have a higher
priority than additive operators) * next the unary operators \texttt{-}
and \texttt{!} * next the access operators \texttt{.} and
\texttt{{[}{]}} (e.g.
\texttt{\{1,2,3\}.x\ \textgreater{}\ 20\ +\ \{4,5,6\}.y} will return the
result of the comparison between the x and y ordinates of the two
points) * and finally the functional operators, which have the highest
priority of all.

\begin{center}\rule{0.5\linewidth}{\linethickness}\end{center}

\section{Using actions as operators}\label{using-actions-as-operators}

Actions defined in species can be used as operators, provided they are
called on the correct agent. The syntax is that of normal functional
operators, but the agent that will perform the action must be added as
the first operand.

For instance, if the following species is defined:

\begin{verbatim}
species spec1 {
        int min(int x, int y) {
                return x > y ? x : y;
        }
}
\end{verbatim}

Any agent instance of spec1 can use \texttt{min} as an operator (if the
action conflicts with an existing operator, a warning will be emitted).
For instance, in the same model, the following line is perfectly
acceptable:

\begin{verbatim}
global {
        init {
                create spec1;
                spec1 my_agent <- spec1[0];
                int the_min <- my_agent min(10,20); // or min(my_agent, 10, 20);
        }
}
\end{verbatim}

If the action doesn't have any operands, the syntax to use is
\texttt{my\_agent\ the\_action()}. Finally, if it does not return a
value, it might still be used but is considering as returning a value of
type \texttt{unknown} (e.g.
\texttt{unknown\ result\ \textless{}-\ my\_agent\ the\_action(op1,\ op2);}).

Note that due to the fact that actions are written by modelers, the
general functional contract is not respected in that case: actions might
perfectly have side effects on their operands (including the agent).

\begin{center}\rule{0.5\linewidth}{\linethickness}\end{center}

\section{Table of Contents}\label{table-of-contents}

\begin{center}\rule{0.5\linewidth}{\linethickness}\end{center}

\section{Operators by categories}\label{operators-by-categories}

\begin{center}\rule{0.5\linewidth}{\linethickness}\end{center}

\subsection{3D}\label{d}

\href{OperatorsBC\#box}{box}, \href{OperatorsBC\#cone3d}{cone3D},
\href{OperatorsBC\#cube}{cube}, \href{OperatorsBC\#cylinder}{cylinder},
\href{OperatorsDH\#dem}{dem}, \href{OperatorsDH\#hexagon}{hexagon},
\href{OperatorsNR\#pyramid}{pyramid},
\href{OperatorsNR\#rgb_to_xyz}{rgb\_to\_xyz},
\href{OperatorsSZ\#set_z}{set\_z}, \href{OperatorsSZ\#sphere}{sphere},
\href{OperatorsSZ\#teapot}{teapot},

\begin{center}\rule{0.5\linewidth}{\linethickness}\end{center}

\subsection{Arithmetic operators}\label{arithmetic-operators}

\href{OperatorsAA\#-}{-}, \href{OperatorsAA\#/}{/},
{[}\textsuperscript{{]}(OperatorsAA\#}), \href{OperatorsAA\#*}{*},
\href{OperatorsAA\#+}{+}, \href{OperatorsAA\#abs}{abs},
\href{OperatorsAA\#acos}{acos}, \href{OperatorsAA\#asin}{asin},
\href{OperatorsAA\#atan}{atan}, \href{OperatorsAA\#atan2}{atan2},
\href{OperatorsBC\#ceil}{ceil}, \href{OperatorsBC\#cos}{cos},
\href{OperatorsBC\#cos_rad}{cos\_rad}, \href{OperatorsDH\#div}{div},
\href{OperatorsDH\#even}{even}, \href{OperatorsDH\#exp}{exp},
\href{OperatorsDH\#fact}{fact}, \href{OperatorsDH\#floor}{floor},
\href{OperatorsDH\#hypot}{hypot},
\href{OperatorsIM\#is_finite}{is\_finite},
\href{OperatorsIM\#is_number}{is\_number}, \href{OperatorsIM\#ln}{ln},
\href{OperatorsIM\#log}{log}, \href{OperatorsIM\#mod}{mod},
\href{OperatorsNR\#round}{round}, \href{OperatorsSZ\#signum}{signum},
\href{OperatorsSZ\#sin}{sin}, \href{OperatorsSZ\#sin_rad}{sin\_rad},
\href{OperatorsSZ\#sqrt}{sqrt}, \href{OperatorsSZ\#tan}{tan},
\href{OperatorsSZ\#tan_rad}{tan\_rad}, \href{OperatorsSZ\#tanh}{tanh},
\href{OperatorsSZ\#with_precision}{with\_precision},

\begin{center}\rule{0.5\linewidth}{\linethickness}\end{center}

\subsection{BDI}\label{bdi}

\href{OperatorsAA\#and}{and}, \href{OperatorsDH\#eval_when}{eval\_when},
\href{OperatorsDH\#get_about}{get\_about},
\href{OperatorsDH\#get_agent}{get\_agent},
\href{OperatorsDH\#get_agent_cause}{get\_agent\_cause},
\href{OperatorsDH\#get_belief_op}{get\_belief\_op},
\href{OperatorsDH\#get_belief_with_name_op}{get\_belief\_with\_name\_op},
\href{OperatorsDH\#get_beliefs_op}{get\_beliefs\_op},
\href{OperatorsDH\#get_beliefs_with_name_op}{get\_beliefs\_with\_name\_op},
\href{OperatorsDH\#get_current_intention_op}{get\_current\_intention\_op},
\href{OperatorsDH\#get_decay}{get\_decay},
\href{OperatorsDH\#get_desire_op}{get\_desire\_op},
\href{OperatorsDH\#get_desire_with_name_op}{get\_desire\_with\_name\_op},
\href{OperatorsDH\#get_desires_op}{get\_desires\_op},
\href{OperatorsDH\#get_desires_with_name_op}{get\_desires\_with\_name\_op},
\href{OperatorsDH\#get_dominance}{get\_dominance},
\href{OperatorsDH\#get_familiarity}{get\_familiarity},
\href{OperatorsDH\#get_ideal_op}{get\_ideal\_op},
\href{OperatorsDH\#get_ideal_with_name_op}{get\_ideal\_with\_name\_op},
\href{OperatorsDH\#get_ideals_op}{get\_ideals\_op},
\href{OperatorsDH\#get_ideals_with_name_op}{get\_ideals\_with\_name\_op},
\href{OperatorsDH\#get_intensity}{get\_intensity},
\href{OperatorsDH\#get_intention_op}{get\_intention\_op},
\href{OperatorsDH\#get_intention_with_name_op}{get\_intention\_with\_name\_op},
\href{OperatorsDH\#get_intentions_op}{get\_intentions\_op},
\href{OperatorsDH\#get_intentions_with_name_op}{get\_intentions\_with\_name\_op},
\href{OperatorsDH\#get_lifetime}{get\_lifetime},
\href{OperatorsDH\#get_liking}{get\_liking},
\href{OperatorsDH\#get_modality}{get\_modality},
\href{OperatorsDH\#get_obligation_op}{get\_obligation\_op},
\href{OperatorsDH\#get_obligation_with_name_op}{get\_obligation\_with\_name\_op},
\href{OperatorsDH\#get_obligations_op}{get\_obligations\_op},
\href{OperatorsDH\#get_obligations_with_name_op}{get\_obligations\_with\_name\_op},
\href{OperatorsDH\#get_plan_name}{get\_plan\_name},
\href{OperatorsDH\#get_predicate}{get\_predicate},
\href{OperatorsDH\#get_solidarity}{get\_solidarity},
\href{OperatorsDH\#get_strength}{get\_strength},
\href{OperatorsDH\#get_super_intention}{get\_super\_intention},
\href{OperatorsDH\#get_trust}{get\_trust},
\href{OperatorsDH\#get_truth}{get\_truth},
\href{OperatorsDH\#get_uncertainties_op}{get\_uncertainties\_op},
\href{OperatorsDH\#get_uncertainties_with_name_op}{get\_uncertainties\_with\_name\_op},
\href{OperatorsDH\#get_uncertainty_op}{get\_uncertainty\_op},
\href{OperatorsDH\#get_uncertainty_with_name_op}{get\_uncertainty\_with\_name\_op},
\href{OperatorsDH\#has_belief_op}{has\_belief\_op},
\href{OperatorsDH\#has_belief_with_name_op}{has\_belief\_with\_name\_op},
\href{OperatorsDH\#has_desire_op}{has\_desire\_op},
\href{OperatorsDH\#has_desire_with_name_op}{has\_desire\_with\_name\_op},
\href{OperatorsDH\#has_ideal_op}{has\_ideal\_op},
\href{OperatorsDH\#has_ideal_with_name_op}{has\_ideal\_with\_name\_op},
\href{OperatorsDH\#has_intention_op}{has\_intention\_op},
\href{OperatorsDH\#has_intention_with_name_op}{has\_intention\_with\_name\_op},
\href{OperatorsDH\#has_obligation_op}{has\_obligation\_op},
\href{OperatorsDH\#has_obligation_with_name_op}{has\_obligation\_with\_name\_op},
\href{OperatorsDH\#has_uncertainty_op}{has\_uncertainty\_op},
\href{OperatorsDH\#has_uncertainty_with_name_op}{has\_uncertainty\_with\_name\_op},
\href{OperatorsNR\#new_emotion}{new\_emotion},
\href{OperatorsNR\#new_mental_state}{new\_mental\_state},
\href{OperatorsNR\#new_predicate}{new\_predicate},
\href{OperatorsNR\#new_social_link}{new\_social\_link},
\href{OperatorsNR\#or}{or}, \href{OperatorsSZ\#set_about}{set\_about},
\href{OperatorsSZ\#set_agent}{set\_agent},
\href{OperatorsSZ\#set_agent_cause}{set\_agent\_cause},
\href{OperatorsSZ\#set_decay}{set\_decay},
\href{OperatorsSZ\#set_dominance}{set\_dominance},
\href{OperatorsSZ\#set_familiarity}{set\_familiarity},
\href{OperatorsSZ\#set_intensity}{set\_intensity},
\href{OperatorsSZ\#set_lifetime}{set\_lifetime},
\href{OperatorsSZ\#set_liking}{set\_liking},
\href{OperatorsSZ\#set_modality}{set\_modality},
\href{OperatorsSZ\#set_predicate}{set\_predicate},
\href{OperatorsSZ\#set_solidarity}{set\_solidarity},
\href{OperatorsSZ\#set_strength}{set\_strength},
\href{OperatorsSZ\#set_trust}{set\_trust},
\href{OperatorsSZ\#set_truth}{set\_truth},
\href{OperatorsSZ\#with_lifetime}{with\_lifetime},
\href{OperatorsSZ\#with_values}{with\_values},

\begin{center}\rule{0.5\linewidth}{\linethickness}\end{center}

\subsection{Casting operators}\label{casting-operators}

\href{OperatorsAA\#as}{as}, \href{OperatorsAA\#as_int}{as\_int},
\href{OperatorsAA\#as_matrix}{as\_matrix},
\href{OperatorsDH\#font}{font}, \href{OperatorsIM\#is}{is},
\href{OperatorsIM\#is_skill}{is\_skill},
\href{OperatorsIM\#list_with}{list\_with},
\href{OperatorsIM\#matrix_with}{matrix\_with},
\href{OperatorsSZ\#species}{species},
\href{OperatorsSZ\#to_gaml}{to\_gaml},
\href{OperatorsSZ\#topology}{topology},

\begin{center}\rule{0.5\linewidth}{\linethickness}\end{center}

\subsection{Color-related operators}\label{color-related-operators}

\href{OperatorsAA\#-}{-}, \href{OperatorsAA\#/}{/},
\href{OperatorsAA\#*}{*}, \href{OperatorsAA\#+}{+},
\href{OperatorsBC\#blend}{blend},
\href{OperatorsBC\#brewer_colors}{brewer\_colors},
\href{OperatorsBC\#brewer_palettes}{brewer\_palettes},
\href{OperatorsDH\#grayscale}{grayscale}, \href{OperatorsDH\#hsb}{hsb},
\href{OperatorsIM\#mean}{mean}, \href{OperatorsIM\#median}{median},
\href{OperatorsNR\#rgb}{rgb}, \href{OperatorsNR\#rnd_color}{rnd\_color},
\href{OperatorsSZ\#sum}{sum},

\begin{center}\rule{0.5\linewidth}{\linethickness}\end{center}

\subsection{Comparison operators}\label{comparison-operators}

\href{OperatorsAA\#!=}{!=}, \href{OperatorsAA\#\%3C}{\textless{}},
\href{OperatorsAA\#\%3C=}{\textless{}=}, \href{OperatorsAA\#=}{=},
\href{OperatorsAA\#\%3E}{\textgreater{}},
\href{OperatorsAA\#\%3E=}{\textgreater{}=},
\href{OperatorsBC\#between}{between},

\begin{center}\rule{0.5\linewidth}{\linethickness}\end{center}

\subsection{Containers-related
operators}\label{containers-related-operators}

\href{OperatorsAA\#-}{-}, \href{OperatorsAA\#::}{::},
\href{OperatorsAA\#+}{+}, \href{OperatorsAA\#accumulate}{accumulate},
\href{OperatorsAA\#among}{among}, \href{OperatorsAA\#at}{at},
\href{OperatorsBC\#collect}{collect},
\href{OperatorsBC\#contains}{contains},
\href{OperatorsBC\#contains_all}{contains\_all},
\href{OperatorsBC\#contains_any}{contains\_any},
\href{OperatorsBC\#count}{count},
\href{OperatorsDH\#distinct}{distinct},
\href{OperatorsDH\#empty}{empty}, \href{OperatorsDH\#every}{every},
\href{OperatorsDH\#first}{first},
\href{OperatorsDH\#first_with}{first\_with},
\href{OperatorsDH\#get}{get}, \href{OperatorsDH\#group_by}{group\_by},
\href{OperatorsIM\#in}{in}, \href{OperatorsIM\#index_by}{index\_by},
\href{OperatorsIM\#inter}{inter},
\href{OperatorsIM\#interleave}{interleave},
\href{OperatorsIM\#internal_at}{internal\_at},
\href{OperatorsIM\#internal_integrated_value}{internal\_integrated\_value},
\href{OperatorsIM\#last}{last},
\href{OperatorsIM\#last_with}{last\_with},
\href{OperatorsIM\#length}{length}, \href{OperatorsIM\#max}{max},
\href{OperatorsIM\#max_of}{max\_of}, \href{OperatorsIM\#mean}{mean},
\href{OperatorsIM\#mean_of}{mean\_of},
\href{OperatorsIM\#median}{median}, \href{OperatorsIM\#min}{min},
\href{OperatorsIM\#min_of}{min\_of}, \href{OperatorsIM\#mul}{mul},
\href{OperatorsNR\#one_of}{one\_of},
\href{OperatorsNR\#product_of}{product\_of},
\href{OperatorsNR\#range}{range}, \href{OperatorsNR\#reverse}{reverse},
\href{OperatorsSZ\#shuffle}{shuffle},
\href{OperatorsSZ\#sort_by}{sort\_by}, \href{OperatorsSZ\#split}{split},
\href{OperatorsSZ\#split_in}{split\_in},
\href{OperatorsSZ\#split_using}{split\_using},
\href{OperatorsSZ\#sum}{sum}, \href{OperatorsSZ\#sum_of}{sum\_of},
\href{OperatorsSZ\#union}{union},
\href{OperatorsSZ\#variance_of}{variance\_of},
\href{OperatorsSZ\#where}{where},
\href{OperatorsSZ\#with_max_of}{with\_max\_of},
\href{OperatorsSZ\#with_min_of}{with\_min\_of},

\begin{center}\rule{0.5\linewidth}{\linethickness}\end{center}

\subsection{Date-related operators}\label{date-related-operators}

\href{OperatorsAA\#-}{-}, \href{OperatorsAA\#!=}{!=},
\href{OperatorsAA\#+}{+}, \href{OperatorsAA\#\%3C}{\textless{}},
\href{OperatorsAA\#\%3C=}{\textless{}=}, \href{OperatorsAA\#=}{=},
\href{OperatorsAA\#\%3E}{\textgreater{}},
\href{OperatorsAA\#\%3E=}{\textgreater{}=},
\href{OperatorsAA\#after}{after}, \href{OperatorsBC\#before}{before},
\href{OperatorsBC\#between}{between}, \href{OperatorsDH\#every}{every},
\href{OperatorsIM\#milliseconds_between}{milliseconds\_between},
\href{OperatorsIM\#minus_days}{minus\_days},
\href{OperatorsIM\#minus_hours}{minus\_hours},
\href{OperatorsIM\#minus_minutes}{minus\_minutes},
\href{OperatorsIM\#minus_months}{minus\_months},
\href{OperatorsIM\#minus_ms}{minus\_ms},
\href{OperatorsIM\#minus_weeks}{minus\_weeks},
\href{OperatorsIM\#minus_years}{minus\_years},
\href{OperatorsIM\#months_between}{months\_between},
\href{OperatorsNR\#plus_days}{plus\_days},
\href{OperatorsNR\#plus_hours}{plus\_hours},
\href{OperatorsNR\#plus_minutes}{plus\_minutes},
\href{OperatorsNR\#plus_months}{plus\_months},
\href{OperatorsNR\#plus_ms}{plus\_ms},
\href{OperatorsNR\#plus_weeks}{plus\_weeks},
\href{OperatorsNR\#plus_years}{plus\_years},
\href{OperatorsSZ\#since}{since}, \href{OperatorsSZ\#to}{to},
\href{OperatorsSZ\#until}{until},
\href{OperatorsSZ\#years_between}{years\_between},

\begin{center}\rule{0.5\linewidth}{\linethickness}\end{center}

\subsection{Dates}\label{dates}

\begin{center}\rule{0.5\linewidth}{\linethickness}\end{center}

\subsection{DescriptiveStatistics}\label{descriptivestatistics}

\href{OperatorsAA\#auto_correlation}{auto\_correlation},
\href{OperatorsBC\#correlation}{correlation},
\href{OperatorsBC\#covariance}{covariance},
\href{OperatorsDH\#durbin_watson}{durbin\_watson},
\href{OperatorsIM\#kurtosis}{kurtosis},
\href{OperatorsIM\#moment}{moment},
\href{OperatorsNR\#quantile}{quantile},
\href{OperatorsNR\#quantile_inverse}{quantile\_inverse},
\href{OperatorsNR\#rank_interpolated}{rank\_interpolated},
\href{OperatorsNR\#rms}{rms}, \href{OperatorsSZ\#skew}{skew},
\href{OperatorsSZ\#variance}{variance},

\begin{center}\rule{0.5\linewidth}{\linethickness}\end{center}

\subsection{Displays}\label{displays}

\href{OperatorsDH\#horizontal}{horizontal},
\href{OperatorsSZ\#stack}{stack},
\href{OperatorsSZ\#vertical}{vertical},

\begin{center}\rule{0.5\linewidth}{\linethickness}\end{center}

\subsection{Distributions}\label{distributions}

\href{OperatorsBC\#binomial_coeff}{binomial\_coeff},
\href{OperatorsBC\#binomial_complemented}{binomial\_complemented},
\href{OperatorsBC\#binomial_sum}{binomial\_sum},
\href{OperatorsBC\#chi_square}{chi\_square},
\href{OperatorsBC\#chi_square_complemented}{chi\_square\_complemented},
\href{OperatorsDH\#gamma_distribution}{gamma\_distribution},
\href{OperatorsDH\#gamma_distribution_complemented}{gamma\_distribution\_complemented},
\href{OperatorsNR\#normal_area}{normal\_area},
\href{OperatorsNR\#normal_density}{normal\_density},
\href{OperatorsNR\#normal_inverse}{normal\_inverse},
\href{OperatorsNR\#pvalue_for_fstat}{pValue\_for\_fStat},
\href{OperatorsNR\#pvalue_for_tstat}{pValue\_for\_tStat},
\href{OperatorsSZ\#student_area}{student\_area},
\href{OperatorsSZ\#student_t_inverse}{student\_t\_inverse},

\begin{center}\rule{0.5\linewidth}{\linethickness}\end{center}

\subsection{Driving operators}\label{driving-operators}

\href{OperatorsAA\#as_driving_graph}{as\_driving\_graph},

\begin{center}\rule{0.5\linewidth}{\linethickness}\end{center}

\subsection{edge}\label{edge}

\href{OperatorsDH\#edge_between}{edge\_between},
\href{OperatorsSZ\#strahler}{strahler},

\begin{center}\rule{0.5\linewidth}{\linethickness}\end{center}

\subsection{EDP-related operators}\label{edp-related-operators}

\href{OperatorsDH\#diff}{diff}, \href{OperatorsDH\#diff2}{diff2},
\href{OperatorsIM\#internal_zero_order_equation}{internal\_zero\_order\_equation},

\begin{center}\rule{0.5\linewidth}{\linethickness}\end{center}

\subsection{Files-related operators}\label{files-related-operators}

\href{OperatorsBC\#crs}{crs},
\href{OperatorsDH\#evaluate_sub_model}{evaluate\_sub\_model},
\href{OperatorsDH\#file}{file},
\href{OperatorsDH\#file_exists}{file\_exists},
\href{OperatorsDH\#folder}{folder}, \href{OperatorsDH\#get}{get},
\href{OperatorsIM\#load_sub_model}{load\_sub\_model},
\href{OperatorsNR\#new_folder}{new\_folder},
\href{OperatorsNR\#osm_file}{osm\_file}, \href{OperatorsNR\#read}{read},
\href{OperatorsSZ\#step_sub_model}{step\_sub\_model},
\href{OperatorsSZ\#writable}{writable},

\begin{center}\rule{0.5\linewidth}{\linethickness}\end{center}

\subsection{FIPA-related operators}\label{fipa-related-operators}

\href{OperatorsBC\#conversation}{conversation},
\href{OperatorsIM\#message}{message},

\begin{center}\rule{0.5\linewidth}{\linethickness}\end{center}

\subsection{GamaMetaType}\label{gamametatype}

\href{OperatorsSZ\#type_of}{type\_of},

\begin{center}\rule{0.5\linewidth}{\linethickness}\end{center}

\subsection{GammaFunction}\label{gammafunction}

\href{OperatorsBC\#beta}{beta}, \href{OperatorsDH\#gamma}{gamma},
\href{OperatorsIM\#incomplete_beta}{incomplete\_beta},
\href{OperatorsIM\#incomplete_gamma}{incomplete\_gamma},
\href{OperatorsIM\#incomplete_gamma_complement}{incomplete\_gamma\_complement},
\href{OperatorsIM\#log_gamma}{log\_gamma},

\begin{center}\rule{0.5\linewidth}{\linethickness}\end{center}

\subsection{Graphs-related operators}\label{graphs-related-operators}

\href{OperatorsAA\#add_edge}{add\_edge},
\href{OperatorsAA\#add_node}{add\_node},
\href{OperatorsAA\#adjacency}{adjacency},
\href{OperatorsAA\#agent_from_geometry}{agent\_from\_geometry},
\href{OperatorsAA\#all_pairs_shortest_path}{all\_pairs\_shortest\_path},
\href{OperatorsAA\#alpha_index}{alpha\_index},
\href{OperatorsAA\#as_distance_graph}{as\_distance\_graph},
\href{OperatorsAA\#as_edge_graph}{as\_edge\_graph},
\href{OperatorsAA\#as_intersection_graph}{as\_intersection\_graph},
\href{OperatorsAA\#as_path}{as\_path},
\href{OperatorsBC\#beta_index}{beta\_index},
\href{OperatorsBC\#betweenness_centrality}{betweenness\_centrality},
\href{OperatorsBC\#biggest_cliques_of}{biggest\_cliques\_of},
\href{OperatorsBC\#connected_components_of}{connected\_components\_of},
\href{OperatorsBC\#connectivity_index}{connectivity\_index},
\href{OperatorsBC\#contains_edge}{contains\_edge},
\href{OperatorsBC\#contains_vertex}{contains\_vertex},
\href{OperatorsDH\#degree_of}{degree\_of},
\href{OperatorsDH\#directed}{directed}, \href{OperatorsDH\#edge}{edge},
\href{OperatorsDH\#edge_between}{edge\_between},
\href{OperatorsDH\#edge_betweenness}{edge\_betweenness},
\href{OperatorsDH\#edges}{edges},
\href{OperatorsDH\#gamma_index}{gamma\_index},
\href{OperatorsDH\#generate_barabasi_albert}{generate\_barabasi\_albert},
\href{OperatorsDH\#generate_complete_graph}{generate\_complete\_graph},
\href{OperatorsDH\#generate_watts_strogatz}{generate\_watts\_strogatz},
\href{OperatorsDH\#grid_cells_to_graph}{grid\_cells\_to\_graph},
\href{OperatorsIM\#in_degree_of}{in\_degree\_of},
\href{OperatorsIM\#in_edges_of}{in\_edges\_of},
\href{OperatorsIM\#layout}{layout},
\href{OperatorsIM\#load_graph_from_file}{load\_graph\_from\_file},
\href{OperatorsIM\#load_shortest_paths}{load\_shortest\_paths},
\href{OperatorsIM\#main_connected_component}{main\_connected\_component},
\href{OperatorsIM\#max_flow_between}{max\_flow\_between},
\href{OperatorsIM\#maximal_cliques_of}{maximal\_cliques\_of},
\href{OperatorsNR\#nb_cycles}{nb\_cycles},
\href{OperatorsNR\#neighbors_of}{neighbors\_of},
\href{OperatorsNR\#node}{node}, \href{OperatorsNR\#nodes}{nodes},
\href{OperatorsNR\#out_degree_of}{out\_degree\_of},
\href{OperatorsNR\#out_edges_of}{out\_edges\_of},
\href{OperatorsNR\#path_between}{path\_between},
\href{OperatorsNR\#paths_between}{paths\_between},
\href{OperatorsNR\#predecessors_of}{predecessors\_of},
\href{OperatorsNR\#remove_node_from}{remove\_node\_from},
\href{OperatorsNR\#rewire_n}{rewire\_n},
\href{OperatorsSZ\#source_of}{source\_of},
\href{OperatorsSZ\#spatial_graph}{spatial\_graph},
\href{OperatorsSZ\#strahler}{strahler},
\href{OperatorsSZ\#successors_of}{successors\_of},
\href{OperatorsSZ\#sum}{sum}, \href{OperatorsSZ\#target_of}{target\_of},
\href{OperatorsSZ\#undirected}{undirected},
\href{OperatorsSZ\#use_cache}{use\_cache},
\href{OperatorsSZ\#weight_of}{weight\_of},
\href{OperatorsSZ\#with_optimizer_type}{with\_optimizer\_type},
\href{OperatorsSZ\#with_weights}{with\_weights},

\begin{center}\rule{0.5\linewidth}{\linethickness}\end{center}

\subsection{Grid-related operators}\label{grid-related-operators}

\href{OperatorsAA\#as_4_grid}{as\_4\_grid},
\href{OperatorsAA\#as_grid}{as\_grid},
\href{OperatorsAA\#as_hexagonal_grid}{as\_hexagonal\_grid},
\href{OperatorsDH\#grid_at}{grid\_at},
\href{OperatorsNR\#path_between}{path\_between},

\begin{center}\rule{0.5\linewidth}{\linethickness}\end{center}

\subsection{Iterator operators}\label{iterator-operators}

\href{OperatorsAA\#accumulate}{accumulate},
\href{OperatorsAA\#as_map}{as\_map},
\href{OperatorsBC\#collect}{collect}, \href{OperatorsBC\#count}{count},
\href{OperatorsBC\#create_map}{create\_map},
\href{OperatorsDH\#distribution_of}{distribution\_of},
\href{OperatorsDH\#distribution_of}{distribution\_of},
\href{OperatorsDH\#distribution_of}{distribution\_of},
\href{OperatorsDH\#distribution2d_of}{distribution2d\_of},
\href{OperatorsDH\#distribution2d_of}{distribution2d\_of},
\href{OperatorsDH\#distribution2d_of}{distribution2d\_of},
\href{OperatorsDH\#first_with}{first\_with},
\href{OperatorsDH\#frequency_of}{frequency\_of},
\href{OperatorsDH\#group_by}{group\_by},
\href{OperatorsIM\#index_by}{index\_by},
\href{OperatorsIM\#last_with}{last\_with},
\href{OperatorsIM\#max_of}{max\_of},
\href{OperatorsIM\#mean_of}{mean\_of},
\href{OperatorsIM\#min_of}{min\_of},
\href{OperatorsNR\#product_of}{product\_of},
\href{OperatorsSZ\#sort_by}{sort\_by},
\href{OperatorsSZ\#sum_of}{sum\_of},
\href{OperatorsSZ\#variance_of}{variance\_of},
\href{OperatorsSZ\#where}{where},
\href{OperatorsSZ\#with_max_of}{with\_max\_of},
\href{OperatorsSZ\#with_min_of}{with\_min\_of},

\begin{center}\rule{0.5\linewidth}{\linethickness}\end{center}

\subsection{List-related operators}\label{list-related-operators}

\href{OperatorsBC\#copy_between}{copy\_between},
\href{OperatorsIM\#index_of}{index\_of},
\href{OperatorsIM\#last_index_of}{last\_index\_of},

\begin{center}\rule{0.5\linewidth}{\linethickness}\end{center}

\subsection{Logical operators}\label{logical-operators}

\href{OperatorsAA\#:}{:}, \href{OperatorsAA\#!}{!},
\href{OperatorsAA\#?}{?}, \href{OperatorsAA\#and}{and},
\href{OperatorsNR\#or}{or}, \href{OperatorsSZ\#xor}{xor},

\begin{center}\rule{0.5\linewidth}{\linethickness}\end{center}

\subsection{Map comparaison operators}\label{map-comparaison-operators}

\href{OperatorsDH\#fuzzy_kappa}{fuzzy\_kappa},
\href{OperatorsDH\#fuzzy_kappa_sim}{fuzzy\_kappa\_sim},
\href{OperatorsIM\#kappa}{kappa},
\href{OperatorsIM\#kappa_sim}{kappa\_sim},
\href{OperatorsNR\#percent_absolute_deviation}{percent\_absolute\_deviation},

\begin{center}\rule{0.5\linewidth}{\linethickness}\end{center}

\subsection{Map-related operators}\label{map-related-operators}

\href{OperatorsAA\#as_map}{as\_map},
\href{OperatorsBC\#create_map}{create\_map},
\href{OperatorsIM\#index_of}{index\_of},
\href{OperatorsIM\#last_index_of}{last\_index\_of},

\begin{center}\rule{0.5\linewidth}{\linethickness}\end{center}

\subsection{Material}\label{material}

\href{OperatorsIM\#material}{material},

\begin{center}\rule{0.5\linewidth}{\linethickness}\end{center}

\subsection{Matrix-related operators}\label{matrix-related-operators}

\href{OperatorsAA\#-}{-}, \href{OperatorsAA\#/}{/},
\href{OperatorsAA\#.}{.}, \href{OperatorsAA\#*}{*},
\href{OperatorsAA\#+}{+},
\href{OperatorsAA\#append_horizontally}{append\_horizontally},
\href{OperatorsAA\#append_vertically}{append\_vertically},
\href{OperatorsBC\#column_at}{column\_at},
\href{OperatorsBC\#columns_list}{columns\_list},
\href{OperatorsDH\#determinant}{determinant},
\href{OperatorsDH\#eigenvalues}{eigenvalues},
\href{OperatorsIM\#index_of}{index\_of},
\href{OperatorsIM\#inverse}{inverse},
\href{OperatorsIM\#last_index_of}{last\_index\_of},
\href{OperatorsNR\#row_at}{row\_at},
\href{OperatorsNR\#rows_list}{rows\_list},
\href{OperatorsSZ\#shuffle}{shuffle}, \href{OperatorsSZ\#trace}{trace},
\href{OperatorsSZ\#transpose}{transpose},

\begin{center}\rule{0.5\linewidth}{\linethickness}\end{center}

\subsection{multicriteria operators}\label{multicriteria-operators}

\href{OperatorsDH\#electre_dm}{electre\_DM},
\href{OperatorsDH\#evidence_theory_dm}{evidence\_theory\_DM},
\href{OperatorsDH\#fuzzy_choquet_dm}{fuzzy\_choquet\_DM},
\href{OperatorsNR\#promethee_dm}{promethee\_DM},
\href{OperatorsSZ\#weighted_means_dm}{weighted\_means\_DM},

\begin{center}\rule{0.5\linewidth}{\linethickness}\end{center}

\subsection{Path-related operators}\label{path-related-operators}

\href{OperatorsAA\#agent_from_geometry}{agent\_from\_geometry},
\href{OperatorsAA\#all_pairs_shortest_path}{all\_pairs\_shortest\_path},
\href{OperatorsAA\#as_path}{as\_path},
\href{OperatorsIM\#load_shortest_paths}{load\_shortest\_paths},
\href{OperatorsIM\#max_flow_between}{max\_flow\_between},
\href{OperatorsNR\#path_between}{path\_between},
\href{OperatorsNR\#path_to}{path\_to},
\href{OperatorsNR\#paths_between}{paths\_between},
\href{OperatorsSZ\#use_cache}{use\_cache},

\begin{center}\rule{0.5\linewidth}{\linethickness}\end{center}

\subsection{Points-related operators}\label{points-related-operators}

\href{OperatorsAA\#-}{-}, \href{OperatorsAA\#/}{/},
\href{OperatorsAA\#*}{*}, \href{OperatorsAA\#+}{+},
\href{OperatorsAA\#\%3C}{\textless{}},
\href{OperatorsAA\#\%3C=}{\textless{}=},
\href{OperatorsAA\#\%3E}{\textgreater{}},
\href{OperatorsAA\#\%3E=}{\textgreater{}=},
\href{OperatorsAA\#add_point}{add\_point},
\href{OperatorsAA\#angle_between}{angle\_between},
\href{OperatorsAA\#any_location_in}{any\_location\_in},
\href{OperatorsBC\#centroid}{centroid},
\href{OperatorsBC\#closest_points_with}{closest\_points\_with},
\href{OperatorsDH\#farthest_point_to}{farthest\_point\_to},
\href{OperatorsDH\#grid_at}{grid\_at}, \href{OperatorsNR\#norm}{norm},
\href{OperatorsNR\#point}{point},
\href{OperatorsNR\#points_along}{points\_along},
\href{OperatorsNR\#points_at}{points\_at},
\href{OperatorsNR\#points_on}{points\_on},

\begin{center}\rule{0.5\linewidth}{\linethickness}\end{center}

\subsection{Random operators}\label{random-operators}

\href{OperatorsBC\#binomial}{binomial}, \href{OperatorsDH\#flip}{flip},
\href{OperatorsDH\#gauss}{gauss},
\href{OperatorsIM\#improved_generator}{improved\_generator},
\href{OperatorsNR\#open_simplex_generator}{open\_simplex\_generator},
\href{OperatorsNR\#poisson}{poisson}, \href{OperatorsNR\#rnd}{rnd},
\href{OperatorsNR\#rnd_choice}{rnd\_choice},
\href{OperatorsSZ\#sample}{sample},
\href{OperatorsSZ\#shuffle}{shuffle},
\href{OperatorsSZ\#simplex_generator}{simplex\_generator},
\href{OperatorsSZ\#skew_gauss}{skew\_gauss},
\href{OperatorsSZ\#truncated_gauss}{truncated\_gauss},

\begin{center}\rule{0.5\linewidth}{\linethickness}\end{center}

\subsection{ReverseOperators}\label{reverseoperators}

\href{OperatorsSZ\#savesimulation}{saveSimulation},
\href{OperatorsSZ\#serialize}{serialize},
\href{OperatorsSZ\#serializeagent}{serializeAgent},
\href{OperatorsSZ\#unserializesimulation}{unSerializeSimulation},
\href{OperatorsSZ\#unserializesimulationfromfile}{unSerializeSimulationFromFile},

\begin{center}\rule{0.5\linewidth}{\linethickness}\end{center}

\subsection{Shape}\label{shape}

\href{OperatorsAA\#arc}{arc}, \href{OperatorsBC\#box}{box},
\href{OperatorsBC\#circle}{circle}, \href{OperatorsBC\#cone}{cone},
\href{OperatorsBC\#cone3d}{cone3D}, \href{OperatorsBC\#cross}{cross},
\href{OperatorsBC\#cube}{cube}, \href{OperatorsBC\#curve}{curve},
\href{OperatorsBC\#cylinder}{cylinder},
\href{OperatorsDH\#ellipse}{ellipse},
\href{OperatorsDH\#envelope}{envelope},
\href{OperatorsDH\#geometry_collection}{geometry\_collection},
\href{OperatorsDH\#hexagon}{hexagon}, \href{OperatorsIM\#line}{line},
\href{OperatorsIM\#link}{link}, \href{OperatorsNR\#plan}{plan},
\href{OperatorsNR\#polygon}{polygon},
\href{OperatorsNR\#polyhedron}{polyhedron},
\href{OperatorsNR\#pyramid}{pyramid},
\href{OperatorsNR\#rectangle}{rectangle},
\href{OperatorsSZ\#sphere}{sphere}, \href{OperatorsSZ\#square}{square},
\href{OperatorsSZ\#squircle}{squircle},
\href{OperatorsSZ\#teapot}{teapot},
\href{OperatorsSZ\#triangle}{triangle},

\begin{center}\rule{0.5\linewidth}{\linethickness}\end{center}

\subsection{Spatial operators}\label{spatial-operators}

\href{OperatorsAA\#-}{-}, \href{OperatorsAA\#*}{*},
\href{OperatorsAA\#+}{+}, \href{OperatorsAA\#add_point}{add\_point},
\href{OperatorsAA\#agent_closest_to}{agent\_closest\_to},
\href{OperatorsAA\#agent_farthest_to}{agent\_farthest\_to},
\href{OperatorsAA\#agents_at_distance}{agents\_at\_distance},
\href{OperatorsAA\#agents_inside}{agents\_inside},
\href{OperatorsAA\#agents_overlapping}{agents\_overlapping},
\href{OperatorsAA\#angle_between}{angle\_between},
\href{OperatorsAA\#any_location_in}{any\_location\_in},
\href{OperatorsAA\#arc}{arc}, \href{OperatorsAA\#around}{around},
\href{OperatorsAA\#as_4_grid}{as\_4\_grid},
\href{OperatorsAA\#as_grid}{as\_grid},
\href{OperatorsAA\#as_hexagonal_grid}{as\_hexagonal\_grid},
\href{OperatorsAA\#at_distance}{at\_distance},
\href{OperatorsAA\#at_location}{at\_location},
\href{OperatorsBC\#box}{box}, \href{OperatorsBC\#centroid}{centroid},
\href{OperatorsBC\#circle}{circle}, \href{OperatorsBC\#clean}{clean},
\href{OperatorsBC\#clean_network}{clean\_network},
\href{OperatorsBC\#closest_points_with}{closest\_points\_with},
\href{OperatorsBC\#closest_to}{closest\_to},
\href{OperatorsBC\#cone}{cone}, \href{OperatorsBC\#cone3d}{cone3D},
\href{OperatorsBC\#convex_hull}{convex\_hull},
\href{OperatorsBC\#covers}{covers}, \href{OperatorsBC\#cross}{cross},
\href{OperatorsBC\#crosses}{crosses}, \href{OperatorsBC\#crs}{crs},
\href{OperatorsBC\#crs_transform}{CRS\_transform},
\href{OperatorsBC\#cube}{cube}, \href{OperatorsBC\#curve}{curve},
\href{OperatorsBC\#cylinder}{cylinder}, \href{OperatorsDH\#dem}{dem},
\href{OperatorsDH\#direction_between}{direction\_between},
\href{OperatorsDH\#disjoint_from}{disjoint\_from},
\href{OperatorsDH\#distance_between}{distance\_between},
\href{OperatorsDH\#distance_to}{distance\_to},
\href{OperatorsDH\#ellipse}{ellipse},
\href{OperatorsDH\#envelope}{envelope},
\href{OperatorsDH\#farthest_point_to}{farthest\_point\_to},
\href{OperatorsDH\#farthest_to}{farthest\_to},
\href{OperatorsDH\#geometry_collection}{geometry\_collection},
\href{OperatorsDH\#gini}{gini}, \href{OperatorsDH\#hexagon}{hexagon},
\href{OperatorsDH\#hierarchical_clustering}{hierarchical\_clustering},
\href{OperatorsIM\#idw}{IDW}, \href{OperatorsIM\#inside}{inside},
\href{OperatorsIM\#inter}{inter},
\href{OperatorsIM\#intersects}{intersects},
\href{OperatorsIM\#line}{line}, \href{OperatorsIM\#link}{link},
\href{OperatorsIM\#masked_by}{masked\_by},
\href{OperatorsIM\#moran}{moran},
\href{OperatorsNR\#neighbors_at}{neighbors\_at},
\href{OperatorsNR\#neighbors_of}{neighbors\_of},
\href{OperatorsNR\#overlapping}{overlapping},
\href{OperatorsNR\#overlaps}{overlaps},
\href{OperatorsNR\#partially_overlaps}{partially\_overlaps},
\href{OperatorsNR\#path_between}{path\_between},
\href{OperatorsNR\#path_to}{path\_to}, \href{OperatorsNR\#plan}{plan},
\href{OperatorsNR\#points_along}{points\_along},
\href{OperatorsNR\#points_at}{points\_at},
\href{OperatorsNR\#points_on}{points\_on},
\href{OperatorsNR\#polygon}{polygon},
\href{OperatorsNR\#polyhedron}{polyhedron},
\href{OperatorsNR\#pyramid}{pyramid},
\href{OperatorsNR\#rectangle}{rectangle},
\href{OperatorsNR\#rgb_to_xyz}{rgb\_to\_xyz},
\href{OperatorsNR\#rotated_by}{rotated\_by},
\href{OperatorsNR\#round}{round},
\href{OperatorsSZ\#scaled_to}{scaled\_to},
\href{OperatorsSZ\#set_z}{set\_z},
\href{OperatorsSZ\#simple_clustering_by_distance}{simple\_clustering\_by\_distance},
\href{OperatorsSZ\#simplification}{simplification},
\href{OperatorsSZ\#skeletonize}{skeletonize},
\href{OperatorsSZ\#smooth}{smooth}, \href{OperatorsSZ\#sphere}{sphere},
\href{OperatorsSZ\#split_at}{split\_at},
\href{OperatorsSZ\#split_geometry}{split\_geometry},
\href{OperatorsSZ\#split_lines}{split\_lines},
\href{OperatorsSZ\#square}{square},
\href{OperatorsSZ\#squircle}{squircle},
\href{OperatorsSZ\#teapot}{teapot},
\href{OperatorsSZ\#to_gama_crs}{to\_GAMA\_CRS},
\href{OperatorsSZ\#to_rectangles}{to\_rectangles},
\href{OperatorsSZ\#to_squares}{to\_squares},
\href{OperatorsSZ\#to_sub_geometries}{to\_sub\_geometries},
\href{OperatorsSZ\#touches}{touches},
\href{OperatorsSZ\#towards}{towards},
\href{OperatorsSZ\#transformed_by}{transformed\_by},
\href{OperatorsSZ\#translated_by}{translated\_by},
\href{OperatorsSZ\#triangle}{triangle},
\href{OperatorsSZ\#triangulate}{triangulate},
\href{OperatorsSZ\#union}{union}, \href{OperatorsSZ\#using}{using},
\href{OperatorsSZ\#voronoi}{voronoi},
\href{OperatorsSZ\#with_precision}{with\_precision},
\href{OperatorsSZ\#without_holes}{without\_holes},

\begin{center}\rule{0.5\linewidth}{\linethickness}\end{center}

\subsection{Spatial properties
operators}\label{spatial-properties-operators}

\href{OperatorsBC\#covers}{covers},
\href{OperatorsBC\#crosses}{crosses},
\href{OperatorsIM\#intersects}{intersects},
\href{OperatorsNR\#partially_overlaps}{partially\_overlaps},
\href{OperatorsSZ\#touches}{touches},

\begin{center}\rule{0.5\linewidth}{\linethickness}\end{center}

\subsection{Spatial queries operators}\label{spatial-queries-operators}

\href{OperatorsAA\#agent_closest_to}{agent\_closest\_to},
\href{OperatorsAA\#agent_farthest_to}{agent\_farthest\_to},
\href{OperatorsAA\#agents_at_distance}{agents\_at\_distance},
\href{OperatorsAA\#agents_inside}{agents\_inside},
\href{OperatorsAA\#agents_overlapping}{agents\_overlapping},
\href{OperatorsAA\#at_distance}{at\_distance},
\href{OperatorsBC\#closest_to}{closest\_to},
\href{OperatorsDH\#farthest_to}{farthest\_to},
\href{OperatorsIM\#inside}{inside},
\href{OperatorsNR\#neighbors_at}{neighbors\_at},
\href{OperatorsNR\#neighbors_of}{neighbors\_of},
\href{OperatorsNR\#overlapping}{overlapping},

\begin{center}\rule{0.5\linewidth}{\linethickness}\end{center}

\subsection{Spatial relations
operators}\label{spatial-relations-operators}

\href{OperatorsDH\#direction_between}{direction\_between},
\href{OperatorsDH\#distance_between}{distance\_between},
\href{OperatorsDH\#distance_to}{distance\_to},
\href{OperatorsNR\#path_between}{path\_between},
\href{OperatorsNR\#path_to}{path\_to},
\href{OperatorsSZ\#towards}{towards},

\begin{center}\rule{0.5\linewidth}{\linethickness}\end{center}

\subsection{Spatial statistical
operators}\label{spatial-statistical-operators}

\href{OperatorsDH\#hierarchical_clustering}{hierarchical\_clustering},
\href{OperatorsSZ\#simple_clustering_by_distance}{simple\_clustering\_by\_distance},

\begin{center}\rule{0.5\linewidth}{\linethickness}\end{center}

\subsection{Spatial transformations
operators}\label{spatial-transformations-operators}

\href{OperatorsAA\#-}{-}, \href{OperatorsAA\#*}{*},
\href{OperatorsAA\#+}{+}, \href{OperatorsAA\#as_4_grid}{as\_4\_grid},
\href{OperatorsAA\#as_grid}{as\_grid},
\href{OperatorsAA\#as_hexagonal_grid}{as\_hexagonal\_grid},
\href{OperatorsAA\#at_location}{at\_location},
\href{OperatorsBC\#clean}{clean},
\href{OperatorsBC\#clean_network}{clean\_network},
\href{OperatorsBC\#convex_hull}{convex\_hull},
\href{OperatorsBC\#crs_transform}{CRS\_transform},
\href{OperatorsNR\#rotated_by}{rotated\_by},
\href{OperatorsSZ\#scaled_to}{scaled\_to},
\href{OperatorsSZ\#simplification}{simplification},
\href{OperatorsSZ\#skeletonize}{skeletonize},
\href{OperatorsSZ\#smooth}{smooth},
\href{OperatorsSZ\#split_geometry}{split\_geometry},
\href{OperatorsSZ\#split_lines}{split\_lines},
\href{OperatorsSZ\#to_gama_crs}{to\_GAMA\_CRS},
\href{OperatorsSZ\#to_rectangles}{to\_rectangles},
\href{OperatorsSZ\#to_squares}{to\_squares},
\href{OperatorsSZ\#to_sub_geometries}{to\_sub\_geometries},
\href{OperatorsSZ\#transformed_by}{transformed\_by},
\href{OperatorsSZ\#translated_by}{translated\_by},
\href{OperatorsSZ\#triangulate}{triangulate},
\href{OperatorsSZ\#voronoi}{voronoi},
\href{OperatorsSZ\#with_precision}{with\_precision},
\href{OperatorsSZ\#without_holes}{without\_holes},

\begin{center}\rule{0.5\linewidth}{\linethickness}\end{center}

\subsection{Species-related operators}\label{species-related-operators}

\href{OperatorsIM\#index_of}{index\_of},
\href{OperatorsIM\#last_index_of}{last\_index\_of},
\href{OperatorsNR\#of_generic_species}{of\_generic\_species},
\href{OperatorsNR\#of_species}{of\_species},

\begin{center}\rule{0.5\linewidth}{\linethickness}\end{center}

\subsection{Statistical operators}\label{statistical-operators}

\href{OperatorsBC\#build}{build}, \href{OperatorsBC\#corr}{corR},
\href{OperatorsDH\#dbscan}{dbscan},
\href{OperatorsDH\#distribution_of}{distribution\_of},
\href{OperatorsDH\#distribution2d_of}{distribution2d\_of},
\href{OperatorsDH\#dtw}{dtw},
\href{OperatorsDH\#frequency_of}{frequency\_of},
\href{OperatorsDH\#gamma_rnd}{gamma\_rnd},
\href{OperatorsDH\#geometric_mean}{geometric\_mean},
\href{OperatorsDH\#gini}{gini},
\href{OperatorsDH\#harmonic_mean}{harmonic\_mean},
\href{OperatorsDH\#hierarchical_clustering}{hierarchical\_clustering},
\href{OperatorsIM\#kmeans}{kmeans},
\href{OperatorsIM\#kurtosis}{kurtosis}, \href{OperatorsIM\#max}{max},
\href{OperatorsIM\#mean}{mean},
\href{OperatorsIM\#mean_deviation}{mean\_deviation},
\href{OperatorsIM\#meanr}{meanR}, \href{OperatorsIM\#median}{median},
\href{OperatorsIM\#min}{min}, \href{OperatorsIM\#moran}{moran},
\href{OperatorsIM\#mul}{mul}, \href{OperatorsNR\#predict}{predict},
\href{OperatorsSZ\#simple_clustering_by_distance}{simple\_clustering\_by\_distance},
\href{OperatorsSZ\#skewness}{skewness},
\href{OperatorsSZ\#split}{split},
\href{OperatorsSZ\#split_in}{split\_in},
\href{OperatorsSZ\#split_using}{split\_using},
\href{OperatorsSZ\#standard_deviation}{standard\_deviation},
\href{OperatorsSZ\#sum}{sum}, \href{OperatorsSZ\#variance}{variance},

\begin{center}\rule{0.5\linewidth}{\linethickness}\end{center}

\subsection{Strings-related operators}\label{strings-related-operators}

\href{OperatorsAA\#+}{+}, \href{OperatorsAA\#\%3C}{\textless{}},
\href{OperatorsAA\#\%3C=}{\textless{}=},
\href{OperatorsAA\#\%3E}{\textgreater{}},
\href{OperatorsAA\#\%3E=}{\textgreater{}=}, \href{OperatorsAA\#at}{at},
\href{OperatorsBC\#char}{char}, \href{OperatorsBC\#contains}{contains},
\href{OperatorsBC\#contains_all}{contains\_all},
\href{OperatorsBC\#contains_any}{contains\_any},
\href{OperatorsBC\#copy_between}{copy\_between},
\href{OperatorsDH\#date}{date}, \href{OperatorsDH\#empty}{empty},
\href{OperatorsDH\#first}{first}, \href{OperatorsIM\#in}{in},
\href{OperatorsIM\#indented_by}{indented\_by},
\href{OperatorsIM\#index_of}{index\_of},
\href{OperatorsIM\#is_number}{is\_number},
\href{OperatorsIM\#last}{last},
\href{OperatorsIM\#last_index_of}{last\_index\_of},
\href{OperatorsIM\#length}{length},
\href{OperatorsIM\#lower_case}{lower\_case},
\href{OperatorsNR\#replace}{replace},
\href{OperatorsNR\#replace_regex}{replace\_regex},
\href{OperatorsNR\#reverse}{reverse},
\href{OperatorsSZ\#sample}{sample},
\href{OperatorsSZ\#shuffle}{shuffle},
\href{OperatorsSZ\#split_with}{split\_with},
\href{OperatorsSZ\#string}{string},
\href{OperatorsSZ\#upper_case}{upper\_case},

\begin{center}\rule{0.5\linewidth}{\linethickness}\end{center}

\subsection{System}\label{system}

\href{OperatorsAA\#.}{.}, \href{OperatorsBC\#command}{command},
\href{OperatorsBC\#copy}{copy}, \href{OperatorsDH\#dead}{dead},
\href{OperatorsDH\#eval_gaml}{eval\_gaml},
\href{OperatorsDH\#every}{every},
\href{OperatorsIM\#is_error}{is\_error},
\href{OperatorsIM\#is_warning}{is\_warning},
\href{OperatorsSZ\#user_input}{user\_input},

\begin{center}\rule{0.5\linewidth}{\linethickness}\end{center}

\subsection{Time-related operators}\label{time-related-operators}

\href{OperatorsDH\#date}{date}, \href{OperatorsSZ\#string}{string},

\begin{center}\rule{0.5\linewidth}{\linethickness}\end{center}

\subsection{Types-related operators}\label{types-related-operators}

\begin{center}\rule{0.5\linewidth}{\linethickness}\end{center}

\subsection{User control operators}\label{user-control-operators}

\href{OperatorsSZ\#user_input}{user\_input},

\begin{center}\rule{0.5\linewidth}{\linethickness}\end{center}

\section{Operators}\label{operators}

\begin{center}\rule{0.5\linewidth}{\linethickness}\end{center}

\subsection{\texorpdfstring{\texttt{-}}{-}}\label{section-1}

\subsubsection{Possible use:}\label{possible-use}

\begin{itemize}
\tightlist
\item
  \textbf{\texttt{-}} (\texttt{float}) ---\textgreater{} \texttt{float}
\item
  \textbf{\texttt{-}} (\texttt{int}) ---\textgreater{} \texttt{int}
\item
  \textbf{\texttt{-}} (\texttt{point}) ---\textgreater{} \texttt{point}
\item
  \texttt{float} \textbf{\texttt{-}} \texttt{matrix} ---\textgreater{}
  \texttt{matrix}
\item
  \textbf{\texttt{-}} (\texttt{float} , \texttt{matrix})
  ---\textgreater{} \texttt{matrix}
\item
  \texttt{float} \textbf{\texttt{-}} \texttt{int} ---\textgreater{}
  \texttt{float}
\item
  \textbf{\texttt{-}} (\texttt{float} , \texttt{int}) ---\textgreater{}
  \texttt{float}
\item
  \texttt{matrix} \textbf{\texttt{-}} \texttt{int} ---\textgreater{}
  \texttt{matrix}
\item
  \textbf{\texttt{-}} (\texttt{matrix} , \texttt{int}) ---\textgreater{}
  \texttt{matrix}
\item
  \texttt{int} \textbf{\texttt{-}} \texttt{float} ---\textgreater{}
  \texttt{float}
\item
  \textbf{\texttt{-}} (\texttt{int} , \texttt{float}) ---\textgreater{}
  \texttt{float}
\item
  \texttt{int} \textbf{\texttt{-}} \texttt{int} ---\textgreater{}
  \texttt{int}
\item
  \textbf{\texttt{-}} (\texttt{int} , \texttt{int}) ---\textgreater{}
  \texttt{int}
\item
  \texttt{container} \textbf{\texttt{-}} \texttt{container}
  ---\textgreater{} \texttt{list}
\item
  \textbf{\texttt{-}} (\texttt{container} , \texttt{container})
  ---\textgreater{} \texttt{list}
\item
  \texttt{matrix} \textbf{\texttt{-}} \texttt{matrix} ---\textgreater{}
  \texttt{matrix}
\item
  \textbf{\texttt{-}} (\texttt{matrix} , \texttt{matrix})
  ---\textgreater{} \texttt{matrix}
\item
  \texttt{date} \textbf{\texttt{-}} \texttt{int} ---\textgreater{}
  \texttt{date}
\item
  \textbf{\texttt{-}} (\texttt{date} , \texttt{int}) ---\textgreater{}
  \texttt{date}
\item
  \texttt{point} \textbf{\texttt{-}} \texttt{point} ---\textgreater{}
  \texttt{point}
\item
  \textbf{\texttt{-}} (\texttt{point} , \texttt{point})
  ---\textgreater{} \texttt{point}
\item
  \texttt{geometry} \textbf{\texttt{-}} \texttt{geometry}
  ---\textgreater{} \texttt{geometry}
\item
  \textbf{\texttt{-}} (\texttt{geometry} , \texttt{geometry})
  ---\textgreater{} \texttt{geometry}
\item
  \texttt{rgb} \textbf{\texttt{-}} \texttt{int} ---\textgreater{}
  \texttt{rgb}
\item
  \textbf{\texttt{-}} (\texttt{rgb} , \texttt{int}) ---\textgreater{}
  \texttt{rgb}
\item
  \texttt{geometry} \textbf{\texttt{-}} \texttt{float} ---\textgreater{}
  \texttt{geometry}
\item
  \textbf{\texttt{-}} (\texttt{geometry} , \texttt{float})
  ---\textgreater{} \texttt{geometry}
\item
  \texttt{point} \textbf{\texttt{-}} \texttt{int} ---\textgreater{}
  \texttt{point}
\item
  \textbf{\texttt{-}} (\texttt{point} , \texttt{int}) ---\textgreater{}
  \texttt{point}
\item
  \texttt{list} \textbf{\texttt{-}} \texttt{unknown} ---\textgreater{}
  \texttt{list}
\item
  \textbf{\texttt{-}} (\texttt{list} , \texttt{unknown})
  ---\textgreater{} \texttt{list}
\item
  \texttt{map} \textbf{\texttt{-}} \texttt{map} ---\textgreater{}
  \texttt{map}
\item
  \textbf{\texttt{-}} (\texttt{map} , \texttt{map}) ---\textgreater{}
  \texttt{map}
\item
  \texttt{date} \textbf{\texttt{-}} \texttt{date} ---\textgreater{}
  \texttt{float}
\item
  \textbf{\texttt{-}} (\texttt{date} , \texttt{date}) ---\textgreater{}
  \texttt{float}
\item
  \texttt{species} \textbf{\texttt{-}} \texttt{agent} ---\textgreater{}
  \texttt{list}
\item
  \textbf{\texttt{-}} (\texttt{species} , \texttt{agent})
  ---\textgreater{} \texttt{list}
\item
  \texttt{rgb} \textbf{\texttt{-}} \texttt{rgb} ---\textgreater{}
  \texttt{rgb}
\item
  \textbf{\texttt{-}} (\texttt{rgb} , \texttt{rgb}) ---\textgreater{}
  \texttt{rgb}
\item
  \texttt{date} \textbf{\texttt{-}} \texttt{float} ---\textgreater{}
  \texttt{date}
\item
  \textbf{\texttt{-}} (\texttt{date} , \texttt{float}) ---\textgreater{}
  \texttt{date}
\item
  \texttt{point} \textbf{\texttt{-}} \texttt{float} ---\textgreater{}
  \texttt{point}
\item
  \textbf{\texttt{-}} (\texttt{point} , \texttt{float})
  ---\textgreater{} \texttt{point}
\item
  \texttt{float} \textbf{\texttt{-}} \texttt{float} ---\textgreater{}
  \texttt{float}
\item
  \textbf{\texttt{-}} (\texttt{float} , \texttt{float})
  ---\textgreater{} \texttt{float}
\item
  \texttt{geometry} \textbf{\texttt{-}}
  \texttt{container\textless{}geometry\textgreater{}} ---\textgreater{}
  \texttt{geometry}
\item
  \textbf{\texttt{-}} (\texttt{geometry} ,
  \texttt{container\textless{}geometry\textgreater{}}) ---\textgreater{}
  \texttt{geometry}
\item
  \texttt{matrix} \textbf{\texttt{-}} \texttt{float} ---\textgreater{}
  \texttt{matrix}
\item
  \textbf{\texttt{-}} (\texttt{matrix} , \texttt{float})
  ---\textgreater{} \texttt{matrix}
\item
  \texttt{map} \textbf{\texttt{-}} \texttt{pair} ---\textgreater{}
  \texttt{map}
\item
  \textbf{\texttt{-}} (\texttt{map} , \texttt{pair}) ---\textgreater{}
  \texttt{map}
\item
  \texttt{int} \textbf{\texttt{-}} \texttt{matrix} ---\textgreater{}
  \texttt{matrix}
\item
  \textbf{\texttt{-}} (\texttt{int} , \texttt{matrix}) ---\textgreater{}
  \texttt{matrix}
\end{itemize}

\subsubsection{Result:}\label{result}

If it is used as an unary operator, it returns the opposite of the
operand. Returns the difference of the two operands.

\subsubsection{Comment:}\label{comment}

The behavior of the operator depends on the type of the operands.

\subsubsection{Special cases:}\label{special-cases}

\begin{itemize}
\tightlist
\item
  if both operands are containers and the right operand is empty, -
  returns the left operand\\
\item
  if the left operand is a species and the right operand is an agent of
  the species, - returns a list containing all the agents of the species
  minus this agent\\
\item
  if both operands are numbers, performs a normal arithmetic difference
  and returns a float if one of them is a float.
\end{itemize}

\begin{verbatim}
 
int var10 <- 1 - 1; // var10 equals 0
\end{verbatim}

\begin{itemize}
\tightlist
\item
  if both operands are containers, returns a new list in which all the
  elements of the right operand have been removed from the left one
\end{itemize}

\begin{verbatim}
 
list<int> var11 <- [1,2,3,4,5,6] - [2,4,9]; // var11 equals [1,3,5,6] 
list<int> var12 <- [1,2,3,4,5,6] - [0,8]; // var12 equals [1,2,3,4,5,6]
\end{verbatim}

\begin{itemize}
\tightlist
\item
  if one of the operands is a date and the other a number, returns a
  date corresponding to the date minus the given number as duration (in
  seconds)
\end{itemize}

\begin{verbatim}
 
date var13 <- date('2000-01-01') - 86400; // var13 equals date('1999-12-31')
\end{verbatim}

\begin{itemize}
\tightlist
\item
  if both operands are points, returns their difference (coordinates per
  coordinates).
\end{itemize}

\begin{verbatim}
 
point var14 <- {1, 2} - {4, 5}; // var14 equals {-3.0, -3.0}
\end{verbatim}

\begin{itemize}
\tightlist
\item
  if both operands are a point, a geometry or an agent, returns the
  geometry resulting from the difference between both geometries
\end{itemize}

\begin{verbatim}
 
geometry var15 <- geom1 - geom2; // var15 equals a geometry corresponding to difference between geom1 and geom2
\end{verbatim}

\begin{itemize}
\tightlist
\item
  if one operand is a color and the other an integer, returns a new
  color resulting from the subtraction of each component of the color
  with the right operand
\end{itemize}

\begin{verbatim}
 
rgb var16 <- rgb([255, 128, 32]) - 3; // var16 equals rgb([252,125,29])
\end{verbatim}

\begin{itemize}
\tightlist
\item
  if the left-hand operand is a geometry and the right-hand operand a
  float, returns a geometry corresponding to the left-hand operand
  (geometry, agent, point) reduced by the right-hand operand distance
\end{itemize}

\begin{verbatim}
 
geometry var17 <- shape - 5; // var17 equals a geometry corresponding to the geometry of the agent applying the operator reduced by a distance of 5
\end{verbatim}

\begin{itemize}
\tightlist
\item
  if the left operand is a list and the right operand is an object of
  any type (except list), - returns a list containing the elements of
  the left operand minus all the occurrences of this object
\end{itemize}

\begin{verbatim}
 
list<int> var18 <- [1,2,3,4,5,6] - 2; // var18 equals [1,3,4,5,6] 
list<int> var19 <- [1,2,3,4,5,6] - 0; // var19 equals [1,2,3,4,5,6]
\end{verbatim}

\begin{itemize}
\tightlist
\item
  if both operands are dates, returns the duration in seconds between
  date2 and date1. To obtain a more precise duration, in milliseconds,
  use milliseconds\_between(date1, date2)
\end{itemize}

\begin{verbatim}
 
float var20 <- date('2000-01-02') - date('2000-01-01'); // var20 equals 86400
\end{verbatim}

\begin{itemize}
\tightlist
\item
  if both operands are colors, returns a new color resulting from the
  subtraction of the two operands, component by component
\end{itemize}

\begin{verbatim}
 
rgb var21 <- rgb([255, 128, 32]) - rgb('red'); // var21 equals rgb([0,128,32])
\end{verbatim}

\begin{itemize}
\tightlist
\item
  if left-hand operand is a point and the right-hand a number, returns a
  new point with each coordinate as the difference of the operand
  coordinate with this number.
\end{itemize}

\begin{verbatim}
 
point var22 <- {1, 2} - 4.5; // var22 equals {-3.5, -2.5, -4.5} 
point var23 <- {1, 2} - 4; // var23 equals {-3.0,-2.0,-4.0}
\end{verbatim}

\begin{itemize}
\tightlist
\item
  if the right-operand is a list of points, geometries or agents,
  returns the geometry resulting from the difference between the
  left-geometry and all of the right-geometries
\end{itemize}

\begin{verbatim}
 
geometry var24 <- rectangle(10,10) - [circle(2), square(2)]; // var24 equals rectangle(10,10) - (circle(2) + square(2))
\end{verbatim}

\begin{itemize}
\tightlist
\item
  if one operand is a matrix and the other a number (float or int),
  performs a normal arithmetic difference of the number with each
  element of the matrix (results are float if the number is a float.
\end{itemize}

\begin{verbatim}
 
matrix var25 <- 3.5 - matrix([[2,5],[3,4]]); // var25 equals matrix([[1.5,-1.5],[0.5,-0.5]])
\end{verbatim}

\subsubsection{Examples:}\label{examples}

\begin{verbatim}
 
float var0 <- 1.0 - 1; // var0 equals 0.0 
float var1 <- 3.7 - 1.2; // var1 equals 2.5 
float var2 <- 3 - 1.2; // var2 equals 1.8 
int var3 <- - (-56); // var3 equals 56 
map var4 <- ['a'::1,'b'::2] - ['b'::2]; // var4 equals ['a'::1] 
map var5 <- ['a'::1,'b'::2] - ['b'::2,'c'::3]; // var5 equals ['a'::1] 
point var6 <- -{3.0,5.0}; // var6 equals {-3.0,-5.0} 
point var7 <- -{1.0,6.0,7.0}; // var7 equals {-1.0,-6.0,-7.0} 
map var8 <- ['a'::1,'b'::2] - ('b'::2); // var8 equals ['a'::1] 
map var9 <- ['a'::1,'b'::2] - ('c'::3); // var9 equals ['a'::1,'b'::2]
\end{verbatim}

\subsubsection{See also:}\label{see-also}

\href{OperatorsAA\#+}{+}, \href{OperatorsAA\#*}{*},
\href{OperatorsAA\#/}{/}, \href{OperatorsAA\#-}{-},
\href{OperatorsIM\#milliseconds_between}{milliseconds\_between},

\begin{center}\rule{0.5\linewidth}{\linethickness}\end{center}

\subsection{\texorpdfstring{\texttt{:}}{:}}\label{section-2}

\subsubsection{Possible use:}\label{possible-use-1}

\begin{itemize}
\tightlist
\item
  \texttt{unknown} \textbf{\texttt{:}} \texttt{unknown}
  ---\textgreater{} \texttt{unknown}
\item
  \textbf{\texttt{:}} (\texttt{unknown} , \texttt{unknown})
  ---\textgreater{} \texttt{unknown}
\end{itemize}

\subsubsection{See also:}\label{see-also-1}

\href{OperatorsAA\#?}{?},

\begin{center}\rule{0.5\linewidth}{\linethickness}\end{center}

\subsection{\texorpdfstring{\texttt{::}}{::}}\label{section-3}

\subsubsection{Possible use:}\label{possible-use-2}

\begin{itemize}
\tightlist
\item
  \texttt{any\ expression} \textbf{\texttt{::}} \texttt{any\ expression}
  ---\textgreater{} \texttt{pair}
\item
  \textbf{\texttt{::}} (\texttt{any\ expression} ,
  \texttt{any\ expression}) ---\textgreater{} \texttt{pair}
\end{itemize}

\subsubsection{Result:}\label{result-1}

produces a new pair combining the left and the right operands

\subsubsection{Special cases:}\label{special-cases-1}

\begin{itemize}
\tightlist
\item
  nil is not acceptable as a key (although it is as a value). If such a
  case happens, :: will throw an appropriate error
\end{itemize}

\begin{center}\rule{0.5\linewidth}{\linethickness}\end{center}

\subsection{\texorpdfstring{\texttt{!}}{!}}\label{section-4}

\subsubsection{Possible use:}\label{possible-use-3}

\begin{itemize}
\tightlist
\item
  \textbf{\texttt{!}} (\texttt{bool}) ---\textgreater{} \texttt{bool}
\end{itemize}

\subsubsection{Result:}\label{result-2}

opposite boolean value.

\subsubsection{Special cases:}\label{special-cases-2}

\begin{itemize}
\tightlist
\item
  if the parameter is not boolean, it is casted to a boolean value.
\end{itemize}

\subsubsection{Examples:}\label{examples-1}

\begin{verbatim}
 
bool var0 <- ! (true); // var0 equals false
\end{verbatim}

\subsubsection{See also:}\label{see-also-2}

\href{OperatorsBC\#bool}{bool}, \href{OperatorsAA\#and}{and},
\href{OperatorsNR\#or}{or},

\begin{center}\rule{0.5\linewidth}{\linethickness}\end{center}

\subsection{\texorpdfstring{\texttt{!=}}{!=}}\label{section-5}

\subsubsection{Possible use:}\label{possible-use-4}

\begin{itemize}
\tightlist
\item
  \texttt{unknown} \textbf{\texttt{!=}} \texttt{unknown}
  ---\textgreater{} \texttt{bool}
\item
  \textbf{\texttt{!=}} (\texttt{unknown} , \texttt{unknown})
  ---\textgreater{} \texttt{bool}
\item
  \texttt{int} \textbf{\texttt{!=}} \texttt{float} ---\textgreater{}
  \texttt{bool}
\item
  \textbf{\texttt{!=}} (\texttt{int} , \texttt{float}) ---\textgreater{}
  \texttt{bool}
\item
  \texttt{date} \textbf{\texttt{!=}} \texttt{date} ---\textgreater{}
  \texttt{bool}
\item
  \textbf{\texttt{!=}} (\texttt{date} , \texttt{date}) ---\textgreater{}
  \texttt{bool}
\item
  \texttt{float} \textbf{\texttt{!=}} \texttt{int} ---\textgreater{}
  \texttt{bool}
\item
  \textbf{\texttt{!=}} (\texttt{float} , \texttt{int}) ---\textgreater{}
  \texttt{bool}
\item
  \texttt{float} \textbf{\texttt{!=}} \texttt{float} ---\textgreater{}
  \texttt{bool}
\item
  \textbf{\texttt{!=}} (\texttt{float} , \texttt{float})
  ---\textgreater{} \texttt{bool}
\end{itemize}

\subsubsection{Result:}\label{result-3}

true if both operands are different, false otherwise

\subsubsection{Examples:}\label{examples-2}

\begin{verbatim}
 
bool var0 <- [2,3] != [2,3]; // var0 equals false 
bool var1 <- [2,4] != [2,3]; // var1 equals true 
bool var2 <- 3 != 3.0; // var2 equals false 
bool var3 <- 4 != 4.7; // var3 equals true 
bool var4 <- #now != #now minus_hours 1; // var4 equals true 
bool var5 <- 3.0 != 3; // var5 equals false 
bool var6 <- 4.7 != 4; // var6 equals true 
bool var7 <- 3.0 != 3.0; // var7 equals false 
bool var8 <- 4.0 != 4.7; // var8 equals true
\end{verbatim}

\subsubsection{See also:}\label{see-also-3}

\href{OperatorsAA\#=}{=}, \href{OperatorsAA\#\%3E}{\textgreater{}},
\href{OperatorsAA\#\%3C}{\textless{}},
\href{OperatorsAA\#\%3E=}{\textgreater{}=},
\href{OperatorsAA\#\%3C=}{\textless{}=},

\begin{center}\rule{0.5\linewidth}{\linethickness}\end{center}

\subsection{\texorpdfstring{\texttt{?}}{?}}\label{section-6}

\subsubsection{Possible use:}\label{possible-use-5}

\begin{itemize}
\tightlist
\item
  \texttt{bool} \textbf{\texttt{?}} \texttt{any\ expression}
  ---\textgreater{} \texttt{unknown}
\item
  \textbf{\texttt{?}} (\texttt{bool} , \texttt{any\ expression})
  ---\textgreater{} \texttt{unknown}
\end{itemize}

\subsubsection{Result:}\label{result-4}

It is used in combination with the : operator: if the left-hand operand
evaluates to true, returns the value of the left-hand operand of the :,
otherwise that of the right-hand operand of the :

\subsubsection{Comment:}\label{comment-1}

These functional tests can be combined together.

\subsubsection{Examples:}\label{examples-3}

\begin{verbatim}
 
list<string> var0 <- [10, 19, 43, 12, 7, 22] collect ((each > 20) ? 'above' : 'below'); // var0 equals ['below', 'below', 'above', 'below', 'below', 'above']rgb col <- (flip(0.3) ? #red : (flip(0.9) ? #blue : #green)); 
\end{verbatim}

\subsubsection{See also:}\label{see-also-4}

\href{OperatorsAA\#:}{:},

\begin{center}\rule{0.5\linewidth}{\linethickness}\end{center}

\subsection{\texorpdfstring{\texttt{/}}{/}}\label{section-7}

\subsubsection{Possible use:}\label{possible-use-6}

\begin{itemize}
\tightlist
\item
  \texttt{point} \textbf{\texttt{/}} \texttt{float} ---\textgreater{}
  \texttt{point}
\item
  \textbf{\texttt{/}} (\texttt{point} , \texttt{float})
  ---\textgreater{} \texttt{point}
\item
  \texttt{float} \textbf{\texttt{/}} \texttt{float} ---\textgreater{}
  \texttt{float}
\item
  \textbf{\texttt{/}} (\texttt{float} , \texttt{float})
  ---\textgreater{} \texttt{float}
\item
  \texttt{int} \textbf{\texttt{/}} \texttt{int} ---\textgreater{}
  \texttt{float}
\item
  \textbf{\texttt{/}} (\texttt{int} , \texttt{int}) ---\textgreater{}
  \texttt{float}
\item
  \texttt{rgb} \textbf{\texttt{/}} \texttt{float} ---\textgreater{}
  \texttt{rgb}
\item
  \textbf{\texttt{/}} (\texttt{rgb} , \texttt{float}) ---\textgreater{}
  \texttt{rgb}
\item
  \texttt{float} \textbf{\texttt{/}} \texttt{int} ---\textgreater{}
  \texttt{float}
\item
  \textbf{\texttt{/}} (\texttt{float} , \texttt{int}) ---\textgreater{}
  \texttt{float}
\item
  \texttt{rgb} \textbf{\texttt{/}} \texttt{int} ---\textgreater{}
  \texttt{rgb}
\item
  \textbf{\texttt{/}} (\texttt{rgb} , \texttt{int}) ---\textgreater{}
  \texttt{rgb}
\item
  \texttt{matrix} \textbf{\texttt{/}} \texttt{int} ---\textgreater{}
  \texttt{matrix}
\item
  \textbf{\texttt{/}} (\texttt{matrix} , \texttt{int}) ---\textgreater{}
  \texttt{matrix}
\item
  \texttt{point} \textbf{\texttt{/}} \texttt{int} ---\textgreater{}
  \texttt{point}
\item
  \textbf{\texttt{/}} (\texttt{point} , \texttt{int}) ---\textgreater{}
  \texttt{point}
\item
  \texttt{matrix} \textbf{\texttt{/}} \texttt{float} ---\textgreater{}
  \texttt{matrix}
\item
  \textbf{\texttt{/}} (\texttt{matrix} , \texttt{float})
  ---\textgreater{} \texttt{matrix}
\item
  \texttt{matrix} \textbf{\texttt{/}} \texttt{matrix} ---\textgreater{}
  \texttt{matrix}
\item
  \textbf{\texttt{/}} (\texttt{matrix} , \texttt{matrix})
  ---\textgreater{} \texttt{matrix}
\item
  \texttt{int} \textbf{\texttt{/}} \texttt{float} ---\textgreater{}
  \texttt{float}
\item
  \textbf{\texttt{/}} (\texttt{int} , \texttt{float}) ---\textgreater{}
  \texttt{float}
\end{itemize}

\subsubsection{Result:}\label{result-5}

Returns the division of the two operands.

\subsubsection{Special cases:}\label{special-cases-3}

\begin{itemize}
\tightlist
\item
  if the right-hand operand is equal to zero, raises a ``Division by
  zero'' exception\\
\item
  if the left operand is a point, returns a new point with coordinates
  divided by the right operand
\end{itemize}

\begin{verbatim}
 
point var0 <- {5, 7.5} / 2.5; // var0 equals {2, 3} 
point var1 <- {2,5} / 4; // var1 equals {0.5,1.25}
\end{verbatim}

\begin{itemize}
\tightlist
\item
  if both operands are numbers (float or int), performs a normal
  arithmetic division and returns a float.
\end{itemize}

\begin{verbatim}
 
float var2 <- 3 / 5.0; // var2 equals 0.6
\end{verbatim}

\begin{itemize}
\tightlist
\item
  if one operand is a color and the other a double, returns a new color
  resulting from the division of each component of the color by the
  right operand. The result on each component is then truncated.
\end{itemize}

\begin{verbatim}
 
rgb var3 <- rgb([255, 128, 32]) / 2.5; // var3 equals rgb([102,51,13])
\end{verbatim}

\begin{itemize}
\tightlist
\item
  if one operand is a color and the other an integer, returns a new
  color resulting from the division of each component of the color by
  the right operand
\end{itemize}

\begin{verbatim}
 
rgb var4 <- rgb([255, 128, 32]) / 2; // var4 equals rgb([127,64,16])
\end{verbatim}

\subsubsection{See also:}\label{see-also-5}

\href{OperatorsAA\#*}{*}, \href{OperatorsAA\#+}{+},
\href{OperatorsAA\#-}{-},

\begin{center}\rule{0.5\linewidth}{\linethickness}\end{center}

\subsection{\texorpdfstring{\texttt{.}}{.}}\label{section-8}

\subsubsection{Possible use:}\label{possible-use-7}

\begin{itemize}
\tightlist
\item
  \texttt{agent} \textbf{\texttt{.}} \texttt{any\ expression}
  ---\textgreater{} \texttt{unknown}
\item
  \textbf{\texttt{.}} (\texttt{agent} , \texttt{any\ expression})
  ---\textgreater{} \texttt{unknown}
\item
  \texttt{matrix} \textbf{\texttt{.}} \texttt{matrix} ---\textgreater{}
  \texttt{matrix}
\item
  \textbf{\texttt{.}} (\texttt{matrix} , \texttt{matrix})
  ---\textgreater{} \texttt{matrix}
\end{itemize}

\subsubsection{Result:}\label{result-6}

It has two different uses: it can be the dot product between 2 matrices
or return an evaluation of the expression (right-hand operand) in the
scope the given agent.

\subsubsection{Special cases:}\label{special-cases-4}

\begin{itemize}
\tightlist
\item
  if the agent is nil or dead, throws an exception\\
\item
  if the left operand is an agent, it evaluates of the expression
  (right-hand operand) in the scope the given agent
\end{itemize}

\begin{verbatim}
 
unknown var0 <- agent1.location; // var0 equals the location of the agent agent1
\end{verbatim}

\begin{itemize}
\tightlist
\item
  if both operands are matrix, returns the dot product of them
\end{itemize}

\begin{verbatim}
 
matrix var1 <- matrix([[1,1],[1,2]]) . matrix([[1,1],[1,2]]); // var1 equals matrix([[2,3],[3,5]])
\end{verbatim}

\begin{center}\rule{0.5\linewidth}{\linethickness}\end{center}

\subsection{\texorpdfstring{\texttt{\^{}}}{\^{}}}\label{section-9}

\subsubsection{Possible use:}\label{possible-use-8}

\begin{itemize}
\tightlist
\item
  \texttt{float} \textbf{\texttt{\^{}}} \texttt{int} ---\textgreater{}
  \texttt{float}
\item
  \textbf{\texttt{\^{}}} (\texttt{float} , \texttt{int})
  ---\textgreater{} \texttt{float}
\item
  \texttt{float} \textbf{\texttt{\^{}}} \texttt{float} ---\textgreater{}
  \texttt{float}
\item
  \textbf{\texttt{\^{}}} (\texttt{float} , \texttt{float})
  ---\textgreater{} \texttt{float}
\item
  \texttt{int} \textbf{\texttt{\^{}}} \texttt{int} ---\textgreater{}
  \texttt{float}
\item
  \textbf{\texttt{\^{}}} (\texttt{int} , \texttt{int}) ---\textgreater{}
  \texttt{float}
\item
  \texttt{int} \textbf{\texttt{\^{}}} \texttt{float} ---\textgreater{}
  \texttt{float}
\item
  \textbf{\texttt{\^{}}} (\texttt{int} , \texttt{float})
  ---\textgreater{} \texttt{float}
\end{itemize}

\subsubsection{Result:}\label{result-7}

Returns the value (always a float) of the left operand raised to the
power of the right operand.

\subsubsection{Special cases:}\label{special-cases-5}

\begin{itemize}
\tightlist
\item
  if the right-hand operand is equal to 0, returns 1\\
\item
  if it is equal to 1, returns the left-hand operand.\\
\item
  Various examples of power
\end{itemize}

\begin{verbatim}
 
float var1 <- 2 ^ 3; // var1 equals 8.0
\end{verbatim}

\subsubsection{Examples:}\label{examples-4}

\begin{verbatim}
 
float var0 <- 4.84 ^ 0.5; // var0 equals 2.2
\end{verbatim}

\subsubsection{See also:}\label{see-also-6}

\href{OperatorsAA\#*}{*}, \href{OperatorsSZ\#sqrt}{sqrt},

\begin{center}\rule{0.5\linewidth}{\linethickness}\end{center}

\subsection{\texorpdfstring{\texttt{@}}{@}}\label{section-10}

Same signification as \href{OperatorsAA\#at}{at}

\begin{center}\rule{0.5\linewidth}{\linethickness}\end{center}

\subsection{\texorpdfstring{\texttt{*}}{*}}\label{section-11}

\subsubsection{Possible use:}\label{possible-use-9}

\begin{itemize}
\tightlist
\item
  \texttt{float} \textbf{\texttt{*}} \texttt{float} ---\textgreater{}
  \texttt{float}
\item
  \textbf{\texttt{*}} (\texttt{float} , \texttt{float})
  ---\textgreater{} \texttt{float}
\item
  \texttt{int} \textbf{\texttt{*}} \texttt{matrix} ---\textgreater{}
  \texttt{matrix}
\item
  \textbf{\texttt{*}} (\texttt{int} , \texttt{matrix}) ---\textgreater{}
  \texttt{matrix}
\item
  \texttt{int} \textbf{\texttt{*}} \texttt{int} ---\textgreater{}
  \texttt{int}
\item
  \textbf{\texttt{*}} (\texttt{int} , \texttt{int}) ---\textgreater{}
  \texttt{int}
\item
  \texttt{point} \textbf{\texttt{*}} \texttt{float} ---\textgreater{}
  \texttt{point}
\item
  \textbf{\texttt{*}} (\texttt{point} , \texttt{float})
  ---\textgreater{} \texttt{point}
\item
  \texttt{rgb} \textbf{\texttt{*}} \texttt{int} ---\textgreater{}
  \texttt{rgb}
\item
  \textbf{\texttt{*}} (\texttt{rgb} , \texttt{int}) ---\textgreater{}
  \texttt{rgb}
\item
  \texttt{int} \textbf{\texttt{*}} \texttt{float} ---\textgreater{}
  \texttt{float}
\item
  \textbf{\texttt{*}} (\texttt{int} , \texttt{float}) ---\textgreater{}
  \texttt{float}
\item
  \texttt{geometry} \textbf{\texttt{*}} \texttt{float} ---\textgreater{}
  \texttt{geometry}
\item
  \textbf{\texttt{*}} (\texttt{geometry} , \texttt{float})
  ---\textgreater{} \texttt{geometry}
\item
  \texttt{geometry} \textbf{\texttt{*}} \texttt{point} ---\textgreater{}
  \texttt{geometry}
\item
  \textbf{\texttt{*}} (\texttt{geometry} , \texttt{point})
  ---\textgreater{} \texttt{geometry}
\item
  \texttt{matrix} \textbf{\texttt{*}} \texttt{float} ---\textgreater{}
  \texttt{matrix}
\item
  \textbf{\texttt{*}} (\texttt{matrix} , \texttt{float})
  ---\textgreater{} \texttt{matrix}
\item
  \texttt{matrix} \textbf{\texttt{*}} \texttt{matrix} ---\textgreater{}
  \texttt{matrix}
\item
  \textbf{\texttt{*}} (\texttt{matrix} , \texttt{matrix})
  ---\textgreater{} \texttt{matrix}
\item
  \texttt{float} \textbf{\texttt{*}} \texttt{matrix} ---\textgreater{}
  \texttt{matrix}
\item
  \textbf{\texttt{*}} (\texttt{float} , \texttt{matrix})
  ---\textgreater{} \texttt{matrix}
\item
  \texttt{matrix} \textbf{\texttt{*}} \texttt{int} ---\textgreater{}
  \texttt{matrix}
\item
  \textbf{\texttt{*}} (\texttt{matrix} , \texttt{int}) ---\textgreater{}
  \texttt{matrix}
\item
  \texttt{float} \textbf{\texttt{*}} \texttt{int} ---\textgreater{}
  \texttt{float}
\item
  \textbf{\texttt{*}} (\texttt{float} , \texttt{int}) ---\textgreater{}
  \texttt{float}
\item
  \texttt{point} \textbf{\texttt{*}} \texttt{int} ---\textgreater{}
  \texttt{point}
\item
  \textbf{\texttt{*}} (\texttt{point} , \texttt{int}) ---\textgreater{}
  \texttt{point}
\item
  \texttt{point} \textbf{\texttt{*}} \texttt{point} ---\textgreater{}
  \texttt{float}
\item
  \textbf{\texttt{*}} (\texttt{point} , \texttt{point})
  ---\textgreater{} \texttt{float}
\end{itemize}

\subsubsection{Result:}\label{result-8}

Returns the product of the two operands.

\subsubsection{Special cases:}\label{special-cases-6}

\begin{itemize}
\tightlist
\item
  if one operand is a matrix and the other a number (float or int),
  performs a normal arithmetic product of the number with each element
  of the matrix (results are float if the number is a float.
\end{itemize}

\begin{verbatim}
matrix<float> m <- (3.5 * matrix([[2,5],[3,4]]));   //m equals matrix([[7.0,17.5],[10.5,14]]) 
\end{verbatim}

\begin{itemize}
\tightlist
\item
  if both operands are numbers (float or int), performs a normal
  arithmetic product and returns a float if one of them is a float.
\end{itemize}

\begin{verbatim}
 
int var2 <- 1 * 1; // var2 equals 1
\end{verbatim}

\begin{itemize}
\tightlist
\item
  if one operand is a color and the other an integer, returns a new
  color resulting from the product of each component of the color with
  the right operand (with a maximum value at 255)
\end{itemize}

\begin{verbatim}
 
rgb var3 <- rgb([255, 128, 32]) * 2; // var3 equals rgb([255,255,64])
\end{verbatim}

\begin{itemize}
\tightlist
\item
  if the left-hand operand is a geometry and the right-hand operand a
  float, returns a geometry corresponding to the left-hand operand
  (geometry, agent, point) scaled by the right-hand operand coefficient
\end{itemize}

\begin{verbatim}
 
geometry var4 <- circle(10) * 2; // var4 equals circle(20) 
geometry var5 <- (circle(10) * 2).location with_precision 9; // var5 equals (circle(20)).location with_precision 9 
float var6 <- (circle(10) * 2).height with_precision 9; // var6 equals (circle(20)).height with_precision 9
\end{verbatim}

\begin{itemize}
\tightlist
\item
  if the left-hand operand is a geometry and the right-hand operand a
  point, returns a geometry corresponding to the left-hand operand
  (geometry, agent, point) scaled by the right-hand operand coefficients
  in the 3 dimensions
\end{itemize}

\begin{verbatim}
 
geometry var7 <- shape * {0.5,0.5,2}; // var7 equals a geometry corresponding to the geometry of the agent applying the operator scaled by a coefficient of 0.5 in x, 0.5 in y and 2 in z
\end{verbatim}

\begin{itemize}
\tightlist
\item
  if the left-hand operator is a point and the right-hand a number,
  returns a point with coordinates multiplied by the number
\end{itemize}

\begin{verbatim}
 
point var8 <- {2,5} * 4; // var8 equals {8.0, 20.0} 
point var9 <- {2, 4} * 2.5; // var9 equals {5.0, 10.0}
\end{verbatim}

\begin{itemize}
\tightlist
\item
  if both operands are points, returns their scalar product
\end{itemize}

\begin{verbatim}
 
float var10 <- {2,5} * {4.5, 5}; // var10 equals 34.0
\end{verbatim}

\subsubsection{Examples:}\label{examples-5}

\begin{verbatim}
 
float var0 <- 2.5 * 2; // var0 equals 5.0
\end{verbatim}

\subsubsection{See also:}\label{see-also-7}

\href{OperatorsAA\#/}{/}, \href{OperatorsAA\#+}{+},
\href{OperatorsAA\#-}{-},

\begin{center}\rule{0.5\linewidth}{\linethickness}\end{center}

\subsection{\texorpdfstring{\texttt{+}}{+}}\label{section-12}

\subsubsection{Possible use:}\label{possible-use-10}

\begin{itemize}
\tightlist
\item
  \texttt{point} \textbf{\texttt{+}} \texttt{point} ---\textgreater{}
  \texttt{point}
\item
  \textbf{\texttt{+}} (\texttt{point} , \texttt{point})
  ---\textgreater{} \texttt{point}
\item
  \texttt{int} \textbf{\texttt{+}} \texttt{matrix} ---\textgreater{}
  \texttt{matrix}
\item
  \textbf{\texttt{+}} (\texttt{int} , \texttt{matrix}) ---\textgreater{}
  \texttt{matrix}
\item
  \texttt{date} \textbf{\texttt{+}} \texttt{int} ---\textgreater{}
  \texttt{date}
\item
  \textbf{\texttt{+}} (\texttt{date} , \texttt{int}) ---\textgreater{}
  \texttt{date}
\item
  \texttt{float} \textbf{\texttt{+}} \texttt{int} ---\textgreater{}
  \texttt{float}
\item
  \textbf{\texttt{+}} (\texttt{float} , \texttt{int}) ---\textgreater{}
  \texttt{float}
\item
  \texttt{matrix} \textbf{\texttt{+}} \texttt{int} ---\textgreater{}
  \texttt{matrix}
\item
  \textbf{\texttt{+}} (\texttt{matrix} , \texttt{int}) ---\textgreater{}
  \texttt{matrix}
\item
  \texttt{matrix} \textbf{\texttt{+}} \texttt{matrix} ---\textgreater{}
  \texttt{matrix}
\item
  \textbf{\texttt{+}} (\texttt{matrix} , \texttt{matrix})
  ---\textgreater{} \texttt{matrix}
\item
  \texttt{float} \textbf{\texttt{+}} \texttt{float} ---\textgreater{}
  \texttt{float}
\item
  \textbf{\texttt{+}} (\texttt{float} , \texttt{float})
  ---\textgreater{} \texttt{float}
\item
  \texttt{rgb} \textbf{\texttt{+}} \texttt{rgb} ---\textgreater{}
  \texttt{rgb}
\item
  \textbf{\texttt{+}} (\texttt{rgb} , \texttt{rgb}) ---\textgreater{}
  \texttt{rgb}
\item
  \texttt{int} \textbf{\texttt{+}} \texttt{int} ---\textgreater{}
  \texttt{int}
\item
  \textbf{\texttt{+}} (\texttt{int} , \texttt{int}) ---\textgreater{}
  \texttt{int}
\item
  \texttt{string} \textbf{\texttt{+}} \texttt{unknown} ---\textgreater{}
  \texttt{string}
\item
  \textbf{\texttt{+}} (\texttt{string} , \texttt{unknown})
  ---\textgreater{} \texttt{string}
\item
  \texttt{map} \textbf{\texttt{+}} \texttt{pair} ---\textgreater{}
  \texttt{map}
\item
  \textbf{\texttt{+}} (\texttt{map} , \texttt{pair}) ---\textgreater{}
  \texttt{map}
\item
  \texttt{container} \textbf{\texttt{+}} \texttt{unknown}
  ---\textgreater{} \texttt{list}
\item
  \textbf{\texttt{+}} (\texttt{container} , \texttt{unknown})
  ---\textgreater{} \texttt{list}
\item
  \texttt{point} \textbf{\texttt{+}} \texttt{float} ---\textgreater{}
  \texttt{point}
\item
  \textbf{\texttt{+}} (\texttt{point} , \texttt{float})
  ---\textgreater{} \texttt{point}
\item
  \texttt{matrix} \textbf{\texttt{+}} \texttt{float} ---\textgreater{}
  \texttt{matrix}
\item
  \textbf{\texttt{+}} (\texttt{matrix} , \texttt{float})
  ---\textgreater{} \texttt{matrix}
\item
  \texttt{geometry} \textbf{\texttt{+}} \texttt{float} ---\textgreater{}
  \texttt{geometry}
\item
  \textbf{\texttt{+}} (\texttt{geometry} , \texttt{float})
  ---\textgreater{} \texttt{geometry}
\item
  \texttt{geometry} \textbf{\texttt{+}} \texttt{geometry}
  ---\textgreater{} \texttt{geometry}
\item
  \textbf{\texttt{+}} (\texttt{geometry} , \texttt{geometry})
  ---\textgreater{} \texttt{geometry}
\item
  \texttt{string} \textbf{\texttt{+}} \texttt{string} ---\textgreater{}
  \texttt{string}
\item
  \textbf{\texttt{+}} (\texttt{string} , \texttt{string})
  ---\textgreater{} \texttt{string}
\item
  \texttt{container} \textbf{\texttt{+}} \texttt{container}
  ---\textgreater{} \texttt{container}
\item
  \textbf{\texttt{+}} (\texttt{container} , \texttt{container})
  ---\textgreater{} \texttt{container}
\item
  \texttt{map} \textbf{\texttt{+}} \texttt{map} ---\textgreater{}
  \texttt{map}
\item
  \textbf{\texttt{+}} (\texttt{map} , \texttt{map}) ---\textgreater{}
  \texttt{map}
\item
  \texttt{point} \textbf{\texttt{+}} \texttt{int} ---\textgreater{}
  \texttt{point}
\item
  \textbf{\texttt{+}} (\texttt{point} , \texttt{int}) ---\textgreater{}
  \texttt{point}
\item
  \texttt{date} \textbf{\texttt{+}} \texttt{float} ---\textgreater{}
  \texttt{date}
\item
  \textbf{\texttt{+}} (\texttt{date} , \texttt{float}) ---\textgreater{}
  \texttt{date}
\item
  \texttt{float} \textbf{\texttt{+}} \texttt{matrix} ---\textgreater{}
  \texttt{matrix}
\item
  \textbf{\texttt{+}} (\texttt{float} , \texttt{matrix})
  ---\textgreater{} \texttt{matrix}
\item
  \texttt{rgb} \textbf{\texttt{+}} \texttt{int} ---\textgreater{}
  \texttt{rgb}
\item
  \textbf{\texttt{+}} (\texttt{rgb} , \texttt{int}) ---\textgreater{}
  \texttt{rgb}
\item
  \texttt{date} \textbf{\texttt{+}} \texttt{string} ---\textgreater{}
  \texttt{string}
\item
  \textbf{\texttt{+}} (\texttt{date} , \texttt{string})
  ---\textgreater{} \texttt{string}
\item
  \texttt{int} \textbf{\texttt{+}} \texttt{float} ---\textgreater{}
  \texttt{float}
\item
  \textbf{\texttt{+}} (\texttt{int} , \texttt{float}) ---\textgreater{}
  \texttt{float}
\item
  \textbf{\texttt{+}} (\texttt{geometry}, \texttt{float}, \texttt{int})
  ---\textgreater{} \texttt{geometry}
\item
  \textbf{\texttt{+}} (\texttt{geometry}, \texttt{float}, \texttt{int},
  \texttt{int}) ---\textgreater{} \texttt{geometry}
\end{itemize}

\subsubsection{Result:}\label{result-9}

Returns the sum, union or concatenation of the two operands.

\subsubsection{Special cases:}\label{special-cases-7}

\begin{itemize}
\tightlist
\item
  if one of the operands is nil, + throws an error\\
\item
  if both operands are species, returns a special type of list called
  meta-population\\
\item
  if both operands are points, returns their sum.
\end{itemize}

\begin{verbatim}
 
point var6 <- {1, 2} + {4, 5}; // var6 equals {5.0, 7.0}
\end{verbatim}

\begin{itemize}
\tightlist
\item
  if the left-hand operand is a geometry and the right-hand operands a
  float and an integer, returns a geometry corresponding to the
  left-hand operand (geometry, agent, point) enlarged by the first
  right-hand operand (distance), using a number of segments equal to the
  second right-hand operand
\end{itemize}

\begin{verbatim}
 
geometry var7 <- circle(5) + (5,32); // var7 equals circle(10)
\end{verbatim}

\begin{itemize}
\tightlist
\item
  if one operand is a matrix and the other a number (float or int),
  performs a normal arithmetic sum of the number with each element of
  the matrix (results are float if the number is a float.
\end{itemize}

\begin{verbatim}
 
matrix var8 <- 3.5 + matrix([[2,5],[3,4]]); // var8 equals matrix([[5.5,8.5],[6.5,7.5]])
\end{verbatim}

\begin{itemize}
\tightlist
\item
  if one of the operands is a date and the other a number, returns a
  date corresponding to the date plus the given number as duration (in
  seconds)
\end{itemize}

\begin{verbatim}
 
date var9 <- date('2000-01-01') + 86400; // var9 equals date('2000-01-02')
\end{verbatim}

\begin{itemize}
\tightlist
\item
  if both operands are colors, returns a new color resulting from the
  sum of the two operands, component by component
\end{itemize}

\begin{verbatim}
 
rgb var10 <- rgb([255, 128, 32]) + rgb('red'); // var10 equals rgb([255,128,32])
\end{verbatim}

\begin{itemize}
\tightlist
\item
  if both operands are numbers (float or int), performs a normal
  arithmetic sum and returns a float if one of them is a float.
\end{itemize}

\begin{verbatim}
 
int var11 <- 1 + 1; // var11 equals 2
\end{verbatim}

\begin{itemize}
\tightlist
\item
  if the left-hand operand is a string, returns the concatenation of the
  two operands (the left-hand one beind casted into a string)
\end{itemize}

\begin{verbatim}
 
string var12 <- "hello " + 12; // var12 equals "hello 12"
\end{verbatim}

\begin{itemize}
\tightlist
\item
  if the left-hand operand is a geometry and the right-hand operands a
  float, an integer and one of \#round, \#square or \#flat, returns a
  geometry corresponding to the left-hand operand (geometry, agent,
  point) enlarged by the first right-hand operand (distance), using a
  number of segments equal to the second right-hand operand and a flat,
  square or round end cap style
\end{itemize}

\begin{verbatim}
 
geometry var13 <- circle(5) + (5,32,#round); // var13 equals circle(10)
\end{verbatim}

\begin{itemize}
\tightlist
\item
  if the right operand is an object of any type (except a container), +
  returns a list of the elements of the left operand, to which this
  object has been added
\end{itemize}

\begin{verbatim}
 
list<int> var14 <- [1,2,3,4,5,6] + 2; // var14 equals [1,2,3,4,5,6,2] 
list<int> var15 <- [1,2,3,4,5,6] + 0; // var15 equals [1,2,3,4,5,6,0]
\end{verbatim}

\begin{itemize}
\tightlist
\item
  if the left-hand operand is a point and the right-hand a number,
  returns a new point with each coordinate as the sum of the operand
  coordinate with this number.
\end{itemize}

\begin{verbatim}
 
point var16 <- {1, 2} + 4; // var16 equals {5.0, 6.0,4.0} 
point var17 <- {1, 2} + 4.5; // var17 equals {5.5, 6.5,4.5}
\end{verbatim}

\begin{itemize}
\tightlist
\item
  if the left-hand operand is a geometry and the right-hand operand a
  float, returns a geometry corresponding to the left-hand operand
  (geometry, agent, point) enlarged by the right-hand operand distance.
  The number of segments used by default is 8 and the end cap style is
  \#round
\end{itemize}

\begin{verbatim}
 
geometry var18 <- circle(5) + 5; // var18 equals circle(10)
\end{verbatim}

\begin{itemize}
\tightlist
\item
  if the right-operand is a point, a geometry or an agent, returns the
  geometry resulting from the union between both geometries
\end{itemize}

\begin{verbatim}
 
geometry var19 <- geom1 + geom2; // var19 equals a geometry corresponding to union between geom1 and geom2
\end{verbatim}

\begin{itemize}
\tightlist
\item
  if both operands are list, +returns the concatenation of both lists.
\end{itemize}

\begin{verbatim}
 
list<int> var20 <- [1,2,3,4,5,6] + [2,4,9]; // var20 equals [1,2,3,4,5,6,2,4,9] 
list<int> var21 <- [1,2,3,4,5,6] + [0,8]; // var21 equals [1,2,3,4,5,6,0,8]
\end{verbatim}

\begin{itemize}
\tightlist
\item
  if one operand is a color and the other an integer, returns a new
  color resulting from the sum of each component of the color with the
  right operand
\end{itemize}

\begin{verbatim}
 
rgb var22 <- rgb([255, 128, 32]) + 3; // var22 equals rgb([255,131,35])
\end{verbatim}

\subsubsection{Examples:}\label{examples-6}

\begin{verbatim}
 
float var0 <- 1.0 + 1; // var0 equals 2.0 
float var1 <- 1.0 + 2.5; // var1 equals 3.5 
map var2 <- ['a'::1,'b'::2] + ('c'::3); // var2 equals ['a'::1,'b'::2,'c'::3] 
map var3 <- ['a'::1,'b'::2] + ('c'::3); // var3 equals ['a'::1,'b'::2,'c'::3] 
map var4 <- ['a'::1,'b'::2] + ['c'::3]; // var4 equals ['a'::1,'b'::2,'c'::3] 
map var5 <- ['a'::1,'b'::2] + [5::3.0]; // var5 equals ['a'::1,'b'::2,5::3.0]
\end{verbatim}

\subsubsection{See also:}\label{see-also-8}

\href{OperatorsAA\#-}{-}, \href{OperatorsAA\#*}{*},
\href{OperatorsAA\#/}{/},

\begin{center}\rule{0.5\linewidth}{\linethickness}\end{center}

\subsection{\texorpdfstring{\texttt{\textless{}}}{\textless{}}}\label{section-13}

\subsubsection{Possible use:}\label{possible-use-11}

\begin{itemize}
\tightlist
\item
  \texttt{string} \textbf{\texttt{\textless{}}} \texttt{string}
  ---\textgreater{} \texttt{bool}
\item
  \textbf{\texttt{\textless{}}} (\texttt{string} , \texttt{string})
  ---\textgreater{} \texttt{bool}
\item
  \texttt{int} \textbf{\texttt{\textless{}}} \texttt{int}
  ---\textgreater{} \texttt{bool}
\item
  \textbf{\texttt{\textless{}}} (\texttt{int} , \texttt{int})
  ---\textgreater{} \texttt{bool}
\item
  \texttt{float} \textbf{\texttt{\textless{}}} \texttt{float}
  ---\textgreater{} \texttt{bool}
\item
  \textbf{\texttt{\textless{}}} (\texttt{float} , \texttt{float})
  ---\textgreater{} \texttt{bool}
\item
  \texttt{float} \textbf{\texttt{\textless{}}} \texttt{int}
  ---\textgreater{} \texttt{bool}
\item
  \textbf{\texttt{\textless{}}} (\texttt{float} , \texttt{int})
  ---\textgreater{} \texttt{bool}
\item
  \texttt{date} \textbf{\texttt{\textless{}}} \texttt{date}
  ---\textgreater{} \texttt{bool}
\item
  \textbf{\texttt{\textless{}}} (\texttt{date} , \texttt{date})
  ---\textgreater{} \texttt{bool}
\item
  \texttt{point} \textbf{\texttt{\textless{}}} \texttt{point}
  ---\textgreater{} \texttt{bool}
\item
  \textbf{\texttt{\textless{}}} (\texttt{point} , \texttt{point})
  ---\textgreater{} \texttt{bool}
\item
  \texttt{int} \textbf{\texttt{\textless{}}} \texttt{float}
  ---\textgreater{} \texttt{bool}
\item
  \textbf{\texttt{\textless{}}} (\texttt{int} , \texttt{float})
  ---\textgreater{} \texttt{bool}
\end{itemize}

\subsubsection{Result:}\label{result-10}

true if the left-hand operand is less than the right-hand operand, false
otherwise.

\subsubsection{Special cases:}\label{special-cases-8}

\begin{itemize}
\tightlist
\item
  if one of the operands is nil, returns false\\
\item
  if both operands are String, uses a lexicographic comparison of two
  strings
\end{itemize}

\begin{verbatim}
 
bool var0 <- 'abc' < 'aeb'; // var0 equals true
\end{verbatim}

\begin{itemize}
\tightlist
\item
  if both operands are points, returns true if and only if the left
  component (x) of the left operand if less than or equal to x of the
  right one and if the right component (y) of the left operand is
  greater than or equal to y of the right one.
\end{itemize}

\begin{verbatim}
 
bool var1 <- {5,7} < {4,6}; // var1 equals false 
bool var2 <- {5,7} < {4,8}; // var2 equals false
\end{verbatim}

\subsubsection{Examples:}\label{examples-7}

\begin{verbatim}
 
bool var3 <- 3 < 7; // var3 equals true 
bool var4 <- 3.5 < 7.6; // var4 equals true 
bool var5 <- 3.5 < 7; // var5 equals true 
bool var6 <- #now < #now minus_hours 1; // var6 equals false 
bool var7 <- 3 < 2.5; // var7 equals false
\end{verbatim}

\subsubsection{See also:}\label{see-also-9}

\href{OperatorsAA\#\%3E}{\textgreater{}},
\href{OperatorsAA\#\%3E=}{\textgreater{}=},
\href{OperatorsAA\#\%3C=}{\textless{}=}, \href{OperatorsAA\#=}{=},
\href{OperatorsAA\#!=}{!=},

\begin{center}\rule{0.5\linewidth}{\linethickness}\end{center}

\subsection{\texorpdfstring{\texttt{\textless{}=}}{\textless{}=}}\label{section-14}

\subsubsection{Possible use:}\label{possible-use-12}

\begin{itemize}
\tightlist
\item
  \texttt{string} \textbf{\texttt{\textless{}=}} \texttt{string}
  ---\textgreater{} \texttt{bool}
\item
  \textbf{\texttt{\textless{}=}} (\texttt{string} , \texttt{string})
  ---\textgreater{} \texttt{bool}
\item
  \texttt{float} \textbf{\texttt{\textless{}=}} \texttt{float}
  ---\textgreater{} \texttt{bool}
\item
  \textbf{\texttt{\textless{}=}} (\texttt{float} , \texttt{float})
  ---\textgreater{} \texttt{bool}
\item
  \texttt{float} \textbf{\texttt{\textless{}=}} \texttt{int}
  ---\textgreater{} \texttt{bool}
\item
  \textbf{\texttt{\textless{}=}} (\texttt{float} , \texttt{int})
  ---\textgreater{} \texttt{bool}
\item
  \texttt{int} \textbf{\texttt{\textless{}=}} \texttt{int}
  ---\textgreater{} \texttt{bool}
\item
  \textbf{\texttt{\textless{}=}} (\texttt{int} , \texttt{int})
  ---\textgreater{} \texttt{bool}
\item
  \texttt{int} \textbf{\texttt{\textless{}=}} \texttt{float}
  ---\textgreater{} \texttt{bool}
\item
  \textbf{\texttt{\textless{}=}} (\texttt{int} , \texttt{float})
  ---\textgreater{} \texttt{bool}
\item
  \texttt{date} \textbf{\texttt{\textless{}=}} \texttt{date}
  ---\textgreater{} \texttt{bool}
\item
  \textbf{\texttt{\textless{}=}} (\texttt{date} , \texttt{date})
  ---\textgreater{} \texttt{bool}
\item
  \texttt{point} \textbf{\texttt{\textless{}=}} \texttt{point}
  ---\textgreater{} \texttt{bool}
\item
  \textbf{\texttt{\textless{}=}} (\texttt{point} , \texttt{point})
  ---\textgreater{} \texttt{bool}
\end{itemize}

\subsubsection{Result:}\label{result-11}

true if the left-hand operand is less or equal than the right-hand
operand, false otherwise.

\subsubsection{Special cases:}\label{special-cases-9}

\begin{itemize}
\tightlist
\item
  if one of the operands is nil, returns false\\
\item
  if both operands are String, uses a lexicographic comparison of two
  strings
\end{itemize}

\begin{verbatim}
 
bool var5 <- 'abc' <= 'aeb'; // var5 equals true
\end{verbatim}

\begin{itemize}
\tightlist
\item
  if both operands are points, returns true if and only if the left
  component (x) of the left operand if less than or equal to x of the
  right one and if the right component (y) of the left operand is
  greater than or equal to y of the right one.
\end{itemize}

\begin{verbatim}
 
bool var6 <- {5,7} <= {4,6}; // var6 equals false 
bool var7 <- {5,7} <= {4,8}; // var7 equals false
\end{verbatim}

\subsubsection{Examples:}\label{examples-8}

\begin{verbatim}
 
bool var0 <- 3.5 <= 3.5; // var0 equals true 
bool var1 <- 7.0 <= 7; // var1 equals true 
bool var2 <- 3 <= 7; // var2 equals true 
bool var3 <- 3 <= 2.5; // var3 equals false 
bool var4 <- #now <= #now minus_hours 1; // var4 equals false
\end{verbatim}

\subsubsection{See also:}\label{see-also-10}

\href{OperatorsAA\#\%3E}{\textgreater{}},
\href{OperatorsAA\#\%3C}{\textless{}},
\href{OperatorsAA\#\%3E=}{\textgreater{}=}, \href{OperatorsAA\#=}{=},
\href{OperatorsAA\#!=}{!=},

\begin{center}\rule{0.5\linewidth}{\linethickness}\end{center}

\subsection{\texorpdfstring{\texttt{\textless{}\textgreater{}}}{\textless{}\textgreater{}}}\label{section-15}

Same signification as \href{OperatorsAA\#!=}{!=}

\begin{center}\rule{0.5\linewidth}{\linethickness}\end{center}

\subsection{\texorpdfstring{\texttt{=}}{=}}\label{section-16}

\subsubsection{Possible use:}\label{possible-use-13}

\begin{itemize}
\tightlist
\item
  \texttt{float} \textbf{\texttt{=}} \texttt{int} ---\textgreater{}
  \texttt{bool}
\item
  \textbf{\texttt{=}} (\texttt{float} , \texttt{int}) ---\textgreater{}
  \texttt{bool}
\item
  \texttt{unknown} \textbf{\texttt{=}} \texttt{unknown}
  ---\textgreater{} \texttt{bool}
\item
  \textbf{\texttt{=}} (\texttt{unknown} , \texttt{unknown})
  ---\textgreater{} \texttt{bool}
\item
  \texttt{int} \textbf{\texttt{=}} \texttt{int} ---\textgreater{}
  \texttt{bool}
\item
  \textbf{\texttt{=}} (\texttt{int} , \texttt{int}) ---\textgreater{}
  \texttt{bool}
\item
  \texttt{float} \textbf{\texttt{=}} \texttt{float} ---\textgreater{}
  \texttt{bool}
\item
  \textbf{\texttt{=}} (\texttt{float} , \texttt{float})
  ---\textgreater{} \texttt{bool}
\item
  \texttt{int} \textbf{\texttt{=}} \texttt{float} ---\textgreater{}
  \texttt{bool}
\item
  \textbf{\texttt{=}} (\texttt{int} , \texttt{float}) ---\textgreater{}
  \texttt{bool}
\item
  \texttt{date} \textbf{\texttt{=}} \texttt{date} ---\textgreater{}
  \texttt{bool}
\item
  \textbf{\texttt{=}} (\texttt{date} , \texttt{date}) ---\textgreater{}
  \texttt{bool}
\end{itemize}

\subsubsection{Result:}\label{result-12}

returns true if both operands are equal, false otherwise returns true if
both operands are equal, false otherwise

\subsubsection{Special cases:}\label{special-cases-10}

\begin{itemize}
\tightlist
\item
  if both operands are any kind of objects, returns true if they are
  identical (i.e., the same object) or equal (comparisons between nil
  values are permitted)
\end{itemize}

\begin{verbatim}
 
bool var0 <- [2,3] = [2,3]; // var0 equals true
\end{verbatim}

\subsubsection{Examples:}\label{examples-9}

\begin{verbatim}
 
bool var1 <- 4.7 = 4; // var1 equals false 
bool var2 <- 4 = 5; // var2 equals false 
bool var3 <- 4.5 = 4.7; // var3 equals false 
bool var4 <- 3 = 3.0; // var4 equals true 
bool var5 <- 4 = 4.7; // var5 equals false 
bool var6 <- #now = #now minus_hours 1; // var6 equals false
\end{verbatim}

\subsubsection{See also:}\label{see-also-11}

\href{OperatorsAA\#!=}{!=}, \href{OperatorsAA\#\%3E}{\textgreater{}},
\href{OperatorsAA\#\%3C}{\textless{}},
\href{OperatorsAA\#\%3E=}{\textgreater{}=},
\href{OperatorsAA\#\%3C=}{\textless{}=},

\begin{center}\rule{0.5\linewidth}{\linethickness}\end{center}

\subsection{\texorpdfstring{\texttt{\textgreater{}}}{\textgreater{}}}\label{section-17}

\subsubsection{Possible use:}\label{possible-use-14}

\begin{itemize}
\tightlist
\item
  \texttt{point} \textbf{\texttt{\textgreater{}}} \texttt{point}
  ---\textgreater{} \texttt{bool}
\item
  \textbf{\texttt{\textgreater{}}} (\texttt{point} , \texttt{point})
  ---\textgreater{} \texttt{bool}
\item
  \texttt{float} \textbf{\texttt{\textgreater{}}} \texttt{int}
  ---\textgreater{} \texttt{bool}
\item
  \textbf{\texttt{\textgreater{}}} (\texttt{float} , \texttt{int})
  ---\textgreater{} \texttt{bool}
\item
  \texttt{string} \textbf{\texttt{\textgreater{}}} \texttt{string}
  ---\textgreater{} \texttt{bool}
\item
  \textbf{\texttt{\textgreater{}}} (\texttt{string} , \texttt{string})
  ---\textgreater{} \texttt{bool}
\item
  \texttt{float} \textbf{\texttt{\textgreater{}}} \texttt{float}
  ---\textgreater{} \texttt{bool}
\item
  \textbf{\texttt{\textgreater{}}} (\texttt{float} , \texttt{float})
  ---\textgreater{} \texttt{bool}
\item
  \texttt{date} \textbf{\texttt{\textgreater{}}} \texttt{date}
  ---\textgreater{} \texttt{bool}
\item
  \textbf{\texttt{\textgreater{}}} (\texttt{date} , \texttt{date})
  ---\textgreater{} \texttt{bool}
\item
  \texttt{int} \textbf{\texttt{\textgreater{}}} \texttt{float}
  ---\textgreater{} \texttt{bool}
\item
  \textbf{\texttt{\textgreater{}}} (\texttt{int} , \texttt{float})
  ---\textgreater{} \texttt{bool}
\item
  \texttt{int} \textbf{\texttt{\textgreater{}}} \texttt{int}
  ---\textgreater{} \texttt{bool}
\item
  \textbf{\texttt{\textgreater{}}} (\texttt{int} , \texttt{int})
  ---\textgreater{} \texttt{bool}
\end{itemize}

\subsubsection{Result:}\label{result-13}

true if the left-hand operand is greater than the right-hand operand,
false otherwise.

\subsubsection{Special cases:}\label{special-cases-11}

\begin{itemize}
\tightlist
\item
  if one of the operands is nil, returns false\\
\item
  if both operands are points, returns true if and only if the left
  component (x) of the left operand if greater than x of the right one
  and if the right component (y) of the left operand is greater than y
  of the right one.
\end{itemize}

\begin{verbatim}
 
bool var0 <- {5,7} > {4,6}; // var0 equals true 
bool var1 <- {5,7} > {4,8}; // var1 equals false
\end{verbatim}

\begin{itemize}
\tightlist
\item
  if both operands are String, uses a lexicographic comparison of two
  strings
\end{itemize}

\begin{verbatim}
 
bool var2 <- 'abc' > 'aeb'; // var2 equals false
\end{verbatim}

\subsubsection{Examples:}\label{examples-10}

\begin{verbatim}
 
bool var3 <- 3.5 > 7; // var3 equals false 
bool var4 <- 3.5 > 7.6; // var4 equals false 
bool var5 <- #now > #now minus_hours 1; // var5 equals true 
bool var6 <- 3 > 2.5; // var6 equals true 
bool var7 <- 3 > 7; // var7 equals false
\end{verbatim}

\subsubsection{See also:}\label{see-also-12}

\href{OperatorsAA\#\%3C}{\textless{}},
\href{OperatorsAA\#\%3E=}{\textgreater{}=},
\href{OperatorsAA\#\%3C=}{\textless{}=}, \href{OperatorsAA\#=}{=},
\href{OperatorsAA\#!=}{!=},

\begin{center}\rule{0.5\linewidth}{\linethickness}\end{center}

\subsection{\texorpdfstring{\texttt{\textgreater{}=}}{\textgreater{}=}}\label{section-18}

\subsubsection{Possible use:}\label{possible-use-15}

\begin{itemize}
\tightlist
\item
  \texttt{float} \textbf{\texttt{\textgreater{}=}} \texttt{int}
  ---\textgreater{} \texttt{bool}
\item
  \textbf{\texttt{\textgreater{}=}} (\texttt{float} , \texttt{int})
  ---\textgreater{} \texttt{bool}
\item
  \texttt{point} \textbf{\texttt{\textgreater{}=}} \texttt{point}
  ---\textgreater{} \texttt{bool}
\item
  \textbf{\texttt{\textgreater{}=}} (\texttt{point} , \texttt{point})
  ---\textgreater{} \texttt{bool}
\item
  \texttt{float} \textbf{\texttt{\textgreater{}=}} \texttt{float}
  ---\textgreater{} \texttt{bool}
\item
  \textbf{\texttt{\textgreater{}=}} (\texttt{float} , \texttt{float})
  ---\textgreater{} \texttt{bool}
\item
  \texttt{int} \textbf{\texttt{\textgreater{}=}} \texttt{int}
  ---\textgreater{} \texttt{bool}
\item
  \textbf{\texttt{\textgreater{}=}} (\texttt{int} , \texttt{int})
  ---\textgreater{} \texttt{bool}
\item
  \texttt{string} \textbf{\texttt{\textgreater{}=}} \texttt{string}
  ---\textgreater{} \texttt{bool}
\item
  \textbf{\texttt{\textgreater{}=}} (\texttt{string} , \texttt{string})
  ---\textgreater{} \texttt{bool}
\item
  \texttt{date} \textbf{\texttt{\textgreater{}=}} \texttt{date}
  ---\textgreater{} \texttt{bool}
\item
  \textbf{\texttt{\textgreater{}=}} (\texttt{date} , \texttt{date})
  ---\textgreater{} \texttt{bool}
\item
  \texttt{int} \textbf{\texttt{\textgreater{}=}} \texttt{float}
  ---\textgreater{} \texttt{bool}
\item
  \textbf{\texttt{\textgreater{}=}} (\texttt{int} , \texttt{float})
  ---\textgreater{} \texttt{bool}
\end{itemize}

\subsubsection{Result:}\label{result-14}

true if the left-hand operand is greater or equal than the right-hand
operand, false otherwise.

\subsubsection{Special cases:}\label{special-cases-12}

\begin{itemize}
\tightlist
\item
  if one of the operands is nil, returns false\\
\item
  if both operands are points, returns true if and only if the left
  component (x) of the left operand if greater or equal than x of the
  right one and if the right component (y) of the left operand is
  greater than or equal to y of the right one.
\end{itemize}

\begin{verbatim}
 
bool var0 <- {5,7} >= {4,6}; // var0 equals true 
bool var1 <- {5,7} >= {4,8}; // var1 equals false
\end{verbatim}

\begin{itemize}
\tightlist
\item
  if both operands are string, uses a lexicographic comparison of the
  two strings
\end{itemize}

\begin{verbatim}
 
bool var2 <- 'abc' >= 'aeb'; // var2 equals false 
bool var3 <- 'abc' >= 'abc'; // var3 equals true
\end{verbatim}

\subsubsection{Examples:}\label{examples-11}

\begin{verbatim}
 
bool var4 <- 3.5 >= 7; // var4 equals false 
bool var5 <- 3.5 >= 3.5; // var5 equals true 
bool var6 <- 3 >= 7; // var6 equals false 
bool var7 <- #now >= #now minus_hours 1; // var7 equals true 
bool var8 <- 3 >= 2.5; // var8 equals true
\end{verbatim}

\subsubsection{See also:}\label{see-also-13}

\href{OperatorsAA\#\%3E}{\textgreater{}},
\href{OperatorsAA\#\%3C}{\textless{}},
\href{OperatorsAA\#\%3C=}{\textless{}=}, \href{OperatorsAA\#=}{=},
\href{OperatorsAA\#!=}{!=},

\begin{center}\rule{0.5\linewidth}{\linethickness}\end{center}

\subsection{\texorpdfstring{\texttt{abs}}{abs}}\label{abs}

\subsubsection{Possible use:}\label{possible-use-16}

\begin{itemize}
\tightlist
\item
  \textbf{\texttt{abs}} (\texttt{float}) ---\textgreater{}
  \texttt{float}
\item
  \textbf{\texttt{abs}} (\texttt{int}) ---\textgreater{} \texttt{int}
\end{itemize}

\subsubsection{Result:}\label{result-15}

Returns the absolute value of the operand (so a positive int or float
depending on the type of the operand).

\subsubsection{Examples:}\label{examples-12}

\begin{verbatim}
 
float var0 <- abs (200 * -1 + 0.5); // var0 equals 199.5 
int var1 <- abs (-10); // var1 equals 10 
int var2 <- abs (10); // var2 equals 10
\end{verbatim}

\begin{center}\rule{0.5\linewidth}{\linethickness}\end{center}

\subsection{\texorpdfstring{\texttt{accumulate}}{accumulate}}\label{accumulate}

\subsubsection{Possible use:}\label{possible-use-17}

\begin{itemize}
\tightlist
\item
  \texttt{container} \textbf{\texttt{accumulate}}
  \texttt{any\ expression} ---\textgreater{} \texttt{list}
\item
  \textbf{\texttt{accumulate}} (\texttt{container} ,
  \texttt{any\ expression}) ---\textgreater{} \texttt{list}
\end{itemize}

\subsubsection{Result:}\label{result-16}

returns a new flat list, in which each element is the evaluation of the
right-hand operand. If this evaluation returns a list, the elements of
this result are added directly to the list returned

\subsubsection{Comment:}\label{comment-2}

accumulate is dedicated to the application of a same computation on each
element of a container (and returns a list). In the right-hand operand,
the keyword each can be used to represent, in turn, each of the
left-hand operand elements.

\subsubsection{Examples:}\label{examples-13}

\begin{verbatim}
 
list var0 <- [a1,a2,a3] accumulate (each neighbors_at 10); // var0 equals a flat list of all the neighbors of these three agents 
list<int> var1 <- [1,2,4] accumulate ([2,4]); // var1 equals [2,4,2,4,2,4] 
list<int> var2 <- [1,2,4] accumulate (each * 2); // var2 equals [2,4,8]
\end{verbatim}

\subsubsection{See also:}\label{see-also-14}

\href{OperatorsBC\#collect}{collect},

\begin{center}\rule{0.5\linewidth}{\linethickness}\end{center}

\subsection{\texorpdfstring{\texttt{acos}}{acos}}\label{acos}

\subsubsection{Possible use:}\label{possible-use-18}

\begin{itemize}
\tightlist
\item
  \textbf{\texttt{acos}} (\texttt{float}) ---\textgreater{}
  \texttt{float}
\item
  \textbf{\texttt{acos}} (\texttt{int}) ---\textgreater{} \texttt{float}
\end{itemize}

\subsubsection{Result:}\label{result-17}

Returns the value (in the interval {[}0,180{]}, in decimal degrees) of
the arccos of the operand (which should be in {[}-1,1{]}).

\subsubsection{Special cases:}\label{special-cases-13}

\begin{itemize}
\tightlist
\item
  if the right-hand operand is outside of the {[}-1,1{]} interval,
  returns NaN
\end{itemize}

\subsubsection{Examples:}\label{examples-14}

\begin{verbatim}
 
float var0 <- acos (0); // var0 equals 90.0
\end{verbatim}

\subsubsection{See also:}\label{see-also-15}

\href{OperatorsAA\#asin}{asin}, \href{OperatorsAA\#atan}{atan},
\href{OperatorsBC\#cos}{cos},

\begin{center}\rule{0.5\linewidth}{\linethickness}\end{center}

\subsection{\texorpdfstring{\texttt{action}}{action}}\label{action}

\subsubsection{Possible use:}\label{possible-use-19}

\begin{itemize}
\tightlist
\item
  \textbf{\texttt{action}} (\texttt{any}) ---\textgreater{}
  \texttt{action}
\end{itemize}

\subsubsection{Result:}\label{result-18}

Casts the operand into the type action

\begin{center}\rule{0.5\linewidth}{\linethickness}\end{center}

\subsection{\texorpdfstring{\texttt{add\_days}}{add\_days}}\label{add_days}

Same signification as \href{OperatorsNR\#plus_days}{plus\_days}

\begin{center}\rule{0.5\linewidth}{\linethickness}\end{center}

\subsection{\texorpdfstring{\texttt{add\_edge}}{add\_edge}}\label{add_edge}

\subsubsection{Possible use:}\label{possible-use-20}

\begin{itemize}
\tightlist
\item
  \texttt{graph} \textbf{\texttt{add\_edge}} \texttt{pair}
  ---\textgreater{} \texttt{graph}
\item
  \textbf{\texttt{add\_edge}} (\texttt{graph} , \texttt{pair})
  ---\textgreater{} \texttt{graph}
\end{itemize}

\subsubsection{Result:}\label{result-19}

add an edge between a source vertex and a target vertex (resp. the left
and the right element of the pair operand)

\subsubsection{Comment:}\label{comment-3}

if the edge already exists, the graph is unchanged

\subsubsection{Examples:}\label{examples-15}

\begin{verbatim}
graph <- graph add_edge (source::target); 
\end{verbatim}

\subsubsection{See also:}\label{see-also-16}

\href{OperatorsAA\#add_node}{add\_node},
\href{OperatorsDH\#graph}{graph},

\begin{center}\rule{0.5\linewidth}{\linethickness}\end{center}

\subsection{\texorpdfstring{\texttt{add\_hours}}{add\_hours}}\label{add_hours}

Same signification as \href{OperatorsNR\#plus_hours}{plus\_hours}

\begin{center}\rule{0.5\linewidth}{\linethickness}\end{center}

\subsection{\texorpdfstring{\texttt{add\_minutes}}{add\_minutes}}\label{add_minutes}

Same signification as \href{OperatorsNR\#plus_minutes}{plus\_minutes}

\begin{center}\rule{0.5\linewidth}{\linethickness}\end{center}

\subsection{\texorpdfstring{\texttt{add\_months}}{add\_months}}\label{add_months}

Same signification as \href{OperatorsNR\#plus_months}{plus\_months}

\begin{center}\rule{0.5\linewidth}{\linethickness}\end{center}

\subsection{\texorpdfstring{\texttt{add\_ms}}{add\_ms}}\label{add_ms}

Same signification as \href{OperatorsNR\#plus_ms}{plus\_ms}

\begin{center}\rule{0.5\linewidth}{\linethickness}\end{center}

\subsection{\texorpdfstring{\texttt{add\_node}}{add\_node}}\label{add_node}

\subsubsection{Possible use:}\label{possible-use-21}

\begin{itemize}
\tightlist
\item
  \texttt{graph} \textbf{\texttt{add\_node}} \texttt{geometry}
  ---\textgreater{} \texttt{graph}
\item
  \textbf{\texttt{add\_node}} (\texttt{graph} , \texttt{geometry})
  ---\textgreater{} \texttt{graph}
\end{itemize}

\subsubsection{Result:}\label{result-20}

adds a node in a graph.

\subsubsection{Examples:}\label{examples-16}

\begin{verbatim}
 
graph var0 <- graph add_node node(0) ; // var0 equals the graph with node(0)
\end{verbatim}

\subsubsection{See also:}\label{see-also-17}

\href{OperatorsAA\#add_edge}{add\_edge},
\href{OperatorsDH\#graph}{graph},

\begin{center}\rule{0.5\linewidth}{\linethickness}\end{center}

\subsection{\texorpdfstring{\texttt{add\_point}}{add\_point}}\label{add_point}

\subsubsection{Possible use:}\label{possible-use-22}

\begin{itemize}
\tightlist
\item
  \texttt{geometry} \textbf{\texttt{add\_point}} \texttt{point}
  ---\textgreater{} \texttt{geometry}
\item
  \textbf{\texttt{add\_point}} (\texttt{geometry} , \texttt{point})
  ---\textgreater{} \texttt{geometry}
\end{itemize}

\subsubsection{Result:}\label{result-21}

A new geometry resulting from the addition of the right point
(coordinate) to the left-hand geometry. Note that adding a point to a
line or polyline will always return a closed contour. Also note that the
position at which the added point will appear in the geometry is not
necessarily the last one, as points are always ordered in a clockwise
fashion in geometries

\subsubsection{Examples:}\label{examples-17}

\begin{verbatim}
 
geometry var0 <- polygon([{10,10},{10,20},{20,20}]) add_point {20,10}; // var0 equals polygon([{10,10},{10,20},{20,20},{20,10}])
\end{verbatim}

\begin{center}\rule{0.5\linewidth}{\linethickness}\end{center}

\subsection{\texorpdfstring{\texttt{add\_seconds}}{add\_seconds}}\label{add_seconds}

Same signification as \href{OperatorsAA\#+}{+}

\begin{center}\rule{0.5\linewidth}{\linethickness}\end{center}

\subsection{\texorpdfstring{\texttt{add\_weeks}}{add\_weeks}}\label{add_weeks}

Same signification as \href{OperatorsNR\#plus_weeks}{plus\_weeks}

\begin{center}\rule{0.5\linewidth}{\linethickness}\end{center}

\subsection{\texorpdfstring{\texttt{add\_years}}{add\_years}}\label{add_years}

Same signification as \href{OperatorsNR\#plus_years}{plus\_years}

\begin{center}\rule{0.5\linewidth}{\linethickness}\end{center}

\subsection{\texorpdfstring{\texttt{adjacency}}{adjacency}}\label{adjacency}

\subsubsection{Possible use:}\label{possible-use-23}

\begin{itemize}
\tightlist
\item
  \textbf{\texttt{adjacency}} (\texttt{graph}) ---\textgreater{}
  \texttt{matrix\textless{}float\textgreater{}}
\end{itemize}

\subsubsection{Result:}\label{result-22}

adjacency matrix of the given graph.

\begin{center}\rule{0.5\linewidth}{\linethickness}\end{center}

\subsection{\texorpdfstring{\texttt{after}}{after}}\label{after}

\subsubsection{Possible use:}\label{possible-use-24}

\begin{itemize}
\tightlist
\item
  \textbf{\texttt{after}} (\texttt{date}) ---\textgreater{}
  \texttt{bool}
\item
  \texttt{any\ expression} \textbf{\texttt{after}} \texttt{date}
  ---\textgreater{} \texttt{bool}
\item
  \textbf{\texttt{after}} (\texttt{any\ expression} , \texttt{date})
  ---\textgreater{} \texttt{bool}
\end{itemize}

\subsubsection{Result:}\label{result-23}

Returns true if the current\_date of the model is strictly after the
date passed in argument. Synonym of `current\_date \textgreater{}
argument'. Can be used in its composed form with 2 arguments to express
the lower boundary for the computation of a frequency. Note that only
dates strictly after this one will be tested against the frequency

\subsubsection{Examples:}\label{examples-18}

\begin{verbatim}
reflex when: after(starting_date) {}    // this reflex will always be run after the first step reflex when: false after(starting date + #10days) {}     // This reflex will not be run after this date. Better to use 'until' or 'before' in that case every(2#days) after (starting_date + 1#day)  // the computation will return true every two days (using the starting_date of the model as the starting point) only for the dates strictly after this starting_date + 1#day 
\end{verbatim}

\begin{center}\rule{0.5\linewidth}{\linethickness}\end{center}

\subsection{\texorpdfstring{\texttt{agent}}{agent}}\label{agent}

\subsubsection{Possible use:}\label{possible-use-25}

\begin{itemize}
\tightlist
\item
  \textbf{\texttt{agent}} (\texttt{any}) ---\textgreater{}
  \texttt{agent}
\end{itemize}

\subsubsection{Result:}\label{result-24}

Casts the operand into the type agent

\begin{center}\rule{0.5\linewidth}{\linethickness}\end{center}

\subsection{\texorpdfstring{\texttt{agent\_closest\_to}}{agent\_closest\_to}}\label{agent_closest_to}

\subsubsection{Possible use:}\label{possible-use-26}

\begin{itemize}
\tightlist
\item
  \textbf{\texttt{agent\_closest\_to}} (\texttt{unknown})
  ---\textgreater{} \texttt{agent}
\end{itemize}

\subsubsection{Result:}\label{result-25}

An agent, the closest to the operand (casted as a geometry).

\subsubsection{Comment:}\label{comment-4}

the distance is computed in the topology of the calling agent (the agent
in which this operator is used), with the distance algorithm specific to
the topology.

\subsubsection{Examples:}\label{examples-19}

\begin{verbatim}
 
agent var0 <- agent_closest_to(self); // var0 equals the closest agent to the agent applying the operator.
\end{verbatim}

\subsubsection{See also:}\label{see-also-18}

\href{OperatorsNR\#neighbors_at}{neighbors\_at},
\href{OperatorsNR\#neighbors_of}{neighbors\_of},
\href{OperatorsAA\#agents_inside}{agents\_inside},
\href{OperatorsAA\#agents_overlapping}{agents\_overlapping},
\href{OperatorsBC\#closest_to}{closest\_to},
\href{OperatorsIM\#inside}{inside},
\href{OperatorsNR\#overlapping}{overlapping},

\begin{center}\rule{0.5\linewidth}{\linethickness}\end{center}

\subsection{\texorpdfstring{\texttt{agent\_farthest\_to}}{agent\_farthest\_to}}\label{agent_farthest_to}

\subsubsection{Possible use:}\label{possible-use-27}

\begin{itemize}
\tightlist
\item
  \textbf{\texttt{agent\_farthest\_to}} (\texttt{unknown})
  ---\textgreater{} \texttt{agent}
\end{itemize}

\subsubsection{Result:}\label{result-26}

An agent, the farthest to the operand (casted as a geometry).

\subsubsection{Comment:}\label{comment-5}

the distance is computed in the topology of the calling agent (the agent
in which this operator is used), with the distance algorithm specific to
the topology.

\subsubsection{Examples:}\label{examples-20}

\begin{verbatim}
 
agent var0 <- agent_farthest_to(self); // var0 equals the farthest agent to the agent applying the operator.
\end{verbatim}

\subsubsection{See also:}\label{see-also-19}

\href{OperatorsNR\#neighbors_at}{neighbors\_at},
\href{OperatorsNR\#neighbors_of}{neighbors\_of},
\href{OperatorsAA\#agents_inside}{agents\_inside},
\href{OperatorsAA\#agents_overlapping}{agents\_overlapping},
\href{OperatorsBC\#closest_to}{closest\_to},
\href{OperatorsIM\#inside}{inside},
\href{OperatorsNR\#overlapping}{overlapping},
\href{OperatorsAA\#agent_closest_to}{agent\_closest\_to},
\href{OperatorsDH\#farthest_to}{farthest\_to},

\begin{center}\rule{0.5\linewidth}{\linethickness}\end{center}

\subsection{\texorpdfstring{\texttt{agent\_from\_geometry}}{agent\_from\_geometry}}\label{agent_from_geometry}

\subsubsection{Possible use:}\label{possible-use-28}

\begin{itemize}
\tightlist
\item
  \texttt{path} \textbf{\texttt{agent\_from\_geometry}}
  \texttt{geometry} ---\textgreater{} \texttt{agent}
\item
  \textbf{\texttt{agent\_from\_geometry}} (\texttt{path} ,
  \texttt{geometry}) ---\textgreater{} \texttt{agent}
\end{itemize}

\subsubsection{Result:}\label{result-27}

returns the agent corresponding to given geometry (right-hand operand)
in the given path (left-hand operand).

\subsubsection{Special cases:}\label{special-cases-14}

\begin{itemize}
\tightlist
\item
  if the left-hand operand is nil, returns nil
\end{itemize}

\subsubsection{Examples:}\label{examples-21}

\begin{verbatim}
geometry line <- one_of(path_followed.segments); road ag <- road(path_followed agent_from_geometry line); 
\end{verbatim}

\subsubsection{See also:}\label{see-also-20}

\href{OperatorsNR\#path}{path},

\begin{center}\rule{0.5\linewidth}{\linethickness}\end{center}

\subsection{\texorpdfstring{\texttt{agents\_at\_distance}}{agents\_at\_distance}}\label{agents_at_distance}

\subsubsection{Possible use:}\label{possible-use-29}

\begin{itemize}
\tightlist
\item
  \textbf{\texttt{agents\_at\_distance}} (\texttt{float})
  ---\textgreater{} \texttt{list}
\end{itemize}

\subsubsection{Result:}\label{result-28}

A list of agents situated at a distance lower than the right argument.

\subsubsection{Examples:}\label{examples-22}

\begin{verbatim}
 
list var0 <- agents_at_distance(20); // var0 equals all the agents (excluding the caller) which distance to the caller is lower than 20
\end{verbatim}

\subsubsection{See also:}\label{see-also-21}

\href{OperatorsNR\#neighbors_at}{neighbors\_at},
\href{OperatorsNR\#neighbors_of}{neighbors\_of},
\href{OperatorsAA\#agent_closest_to}{agent\_closest\_to},
\href{OperatorsAA\#agents_inside}{agents\_inside},
\href{OperatorsBC\#closest_to}{closest\_to},
\href{OperatorsIM\#inside}{inside},
\href{OperatorsNR\#overlapping}{overlapping},
\href{OperatorsAA\#at_distance}{at\_distance},

\begin{center}\rule{0.5\linewidth}{\linethickness}\end{center}

\subsection{\texorpdfstring{\texttt{agents\_inside}}{agents\_inside}}\label{agents_inside}

\subsubsection{Possible use:}\label{possible-use-30}

\begin{itemize}
\tightlist
\item
  \textbf{\texttt{agents\_inside}} (\texttt{unknown}) ---\textgreater{}
  \texttt{list\textless{}agent\textgreater{}}
\end{itemize}

\subsubsection{Result:}\label{result-29}

A list of agents covered by the operand (casted as a geometry).

\subsubsection{Examples:}\label{examples-23}

\begin{verbatim}
 
list<agent> var0 <- agents_inside(self); // var0 equals the agents that are covered by the shape of the agent applying the operator.
\end{verbatim}

\subsubsection{See also:}\label{see-also-22}

\href{OperatorsAA\#agent_closest_to}{agent\_closest\_to},
\href{OperatorsAA\#agents_overlapping}{agents\_overlapping},
\href{OperatorsBC\#closest_to}{closest\_to},
\href{OperatorsIM\#inside}{inside},
\href{OperatorsNR\#overlapping}{overlapping},

\begin{center}\rule{0.5\linewidth}{\linethickness}\end{center}

\subsection{\texorpdfstring{\texttt{agents\_overlapping}}{agents\_overlapping}}\label{agents_overlapping}

\subsubsection{Possible use:}\label{possible-use-31}

\begin{itemize}
\tightlist
\item
  \textbf{\texttt{agents\_overlapping}} (\texttt{unknown})
  ---\textgreater{} \texttt{list\textless{}agent\textgreater{}}
\end{itemize}

\subsubsection{Result:}\label{result-30}

A list of agents overlapping the operand (casted as a geometry).

\subsubsection{Examples:}\label{examples-24}

\begin{verbatim}
 
list<agent> var0 <- agents_overlapping(self); // var0 equals the agents that overlap the shape of the agent applying the operator.
\end{verbatim}

\subsubsection{See also:}\label{see-also-23}

\href{OperatorsNR\#neighbors_at}{neighbors\_at},
\href{OperatorsNR\#neighbors_of}{neighbors\_of},
\href{OperatorsAA\#agent_closest_to}{agent\_closest\_to},
\href{OperatorsAA\#agents_inside}{agents\_inside},
\href{OperatorsBC\#closest_to}{closest\_to},
\href{OperatorsIM\#inside}{inside},
\href{OperatorsNR\#overlapping}{overlapping},
\href{OperatorsAA\#at_distance}{at\_distance},

\begin{center}\rule{0.5\linewidth}{\linethickness}\end{center}

\subsection{\texorpdfstring{\texttt{all\_pairs\_shortest\_path}}{all\_pairs\_shortest\_path}}\label{all_pairs_shortest_path}

\subsubsection{Possible use:}\label{possible-use-32}

\begin{itemize}
\tightlist
\item
  \textbf{\texttt{all\_pairs\_shortest\_path}} (\texttt{graph})
  ---\textgreater{} \texttt{matrix\textless{}int\textgreater{}}
\end{itemize}

\subsubsection{Result:}\label{result-31}

returns the successor matrix of shortest paths between all node pairs
(rows: source, columns: target): a cell (i,j) will thus contains the
next node in the shortest path between i and j.

\subsubsection{Examples:}\label{examples-25}

\begin{verbatim}
 
matrix<int> var0 <- all_pairs_shortest_paths(my_graph); // var0 equals shortest_paths_matrix will contain all pairs of shortest paths
\end{verbatim}

\begin{center}\rule{0.5\linewidth}{\linethickness}\end{center}

\subsection{\texorpdfstring{\texttt{alpha\_index}}{alpha\_index}}\label{alpha_index}

\subsubsection{Possible use:}\label{possible-use-33}

\begin{itemize}
\tightlist
\item
  \textbf{\texttt{alpha\_index}} (\texttt{graph}) ---\textgreater{}
  \texttt{float}
\end{itemize}

\subsubsection{Result:}\label{result-32}

returns the alpha index of the graph (measure of connectivity which
evaluates the number of cycles in a graph in comparison with the maximum
number of cycles. The higher the alpha index, the more a network is
connected: alpha = nb\_cycles / (2\texttt{*}S-5) - planar graph)

\subsubsection{Examples:}\label{examples-26}

\begin{verbatim}
 
float var1 <- alpha_index(graphEpidemio); // var1 equals the alpha index of the graph
\end{verbatim}

\subsubsection{See also:}\label{see-also-24}

\href{OperatorsBC\#beta_index}{beta\_index},
\href{OperatorsDH\#gamma_index}{gamma\_index},
\href{OperatorsNR\#nb_cycles}{nb\_cycles},
\href{OperatorsBC\#connectivity_index}{connectivity\_index},

\begin{center}\rule{0.5\linewidth}{\linethickness}\end{center}

\subsection{\texorpdfstring{\texttt{among}}{among}}\label{among}

\subsubsection{Possible use:}\label{possible-use-34}

\begin{itemize}
\tightlist
\item
  \texttt{int} \textbf{\texttt{among}} \texttt{container}
  ---\textgreater{} \texttt{list}
\item
  \textbf{\texttt{among}} (\texttt{int} , \texttt{container})
  ---\textgreater{} \texttt{list}
\end{itemize}

\subsubsection{Result:}\label{result-33}

Returns a list of length the value of the left-hand operand, containing
random elements from the right-hand operand. As of GAMA 1.6, the order
in which the elements are returned can be different than the order in
which they appear in the right-hand container

\subsubsection{Special cases:}\label{special-cases-15}

\begin{itemize}
\tightlist
\item
  if the right-hand operand is empty, among returns a new empty list. If
  it is nil, it throws an error.\\
\item
  if the left-hand operand is greater than the length of the right-hand
  operand, among returns the right-hand operand (converted as a list).
  If it is smaller or equal to zero, it returns an empty list
\end{itemize}

\subsubsection{Examples:}\label{examples-27}

\begin{verbatim}
 
list<int> var0 <- 3 among [1,2,4,3,5,7,6,8]; // var0 equals [1,2,8] (for example) 
list var1 <- 3 among g2; // var1 equals [node6,node11,node7] 
list var2 <- 3 among list(node); // var2 equals [node1,node11,node4] 
list<int> var3 <- 1 among [1::2,3::4]; // var3 equals 2 or 4
\end{verbatim}

\begin{center}\rule{0.5\linewidth}{\linethickness}\end{center}

\subsection{\texorpdfstring{\texttt{and}}{and}}\label{and}

\subsubsection{Possible use:}\label{possible-use-35}

\begin{itemize}
\tightlist
\item
  \texttt{bool} \textbf{\texttt{and}} \texttt{any\ expression}
  ---\textgreater{} \texttt{bool}
\item
  \textbf{\texttt{and}} (\texttt{bool} , \texttt{any\ expression})
  ---\textgreater{} \texttt{bool}
\end{itemize}

\subsubsection{Result:}\label{result-34}

a bool value, equal to the logical and between the left-hand operand and
the right-hand operand.

\subsubsection{Comment:}\label{comment-6}

both operands are always casted to bool before applying the operator.
Thus, an expression like (1 and 0) is accepted and returns false.

\subsubsection{See also:}\label{see-also-25}

\href{OperatorsBC\#bool}{bool}, \href{OperatorsNR\#or}{or},
\href{OperatorsAA\#!}{!},

\begin{center}\rule{0.5\linewidth}{\linethickness}\end{center}

\subsection{\texorpdfstring{\texttt{and}}{and}}\label{and-1}

\subsubsection{Possible use:}\label{possible-use-36}

\begin{itemize}
\tightlist
\item
  \texttt{predicate} \textbf{\texttt{and}} \texttt{predicate}
  ---\textgreater{} \texttt{predicate}
\item
  \textbf{\texttt{and}} (\texttt{predicate} , \texttt{predicate})
  ---\textgreater{} \texttt{predicate}
\end{itemize}

\subsubsection{Result:}\label{result-35}

create a new predicate from two others by including them as
subintentions

\subsubsection{Examples:}\label{examples-28}

\begin{verbatim}
predicate1 and predicate2 
\end{verbatim}

\begin{center}\rule{0.5\linewidth}{\linethickness}\end{center}

\subsection{\texorpdfstring{\texttt{angle\_between}}{angle\_between}}\label{angle_between}

\subsubsection{Possible use:}\label{possible-use-37}

\begin{itemize}
\tightlist
\item
  \textbf{\texttt{angle\_between}} (\texttt{point}, \texttt{point},
  \texttt{point}) ---\textgreater{} \texttt{float}
\end{itemize}

\subsubsection{Result:}\label{result-36}

the angle between vectors P0P1 and P0P2 (P0, P1, P2 being the three
point operands)

\subsubsection{Examples:}\label{examples-29}

\begin{verbatim}
 
float var0 <- angle_between({5,5},{10,5},{5,10}); // var0 equals 90
\end{verbatim}

\begin{center}\rule{0.5\linewidth}{\linethickness}\end{center}

\subsection{\texorpdfstring{\texttt{any}}{any}}\label{any}

Same signification as \href{OperatorsNR\#one_of}{one\_of}

\begin{center}\rule{0.5\linewidth}{\linethickness}\end{center}

\subsection{\texorpdfstring{\texttt{any\_location\_in}}{any\_location\_in}}\label{any_location_in}

\subsubsection{Possible use:}\label{possible-use-38}

\begin{itemize}
\tightlist
\item
  \textbf{\texttt{any\_location\_in}} (\texttt{geometry})
  ---\textgreater{} \texttt{point}
\end{itemize}

\subsubsection{Result:}\label{result-37}

A point inside (or touching) the operand-geometry.

\subsubsection{Examples:}\label{examples-30}

\begin{verbatim}
 
point var0 <- any_location_in(square(5)); // var0 equals a point in the square, for example : {3,4.6}.
\end{verbatim}

\subsubsection{See also:}\label{see-also-26}

\href{OperatorsBC\#closest_points_with}{closest\_points\_with},
\href{OperatorsDH\#farthest_point_to}{farthest\_point\_to},
\href{OperatorsNR\#points_at}{points\_at},

\begin{center}\rule{0.5\linewidth}{\linethickness}\end{center}

\subsection{\texorpdfstring{\texttt{any\_point\_in}}{any\_point\_in}}\label{any_point_in}

Same signification as
\href{OperatorsAA\#any_location_in}{any\_location\_in}

\begin{center}\rule{0.5\linewidth}{\linethickness}\end{center}

\subsection{\texorpdfstring{\texttt{append\_horizontally}}{append\_horizontally}}\label{append_horizontally}

\subsubsection{Possible use:}\label{possible-use-39}

\begin{itemize}
\tightlist
\item
  \texttt{matrix} \textbf{\texttt{append\_horizontally}} \texttt{matrix}
  ---\textgreater{} \texttt{matrix}
\item
  \textbf{\texttt{append\_horizontally}} (\texttt{matrix} ,
  \texttt{matrix}) ---\textgreater{} \texttt{matrix}
\item
  \texttt{matrix} \textbf{\texttt{append\_horizontally}} \texttt{matrix}
  ---\textgreater{} \texttt{matrix}
\item
  \textbf{\texttt{append\_horizontally}} (\texttt{matrix} ,
  \texttt{matrix}) ---\textgreater{} \texttt{matrix}
\end{itemize}

\subsubsection{Result:}\label{result-38}

A matrix resulting from the concatenation of the rows of the two given
matrices. If not both numerical or both object matrices, returns the
first matrix.

\subsubsection{Examples:}\label{examples-31}

\begin{verbatim}
 
matrix var0 <- matrix([[1.0,2.0],[3.0,4.0]]) append_horizontally matrix([[1,2],[3,4]]); // var0 equals matrix([[1.0,2.0],[3.0,4.0],[1.0,2.0],[3.0,4.0]])
\end{verbatim}

\begin{center}\rule{0.5\linewidth}{\linethickness}\end{center}

\subsection{\texorpdfstring{\texttt{append\_vertically}}{append\_vertically}}\label{append_vertically}

\subsubsection{Possible use:}\label{possible-use-40}

\begin{itemize}
\tightlist
\item
  \texttt{matrix} \textbf{\texttt{append\_vertically}} \texttt{matrix}
  ---\textgreater{} \texttt{matrix}
\item
  \textbf{\texttt{append\_vertically}} (\texttt{matrix} ,
  \texttt{matrix}) ---\textgreater{} \texttt{matrix}
\item
  \texttt{matrix} \textbf{\texttt{append\_vertically}} \texttt{matrix}
  ---\textgreater{} \texttt{matrix}
\item
  \textbf{\texttt{append\_vertically}} (\texttt{matrix} ,
  \texttt{matrix}) ---\textgreater{} \texttt{matrix}
\end{itemize}

\subsubsection{Result:}\label{result-39}

A matrix resulting from the concatenation of the columns of the two
given matrices. If not both numerical or both object matrices, returns
the first matrix.

\subsubsection{Examples:}\label{examples-32}

\begin{verbatim}
 
matrix var0 <- matrix([[1,2],[3,4]]) append_vertically matrix([[1,2],[3,4]]); // var0 equals matrix([[1,2,1,2],[3,4,3,4]])
\end{verbatim}

\begin{center}\rule{0.5\linewidth}{\linethickness}\end{center}

\subsection{\texorpdfstring{\texttt{arc}}{arc}}\label{arc}

\subsubsection{Possible use:}\label{possible-use-41}

\begin{itemize}
\tightlist
\item
  \textbf{\texttt{arc}} (\texttt{float}, \texttt{float}, \texttt{float})
  ---\textgreater{} \texttt{geometry}
\item
  \textbf{\texttt{arc}} (\texttt{float}, \texttt{float}, \texttt{float},
  \texttt{bool}) ---\textgreater{} \texttt{geometry}
\end{itemize}

\subsubsection{Result:}\label{result-40}

An arc, which radius is equal to the first operand, heading to the
second and amplitude the third An arc, which radius is equal to the
first operand, heading to the second, amplitude to the third and a
boolean indicating whether to return a linestring or a polygon to the
fourth

\subsubsection{Comment:}\label{comment-7}

the center of the arc is by default the location of the current agent in
which has been called this operator. This operator returns a polygon by
default.the center of the arc is by default the location of the current
agent in which has been called this operator.

\subsubsection{Special cases:}\label{special-cases-16}

\begin{itemize}
\tightlist
\item
  returns a point if the radius operand is lower or equal to 0.\\
\item
  returns a point if the radius operand is lower or equal to 0.
\end{itemize}

\subsubsection{Examples:}\label{examples-33}

\begin{verbatim}
 
geometry var0 <- arc(4,45,90); // var0 equals a geometry as an arc of radius 4, in a direction of 45° and an amplitude of 90° 
geometry var1 <- arc(4,45,90, false); // var1 equals a geometry as an arc of radius 4, in a direction of 45° and an amplitude of 90°, which only contains the points on the arc
\end{verbatim}

\subsubsection{See also:}\label{see-also-27}

\href{OperatorsAA\#around}{around}, \href{OperatorsBC\#cone}{cone},
\href{OperatorsIM\#line}{line}, \href{OperatorsIM\#link}{link},
\href{OperatorsNR\#norm}{norm}, \href{OperatorsNR\#point}{point},
\href{OperatorsNR\#polygon}{polygon},
\href{OperatorsNR\#polyline}{polyline},
\href{OperatorsSZ\#super_ellipse}{super\_ellipse},
\href{OperatorsNR\#rectangle}{rectangle},
\href{OperatorsSZ\#square}{square}, \href{OperatorsBC\#circle}{circle},
\href{OperatorsDH\#ellipse}{ellipse},
\href{OperatorsSZ\#triangle}{triangle},

\begin{center}\rule{0.5\linewidth}{\linethickness}\end{center}

\subsection{\texorpdfstring{\texttt{around}}{around}}\label{around}

\subsubsection{Possible use:}\label{possible-use-42}

\begin{itemize}
\tightlist
\item
  \texttt{float} \textbf{\texttt{around}} \texttt{unknown}
  ---\textgreater{} \texttt{geometry}
\item
  \textbf{\texttt{around}} (\texttt{float} , \texttt{unknown})
  ---\textgreater{} \texttt{geometry}
\end{itemize}

\subsubsection{Result:}\label{result-41}

A geometry resulting from the difference between a buffer around the
right-operand casted in geometry at a distance left-operand
(right-operand buffer left-operand) and the right-operand casted as
geometry.

\subsubsection{Special cases:}\label{special-cases-17}

\begin{itemize}
\tightlist
\item
  returns a circle geometry of radius right-operand if the left-operand
  is nil
\end{itemize}

\subsubsection{Examples:}\label{examples-34}

\begin{verbatim}
 
geometry var0 <- 10 around circle(5); // var0 equals the ring geometry between 5 and 10.
\end{verbatim}

\subsubsection{See also:}\label{see-also-28}

\href{OperatorsBC\#circle}{circle}, \href{OperatorsBC\#cone}{cone},
\href{OperatorsIM\#line}{line}, \href{OperatorsIM\#link}{link},
\href{OperatorsNR\#norm}{norm}, \href{OperatorsNR\#point}{point},
\href{OperatorsNR\#polygon}{polygon},
\href{OperatorsNR\#polyline}{polyline},
\href{OperatorsNR\#rectangle}{rectangle},
\href{OperatorsSZ\#square}{square},
\href{OperatorsSZ\#triangle}{triangle},

\begin{center}\rule{0.5\linewidth}{\linethickness}\end{center}

\subsection{\texorpdfstring{\texttt{as}}{as}}\label{as}

\subsubsection{Possible use:}\label{possible-use-43}

\begin{itemize}
\tightlist
\item
  \texttt{unknown} \textbf{\texttt{as}} \texttt{msi.gaml.types.IType}
  ---\textgreater{} \texttt{unknown}
\item
  \textbf{\texttt{as}} (\texttt{unknown} ,
  \texttt{msi.gaml.types.IType}) ---\textgreater{} \texttt{unknown}
\end{itemize}

\subsubsection{Result:}\label{result-42}

casting of the first argument into a given type

\subsubsection{Comment:}\label{comment-8}

It is equivalent to the application of the type operator on the left
operand.

\subsubsection{Examples:}\label{examples-35}

\begin{verbatim}
 
int var0 <- 3.5 as int; // var0 equals int(3.5)
\end{verbatim}

\begin{center}\rule{0.5\linewidth}{\linethickness}\end{center}

\subsection{\texorpdfstring{\texttt{as\_4\_grid}}{as\_4\_grid}}\label{as_4_grid}

\subsubsection{Possible use:}\label{possible-use-44}

\begin{itemize}
\tightlist
\item
  \texttt{geometry} \textbf{\texttt{as\_4\_grid}} \texttt{point}
  ---\textgreater{} \texttt{matrix}
\item
  \textbf{\texttt{as\_4\_grid}} (\texttt{geometry} , \texttt{point})
  ---\textgreater{} \texttt{matrix}
\end{itemize}

\subsubsection{Result:}\label{result-43}

A matrix of square geometries (grid with 4-neighborhood) with dimension
given by the right-hand operand (\{nb\_cols, nb\_lines\}) corresponding
to the square tessellation of the left-hand operand geometry (geometry,
agent)

\subsubsection{Examples:}\label{examples-36}

\begin{verbatim}
 
matrix var0 <- self as_4_grid {10, 5}; // var0 equals the matrix of square geometries (grid with 4-neighborhood) with 10 columns and 5 lines corresponding to the square tessellation of the geometry of the agent applying the operator.
\end{verbatim}

\subsubsection{See also:}\label{see-also-29}

\href{OperatorsAA\#as_grid}{as\_grid},
\href{OperatorsAA\#as_hexagonal_grid}{as\_hexagonal\_grid},

\begin{center}\rule{0.5\linewidth}{\linethickness}\end{center}

\subsection{\texorpdfstring{\texttt{as\_distance\_graph}}{as\_distance\_graph}}\label{as_distance_graph}

\subsubsection{Possible use:}\label{possible-use-45}

\begin{itemize}
\tightlist
\item
  \texttt{container} \textbf{\texttt{as\_distance\_graph}} \texttt{map}
  ---\textgreater{} \texttt{graph}
\item
  \textbf{\texttt{as\_distance\_graph}} (\texttt{container} ,
  \texttt{map}) ---\textgreater{} \texttt{graph}
\item
  \texttt{container} \textbf{\texttt{as\_distance\_graph}}
  \texttt{float} ---\textgreater{} \texttt{graph}
\item
  \textbf{\texttt{as\_distance\_graph}} (\texttt{container} ,
  \texttt{float}) ---\textgreater{} \texttt{graph}
\item
  \textbf{\texttt{as\_distance\_graph}} (\texttt{container},
  \texttt{float}, \texttt{species}) ---\textgreater{} \texttt{graph}
\end{itemize}

\subsubsection{Result:}\label{result-44}

creates a graph from a list of vertices (left-hand operand). An edge is
created between each pair of vertices close enough (less than a
distance, right-hand operand).

\subsubsection{Comment:}\label{comment-9}

as\_distance\_graph is more efficient for a list of points than
as\_intersection\_graph.

\subsubsection{Examples:}\label{examples-37}

\begin{verbatim}
list(ant) as_distance_graph 3.0 
\end{verbatim}

\subsubsection{See also:}\label{see-also-30}

\href{OperatorsAA\#as_intersection_graph}{as\_intersection\_graph},
\href{OperatorsAA\#as_edge_graph}{as\_edge\_graph},

\begin{center}\rule{0.5\linewidth}{\linethickness}\end{center}

\subsection{\texorpdfstring{\texttt{as\_driving\_graph}}{as\_driving\_graph}}\label{as_driving_graph}

\subsubsection{Possible use:}\label{possible-use-46}

\begin{itemize}
\tightlist
\item
  \texttt{container} \textbf{\texttt{as\_driving\_graph}}
  \texttt{container} ---\textgreater{} \texttt{graph}
\item
  \textbf{\texttt{as\_driving\_graph}} (\texttt{container} ,
  \texttt{container}) ---\textgreater{} \texttt{graph}
\end{itemize}

\subsubsection{Result:}\label{result-45}

creates a graph from the list/map of edges given as operand and connect
the node to the edge

\subsubsection{Examples:}\label{examples-38}

\begin{verbatim}
as_driving_graph(road,node)  --:  build a graph while using the road agents as edges and the node agents as nodes 
\end{verbatim}

\subsubsection{See also:}\label{see-also-31}

\href{OperatorsAA\#as_intersection_graph}{as\_intersection\_graph},
\href{OperatorsAA\#as_distance_graph}{as\_distance\_graph},
\href{OperatorsAA\#as_edge_graph}{as\_edge\_graph},

\begin{center}\rule{0.5\linewidth}{\linethickness}\end{center}

\subsection{\texorpdfstring{\texttt{as\_edge\_graph}}{as\_edge\_graph}}\label{as_edge_graph}

\subsubsection{Possible use:}\label{possible-use-47}

\begin{itemize}
\tightlist
\item
  \textbf{\texttt{as\_edge\_graph}} (\texttt{map}) ---\textgreater{}
  \texttt{graph}
\item
  \textbf{\texttt{as\_edge\_graph}} (\texttt{container})
  ---\textgreater{} \texttt{graph}
\item
  \texttt{container} \textbf{\texttt{as\_edge\_graph}} \texttt{float}
  ---\textgreater{} \texttt{graph}
\item
  \textbf{\texttt{as\_edge\_graph}} (\texttt{container} ,
  \texttt{float}) ---\textgreater{} \texttt{graph}
\end{itemize}

\subsubsection{Result:}\label{result-46}

creates a graph from the list/map of edges given as operand

\subsubsection{Special cases:}\label{special-cases-18}

\begin{itemize}
\tightlist
\item
  if the operand is a list and a tolerance (max distance in meters to
  consider that 2 points are the same node) is given, the graph will be
  built with elements of the list as edges and two edges will be
  connected by a node if the distance between their extremity (first or
  last points) are at distance lower or equal to the tolerance
\end{itemize}

\begin{verbatim}
 
graph var0 <- as_edge_graph([line([{1,5},{12,45}]),line([{13,45},{34,56}])],1); // var0 equals a graph with two edges and three vertices
\end{verbatim}

\begin{itemize}
\tightlist
\item
  if the operand is a map, the graph will be built by creating edges
  from pairs of the map
\end{itemize}

\begin{verbatim}
 
graph var1 <- as_edge_graph([{1,5}::{12,45},{12,45}::{34,56}]); // var1 equals a graph with these three vertices and two edges
\end{verbatim}

\begin{itemize}
\tightlist
\item
  if the operand is a list, the graph will be built with elements of the
  list as edges
\end{itemize}

\begin{verbatim}
 
graph var2 <- as_edge_graph([line([{1,5},{12,45}]),line([{12,45},{34,56}])]); // var2 equals a graph with two edges and three vertices
\end{verbatim}

\subsubsection{See also:}\label{see-also-32}

\href{OperatorsAA\#as_intersection_graph}{as\_intersection\_graph},
\href{OperatorsAA\#as_distance_graph}{as\_distance\_graph},

\begin{center}\rule{0.5\linewidth}{\linethickness}\end{center}

\subsection{\texorpdfstring{\texttt{as\_grid}}{as\_grid}}\label{as_grid}

\subsubsection{Possible use:}\label{possible-use-48}

\begin{itemize}
\tightlist
\item
  \texttt{geometry} \textbf{\texttt{as\_grid}} \texttt{point}
  ---\textgreater{} \texttt{matrix}
\item
  \textbf{\texttt{as\_grid}} (\texttt{geometry} , \texttt{point})
  ---\textgreater{} \texttt{matrix}
\end{itemize}

\subsubsection{Result:}\label{result-47}

A matrix of square geometries (grid with 8-neighborhood) with dimension
given by the right-hand operand (\{nb\_cols, nb\_lines\}) corresponding
to the square tessellation of the left-hand operand geometry (geometry,
agent)

\subsubsection{Examples:}\label{examples-39}

\begin{verbatim}
 
matrix var0 <- self as_grid {10, 5}; // var0 equals a matrix of square geometries (grid with 8-neighborhood) with 10 columns and 5 lines corresponding to the square tessellation of the geometry of the agent applying the operator.
\end{verbatim}

\subsubsection{See also:}\label{see-also-33}

\href{OperatorsAA\#as_4_grid}{as\_4\_grid},
\href{OperatorsAA\#as_hexagonal_grid}{as\_hexagonal\_grid},

\begin{center}\rule{0.5\linewidth}{\linethickness}\end{center}

\subsection{\texorpdfstring{\texttt{as\_hexagonal\_grid}}{as\_hexagonal\_grid}}\label{as_hexagonal_grid}

\subsubsection{Possible use:}\label{possible-use-49}

\begin{itemize}
\tightlist
\item
  \texttt{geometry} \textbf{\texttt{as\_hexagonal\_grid}} \texttt{point}
  ---\textgreater{} \texttt{list\textless{}geometry\textgreater{}}
\item
  \textbf{\texttt{as\_hexagonal\_grid}} (\texttt{geometry} ,
  \texttt{point}) ---\textgreater{}
  \texttt{list\textless{}geometry\textgreater{}}
\end{itemize}

\subsubsection{Result:}\label{result-48}

A list of geometries (hexagonal) corresponding to the hexagonal
tesselation of the first operand geometry

\subsubsection{Examples:}\label{examples-40}

\begin{verbatim}
 
list<geometry> var0 <- self as_hexagonal_grid {10, 5}; // var0 equals list of geometries (hexagonal) corresponding to the hexagonal tesselation of the first operand geometry
\end{verbatim}

\subsubsection{See also:}\label{see-also-34}

\href{OperatorsAA\#as_4_grid}{as\_4\_grid},
\href{OperatorsAA\#as_grid}{as\_grid},

\begin{center}\rule{0.5\linewidth}{\linethickness}\end{center}

\subsection{\texorpdfstring{\texttt{as\_int}}{as\_int}}\label{as_int}

\subsubsection{Possible use:}\label{possible-use-50}

\begin{itemize}
\tightlist
\item
  \texttt{string} \textbf{\texttt{as\_int}} \texttt{int}
  ---\textgreater{} \texttt{int}
\item
  \textbf{\texttt{as\_int}} (\texttt{string} , \texttt{int})
  ---\textgreater{} \texttt{int}
\end{itemize}

\subsubsection{Result:}\label{result-49}

parses the string argument as a signed integer in the radix specified by
the second argument.

\subsubsection{Special cases:}\label{special-cases-19}

\begin{itemize}
\tightlist
\item
  if the left operand is nil or empty, as\_int returns 0\\
\item
  if the left operand does not represent an integer in the specified
  radix, as\_int throws an exception
\end{itemize}

\subsubsection{Examples:}\label{examples-41}

\begin{verbatim}
 
int var0 <- '20' as_int 10; // var0 equals 20 
int var1 <- '20' as_int 8; // var1 equals 16 
int var2 <- '20' as_int 16; // var2 equals 32 
int var3 <- '1F' as_int 16; // var3 equals 31 
int var4 <- 'hello' as_int 32; // var4 equals 18306744
\end{verbatim}

\subsubsection{See also:}\label{see-also-35}

\href{OperatorsIM\#int}{int},

\begin{center}\rule{0.5\linewidth}{\linethickness}\end{center}

\subsection{\texorpdfstring{\texttt{as\_intersection\_graph}}{as\_intersection\_graph}}\label{as_intersection_graph}

\subsubsection{Possible use:}\label{possible-use-51}

\begin{itemize}
\tightlist
\item
  \texttt{container} \textbf{\texttt{as\_intersection\_graph}}
  \texttt{float} ---\textgreater{} \texttt{graph}
\item
  \textbf{\texttt{as\_intersection\_graph}} (\texttt{container} ,
  \texttt{float}) ---\textgreater{} \texttt{graph}
\end{itemize}

\subsubsection{Result:}\label{result-50}

creates a graph from a list of vertices (left-hand operand). An edge is
created between each pair of vertices with an intersection (with a given
tolerance).

\subsubsection{Comment:}\label{comment-10}

as\_intersection\_graph is more efficient for a list of geometries (but
less accurate) than as\_distance\_graph.

\subsubsection{Examples:}\label{examples-42}

\begin{verbatim}
list(ant) as_intersection_graph 0.5 
\end{verbatim}

\subsubsection{See also:}\label{see-also-36}

\href{OperatorsAA\#as_distance_graph}{as\_distance\_graph},
\href{OperatorsAA\#as_edge_graph}{as\_edge\_graph},

\begin{center}\rule{0.5\linewidth}{\linethickness}\end{center}

\subsection{\texorpdfstring{\texttt{as\_map}}{as\_map}}\label{as_map}

\subsubsection{Possible use:}\label{possible-use-52}

\begin{itemize}
\tightlist
\item
  \texttt{container} \textbf{\texttt{as\_map}} \texttt{any\ expression}
  ---\textgreater{} \texttt{map}
\item
  \textbf{\texttt{as\_map}} (\texttt{container} ,
  \texttt{any\ expression}) ---\textgreater{} \texttt{map}
\end{itemize}

\subsubsection{Result:}\label{result-51}

produces a new map from the evaluation of the right-hand operand for
each element of the left-hand operand

\subsubsection{Comment:}\label{comment-11}

the right-hand operand should be a pair

\subsubsection{Special cases:}\label{special-cases-20}

\begin{itemize}
\tightlist
\item
  if the left-hand operand is nil, as\_map throws an error.
\end{itemize}

\subsubsection{Examples:}\label{examples-43}

\begin{verbatim}
 
map<int,int> var0 <- [1,2,3,4,5,6,7,8] as_map (each::(each * 2)); // var0 equals [1::2, 2::4, 3::6, 4::8, 5::10, 6::12, 7::14, 8::16] 
map<int,int> var1 <- [1::2,3::4,5::6] as_map (each::(each * 2)); // var1 equals [2::4, 4::8, 6::12] 
\end{verbatim}

\begin{center}\rule{0.5\linewidth}{\linethickness}\end{center}

\subsection{\texorpdfstring{\texttt{as\_matrix}}{as\_matrix}}\label{as_matrix}

\subsubsection{Possible use:}\label{possible-use-53}

\begin{itemize}
\tightlist
\item
  \texttt{unknown} \textbf{\texttt{as\_matrix}} \texttt{point}
  ---\textgreater{} \texttt{matrix}
\item
  \textbf{\texttt{as\_matrix}} (\texttt{unknown} , \texttt{point})
  ---\textgreater{} \texttt{matrix}
\end{itemize}

\subsubsection{Result:}\label{result-52}

casts the left operand into a matrix with right operand as preferred
size

\subsubsection{Comment:}\label{comment-12}

This operator is very useful to cast a file containing raster data into
a matrix.Note that both components of the right operand point should be
positive, otherwise an exception is raised.The operator as\_matrix
creates a matrix of preferred size. It fills in it with elements of the
left operand until the matrix is full If the size is to short, some
elements will be omitted. Matrix remaining elements will be filled in by
nil.

\subsubsection{Special cases:}\label{special-cases-21}

\begin{itemize}
\tightlist
\item
  if the right operand is nil, as\_matrix is equivalent to the matrix
  operator
\end{itemize}

\subsubsection{See also:}\label{see-also-37}

\href{OperatorsIM\#matrix}{matrix},

\begin{center}\rule{0.5\linewidth}{\linethickness}\end{center}

\subsection{\texorpdfstring{\texttt{as\_path}}{as\_path}}\label{as_path}

\subsubsection{Possible use:}\label{possible-use-54}

\begin{itemize}
\tightlist
\item
  \texttt{list\textless{}geometry\textgreater{}}
  \textbf{\texttt{as\_path}} \texttt{graph} ---\textgreater{}
  \texttt{path}
\item
  \textbf{\texttt{as\_path}}
  (\texttt{list\textless{}geometry\textgreater{}} , \texttt{graph})
  ---\textgreater{} \texttt{path}
\end{itemize}

\subsubsection{Result:}\label{result-53}

create a graph path from the list of shape

\subsubsection{Examples:}\label{examples-44}

\begin{verbatim}
 
path var0 <- [road1,road2,road3] as_path my_graph; // var0 equals a path road1->road2->road3 of my_graph
\end{verbatim}

\begin{center}\rule{0.5\linewidth}{\linethickness}\end{center}

\subsection{\texorpdfstring{\texttt{asin}}{asin}}\label{asin}

\subsubsection{Possible use:}\label{possible-use-55}

\begin{itemize}
\tightlist
\item
  \textbf{\texttt{asin}} (\texttt{float}) ---\textgreater{}
  \texttt{float}
\item
  \textbf{\texttt{asin}} (\texttt{int}) ---\textgreater{} \texttt{float}
\end{itemize}

\subsubsection{Result:}\label{result-54}

the arcsin of the operand

\subsubsection{Special cases:}\label{special-cases-22}

\begin{itemize}
\tightlist
\item
  if the right-hand operand is outside of the {[}-1,1{]} interval,
  returns NaN
\end{itemize}

\subsubsection{Examples:}\label{examples-45}

\begin{verbatim}
 
float var0 <- asin (0); // var0 equals 0.0 
float var1 <- asin (90); // var1 equals #nan
\end{verbatim}

\subsubsection{See also:}\label{see-also-38}

\href{OperatorsAA\#acos}{acos}, \href{OperatorsAA\#atan}{atan},
\href{OperatorsSZ\#sin}{sin},

\begin{center}\rule{0.5\linewidth}{\linethickness}\end{center}

\subsection{\texorpdfstring{\texttt{at}}{at}}\label{at}

\subsubsection{Possible use:}\label{possible-use-56}

\begin{itemize}
\tightlist
\item
  \texttt{container\textless{}KeyType,ValueType\textgreater{}}
  \textbf{\texttt{at}} \texttt{KeyType} ---\textgreater{}
  \texttt{ValueType}
\item
  \textbf{\texttt{at}}
  (\texttt{container\textless{}KeyType,ValueType\textgreater{}} ,
  \texttt{KeyType}) ---\textgreater{} \texttt{ValueType}
\item
  \texttt{string} \textbf{\texttt{at}} \texttt{int} ---\textgreater{}
  \texttt{string}
\item
  \textbf{\texttt{at}} (\texttt{string} , \texttt{int})
  ---\textgreater{} \texttt{string}
\end{itemize}

\subsubsection{Result:}\label{result-55}

the element at the right operand index of the container

\subsubsection{Comment:}\label{comment-13}

The first element of the container is located at the index 0. In
addition, if the user tries to get the element at an index higher or
equals than the length of the container, he will get an
IndexOutOfBoundException.The at operator behavior depends on the nature
of the operand

\subsubsection{Special cases:}\label{special-cases-23}

\begin{itemize}
\tightlist
\item
  if it is a file, at returns the element of the file content at the
  index specified by the right operand\\
\item
  if it is a population, at returns the agent at the index specified by
  the right operand\\
\item
  if it is a graph and if the right operand is a node, at returns the in
  and out edges corresponding to that node\\
\item
  if it is a graph and if the right operand is an edge, at returns the
  pair node\_out::node\_in of the edge\\
\item
  if it is a graph and if the right operand is a pair node1::node2, at
  returns the edge from node1 to node2 in the graph\\
\item
  if it is a list or a matrix, at returns the element at the index
  specified by the right operand
\end{itemize}

\begin{verbatim}
 
int var0 <- [1, 2, 3] at 2; // var0 equals 3 
point var1 <- [{1,2}, {3,4}, {5,6}] at 0; // var1 equals {1.0,2.0}
\end{verbatim}

\subsubsection{Examples:}\label{examples-46}

\begin{verbatim}
 
string var2 <- 'abcdef' at 0; // var2 equals 'a'
\end{verbatim}

\subsubsection{See also:}\label{see-also-39}

\href{OperatorsBC\#contains_all}{contains\_all},
\href{OperatorsBC\#contains_any}{contains\_any},

\begin{center}\rule{0.5\linewidth}{\linethickness}\end{center}

\subsection{\texorpdfstring{\texttt{at\_distance}}{at\_distance}}\label{at_distance}

\subsubsection{Possible use:}\label{possible-use-57}

\begin{itemize}
\tightlist
\item
  \texttt{container\textless{}agent\textgreater{}}
  \textbf{\texttt{at\_distance}} \texttt{float} ---\textgreater{}
  \texttt{list\textless{}geometry\textgreater{}}
\item
  \textbf{\texttt{at\_distance}}
  (\texttt{container\textless{}agent\textgreater{}} , \texttt{float})
  ---\textgreater{} \texttt{list\textless{}geometry\textgreater{}}
\end{itemize}

\subsubsection{Result:}\label{result-56}

A list of agents or geometries among the left-operand list that are
located at a distance \textless{}= the right operand from the caller
agent (in its topology)

\subsubsection{Examples:}\label{examples-47}

\begin{verbatim}
 
list<geometry> var0 <- [ag1, ag2, ag3] at_distance 20; // var0 equals the agents of the list located at a distance <= 20 from the caller agent (in the same order).
\end{verbatim}

\subsubsection{See also:}\label{see-also-40}

\href{OperatorsNR\#neighbors_at}{neighbors\_at},
\href{OperatorsNR\#neighbors_of}{neighbors\_of},
\href{OperatorsAA\#agent_closest_to}{agent\_closest\_to},
\href{OperatorsAA\#agents_inside}{agents\_inside},
\href{OperatorsBC\#closest_to}{closest\_to},
\href{OperatorsIM\#inside}{inside},
\href{OperatorsNR\#overlapping}{overlapping},

\begin{center}\rule{0.5\linewidth}{\linethickness}\end{center}

\subsection{\texorpdfstring{\texttt{at\_location}}{at\_location}}\label{at_location}

\subsubsection{Possible use:}\label{possible-use-58}

\begin{itemize}
\tightlist
\item
  \texttt{geometry} \textbf{\texttt{at\_location}} \texttt{point}
  ---\textgreater{} \texttt{geometry}
\item
  \textbf{\texttt{at\_location}} (\texttt{geometry} , \texttt{point})
  ---\textgreater{} \texttt{geometry}
\end{itemize}

\subsubsection{Result:}\label{result-57}

A geometry resulting from the tran of a translation to the right-hand
operand point of the left-hand operand (geometry, agent, point)

\subsubsection{Examples:}\label{examples-48}

\begin{verbatim}
 
geometry var0 <- self at_location {10, 20}; // var0 equals the geometry resulting from a translation to the location {10, 20} of the left-hand geometry (or agent).
\end{verbatim}

\begin{center}\rule{0.5\linewidth}{\linethickness}\end{center}

\subsection{\texorpdfstring{\texttt{atan}}{atan}}\label{atan}

\subsubsection{Possible use:}\label{possible-use-59}

\begin{itemize}
\tightlist
\item
  \textbf{\texttt{atan}} (\texttt{float}) ---\textgreater{}
  \texttt{float}
\item
  \textbf{\texttt{atan}} (\texttt{int}) ---\textgreater{} \texttt{float}
\end{itemize}

\subsubsection{Result:}\label{result-58}

Returns the value (in the interval {[}-90,90{]}, in decimal degrees) of
the arctan of the operand (which can be any real number).

\subsubsection{Examples:}\label{examples-49}

\begin{verbatim}
 
float var0 <- atan (1); // var0 equals 45.0
\end{verbatim}

\subsubsection{See also:}\label{see-also-41}

\href{OperatorsAA\#acos}{acos}, \href{OperatorsAA\#asin}{asin},
\href{OperatorsSZ\#tan}{tan},

\begin{center}\rule{0.5\linewidth}{\linethickness}\end{center}

\subsection{\texorpdfstring{\texttt{atan2}}{atan2}}\label{atan2}

\subsubsection{Possible use:}\label{possible-use-60}

\begin{itemize}
\tightlist
\item
  \texttt{float} \textbf{\texttt{atan2}} \texttt{float}
  ---\textgreater{} \texttt{float}
\item
  \textbf{\texttt{atan2}} (\texttt{float} , \texttt{float})
  ---\textgreater{} \texttt{float}
\end{itemize}

\subsubsection{Result:}\label{result-59}

the atan2 value of the two operands.

\subsubsection{Comment:}\label{comment-14}

The function atan2 is the arctangent function with two arguments. The
purpose of using two arguments instead of one is to gather information
on the signs of the inputs in order to return the appropriate quadrant
of the computed angle, which is not possible for the single-argument
arctangent function.

\subsubsection{Examples:}\label{examples-50}

\begin{verbatim}
 
float var0 <- atan2 (0,0); // var0 equals 0.0
\end{verbatim}

\subsubsection{See also:}\label{see-also-42}

\href{OperatorsAA\#atan}{atan}, \href{OperatorsAA\#acos}{acos},
\href{OperatorsAA\#asin}{asin},

\begin{center}\rule{0.5\linewidth}{\linethickness}\end{center}

\subsection{\texorpdfstring{\texttt{attributes}}{attributes}}\label{attributes}

\subsubsection{Possible use:}\label{possible-use-61}

\begin{itemize}
\tightlist
\item
  \textbf{\texttt{attributes}} (\texttt{any}) ---\textgreater{}
  \texttt{attributes}
\end{itemize}

\subsubsection{Result:}\label{result-60}

Casts the operand into the type attributes

\begin{center}\rule{0.5\linewidth}{\linethickness}\end{center}

\subsection{\texorpdfstring{\texttt{auto\_correlation}}{auto\_correlation}}\label{auto_correlation}

\subsubsection{Possible use:}\label{possible-use-62}

\begin{itemize}
\tightlist
\item
  \texttt{container} \textbf{\texttt{auto\_correlation}} \texttt{int}
  ---\textgreater{} \texttt{float}
\item
  \textbf{\texttt{auto\_correlation}} (\texttt{container} ,
  \texttt{int}) ---\textgreater{} \texttt{float}
\end{itemize}

\subsubsection{Result:}\label{result-61}

Returns the auto-correlation of a data sequence

\chapter{Operators (B to C)}\label{operators-b-to-c}

\begin{center}\rule{0.5\linewidth}{\linethickness}\end{center}

\textbf{This file is automatically generated from java files. Do Not
Edit It.}

\begin{center}\rule{0.5\linewidth}{\linethickness}\end{center}

\section{Definition}\label{definition-1}

Operators in the GAML language are used to compose complex expressions.
An operator performs a function on one, two, or n operands (which are
other expressions and thus may be themselves composed of operators) and
returns the result of this function.

Most of them use a classical prefixed functional syntax (i.e.
\texttt{operator\_name(operand1,\ operand2,\ operand3)}, see below),
with the exception of arithmetic (e.g. \texttt{+}, \texttt{/}), logical
(\texttt{and}, \texttt{or}), comparison (e.g. \texttt{\textgreater{}},
\texttt{\textless{}}), access (\texttt{.}, \texttt{{[}..{]}}) and pair
(\texttt{::}) operators, which require an infixed notation (i.e.
\texttt{operand1\ operator\_symbol\ operand1}).

The ternary functional if-else operator, \texttt{?\ :}, uses a special
infixed syntax composed with two symbols (e.g.
\texttt{operand1\ ?\ operand2\ :\ operand3}). Two unary operators
(\texttt{-} and \texttt{!}) use a traditional prefixed syntax that does
not require parentheses unless the operand is itself a complex
expression (e.g. \texttt{-\ 10}, \texttt{!\ (operand1\ or\ operand2)}).

Finally, special constructor operators (\texttt{\{...\}} for
constructing points, \texttt{{[}...{]}} for constructing lists and maps)
will require their operands to be placed between their two symbols (e.g.
\texttt{\{1,2,3\}}, \texttt{{[}operand1,\ operand2,\ ...,\ operandn{]}}
or \texttt{{[}key1::value1,\ key2::value2...\ keyn::valuen{]}}).

With the exception of these special cases above, the following rules
apply to the syntax of operators: * if they only have one operand, the
functional prefixed syntax is mandatory (e.g.
\texttt{operator\_name(operand1)}) * if they have two arguments, either
the functional prefixed syntax (e.g.
\texttt{operator\_name(operand1,\ operand2)}) or the infixed syntax
(e.g. \texttt{operand1\ operator\_name\ operand2}) can be used. * if
they have more than two arguments, either the functional prefixed syntax
(e.g. \texttt{operator\_name(operand1,\ operand2,\ ...,\ operand)}) or a
special infixed syntax with the first operand on the left-hand side of
the operator name (e.g.
\texttt{operand1\ operator\_name(operand2,\ ...,\ operand)}) can be
used.

All of these alternative syntaxes are completely equivalent.

Operators in GAML are purely functional, i.e.~they are guaranteed to not
have any side effects on their operands. For instance, the
\texttt{shuffle} operator, which randomizes the positions of elements in
a list, does not modify its list operand but returns a new shuffled
list.

\section{\texorpdfstring{}{ }}\label{section-19}

\section{Priority between operators}\label{priority-between-operators-1}

The priority of operators determines, in the case of complex expressions
composed of several operators, which one(s) will be evaluated first.

GAML follows in general the traditional priorities attributed to
arithmetic, boolean, comparison operators, with some twists. Namely: *
the constructor operators, like \texttt{::}, used to compose pairs of
operands, have the lowest priority of all operators (e.g.
\texttt{a\ \textgreater{}\ b\ ::\ b\ \textgreater{}\ c} will return a
pair of boolean values, which means that the two comparisons are
evaluated before the operator applies. Similarly,
\texttt{{[}a\ \textgreater{}\ 10,\ b\ \textgreater{}\ 5{]}} will return
a list of boolean values. * it is followed by the \texttt{?:} operator,
the functional if-else (e.g.
\texttt{a\ \textgreater{}\ b\ ?\ a\ +\ 10\ :\ a\ -\ 10} will return the
result of the if-else). * next are the logical operators, \texttt{and}
and \texttt{or} (e.g.
\texttt{a\ \textgreater{}\ b\ or\ b\ \textgreater{}\ c} will return the
value of the test) * next are the comparison operators (i.e.
\texttt{\textgreater{}}, \texttt{\textless{}}, \texttt{\textless{}=},
\texttt{\textgreater{}=}, \texttt{=}, \texttt{!=}) * next the arithmetic
operators in their logical order (multiplicative operators have a higher
priority than additive operators) * next the unary operators \texttt{-}
and \texttt{!} * next the access operators \texttt{.} and
\texttt{{[}{]}} (e.g.
\texttt{\{1,2,3\}.x\ \textgreater{}\ 20\ +\ \{4,5,6\}.y} will return the
result of the comparison between the x and y ordinates of the two
points) * and finally the functional operators, which have the highest
priority of all.

\begin{center}\rule{0.5\linewidth}{\linethickness}\end{center}

\section{Using actions as operators}\label{using-actions-as-operators-1}

Actions defined in species can be used as operators, provided they are
called on the correct agent. The syntax is that of normal functional
operators, but the agent that will perform the action must be added as
the first operand.

For instance, if the following species is defined:

\begin{verbatim}
species spec1 {
        int min(int x, int y) {
                return x > y ? x : y;
        }
}
\end{verbatim}

Any agent instance of spec1 can use \texttt{min} as an operator (if the
action conflicts with an existing operator, a warning will be emitted).
For instance, in the same model, the following line is perfectly
acceptable:

\begin{verbatim}
global {
        init {
                create spec1;
                spec1 my_agent <- spec1[0];
                int the_min <- my_agent min(10,20); // or min(my_agent, 10, 20);
        }
}
\end{verbatim}

If the action doesn't have any operands, the syntax to use is
\texttt{my\_agent\ the\_action()}. Finally, if it does not return a
value, it might still be used but is considering as returning a value of
type \texttt{unknown} (e.g.
\texttt{unknown\ result\ \textless{}-\ my\_agent\ the\_action(op1,\ op2);}).

Note that due to the fact that actions are written by modelers, the
general functional contract is not respected in that case: actions might
perfectly have side effects on their operands (including the agent).

\begin{center}\rule{0.5\linewidth}{\linethickness}\end{center}

\section{Table of Contents}\label{table-of-contents-1}

\begin{center}\rule{0.5\linewidth}{\linethickness}\end{center}

\section{Operators by categories}\label{operators-by-categories-1}

\begin{center}\rule{0.5\linewidth}{\linethickness}\end{center}

\subsection{3D}\label{d-1}

\href{OperatorsBC\#box}{box}, \href{OperatorsBC\#cone3d}{cone3D},
\href{OperatorsBC\#cube}{cube}, \href{OperatorsBC\#cylinder}{cylinder},
\href{OperatorsDH\#dem}{dem}, \href{OperatorsDH\#hexagon}{hexagon},
\href{OperatorsNR\#pyramid}{pyramid},
\href{OperatorsNR\#rgb_to_xyz}{rgb\_to\_xyz},
\href{OperatorsSZ\#set_z}{set\_z}, \href{OperatorsSZ\#sphere}{sphere},
\href{OperatorsSZ\#teapot}{teapot},

\begin{center}\rule{0.5\linewidth}{\linethickness}\end{center}

\subsection{Arithmetic operators}\label{arithmetic-operators-1}

\href{OperatorsAA\#-}{-}, \href{OperatorsAA\#/}{/},
{[}\textsuperscript{{]}(OperatorsAA\#}), \href{OperatorsAA\#*}{*},
\href{OperatorsAA\#+}{+}, \href{OperatorsAA\#abs}{abs},
\href{OperatorsAA\#acos}{acos}, \href{OperatorsAA\#asin}{asin},
\href{OperatorsAA\#atan}{atan}, \href{OperatorsAA\#atan2}{atan2},
\href{OperatorsBC\#ceil}{ceil}, \href{OperatorsBC\#cos}{cos},
\href{OperatorsBC\#cos_rad}{cos\_rad}, \href{OperatorsDH\#div}{div},
\href{OperatorsDH\#even}{even}, \href{OperatorsDH\#exp}{exp},
\href{OperatorsDH\#fact}{fact}, \href{OperatorsDH\#floor}{floor},
\href{OperatorsDH\#hypot}{hypot},
\href{OperatorsIM\#is_finite}{is\_finite},
\href{OperatorsIM\#is_number}{is\_number}, \href{OperatorsIM\#ln}{ln},
\href{OperatorsIM\#log}{log}, \href{OperatorsIM\#mod}{mod},
\href{OperatorsNR\#round}{round}, \href{OperatorsSZ\#signum}{signum},
\href{OperatorsSZ\#sin}{sin}, \href{OperatorsSZ\#sin_rad}{sin\_rad},
\href{OperatorsSZ\#sqrt}{sqrt}, \href{OperatorsSZ\#tan}{tan},
\href{OperatorsSZ\#tan_rad}{tan\_rad}, \href{OperatorsSZ\#tanh}{tanh},
\href{OperatorsSZ\#with_precision}{with\_precision},

\begin{center}\rule{0.5\linewidth}{\linethickness}\end{center}

\subsection{BDI}\label{bdi-1}

\href{OperatorsAA\#and}{and}, \href{OperatorsDH\#eval_when}{eval\_when},
\href{OperatorsDH\#get_about}{get\_about},
\href{OperatorsDH\#get_agent}{get\_agent},
\href{OperatorsDH\#get_agent_cause}{get\_agent\_cause},
\href{OperatorsDH\#get_belief_op}{get\_belief\_op},
\href{OperatorsDH\#get_belief_with_name_op}{get\_belief\_with\_name\_op},
\href{OperatorsDH\#get_beliefs_op}{get\_beliefs\_op},
\href{OperatorsDH\#get_beliefs_with_name_op}{get\_beliefs\_with\_name\_op},
\href{OperatorsDH\#get_current_intention_op}{get\_current\_intention\_op},
\href{OperatorsDH\#get_decay}{get\_decay},
\href{OperatorsDH\#get_desire_op}{get\_desire\_op},
\href{OperatorsDH\#get_desire_with_name_op}{get\_desire\_with\_name\_op},
\href{OperatorsDH\#get_desires_op}{get\_desires\_op},
\href{OperatorsDH\#get_desires_with_name_op}{get\_desires\_with\_name\_op},
\href{OperatorsDH\#get_dominance}{get\_dominance},
\href{OperatorsDH\#get_familiarity}{get\_familiarity},
\href{OperatorsDH\#get_ideal_op}{get\_ideal\_op},
\href{OperatorsDH\#get_ideal_with_name_op}{get\_ideal\_with\_name\_op},
\href{OperatorsDH\#get_ideals_op}{get\_ideals\_op},
\href{OperatorsDH\#get_ideals_with_name_op}{get\_ideals\_with\_name\_op},
\href{OperatorsDH\#get_intensity}{get\_intensity},
\href{OperatorsDH\#get_intention_op}{get\_intention\_op},
\href{OperatorsDH\#get_intention_with_name_op}{get\_intention\_with\_name\_op},
\href{OperatorsDH\#get_intentions_op}{get\_intentions\_op},
\href{OperatorsDH\#get_intentions_with_name_op}{get\_intentions\_with\_name\_op},
\href{OperatorsDH\#get_lifetime}{get\_lifetime},
\href{OperatorsDH\#get_liking}{get\_liking},
\href{OperatorsDH\#get_modality}{get\_modality},
\href{OperatorsDH\#get_obligation_op}{get\_obligation\_op},
\href{OperatorsDH\#get_obligation_with_name_op}{get\_obligation\_with\_name\_op},
\href{OperatorsDH\#get_obligations_op}{get\_obligations\_op},
\href{OperatorsDH\#get_obligations_with_name_op}{get\_obligations\_with\_name\_op},
\href{OperatorsDH\#get_plan_name}{get\_plan\_name},
\href{OperatorsDH\#get_predicate}{get\_predicate},
\href{OperatorsDH\#get_solidarity}{get\_solidarity},
\href{OperatorsDH\#get_strength}{get\_strength},
\href{OperatorsDH\#get_super_intention}{get\_super\_intention},
\href{OperatorsDH\#get_trust}{get\_trust},
\href{OperatorsDH\#get_truth}{get\_truth},
\href{OperatorsDH\#get_uncertainties_op}{get\_uncertainties\_op},
\href{OperatorsDH\#get_uncertainties_with_name_op}{get\_uncertainties\_with\_name\_op},
\href{OperatorsDH\#get_uncertainty_op}{get\_uncertainty\_op},
\href{OperatorsDH\#get_uncertainty_with_name_op}{get\_uncertainty\_with\_name\_op},
\href{OperatorsDH\#has_belief_op}{has\_belief\_op},
\href{OperatorsDH\#has_belief_with_name_op}{has\_belief\_with\_name\_op},
\href{OperatorsDH\#has_desire_op}{has\_desire\_op},
\href{OperatorsDH\#has_desire_with_name_op}{has\_desire\_with\_name\_op},
\href{OperatorsDH\#has_ideal_op}{has\_ideal\_op},
\href{OperatorsDH\#has_ideal_with_name_op}{has\_ideal\_with\_name\_op},
\href{OperatorsDH\#has_intention_op}{has\_intention\_op},
\href{OperatorsDH\#has_intention_with_name_op}{has\_intention\_with\_name\_op},
\href{OperatorsDH\#has_obligation_op}{has\_obligation\_op},
\href{OperatorsDH\#has_obligation_with_name_op}{has\_obligation\_with\_name\_op},
\href{OperatorsDH\#has_uncertainty_op}{has\_uncertainty\_op},
\href{OperatorsDH\#has_uncertainty_with_name_op}{has\_uncertainty\_with\_name\_op},
\href{OperatorsNR\#new_emotion}{new\_emotion},
\href{OperatorsNR\#new_mental_state}{new\_mental\_state},
\href{OperatorsNR\#new_predicate}{new\_predicate},
\href{OperatorsNR\#new_social_link}{new\_social\_link},
\href{OperatorsNR\#or}{or}, \href{OperatorsSZ\#set_about}{set\_about},
\href{OperatorsSZ\#set_agent}{set\_agent},
\href{OperatorsSZ\#set_agent_cause}{set\_agent\_cause},
\href{OperatorsSZ\#set_decay}{set\_decay},
\href{OperatorsSZ\#set_dominance}{set\_dominance},
\href{OperatorsSZ\#set_familiarity}{set\_familiarity},
\href{OperatorsSZ\#set_intensity}{set\_intensity},
\href{OperatorsSZ\#set_lifetime}{set\_lifetime},
\href{OperatorsSZ\#set_liking}{set\_liking},
\href{OperatorsSZ\#set_modality}{set\_modality},
\href{OperatorsSZ\#set_predicate}{set\_predicate},
\href{OperatorsSZ\#set_solidarity}{set\_solidarity},
\href{OperatorsSZ\#set_strength}{set\_strength},
\href{OperatorsSZ\#set_trust}{set\_trust},
\href{OperatorsSZ\#set_truth}{set\_truth},
\href{OperatorsSZ\#with_lifetime}{with\_lifetime},
\href{OperatorsSZ\#with_values}{with\_values},

\begin{center}\rule{0.5\linewidth}{\linethickness}\end{center}

\subsection{Casting operators}\label{casting-operators-1}

\href{OperatorsAA\#as}{as}, \href{OperatorsAA\#as_int}{as\_int},
\href{OperatorsAA\#as_matrix}{as\_matrix},
\href{OperatorsDH\#font}{font}, \href{OperatorsIM\#is}{is},
\href{OperatorsIM\#is_skill}{is\_skill},
\href{OperatorsIM\#list_with}{list\_with},
\href{OperatorsIM\#matrix_with}{matrix\_with},
\href{OperatorsSZ\#species}{species},
\href{OperatorsSZ\#to_gaml}{to\_gaml},
\href{OperatorsSZ\#topology}{topology},

\begin{center}\rule{0.5\linewidth}{\linethickness}\end{center}

\subsection{Color-related operators}\label{color-related-operators-1}

\href{OperatorsAA\#-}{-}, \href{OperatorsAA\#/}{/},
\href{OperatorsAA\#*}{*}, \href{OperatorsAA\#+}{+},
\href{OperatorsBC\#blend}{blend},
\href{OperatorsBC\#brewer_colors}{brewer\_colors},
\href{OperatorsBC\#brewer_palettes}{brewer\_palettes},
\href{OperatorsDH\#grayscale}{grayscale}, \href{OperatorsDH\#hsb}{hsb},
\href{OperatorsIM\#mean}{mean}, \href{OperatorsIM\#median}{median},
\href{OperatorsNR\#rgb}{rgb}, \href{OperatorsNR\#rnd_color}{rnd\_color},
\href{OperatorsSZ\#sum}{sum},

\begin{center}\rule{0.5\linewidth}{\linethickness}\end{center}

\subsection{Comparison operators}\label{comparison-operators-1}

\href{OperatorsAA\#!=}{!=}, \href{OperatorsAA\#\%3C}{\textless{}},
\href{OperatorsAA\#\%3C=}{\textless{}=}, \href{OperatorsAA\#=}{=},
\href{OperatorsAA\#\%3E}{\textgreater{}},
\href{OperatorsAA\#\%3E=}{\textgreater{}=},
\href{OperatorsBC\#between}{between},

\begin{center}\rule{0.5\linewidth}{\linethickness}\end{center}

\subsection{Containers-related
operators}\label{containers-related-operators-1}

\href{OperatorsAA\#-}{-}, \href{OperatorsAA\#::}{::},
\href{OperatorsAA\#+}{+}, \href{OperatorsAA\#accumulate}{accumulate},
\href{OperatorsAA\#among}{among}, \href{OperatorsAA\#at}{at},
\href{OperatorsBC\#collect}{collect},
\href{OperatorsBC\#contains}{contains},
\href{OperatorsBC\#contains_all}{contains\_all},
\href{OperatorsBC\#contains_any}{contains\_any},
\href{OperatorsBC\#count}{count},
\href{OperatorsDH\#distinct}{distinct},
\href{OperatorsDH\#empty}{empty}, \href{OperatorsDH\#every}{every},
\href{OperatorsDH\#first}{first},
\href{OperatorsDH\#first_with}{first\_with},
\href{OperatorsDH\#get}{get}, \href{OperatorsDH\#group_by}{group\_by},
\href{OperatorsIM\#in}{in}, \href{OperatorsIM\#index_by}{index\_by},
\href{OperatorsIM\#inter}{inter},
\href{OperatorsIM\#interleave}{interleave},
\href{OperatorsIM\#internal_at}{internal\_at},
\href{OperatorsIM\#internal_integrated_value}{internal\_integrated\_value},
\href{OperatorsIM\#last}{last},
\href{OperatorsIM\#last_with}{last\_with},
\href{OperatorsIM\#length}{length}, \href{OperatorsIM\#max}{max},
\href{OperatorsIM\#max_of}{max\_of}, \href{OperatorsIM\#mean}{mean},
\href{OperatorsIM\#mean_of}{mean\_of},
\href{OperatorsIM\#median}{median}, \href{OperatorsIM\#min}{min},
\href{OperatorsIM\#min_of}{min\_of}, \href{OperatorsIM\#mul}{mul},
\href{OperatorsNR\#one_of}{one\_of},
\href{OperatorsNR\#product_of}{product\_of},
\href{OperatorsNR\#range}{range}, \href{OperatorsNR\#reverse}{reverse},
\href{OperatorsSZ\#shuffle}{shuffle},
\href{OperatorsSZ\#sort_by}{sort\_by}, \href{OperatorsSZ\#split}{split},
\href{OperatorsSZ\#split_in}{split\_in},
\href{OperatorsSZ\#split_using}{split\_using},
\href{OperatorsSZ\#sum}{sum}, \href{OperatorsSZ\#sum_of}{sum\_of},
\href{OperatorsSZ\#union}{union},
\href{OperatorsSZ\#variance_of}{variance\_of},
\href{OperatorsSZ\#where}{where},
\href{OperatorsSZ\#with_max_of}{with\_max\_of},
\href{OperatorsSZ\#with_min_of}{with\_min\_of},

\begin{center}\rule{0.5\linewidth}{\linethickness}\end{center}

\subsection{Date-related operators}\label{date-related-operators-1}

\href{OperatorsAA\#-}{-}, \href{OperatorsAA\#!=}{!=},
\href{OperatorsAA\#+}{+}, \href{OperatorsAA\#\%3C}{\textless{}},
\href{OperatorsAA\#\%3C=}{\textless{}=}, \href{OperatorsAA\#=}{=},
\href{OperatorsAA\#\%3E}{\textgreater{}},
\href{OperatorsAA\#\%3E=}{\textgreater{}=},
\href{OperatorsAA\#after}{after}, \href{OperatorsBC\#before}{before},
\href{OperatorsBC\#between}{between}, \href{OperatorsDH\#every}{every},
\href{OperatorsIM\#milliseconds_between}{milliseconds\_between},
\href{OperatorsIM\#minus_days}{minus\_days},
\href{OperatorsIM\#minus_hours}{minus\_hours},
\href{OperatorsIM\#minus_minutes}{minus\_minutes},
\href{OperatorsIM\#minus_months}{minus\_months},
\href{OperatorsIM\#minus_ms}{minus\_ms},
\href{OperatorsIM\#minus_weeks}{minus\_weeks},
\href{OperatorsIM\#minus_years}{minus\_years},
\href{OperatorsIM\#months_between}{months\_between},
\href{OperatorsNR\#plus_days}{plus\_days},
\href{OperatorsNR\#plus_hours}{plus\_hours},
\href{OperatorsNR\#plus_minutes}{plus\_minutes},
\href{OperatorsNR\#plus_months}{plus\_months},
\href{OperatorsNR\#plus_ms}{plus\_ms},
\href{OperatorsNR\#plus_weeks}{plus\_weeks},
\href{OperatorsNR\#plus_years}{plus\_years},
\href{OperatorsSZ\#since}{since}, \href{OperatorsSZ\#to}{to},
\href{OperatorsSZ\#until}{until},
\href{OperatorsSZ\#years_between}{years\_between},

\begin{center}\rule{0.5\linewidth}{\linethickness}\end{center}

\subsection{Dates}\label{dates-1}

\begin{center}\rule{0.5\linewidth}{\linethickness}\end{center}

\subsection{DescriptiveStatistics}\label{descriptivestatistics-1}

\href{OperatorsAA\#auto_correlation}{auto\_correlation},
\href{OperatorsBC\#correlation}{correlation},
\href{OperatorsBC\#covariance}{covariance},
\href{OperatorsDH\#durbin_watson}{durbin\_watson},
\href{OperatorsIM\#kurtosis}{kurtosis},
\href{OperatorsIM\#moment}{moment},
\href{OperatorsNR\#quantile}{quantile},
\href{OperatorsNR\#quantile_inverse}{quantile\_inverse},
\href{OperatorsNR\#rank_interpolated}{rank\_interpolated},
\href{OperatorsNR\#rms}{rms}, \href{OperatorsSZ\#skew}{skew},
\href{OperatorsSZ\#variance}{variance},

\begin{center}\rule{0.5\linewidth}{\linethickness}\end{center}

\subsection{Displays}\label{displays-1}

\href{OperatorsDH\#horizontal}{horizontal},
\href{OperatorsSZ\#stack}{stack},
\href{OperatorsSZ\#vertical}{vertical},

\begin{center}\rule{0.5\linewidth}{\linethickness}\end{center}

\subsection{Distributions}\label{distributions-1}

\href{OperatorsBC\#binomial_coeff}{binomial\_coeff},
\href{OperatorsBC\#binomial_complemented}{binomial\_complemented},
\href{OperatorsBC\#binomial_sum}{binomial\_sum},
\href{OperatorsBC\#chi_square}{chi\_square},
\href{OperatorsBC\#chi_square_complemented}{chi\_square\_complemented},
\href{OperatorsDH\#gamma_distribution}{gamma\_distribution},
\href{OperatorsDH\#gamma_distribution_complemented}{gamma\_distribution\_complemented},
\href{OperatorsNR\#normal_area}{normal\_area},
\href{OperatorsNR\#normal_density}{normal\_density},
\href{OperatorsNR\#normal_inverse}{normal\_inverse},
\href{OperatorsNR\#pvalue_for_fstat}{pValue\_for\_fStat},
\href{OperatorsNR\#pvalue_for_tstat}{pValue\_for\_tStat},
\href{OperatorsSZ\#student_area}{student\_area},
\href{OperatorsSZ\#student_t_inverse}{student\_t\_inverse},

\begin{center}\rule{0.5\linewidth}{\linethickness}\end{center}

\subsection{Driving operators}\label{driving-operators-1}

\href{OperatorsAA\#as_driving_graph}{as\_driving\_graph},

\begin{center}\rule{0.5\linewidth}{\linethickness}\end{center}

\subsection{edge}\label{edge-1}

\href{OperatorsDH\#edge_between}{edge\_between},
\href{OperatorsSZ\#strahler}{strahler},

\begin{center}\rule{0.5\linewidth}{\linethickness}\end{center}

\subsection{EDP-related operators}\label{edp-related-operators-1}

\href{OperatorsDH\#diff}{diff}, \href{OperatorsDH\#diff2}{diff2},
\href{OperatorsIM\#internal_zero_order_equation}{internal\_zero\_order\_equation},

\begin{center}\rule{0.5\linewidth}{\linethickness}\end{center}

\subsection{Files-related operators}\label{files-related-operators-1}

\href{OperatorsBC\#crs}{crs},
\href{OperatorsDH\#evaluate_sub_model}{evaluate\_sub\_model},
\href{OperatorsDH\#file}{file},
\href{OperatorsDH\#file_exists}{file\_exists},
\href{OperatorsDH\#folder}{folder}, \href{OperatorsDH\#get}{get},
\href{OperatorsIM\#load_sub_model}{load\_sub\_model},
\href{OperatorsNR\#new_folder}{new\_folder},
\href{OperatorsNR\#osm_file}{osm\_file}, \href{OperatorsNR\#read}{read},
\href{OperatorsSZ\#step_sub_model}{step\_sub\_model},
\href{OperatorsSZ\#writable}{writable},

\begin{center}\rule{0.5\linewidth}{\linethickness}\end{center}

\subsection{FIPA-related operators}\label{fipa-related-operators-1}

\href{OperatorsBC\#conversation}{conversation},
\href{OperatorsIM\#message}{message},

\begin{center}\rule{0.5\linewidth}{\linethickness}\end{center}

\subsection{GamaMetaType}\label{gamametatype-1}

\href{OperatorsSZ\#type_of}{type\_of},

\begin{center}\rule{0.5\linewidth}{\linethickness}\end{center}

\subsection{GammaFunction}\label{gammafunction-1}

\href{OperatorsBC\#beta}{beta}, \href{OperatorsDH\#gamma}{gamma},
\href{OperatorsIM\#incomplete_beta}{incomplete\_beta},
\href{OperatorsIM\#incomplete_gamma}{incomplete\_gamma},
\href{OperatorsIM\#incomplete_gamma_complement}{incomplete\_gamma\_complement},
\href{OperatorsIM\#log_gamma}{log\_gamma},

\begin{center}\rule{0.5\linewidth}{\linethickness}\end{center}

\subsection{Graphs-related operators}\label{graphs-related-operators-1}

\href{OperatorsAA\#add_edge}{add\_edge},
\href{OperatorsAA\#add_node}{add\_node},
\href{OperatorsAA\#adjacency}{adjacency},
\href{OperatorsAA\#agent_from_geometry}{agent\_from\_geometry},
\href{OperatorsAA\#all_pairs_shortest_path}{all\_pairs\_shortest\_path},
\href{OperatorsAA\#alpha_index}{alpha\_index},
\href{OperatorsAA\#as_distance_graph}{as\_distance\_graph},
\href{OperatorsAA\#as_edge_graph}{as\_edge\_graph},
\href{OperatorsAA\#as_intersection_graph}{as\_intersection\_graph},
\href{OperatorsAA\#as_path}{as\_path},
\href{OperatorsBC\#beta_index}{beta\_index},
\href{OperatorsBC\#betweenness_centrality}{betweenness\_centrality},
\href{OperatorsBC\#biggest_cliques_of}{biggest\_cliques\_of},
\href{OperatorsBC\#connected_components_of}{connected\_components\_of},
\href{OperatorsBC\#connectivity_index}{connectivity\_index},
\href{OperatorsBC\#contains_edge}{contains\_edge},
\href{OperatorsBC\#contains_vertex}{contains\_vertex},
\href{OperatorsDH\#degree_of}{degree\_of},
\href{OperatorsDH\#directed}{directed}, \href{OperatorsDH\#edge}{edge},
\href{OperatorsDH\#edge_between}{edge\_between},
\href{OperatorsDH\#edge_betweenness}{edge\_betweenness},
\href{OperatorsDH\#edges}{edges},
\href{OperatorsDH\#gamma_index}{gamma\_index},
\href{OperatorsDH\#generate_barabasi_albert}{generate\_barabasi\_albert},
\href{OperatorsDH\#generate_complete_graph}{generate\_complete\_graph},
\href{OperatorsDH\#generate_watts_strogatz}{generate\_watts\_strogatz},
\href{OperatorsDH\#grid_cells_to_graph}{grid\_cells\_to\_graph},
\href{OperatorsIM\#in_degree_of}{in\_degree\_of},
\href{OperatorsIM\#in_edges_of}{in\_edges\_of},
\href{OperatorsIM\#layout}{layout},
\href{OperatorsIM\#load_graph_from_file}{load\_graph\_from\_file},
\href{OperatorsIM\#load_shortest_paths}{load\_shortest\_paths},
\href{OperatorsIM\#main_connected_component}{main\_connected\_component},
\href{OperatorsIM\#max_flow_between}{max\_flow\_between},
\href{OperatorsIM\#maximal_cliques_of}{maximal\_cliques\_of},
\href{OperatorsNR\#nb_cycles}{nb\_cycles},
\href{OperatorsNR\#neighbors_of}{neighbors\_of},
\href{OperatorsNR\#node}{node}, \href{OperatorsNR\#nodes}{nodes},
\href{OperatorsNR\#out_degree_of}{out\_degree\_of},
\href{OperatorsNR\#out_edges_of}{out\_edges\_of},
\href{OperatorsNR\#path_between}{path\_between},
\href{OperatorsNR\#paths_between}{paths\_between},
\href{OperatorsNR\#predecessors_of}{predecessors\_of},
\href{OperatorsNR\#remove_node_from}{remove\_node\_from},
\href{OperatorsNR\#rewire_n}{rewire\_n},
\href{OperatorsSZ\#source_of}{source\_of},
\href{OperatorsSZ\#spatial_graph}{spatial\_graph},
\href{OperatorsSZ\#strahler}{strahler},
\href{OperatorsSZ\#successors_of}{successors\_of},
\href{OperatorsSZ\#sum}{sum}, \href{OperatorsSZ\#target_of}{target\_of},
\href{OperatorsSZ\#undirected}{undirected},
\href{OperatorsSZ\#use_cache}{use\_cache},
\href{OperatorsSZ\#weight_of}{weight\_of},
\href{OperatorsSZ\#with_optimizer_type}{with\_optimizer\_type},
\href{OperatorsSZ\#with_weights}{with\_weights},

\begin{center}\rule{0.5\linewidth}{\linethickness}\end{center}

\subsection{Grid-related operators}\label{grid-related-operators-1}

\href{OperatorsAA\#as_4_grid}{as\_4\_grid},
\href{OperatorsAA\#as_grid}{as\_grid},
\href{OperatorsAA\#as_hexagonal_grid}{as\_hexagonal\_grid},
\href{OperatorsDH\#grid_at}{grid\_at},
\href{OperatorsNR\#path_between}{path\_between},

\begin{center}\rule{0.5\linewidth}{\linethickness}\end{center}

\subsection{Iterator operators}\label{iterator-operators-1}

\href{OperatorsAA\#accumulate}{accumulate},
\href{OperatorsAA\#as_map}{as\_map},
\href{OperatorsBC\#collect}{collect}, \href{OperatorsBC\#count}{count},
\href{OperatorsBC\#create_map}{create\_map},
\href{OperatorsDH\#distribution_of}{distribution\_of},
\href{OperatorsDH\#distribution_of}{distribution\_of},
\href{OperatorsDH\#distribution_of}{distribution\_of},
\href{OperatorsDH\#distribution2d_of}{distribution2d\_of},
\href{OperatorsDH\#distribution2d_of}{distribution2d\_of},
\href{OperatorsDH\#distribution2d_of}{distribution2d\_of},
\href{OperatorsDH\#first_with}{first\_with},
\href{OperatorsDH\#frequency_of}{frequency\_of},
\href{OperatorsDH\#group_by}{group\_by},
\href{OperatorsIM\#index_by}{index\_by},
\href{OperatorsIM\#last_with}{last\_with},
\href{OperatorsIM\#max_of}{max\_of},
\href{OperatorsIM\#mean_of}{mean\_of},
\href{OperatorsIM\#min_of}{min\_of},
\href{OperatorsNR\#product_of}{product\_of},
\href{OperatorsSZ\#sort_by}{sort\_by},
\href{OperatorsSZ\#sum_of}{sum\_of},
\href{OperatorsSZ\#variance_of}{variance\_of},
\href{OperatorsSZ\#where}{where},
\href{OperatorsSZ\#with_max_of}{with\_max\_of},
\href{OperatorsSZ\#with_min_of}{with\_min\_of},

\begin{center}\rule{0.5\linewidth}{\linethickness}\end{center}

\subsection{List-related operators}\label{list-related-operators-1}

\href{OperatorsBC\#copy_between}{copy\_between},
\href{OperatorsIM\#index_of}{index\_of},
\href{OperatorsIM\#last_index_of}{last\_index\_of},

\begin{center}\rule{0.5\linewidth}{\linethickness}\end{center}

\subsection{Logical operators}\label{logical-operators-1}

\href{OperatorsAA\#:}{:}, \href{OperatorsAA\#!}{!},
\href{OperatorsAA\#?}{?}, \href{OperatorsAA\#and}{and},
\href{OperatorsNR\#or}{or}, \href{OperatorsSZ\#xor}{xor},

\begin{center}\rule{0.5\linewidth}{\linethickness}\end{center}

\subsection{Map comparaison
operators}\label{map-comparaison-operators-1}

\href{OperatorsDH\#fuzzy_kappa}{fuzzy\_kappa},
\href{OperatorsDH\#fuzzy_kappa_sim}{fuzzy\_kappa\_sim},
\href{OperatorsIM\#kappa}{kappa},
\href{OperatorsIM\#kappa_sim}{kappa\_sim},
\href{OperatorsNR\#percent_absolute_deviation}{percent\_absolute\_deviation},

\begin{center}\rule{0.5\linewidth}{\linethickness}\end{center}

\subsection{Map-related operators}\label{map-related-operators-1}

\href{OperatorsAA\#as_map}{as\_map},
\href{OperatorsBC\#create_map}{create\_map},
\href{OperatorsIM\#index_of}{index\_of},
\href{OperatorsIM\#last_index_of}{last\_index\_of},

\begin{center}\rule{0.5\linewidth}{\linethickness}\end{center}

\subsection{Material}\label{material-1}

\href{OperatorsIM\#material}{material},

\begin{center}\rule{0.5\linewidth}{\linethickness}\end{center}

\subsection{Matrix-related operators}\label{matrix-related-operators-1}

\href{OperatorsAA\#-}{-}, \href{OperatorsAA\#/}{/},
\href{OperatorsAA\#.}{.}, \href{OperatorsAA\#*}{*},
\href{OperatorsAA\#+}{+},
\href{OperatorsAA\#append_horizontally}{append\_horizontally},
\href{OperatorsAA\#append_vertically}{append\_vertically},
\href{OperatorsBC\#column_at}{column\_at},
\href{OperatorsBC\#columns_list}{columns\_list},
\href{OperatorsDH\#determinant}{determinant},
\href{OperatorsDH\#eigenvalues}{eigenvalues},
\href{OperatorsIM\#index_of}{index\_of},
\href{OperatorsIM\#inverse}{inverse},
\href{OperatorsIM\#last_index_of}{last\_index\_of},
\href{OperatorsNR\#row_at}{row\_at},
\href{OperatorsNR\#rows_list}{rows\_list},
\href{OperatorsSZ\#shuffle}{shuffle}, \href{OperatorsSZ\#trace}{trace},
\href{OperatorsSZ\#transpose}{transpose},

\begin{center}\rule{0.5\linewidth}{\linethickness}\end{center}

\subsection{multicriteria operators}\label{multicriteria-operators-1}

\href{OperatorsDH\#electre_dm}{electre\_DM},
\href{OperatorsDH\#evidence_theory_dm}{evidence\_theory\_DM},
\href{OperatorsDH\#fuzzy_choquet_dm}{fuzzy\_choquet\_DM},
\href{OperatorsNR\#promethee_dm}{promethee\_DM},
\href{OperatorsSZ\#weighted_means_dm}{weighted\_means\_DM},

\begin{center}\rule{0.5\linewidth}{\linethickness}\end{center}

\subsection{Path-related operators}\label{path-related-operators-1}

\href{OperatorsAA\#agent_from_geometry}{agent\_from\_geometry},
\href{OperatorsAA\#all_pairs_shortest_path}{all\_pairs\_shortest\_path},
\href{OperatorsAA\#as_path}{as\_path},
\href{OperatorsIM\#load_shortest_paths}{load\_shortest\_paths},
\href{OperatorsIM\#max_flow_between}{max\_flow\_between},
\href{OperatorsNR\#path_between}{path\_between},
\href{OperatorsNR\#path_to}{path\_to},
\href{OperatorsNR\#paths_between}{paths\_between},
\href{OperatorsSZ\#use_cache}{use\_cache},

\begin{center}\rule{0.5\linewidth}{\linethickness}\end{center}

\subsection{Points-related operators}\label{points-related-operators-1}

\href{OperatorsAA\#-}{-}, \href{OperatorsAA\#/}{/},
\href{OperatorsAA\#*}{*}, \href{OperatorsAA\#+}{+},
\href{OperatorsAA\#\%3C}{\textless{}},
\href{OperatorsAA\#\%3C=}{\textless{}=},
\href{OperatorsAA\#\%3E}{\textgreater{}},
\href{OperatorsAA\#\%3E=}{\textgreater{}=},
\href{OperatorsAA\#add_point}{add\_point},
\href{OperatorsAA\#angle_between}{angle\_between},
\href{OperatorsAA\#any_location_in}{any\_location\_in},
\href{OperatorsBC\#centroid}{centroid},
\href{OperatorsBC\#closest_points_with}{closest\_points\_with},
\href{OperatorsDH\#farthest_point_to}{farthest\_point\_to},
\href{OperatorsDH\#grid_at}{grid\_at}, \href{OperatorsNR\#norm}{norm},
\href{OperatorsNR\#point}{point},
\href{OperatorsNR\#points_along}{points\_along},
\href{OperatorsNR\#points_at}{points\_at},
\href{OperatorsNR\#points_on}{points\_on},

\begin{center}\rule{0.5\linewidth}{\linethickness}\end{center}

\subsection{Random operators}\label{random-operators-1}

\href{OperatorsBC\#binomial}{binomial}, \href{OperatorsDH\#flip}{flip},
\href{OperatorsDH\#gauss}{gauss},
\href{OperatorsIM\#improved_generator}{improved\_generator},
\href{OperatorsNR\#open_simplex_generator}{open\_simplex\_generator},
\href{OperatorsNR\#poisson}{poisson}, \href{OperatorsNR\#rnd}{rnd},
\href{OperatorsNR\#rnd_choice}{rnd\_choice},
\href{OperatorsSZ\#sample}{sample},
\href{OperatorsSZ\#shuffle}{shuffle},
\href{OperatorsSZ\#simplex_generator}{simplex\_generator},
\href{OperatorsSZ\#skew_gauss}{skew\_gauss},
\href{OperatorsSZ\#truncated_gauss}{truncated\_gauss},

\begin{center}\rule{0.5\linewidth}{\linethickness}\end{center}

\subsection{ReverseOperators}\label{reverseoperators-1}

\href{OperatorsSZ\#savesimulation}{saveSimulation},
\href{OperatorsSZ\#serialize}{serialize},
\href{OperatorsSZ\#serializeagent}{serializeAgent},
\href{OperatorsSZ\#unserializesimulation}{unSerializeSimulation},
\href{OperatorsSZ\#unserializesimulationfromfile}{unSerializeSimulationFromFile},

\begin{center}\rule{0.5\linewidth}{\linethickness}\end{center}

\subsection{Shape}\label{shape-1}

\href{OperatorsAA\#arc}{arc}, \href{OperatorsBC\#box}{box},
\href{OperatorsBC\#circle}{circle}, \href{OperatorsBC\#cone}{cone},
\href{OperatorsBC\#cone3d}{cone3D}, \href{OperatorsBC\#cross}{cross},
\href{OperatorsBC\#cube}{cube}, \href{OperatorsBC\#curve}{curve},
\href{OperatorsBC\#cylinder}{cylinder},
\href{OperatorsDH\#ellipse}{ellipse},
\href{OperatorsDH\#envelope}{envelope},
\href{OperatorsDH\#geometry_collection}{geometry\_collection},
\href{OperatorsDH\#hexagon}{hexagon}, \href{OperatorsIM\#line}{line},
\href{OperatorsIM\#link}{link}, \href{OperatorsNR\#plan}{plan},
\href{OperatorsNR\#polygon}{polygon},
\href{OperatorsNR\#polyhedron}{polyhedron},
\href{OperatorsNR\#pyramid}{pyramid},
\href{OperatorsNR\#rectangle}{rectangle},
\href{OperatorsSZ\#sphere}{sphere}, \href{OperatorsSZ\#square}{square},
\href{OperatorsSZ\#squircle}{squircle},
\href{OperatorsSZ\#teapot}{teapot},
\href{OperatorsSZ\#triangle}{triangle},

\begin{center}\rule{0.5\linewidth}{\linethickness}\end{center}

\subsection{Spatial operators}\label{spatial-operators-1}

\href{OperatorsAA\#-}{-}, \href{OperatorsAA\#*}{*},
\href{OperatorsAA\#+}{+}, \href{OperatorsAA\#add_point}{add\_point},
\href{OperatorsAA\#agent_closest_to}{agent\_closest\_to},
\href{OperatorsAA\#agent_farthest_to}{agent\_farthest\_to},
\href{OperatorsAA\#agents_at_distance}{agents\_at\_distance},
\href{OperatorsAA\#agents_inside}{agents\_inside},
\href{OperatorsAA\#agents_overlapping}{agents\_overlapping},
\href{OperatorsAA\#angle_between}{angle\_between},
\href{OperatorsAA\#any_location_in}{any\_location\_in},
\href{OperatorsAA\#arc}{arc}, \href{OperatorsAA\#around}{around},
\href{OperatorsAA\#as_4_grid}{as\_4\_grid},
\href{OperatorsAA\#as_grid}{as\_grid},
\href{OperatorsAA\#as_hexagonal_grid}{as\_hexagonal\_grid},
\href{OperatorsAA\#at_distance}{at\_distance},
\href{OperatorsAA\#at_location}{at\_location},
\href{OperatorsBC\#box}{box}, \href{OperatorsBC\#centroid}{centroid},
\href{OperatorsBC\#circle}{circle}, \href{OperatorsBC\#clean}{clean},
\href{OperatorsBC\#clean_network}{clean\_network},
\href{OperatorsBC\#closest_points_with}{closest\_points\_with},
\href{OperatorsBC\#closest_to}{closest\_to},
\href{OperatorsBC\#cone}{cone}, \href{OperatorsBC\#cone3d}{cone3D},
\href{OperatorsBC\#convex_hull}{convex\_hull},
\href{OperatorsBC\#covers}{covers}, \href{OperatorsBC\#cross}{cross},
\href{OperatorsBC\#crosses}{crosses}, \href{OperatorsBC\#crs}{crs},
\href{OperatorsBC\#crs_transform}{CRS\_transform},
\href{OperatorsBC\#cube}{cube}, \href{OperatorsBC\#curve}{curve},
\href{OperatorsBC\#cylinder}{cylinder}, \href{OperatorsDH\#dem}{dem},
\href{OperatorsDH\#direction_between}{direction\_between},
\href{OperatorsDH\#disjoint_from}{disjoint\_from},
\href{OperatorsDH\#distance_between}{distance\_between},
\href{OperatorsDH\#distance_to}{distance\_to},
\href{OperatorsDH\#ellipse}{ellipse},
\href{OperatorsDH\#envelope}{envelope},
\href{OperatorsDH\#farthest_point_to}{farthest\_point\_to},
\href{OperatorsDH\#farthest_to}{farthest\_to},
\href{OperatorsDH\#geometry_collection}{geometry\_collection},
\href{OperatorsDH\#gini}{gini}, \href{OperatorsDH\#hexagon}{hexagon},
\href{OperatorsDH\#hierarchical_clustering}{hierarchical\_clustering},
\href{OperatorsIM\#idw}{IDW}, \href{OperatorsIM\#inside}{inside},
\href{OperatorsIM\#inter}{inter},
\href{OperatorsIM\#intersects}{intersects},
\href{OperatorsIM\#line}{line}, \href{OperatorsIM\#link}{link},
\href{OperatorsIM\#masked_by}{masked\_by},
\href{OperatorsIM\#moran}{moran},
\href{OperatorsNR\#neighbors_at}{neighbors\_at},
\href{OperatorsNR\#neighbors_of}{neighbors\_of},
\href{OperatorsNR\#overlapping}{overlapping},
\href{OperatorsNR\#overlaps}{overlaps},
\href{OperatorsNR\#partially_overlaps}{partially\_overlaps},
\href{OperatorsNR\#path_between}{path\_between},
\href{OperatorsNR\#path_to}{path\_to}, \href{OperatorsNR\#plan}{plan},
\href{OperatorsNR\#points_along}{points\_along},
\href{OperatorsNR\#points_at}{points\_at},
\href{OperatorsNR\#points_on}{points\_on},
\href{OperatorsNR\#polygon}{polygon},
\href{OperatorsNR\#polyhedron}{polyhedron},
\href{OperatorsNR\#pyramid}{pyramid},
\href{OperatorsNR\#rectangle}{rectangle},
\href{OperatorsNR\#rgb_to_xyz}{rgb\_to\_xyz},
\href{OperatorsNR\#rotated_by}{rotated\_by},
\href{OperatorsNR\#round}{round},
\href{OperatorsSZ\#scaled_to}{scaled\_to},
\href{OperatorsSZ\#set_z}{set\_z},
\href{OperatorsSZ\#simple_clustering_by_distance}{simple\_clustering\_by\_distance},
\href{OperatorsSZ\#simplification}{simplification},
\href{OperatorsSZ\#skeletonize}{skeletonize},
\href{OperatorsSZ\#smooth}{smooth}, \href{OperatorsSZ\#sphere}{sphere},
\href{OperatorsSZ\#split_at}{split\_at},
\href{OperatorsSZ\#split_geometry}{split\_geometry},
\href{OperatorsSZ\#split_lines}{split\_lines},
\href{OperatorsSZ\#square}{square},
\href{OperatorsSZ\#squircle}{squircle},
\href{OperatorsSZ\#teapot}{teapot},
\href{OperatorsSZ\#to_gama_crs}{to\_GAMA\_CRS},
\href{OperatorsSZ\#to_rectangles}{to\_rectangles},
\href{OperatorsSZ\#to_squares}{to\_squares},
\href{OperatorsSZ\#to_sub_geometries}{to\_sub\_geometries},
\href{OperatorsSZ\#touches}{touches},
\href{OperatorsSZ\#towards}{towards},
\href{OperatorsSZ\#transformed_by}{transformed\_by},
\href{OperatorsSZ\#translated_by}{translated\_by},
\href{OperatorsSZ\#triangle}{triangle},
\href{OperatorsSZ\#triangulate}{triangulate},
\href{OperatorsSZ\#union}{union}, \href{OperatorsSZ\#using}{using},
\href{OperatorsSZ\#voronoi}{voronoi},
\href{OperatorsSZ\#with_precision}{with\_precision},
\href{OperatorsSZ\#without_holes}{without\_holes},

\begin{center}\rule{0.5\linewidth}{\linethickness}\end{center}

\subsection{Spatial properties
operators}\label{spatial-properties-operators-1}

\href{OperatorsBC\#covers}{covers},
\href{OperatorsBC\#crosses}{crosses},
\href{OperatorsIM\#intersects}{intersects},
\href{OperatorsNR\#partially_overlaps}{partially\_overlaps},
\href{OperatorsSZ\#touches}{touches},

\begin{center}\rule{0.5\linewidth}{\linethickness}\end{center}

\subsection{Spatial queries
operators}\label{spatial-queries-operators-1}

\href{OperatorsAA\#agent_closest_to}{agent\_closest\_to},
\href{OperatorsAA\#agent_farthest_to}{agent\_farthest\_to},
\href{OperatorsAA\#agents_at_distance}{agents\_at\_distance},
\href{OperatorsAA\#agents_inside}{agents\_inside},
\href{OperatorsAA\#agents_overlapping}{agents\_overlapping},
\href{OperatorsAA\#at_distance}{at\_distance},
\href{OperatorsBC\#closest_to}{closest\_to},
\href{OperatorsDH\#farthest_to}{farthest\_to},
\href{OperatorsIM\#inside}{inside},
\href{OperatorsNR\#neighbors_at}{neighbors\_at},
\href{OperatorsNR\#neighbors_of}{neighbors\_of},
\href{OperatorsNR\#overlapping}{overlapping},

\begin{center}\rule{0.5\linewidth}{\linethickness}\end{center}

\subsection{Spatial relations
operators}\label{spatial-relations-operators-1}

\href{OperatorsDH\#direction_between}{direction\_between},
\href{OperatorsDH\#distance_between}{distance\_between},
\href{OperatorsDH\#distance_to}{distance\_to},
\href{OperatorsNR\#path_between}{path\_between},
\href{OperatorsNR\#path_to}{path\_to},
\href{OperatorsSZ\#towards}{towards},

\begin{center}\rule{0.5\linewidth}{\linethickness}\end{center}

\subsection{Spatial statistical
operators}\label{spatial-statistical-operators-1}

\href{OperatorsDH\#hierarchical_clustering}{hierarchical\_clustering},
\href{OperatorsSZ\#simple_clustering_by_distance}{simple\_clustering\_by\_distance},

\begin{center}\rule{0.5\linewidth}{\linethickness}\end{center}

\subsection{Spatial transformations
operators}\label{spatial-transformations-operators-1}

\href{OperatorsAA\#-}{-}, \href{OperatorsAA\#*}{*},
\href{OperatorsAA\#+}{+}, \href{OperatorsAA\#as_4_grid}{as\_4\_grid},
\href{OperatorsAA\#as_grid}{as\_grid},
\href{OperatorsAA\#as_hexagonal_grid}{as\_hexagonal\_grid},
\href{OperatorsAA\#at_location}{at\_location},
\href{OperatorsBC\#clean}{clean},
\href{OperatorsBC\#clean_network}{clean\_network},
\href{OperatorsBC\#convex_hull}{convex\_hull},
\href{OperatorsBC\#crs_transform}{CRS\_transform},
\href{OperatorsNR\#rotated_by}{rotated\_by},
\href{OperatorsSZ\#scaled_to}{scaled\_to},
\href{OperatorsSZ\#simplification}{simplification},
\href{OperatorsSZ\#skeletonize}{skeletonize},
\href{OperatorsSZ\#smooth}{smooth},
\href{OperatorsSZ\#split_geometry}{split\_geometry},
\href{OperatorsSZ\#split_lines}{split\_lines},
\href{OperatorsSZ\#to_gama_crs}{to\_GAMA\_CRS},
\href{OperatorsSZ\#to_rectangles}{to\_rectangles},
\href{OperatorsSZ\#to_squares}{to\_squares},
\href{OperatorsSZ\#to_sub_geometries}{to\_sub\_geometries},
\href{OperatorsSZ\#transformed_by}{transformed\_by},
\href{OperatorsSZ\#translated_by}{translated\_by},
\href{OperatorsSZ\#triangulate}{triangulate},
\href{OperatorsSZ\#voronoi}{voronoi},
\href{OperatorsSZ\#with_precision}{with\_precision},
\href{OperatorsSZ\#without_holes}{without\_holes},

\begin{center}\rule{0.5\linewidth}{\linethickness}\end{center}

\subsection{Species-related
operators}\label{species-related-operators-1}

\href{OperatorsIM\#index_of}{index\_of},
\href{OperatorsIM\#last_index_of}{last\_index\_of},
\href{OperatorsNR\#of_generic_species}{of\_generic\_species},
\href{OperatorsNR\#of_species}{of\_species},

\begin{center}\rule{0.5\linewidth}{\linethickness}\end{center}

\subsection{Statistical operators}\label{statistical-operators-1}

\href{OperatorsBC\#build}{build}, \href{OperatorsBC\#corr}{corR},
\href{OperatorsDH\#dbscan}{dbscan},
\href{OperatorsDH\#distribution_of}{distribution\_of},
\href{OperatorsDH\#distribution2d_of}{distribution2d\_of},
\href{OperatorsDH\#dtw}{dtw},
\href{OperatorsDH\#frequency_of}{frequency\_of},
\href{OperatorsDH\#gamma_rnd}{gamma\_rnd},
\href{OperatorsDH\#geometric_mean}{geometric\_mean},
\href{OperatorsDH\#gini}{gini},
\href{OperatorsDH\#harmonic_mean}{harmonic\_mean},
\href{OperatorsDH\#hierarchical_clustering}{hierarchical\_clustering},
\href{OperatorsIM\#kmeans}{kmeans},
\href{OperatorsIM\#kurtosis}{kurtosis}, \href{OperatorsIM\#max}{max},
\href{OperatorsIM\#mean}{mean},
\href{OperatorsIM\#mean_deviation}{mean\_deviation},
\href{OperatorsIM\#meanr}{meanR}, \href{OperatorsIM\#median}{median},
\href{OperatorsIM\#min}{min}, \href{OperatorsIM\#moran}{moran},
\href{OperatorsIM\#mul}{mul}, \href{OperatorsNR\#predict}{predict},
\href{OperatorsSZ\#simple_clustering_by_distance}{simple\_clustering\_by\_distance},
\href{OperatorsSZ\#skewness}{skewness},
\href{OperatorsSZ\#split}{split},
\href{OperatorsSZ\#split_in}{split\_in},
\href{OperatorsSZ\#split_using}{split\_using},
\href{OperatorsSZ\#standard_deviation}{standard\_deviation},
\href{OperatorsSZ\#sum}{sum}, \href{OperatorsSZ\#variance}{variance},

\begin{center}\rule{0.5\linewidth}{\linethickness}\end{center}

\subsection{Strings-related
operators}\label{strings-related-operators-1}

\href{OperatorsAA\#+}{+}, \href{OperatorsAA\#\%3C}{\textless{}},
\href{OperatorsAA\#\%3C=}{\textless{}=},
\href{OperatorsAA\#\%3E}{\textgreater{}},
\href{OperatorsAA\#\%3E=}{\textgreater{}=}, \href{OperatorsAA\#at}{at},
\href{OperatorsBC\#char}{char}, \href{OperatorsBC\#contains}{contains},
\href{OperatorsBC\#contains_all}{contains\_all},
\href{OperatorsBC\#contains_any}{contains\_any},
\href{OperatorsBC\#copy_between}{copy\_between},
\href{OperatorsDH\#date}{date}, \href{OperatorsDH\#empty}{empty},
\href{OperatorsDH\#first}{first}, \href{OperatorsIM\#in}{in},
\href{OperatorsIM\#indented_by}{indented\_by},
\href{OperatorsIM\#index_of}{index\_of},
\href{OperatorsIM\#is_number}{is\_number},
\href{OperatorsIM\#last}{last},
\href{OperatorsIM\#last_index_of}{last\_index\_of},
\href{OperatorsIM\#length}{length},
\href{OperatorsIM\#lower_case}{lower\_case},
\href{OperatorsNR\#replace}{replace},
\href{OperatorsNR\#replace_regex}{replace\_regex},
\href{OperatorsNR\#reverse}{reverse},
\href{OperatorsSZ\#sample}{sample},
\href{OperatorsSZ\#shuffle}{shuffle},
\href{OperatorsSZ\#split_with}{split\_with},
\href{OperatorsSZ\#string}{string},
\href{OperatorsSZ\#upper_case}{upper\_case},

\begin{center}\rule{0.5\linewidth}{\linethickness}\end{center}

\subsection{System}\label{system-1}

\href{OperatorsAA\#.}{.}, \href{OperatorsBC\#command}{command},
\href{OperatorsBC\#copy}{copy}, \href{OperatorsDH\#dead}{dead},
\href{OperatorsDH\#eval_gaml}{eval\_gaml},
\href{OperatorsDH\#every}{every},
\href{OperatorsIM\#is_error}{is\_error},
\href{OperatorsIM\#is_warning}{is\_warning},
\href{OperatorsSZ\#user_input}{user\_input},

\begin{center}\rule{0.5\linewidth}{\linethickness}\end{center}

\subsection{Time-related operators}\label{time-related-operators-1}

\href{OperatorsDH\#date}{date}, \href{OperatorsSZ\#string}{string},

\begin{center}\rule{0.5\linewidth}{\linethickness}\end{center}

\subsection{Types-related operators}\label{types-related-operators-1}

\begin{center}\rule{0.5\linewidth}{\linethickness}\end{center}

\subsection{User control operators}\label{user-control-operators-1}

\href{OperatorsSZ\#user_input}{user\_input},

\begin{center}\rule{0.5\linewidth}{\linethickness}\end{center}

\section{Operators}\label{operators-1}

\begin{center}\rule{0.5\linewidth}{\linethickness}\end{center}

\subsection{\texorpdfstring{\texttt{BDIPlan}}{BDIPlan}}\label{bdiplan}

\subsubsection{Possible use:}\label{possible-use-63}

\begin{itemize}
\tightlist
\item
  \textbf{\texttt{BDIPlan}} (\texttt{any}) ---\textgreater{}
  \texttt{BDIPlan}
\end{itemize}

\subsubsection{Result:}\label{result-62}

Casts the operand into the type BDIPlan

\begin{center}\rule{0.5\linewidth}{\linethickness}\end{center}

\subsection{\texorpdfstring{\texttt{before}}{before}}\label{before}

\subsubsection{Possible use:}\label{possible-use-64}

\begin{itemize}
\tightlist
\item
  \textbf{\texttt{before}} (\texttt{date}) ---\textgreater{}
  \texttt{bool}
\item
  \texttt{any\ expression} \textbf{\texttt{before}} \texttt{date}
  ---\textgreater{} \texttt{bool}
\item
  \textbf{\texttt{before}} (\texttt{any\ expression} , \texttt{date})
  ---\textgreater{} \texttt{bool}
\end{itemize}

\subsubsection{Result:}\label{result-63}

Returns true if the current\_date of the model is strictly before the
date passed in argument. Synonym of `current\_date \textless{} argument'

\subsubsection{Examples:}\label{examples-51}

\begin{verbatim}
reflex when: before(starting_date) {}   // this reflex will never be run 
\end{verbatim}

\begin{center}\rule{0.5\linewidth}{\linethickness}\end{center}

\subsection{\texorpdfstring{\texttt{beta}}{beta}}\label{beta}

\subsubsection{Possible use:}\label{possible-use-65}

\begin{itemize}
\tightlist
\item
  \texttt{float} \textbf{\texttt{beta}} \texttt{float} ---\textgreater{}
  \texttt{float}
\item
  \textbf{\texttt{beta}} (\texttt{float} , \texttt{float})
  ---\textgreater{} \texttt{float}
\end{itemize}

\subsubsection{Result:}\label{result-64}

Returns the beta function with arguments a, b.

\begin{center}\rule{0.5\linewidth}{\linethickness}\end{center}

\subsection{\texorpdfstring{\texttt{beta\_index}}{beta\_index}}\label{beta_index}

\subsubsection{Possible use:}\label{possible-use-66}

\begin{itemize}
\tightlist
\item
  \textbf{\texttt{beta\_index}} (\texttt{graph}) ---\textgreater{}
  \texttt{float}
\end{itemize}

\subsubsection{Result:}\label{result-65}

returns the beta index of the graph (Measures the level of connectivity
in a graph and is expressed by the relationship between the number of
links (e) over the number of nodes (v) : beta = e/v.

\subsubsection{Examples:}\label{examples-52}

\begin{verbatim}
graph graphEpidemio <- graph([]);  
float var1 <- beta_index(graphEpidemio); // var1 equals the beta index of the graph
\end{verbatim}

\subsubsection{See also:}\label{see-also-43}

\href{OperatorsAA\#alpha_index}{alpha\_index},
\href{OperatorsDH\#gamma_index}{gamma\_index},
\href{OperatorsNR\#nb_cycles}{nb\_cycles},
\href{OperatorsBC\#connectivity_index}{connectivity\_index},

\begin{center}\rule{0.5\linewidth}{\linethickness}\end{center}

\subsection{\texorpdfstring{\texttt{between}}{between}}\label{between}

\subsubsection{Possible use:}\label{possible-use-67}

\begin{itemize}
\tightlist
\item
  \texttt{date} \textbf{\texttt{between}} \texttt{date}
  ---\textgreater{} \texttt{bool}
\item
  \textbf{\texttt{between}} (\texttt{date} , \texttt{date})
  ---\textgreater{} \texttt{bool}
\item
  \textbf{\texttt{between}} (\texttt{float}, \texttt{float},
  \texttt{float}) ---\textgreater{} \texttt{bool}
\item
  \textbf{\texttt{between}} (\texttt{date}, \texttt{date},
  \texttt{date}) ---\textgreater{} \texttt{bool}
\item
  \textbf{\texttt{between}} (\texttt{any\ expression}, \texttt{date},
  \texttt{date}) ---\textgreater{} \texttt{bool}
\item
  \textbf{\texttt{between}} (\texttt{int}, \texttt{int}, \texttt{int})
  ---\textgreater{} \texttt{bool}
\end{itemize}

\subsubsection{Result:}\label{result-66}

returns true if the first float operand is bigger than the second float
operand and smaller than the third float operand

returns true the first integer operand is bigger than the second integer
operand and smaller than the third integer operand

\subsubsection{Special cases:}\label{special-cases-24}

\begin{itemize}
\tightlist
\item
  returns true if the first operand is between the two dates passed in
  arguments (both exclusive). The version with 2 arguments compares the
  current\_date with the 2 others
\end{itemize}

\begin{verbatim}
 
bool var0 <- (date('2016-01-01') between(date('2000-01-01'), date('2020-02-02'))); // var0 equals true// // will return true if the current_date of the model is in_between the 2 between(date('2000-01-01'), date('2020-02-02')) 
\end{verbatim}

\begin{itemize}
\tightlist
\item
  returns true if the first operand is between the two dates passed in
  arguments (both exclusive). Can be combined with `every' to express a
  frequency between two dates
\end{itemize}

\begin{verbatim}
 
bool var3 <- (date('2016-01-01') between(date('2000-01-01'), date('2020-02-02'))); // var3 equals true// will return true every new day between these two dates, taking the first one as the starting point every(#day between(date('2000-01-01'), date('2020-02-02')))  
\end{verbatim}

\subsubsection{Examples:}\label{examples-53}

\begin{verbatim}
 
bool var6 <- between(5.0, 1.0, 10.0); // var6 equals true 
bool var7 <- between(5, 1, 10); // var7 equals true
\end{verbatim}

\begin{center}\rule{0.5\linewidth}{\linethickness}\end{center}

\subsection{\texorpdfstring{\texttt{betweenness\_centrality}}{betweenness\_centrality}}\label{betweenness_centrality}

\subsubsection{Possible use:}\label{possible-use-68}

\begin{itemize}
\tightlist
\item
  \textbf{\texttt{betweenness\_centrality}} (\texttt{graph})
  ---\textgreater{} \texttt{map}
\end{itemize}

\subsubsection{Result:}\label{result-67}

returns a map containing for each vertex (key), its betweenness
centrality (value): number of shortest paths passing through each vertex

\subsubsection{Examples:}\label{examples-54}

\begin{verbatim}
graph graphEpidemio <- graph([]);  
map var1 <- betweenness_centrality(graphEpidemio); // var1 equals the betweenness centrality index of the graph
\end{verbatim}

\begin{center}\rule{0.5\linewidth}{\linethickness}\end{center}

\subsection{\texorpdfstring{\texttt{biggest\_cliques\_of}}{biggest\_cliques\_of}}\label{biggest_cliques_of}

\subsubsection{Possible use:}\label{possible-use-69}

\begin{itemize}
\tightlist
\item
  \textbf{\texttt{biggest\_cliques\_of}} (\texttt{graph})
  ---\textgreater{} \texttt{list\textless{}list\textgreater{}}
\end{itemize}

\subsubsection{Result:}\label{result-68}

returns the biggest cliques of a graph using the Bron-Kerbosch clique
detection algorithm

\subsubsection{Examples:}\label{examples-55}

\begin{verbatim}
graph my_graph <- graph([]);  
list<list> var1 <- biggest_cliques_of (my_graph); // var1 equals the list of the biggest cliques as list
\end{verbatim}

\subsubsection{See also:}\label{see-also-44}

\href{OperatorsIM\#maximal_cliques_of}{maximal\_cliques\_of},

\begin{center}\rule{0.5\linewidth}{\linethickness}\end{center}

\subsection{\texorpdfstring{\texttt{binomial}}{binomial}}\label{binomial}

\subsubsection{Possible use:}\label{possible-use-70}

\begin{itemize}
\tightlist
\item
  \texttt{int} \textbf{\texttt{binomial}} \texttt{float}
  ---\textgreater{} \texttt{int}
\item
  \textbf{\texttt{binomial}} (\texttt{int} , \texttt{float})
  ---\textgreater{} \texttt{int}
\end{itemize}

\subsubsection{Result:}\label{result-69}

A value from a random variable following a binomial distribution. The
operands represent the number of experiments n and the success
probability p.

\subsubsection{Comment:}\label{comment-15}

The binomial distribution is the discrete probability distribution of
the number of successes in a sequence of n independent yes/no
experiments, each of which yields success with probability p,
cf.~Binomial distribution on Wikipedia.

\subsubsection{Examples:}\label{examples-56}

\begin{verbatim}
 
int var0 <- binomial(15,0.6); // var0 equals a random positive integer
\end{verbatim}

\subsubsection{See also:}\label{see-also-45}

\href{OperatorsNR\#poisson}{poisson}, \href{OperatorsDH\#gauss}{gauss},

\begin{center}\rule{0.5\linewidth}{\linethickness}\end{center}

\subsection{\texorpdfstring{\texttt{binomial\_coeff}}{binomial\_coeff}}\label{binomial_coeff}

\subsubsection{Possible use:}\label{possible-use-71}

\begin{itemize}
\tightlist
\item
  \texttt{int} \textbf{\texttt{binomial\_coeff}} \texttt{int}
  ---\textgreater{} \texttt{float}
\item
  \textbf{\texttt{binomial\_coeff}} (\texttt{int} , \texttt{int})
  ---\textgreater{} \texttt{float}
\end{itemize}

\subsubsection{Result:}\label{result-70}

Returns n choose k as a double. Note the integerization of the double
return value.

\begin{center}\rule{0.5\linewidth}{\linethickness}\end{center}

\subsection{\texorpdfstring{\texttt{binomial\_complemented}}{binomial\_complemented}}\label{binomial_complemented}

\subsubsection{Possible use:}\label{possible-use-72}

\begin{itemize}
\tightlist
\item
  \textbf{\texttt{binomial\_complemented}} (\texttt{int}, \texttt{int},
  \texttt{float}) ---\textgreater{} \texttt{float}
\end{itemize}

\subsubsection{Result:}\label{result-71}

Returns the sum of the terms k+1 through n of the Binomial probability
density, where n is the number of trials and P is the probability of
success in the range 0 to 1.

\begin{center}\rule{0.5\linewidth}{\linethickness}\end{center}

\subsection{\texorpdfstring{\texttt{binomial\_sum}}{binomial\_sum}}\label{binomial_sum}

\subsubsection{Possible use:}\label{possible-use-73}

\begin{itemize}
\tightlist
\item
  \textbf{\texttt{binomial\_sum}} (\texttt{int}, \texttt{int},
  \texttt{float}) ---\textgreater{} \texttt{float}
\end{itemize}

\subsubsection{Result:}\label{result-72}

Returns the sum of the terms 0 through k of the Binomial probability
density, where n is the number of trials and p is the probability of
success in the range 0 to 1.

\begin{center}\rule{0.5\linewidth}{\linethickness}\end{center}

\subsection{\texorpdfstring{\texttt{blend}}{blend}}\label{blend}

\subsubsection{Possible use:}\label{possible-use-74}

\begin{itemize}
\tightlist
\item
  \texttt{rgb} \textbf{\texttt{blend}} \texttt{rgb} ---\textgreater{}
  \texttt{rgb}
\item
  \textbf{\texttt{blend}} (\texttt{rgb} , \texttt{rgb})
  ---\textgreater{} \texttt{rgb}
\item
  \textbf{\texttt{blend}} (\texttt{rgb}, \texttt{rgb}, \texttt{float})
  ---\textgreater{} \texttt{rgb}
\end{itemize}

\subsubsection{Result:}\label{result-73}

Blend two colors with an optional ratio (c1 \texttt{*} r + c2 \texttt{*}
(1 - r)) between 0 and 1

\subsubsection{Special cases:}\label{special-cases-25}

\begin{itemize}
\tightlist
\item
  If the ratio is omitted, an even blend is done
\end{itemize}

\begin{verbatim}
 
rgb var1 <- blend(#red, #blue); // var1 equals to a color very close to the purple
\end{verbatim}

\subsubsection{Examples:}\label{examples-57}

\begin{verbatim}
 
rgb var3 <- blend(#red, #blue, 0.3); // var3 equals to a color between the purple and the blue
\end{verbatim}

\subsubsection{See also:}\label{see-also-46}

\href{OperatorsNR\#rgb}{rgb}, \href{OperatorsDH\#hsb}{hsb},

\begin{center}\rule{0.5\linewidth}{\linethickness}\end{center}

\subsection{\texorpdfstring{\texttt{bool}}{bool}}\label{bool}

\subsubsection{Possible use:}\label{possible-use-75}

\begin{itemize}
\tightlist
\item
  \textbf{\texttt{bool}} (\texttt{any}) ---\textgreater{} \texttt{bool}
\end{itemize}

\subsubsection{Result:}\label{result-74}

Casts the operand into the type bool

\begin{center}\rule{0.5\linewidth}{\linethickness}\end{center}

\subsection{\texorpdfstring{\texttt{box}}{box}}\label{box}

\subsubsection{Possible use:}\label{possible-use-76}

\begin{itemize}
\tightlist
\item
  \textbf{\texttt{box}} (\texttt{point}) ---\textgreater{}
  \texttt{geometry}
\item
  \textbf{\texttt{box}} (\texttt{float}, \texttt{float}, \texttt{float})
  ---\textgreater{} \texttt{geometry}
\end{itemize}

\subsubsection{Result:}\label{result-75}

A box geometry which side sizes are given by the operands.

\subsubsection{Comment:}\label{comment-16}

the center of the box is by default the location of the current agent in
which has been called this operator.the center of the box is by default
the location of the current agent in which has been called this
operator.

\subsubsection{Special cases:}\label{special-cases-26}

\begin{itemize}
\tightlist
\item
  returns nil if the operand is nil.\\
\item
  returns nil if the operand is nil.
\end{itemize}

\subsubsection{Examples:}\label{examples-58}

\begin{verbatim}
 
geometry var0 <- box(10, 5 , 5); // var0 equals a geometry as a rectangle with width = 10, height = 5 depth= 5. 
geometry var1 <- box({10, 5 , 5}); // var1 equals a geometry as a rectangle with width = 10, height = 5 depth= 5.
\end{verbatim}

\subsubsection{See also:}\label{see-also-47}

\href{OperatorsAA\#around}{around}, \href{OperatorsBC\#circle}{circle},
\href{OperatorsSZ\#sphere}{sphere}, \href{OperatorsBC\#cone}{cone},
\href{OperatorsIM\#line}{line}, \href{OperatorsIM\#link}{link},
\href{OperatorsNR\#norm}{norm}, \href{OperatorsNR\#point}{point},
\href{OperatorsNR\#polygon}{polygon},
\href{OperatorsNR\#polyline}{polyline},
\href{OperatorsSZ\#square}{square}, \href{OperatorsBC\#cube}{cube},
\href{OperatorsSZ\#triangle}{triangle},

\begin{center}\rule{0.5\linewidth}{\linethickness}\end{center}

\subsection{\texorpdfstring{\texttt{brewer\_colors}}{brewer\_colors}}\label{brewer_colors}

\subsubsection{Possible use:}\label{possible-use-77}

\begin{itemize}
\tightlist
\item
  \textbf{\texttt{brewer\_colors}} (\texttt{string}) ---\textgreater{}
  \texttt{list\textless{}rgb\textgreater{}}
\item
  \texttt{string} \textbf{\texttt{brewer\_colors}} \texttt{int}
  ---\textgreater{} \texttt{list\textless{}rgb\textgreater{}}
\item
  \textbf{\texttt{brewer\_colors}} (\texttt{string} , \texttt{int})
  ---\textgreater{} \texttt{list\textless{}rgb\textgreater{}}
\end{itemize}

\subsubsection{Result:}\label{result-76}

Build a list of colors of a given type (see website
\url{http://colorbrewer2.org/}) with a given number of classes Build a
list of colors of a given type (see website
\url{http://colorbrewer2.org/})

\subsubsection{Examples:}\label{examples-59}

\begin{verbatim}
 
list<rgb> var0 <- list<rgb> colors <- brewer_colors("Pastel1", 10);; // var0 equals a list of 10 sequential colors 
list<rgb> var1 <- list<rgb> colors <- brewer_colors("OrRd");; // var1 equals a list of 6 blue colors
\end{verbatim}

\subsubsection{See also:}\label{see-also-48}

\href{OperatorsBC\#brewer_palettes}{brewer\_palettes},

\begin{center}\rule{0.5\linewidth}{\linethickness}\end{center}

\subsection{\texorpdfstring{\texttt{brewer\_palettes}}{brewer\_palettes}}\label{brewer_palettes}

\subsubsection{Possible use:}\label{possible-use-78}

\begin{itemize}
\tightlist
\item
  \textbf{\texttt{brewer\_palettes}} (\texttt{int}) ---\textgreater{}
  \texttt{list\textless{}string\textgreater{}}
\item
  \texttt{int} \textbf{\texttt{brewer\_palettes}} \texttt{int}
  ---\textgreater{} \texttt{list\textless{}string\textgreater{}}
\item
  \textbf{\texttt{brewer\_palettes}} (\texttt{int} , \texttt{int})
  ---\textgreater{} \texttt{list\textless{}string\textgreater{}}
\end{itemize}

\subsubsection{Result:}\label{result-77}

returns the list a palette with a given min number of classes and max
number of classes) returns the list a palette with a given min number of
classes and max number of classes)

\subsubsection{Examples:}\label{examples-60}

\begin{verbatim}
 
list<string> var0 <- list<rgb> colors <- brewer_palettes(5,10);; // var0 equals a list of palettes that are composed of a min of 5 colors and a max of 10 colors 
list<string> var1 <- list<rgb> colors <- brewer_palettes();; // var1 equals a list of palettes that are composed of a min of 5 colors
\end{verbatim}

\subsubsection{See also:}\label{see-also-49}

\href{OperatorsBC\#brewer_colors}{brewer\_colors},

\begin{center}\rule{0.5\linewidth}{\linethickness}\end{center}

\subsection{\texorpdfstring{\texttt{buffer}}{buffer}}\label{buffer}

Same signification as \href{OperatorsAA\#+}{+}

\begin{center}\rule{0.5\linewidth}{\linethickness}\end{center}

\subsection{\texorpdfstring{\texttt{build}}{build}}\label{build}

\subsubsection{Possible use:}\label{possible-use-79}

\begin{itemize}
\tightlist
\item
  \textbf{\texttt{build}}
  (\texttt{matrix\textless{}float\textgreater{}}) ---\textgreater{}
  \texttt{regression}
\item
  \texttt{matrix\textless{}float\textgreater{}} \textbf{\texttt{build}}
  \texttt{string} ---\textgreater{} \texttt{regression}
\item
  \textbf{\texttt{build}} (\texttt{matrix\textless{}float\textgreater{}}
  , \texttt{string}) ---\textgreater{} \texttt{regression}
\end{itemize}

\subsubsection{Result:}\label{result-78}

returns the regression build from the matrix data (a row = an instance,
the last value of each line is the y value) while using the given method
(``GLS'' or ``OLS''). Usage: build(data,method) returns the regression
build from the matrix data (a row = an instance, the last value of each
line is the y value) while using the given ordinary least squares
method. Usage: build(data)

\subsubsection{Examples:}\label{examples-61}

\begin{verbatim}
build(matrix([[1,2,3,4],[2,3,4,2]]),"GLS") matrix([[1,2,3,4],[2,3,4,2]]) 
\end{verbatim}

\begin{center}\rule{0.5\linewidth}{\linethickness}\end{center}

\subsection{\texorpdfstring{\texttt{ceil}}{ceil}}\label{ceil}

\subsubsection{Possible use:}\label{possible-use-80}

\begin{itemize}
\tightlist
\item
  \textbf{\texttt{ceil}} (\texttt{float}) ---\textgreater{}
  \texttt{float}
\end{itemize}

\subsubsection{Result:}\label{result-79}

Maps the operand to the smallest following integer, i.e.~the smallest
integer not less than x.

\subsubsection{Examples:}\label{examples-62}

\begin{verbatim}
 
float var0 <- ceil(3); // var0 equals 3.0 
float var1 <- ceil(3.5); // var1 equals 4.0 
float var2 <- ceil(-4.7); // var2 equals -4.0
\end{verbatim}

\subsubsection{See also:}\label{see-also-50}

\href{OperatorsDH\#floor}{floor}, \href{OperatorsNR\#round}{round},

\begin{center}\rule{0.5\linewidth}{\linethickness}\end{center}

\subsection{\texorpdfstring{\texttt{centroid}}{centroid}}\label{centroid}

\subsubsection{Possible use:}\label{possible-use-81}

\begin{itemize}
\tightlist
\item
  \textbf{\texttt{centroid}} (\texttt{geometry}) ---\textgreater{}
  \texttt{point}
\end{itemize}

\subsubsection{Result:}\label{result-80}

Centroid (weighted sum of the centroids of a decomposition of the area
into triangles) of the operand-geometry. Can be different to the
location of the geometry

\subsubsection{Examples:}\label{examples-63}

\begin{verbatim}
 
point var0 <- centroid(world); // var0 equals the centroid of the square, for example : {50.0,50.0}.
\end{verbatim}

\subsubsection{See also:}\label{see-also-51}

\href{OperatorsAA\#any_location_in}{any\_location\_in},
\href{OperatorsBC\#closest_points_with}{closest\_points\_with},
\href{OperatorsDH\#farthest_point_to}{farthest\_point\_to},
\href{OperatorsNR\#points_at}{points\_at},

\begin{center}\rule{0.5\linewidth}{\linethickness}\end{center}

\subsection{\texorpdfstring{\texttt{char}}{char}}\label{char}

\subsubsection{Possible use:}\label{possible-use-82}

\begin{itemize}
\tightlist
\item
  \textbf{\texttt{char}} (\texttt{int}) ---\textgreater{}
  \texttt{string}
\end{itemize}

\subsubsection{Special cases:}\label{special-cases-27}

\begin{itemize}
\tightlist
\item
  converts ACSII integer value to character
\end{itemize}

\begin{verbatim}
 
string var0 <- char (34); // var0 equals '"'
\end{verbatim}

\begin{center}\rule{0.5\linewidth}{\linethickness}\end{center}

\subsection{\texorpdfstring{\texttt{chi\_square}}{chi\_square}}\label{chi_square}

\subsubsection{Possible use:}\label{possible-use-83}

\begin{itemize}
\tightlist
\item
  \texttt{float} \textbf{\texttt{chi\_square}} \texttt{float}
  ---\textgreater{} \texttt{float}
\item
  \textbf{\texttt{chi\_square}} (\texttt{float} , \texttt{float})
  ---\textgreater{} \texttt{float}
\end{itemize}

\subsubsection{Result:}\label{result-81}

Returns the area under the left hand tail (from 0 to x) of the Chi
square probability density function with df degrees of freedom.

\begin{center}\rule{0.5\linewidth}{\linethickness}\end{center}

\subsection{\texorpdfstring{\texttt{chi\_square\_complemented}}{chi\_square\_complemented}}\label{chi_square_complemented}

\subsubsection{Possible use:}\label{possible-use-84}

\begin{itemize}
\tightlist
\item
  \texttt{float} \textbf{\texttt{chi\_square\_complemented}}
  \texttt{float} ---\textgreater{} \texttt{float}
\item
  \textbf{\texttt{chi\_square\_complemented}} (\texttt{float} ,
  \texttt{float}) ---\textgreater{} \texttt{float}
\end{itemize}

\subsubsection{Result:}\label{result-82}

Returns the area under the right hand tail (from x to infinity) of the
Chi square probability density function with df degrees of freedom.

\begin{center}\rule{0.5\linewidth}{\linethickness}\end{center}

\subsection{\texorpdfstring{\texttt{circle}}{circle}}\label{circle}

\subsubsection{Possible use:}\label{possible-use-85}

\begin{itemize}
\tightlist
\item
  \textbf{\texttt{circle}} (\texttt{float}) ---\textgreater{}
  \texttt{geometry}
\item
  \texttt{float} \textbf{\texttt{circle}} \texttt{point}
  ---\textgreater{} \texttt{geometry}
\item
  \textbf{\texttt{circle}} (\texttt{float} , \texttt{point})
  ---\textgreater{} \texttt{geometry}
\end{itemize}

\subsubsection{Result:}\label{result-83}

A circle geometry which radius is equal to the first operand, and the
center has the location equal to the second operand. A circle geometry
which radius is equal to the operand.

\subsubsection{Comment:}\label{comment-17}

the center of the circle is by default the location of the current agent
in which has been called this operator.

\subsubsection{Special cases:}\label{special-cases-28}

\begin{itemize}
\tightlist
\item
  returns a point if the operand is lower or equal to 0.\\
\item
  returns a point if the operand is lower or equal to 0.
\end{itemize}

\subsubsection{Examples:}\label{examples-64}

\begin{verbatim}
 
geometry var0 <- circle(10,{80,30}); // var0 equals a geometry as a circle of radius 10, the center will be in the location {80,30}. 
geometry var1 <- circle(10); // var1 equals a geometry as a circle of radius 10.
\end{verbatim}

\subsubsection{See also:}\label{see-also-52}

\href{OperatorsAA\#around}{around}, \href{OperatorsBC\#cone}{cone},
\href{OperatorsIM\#line}{line}, \href{OperatorsIM\#link}{link},
\href{OperatorsNR\#norm}{norm}, \href{OperatorsNR\#point}{point},
\href{OperatorsNR\#polygon}{polygon},
\href{OperatorsNR\#polyline}{polyline},
\href{OperatorsNR\#rectangle}{rectangle},
\href{OperatorsSZ\#square}{square},
\href{OperatorsSZ\#triangle}{triangle},

\begin{center}\rule{0.5\linewidth}{\linethickness}\end{center}

\subsection{\texorpdfstring{\texttt{clean}}{clean}}\label{clean}

\subsubsection{Possible use:}\label{possible-use-86}

\begin{itemize}
\tightlist
\item
  \textbf{\texttt{clean}} (\texttt{geometry}) ---\textgreater{}
  \texttt{geometry}
\end{itemize}

\subsubsection{Result:}\label{result-84}

A geometry corresponding to the cleaning of the operand (geometry,
agent, point)

\subsubsection{Comment:}\label{comment-18}

The cleaning corresponds to a buffer with a distance of 0.0

\subsubsection{Examples:}\label{examples-65}

\begin{verbatim}
 
geometry var0 <- clean(self); // var0 equals returns the geometry resulting from the cleaning of the geometry of the agent applying the operator.
\end{verbatim}

\begin{center}\rule{0.5\linewidth}{\linethickness}\end{center}

\subsection{\texorpdfstring{\texttt{clean\_network}}{clean\_network}}\label{clean_network}

\subsubsection{Possible use:}\label{possible-use-87}

\begin{itemize}
\tightlist
\item
  \textbf{\texttt{clean\_network}}
  (\texttt{list\textless{}geometry\textgreater{}}, \texttt{float},
  \texttt{bool}, \texttt{bool}) ---\textgreater{}
  \texttt{list\textless{}geometry\textgreater{}}
\end{itemize}

\subsubsection{Result:}\label{result-85}

A list of polylines corresponding to the cleaning of the first operand
(list of polyline geometry or agents), considering the tolerance
distance given by the second operand; the third operator is used to
define if the operator should as well split the lines at their
intersections(true to split the lines); the last operandis used to
specify if the operator should as well keep only the main connected
component of the network. Usage: clean\_network(lines:list of geometries
or agents, tolerance: float, split\_lines: bool,
keepMainConnectedComponent: bool)

\subsubsection{Comment:}\label{comment-19}

The cleaned set of polylines

\subsubsection{Examples:}\label{examples-66}

\begin{verbatim}
 
list<geometry> var0 <- clean_network(my_road_shapefile.contents, 1.0, true, false); // var0 equals returns the list of polulines resulting from the cleaning of the geometry of the agent applying the operator with a tolerance of 1m, and splitting the lines at their intersections.
\end{verbatim}

\begin{center}\rule{0.5\linewidth}{\linethickness}\end{center}

\subsection{\texorpdfstring{\texttt{closest\_points\_with}}{closest\_points\_with}}\label{closest_points_with}

\subsubsection{Possible use:}\label{possible-use-88}

\begin{itemize}
\tightlist
\item
  \texttt{geometry} \textbf{\texttt{closest\_points\_with}}
  \texttt{geometry} ---\textgreater{}
  \texttt{list\textless{}point\textgreater{}}
\item
  \textbf{\texttt{closest\_points\_with}} (\texttt{geometry} ,
  \texttt{geometry}) ---\textgreater{}
  \texttt{list\textless{}point\textgreater{}}
\end{itemize}

\subsubsection{Result:}\label{result-86}

A list of two closest points between the two geometries.

\subsubsection{Examples:}\label{examples-67}

\begin{verbatim}
 
list<point> var0 <- geom1 closest_points_with(geom2); // var0 equals [pt1, pt2] with pt1 the closest point of geom1 to geom2 and pt1 the closest point of geom2 to geom1
\end{verbatim}

\subsubsection{See also:}\label{see-also-53}

\href{OperatorsAA\#any_location_in}{any\_location\_in},
\href{OperatorsAA\#any_point_in}{any\_point\_in},
\href{OperatorsDH\#farthest_point_to}{farthest\_point\_to},
\href{OperatorsNR\#points_at}{points\_at},

\begin{center}\rule{0.5\linewidth}{\linethickness}\end{center}

\subsection{\texorpdfstring{\texttt{closest\_to}}{closest\_to}}\label{closest_to}

\subsubsection{Possible use:}\label{possible-use-89}

\begin{itemize}
\tightlist
\item
  \texttt{container\textless{}agent\textgreater{}}
  \textbf{\texttt{closest\_to}} \texttt{geometry} ---\textgreater{}
  \texttt{geometry}
\item
  \textbf{\texttt{closest\_to}}
  (\texttt{container\textless{}agent\textgreater{}} , \texttt{geometry})
  ---\textgreater{} \texttt{geometry}
\end{itemize}

\subsubsection{Result:}\label{result-87}

An agent or a geometry among the left-operand list of agents, species or
meta-population (addition of species), the closest to the operand
(casted as a geometry).

\subsubsection{Comment:}\label{comment-20}

the distance is computed in the topology of the calling agent (the agent
in which this operator is used), with the distance algorithm specific to
the topology.

\subsubsection{Examples:}\label{examples-68}

\begin{verbatim}
 
geometry var0 <- [ag1, ag2, ag3] closest_to(self); // var0 equals return the closest agent among ag1, ag2 and ag3 to the agent applying the operator.(species1 + species2) closest_to self 
\end{verbatim}

\subsubsection{See also:}\label{see-also-54}

\href{OperatorsNR\#neighbors_at}{neighbors\_at},
\href{OperatorsNR\#neighbors_of}{neighbors\_of},
\href{OperatorsIM\#inside}{inside},
\href{OperatorsNR\#overlapping}{overlapping},
\href{OperatorsAA\#agents_overlapping}{agents\_overlapping},
\href{OperatorsAA\#agents_inside}{agents\_inside},
\href{OperatorsAA\#agent_closest_to}{agent\_closest\_to},

\begin{center}\rule{0.5\linewidth}{\linethickness}\end{center}

\subsection{\texorpdfstring{\texttt{collect}}{collect}}\label{collect}

\subsubsection{Possible use:}\label{possible-use-90}

\begin{itemize}
\tightlist
\item
  \texttt{container} \textbf{\texttt{collect}} \texttt{any\ expression}
  ---\textgreater{} \texttt{list}
\item
  \textbf{\texttt{collect}} (\texttt{container} ,
  \texttt{any\ expression}) ---\textgreater{} \texttt{list}
\end{itemize}

\subsubsection{Result:}\label{result-88}

returns a new list, in which each element is the evaluation of the
right-hand operand.

\subsubsection{Comment:}\label{comment-21}

collect is similar to accumulate except that accumulate always produces
flat lists if the right-hand operand returns a list.In addition, collect
can be applied to any container.

\subsubsection{Special cases:}\label{special-cases-29}

\begin{itemize}
\tightlist
\item
  if the left-hand operand is nil, collect throws an error
\end{itemize}

\subsubsection{Examples:}\label{examples-69}

\begin{verbatim}
 
list var0 <- [1,2,4] collect (each *2); // var0 equals [2,4,8] 
list var1 <- [1,2,4] collect ([2,4]); // var1 equals [[2,4],[2,4],[2,4]] 
list var2 <- [1::2, 3::4, 5::6] collect (each + 2); // var2 equals [4,6,8] 
list var3 <- (list(node) collect (node(each).location.x * 2); // var3 equals the list of nodes with their x multiplied by 2
\end{verbatim}

\subsubsection{See also:}\label{see-also-55}

\href{OperatorsAA\#accumulate}{accumulate},

\begin{center}\rule{0.5\linewidth}{\linethickness}\end{center}

\subsection{\texorpdfstring{\texttt{column\_at}}{column\_at}}\label{column_at}

\subsubsection{Possible use:}\label{possible-use-91}

\begin{itemize}
\tightlist
\item
  \texttt{matrix} \textbf{\texttt{column\_at}} \texttt{int}
  ---\textgreater{} \texttt{list}
\item
  \textbf{\texttt{column\_at}} (\texttt{matrix} , \texttt{int})
  ---\textgreater{} \texttt{list}
\end{itemize}

\subsubsection{Result:}\label{result-89}

returns the column at a num\_col (right-hand operand)

\subsubsection{Examples:}\label{examples-70}

\begin{verbatim}
 
list var0 <- matrix([["el11","el12","el13"],["el21","el22","el23"],["el31","el32","el33"]]) column_at 2; // var0 equals ["el31","el32","el33"]
\end{verbatim}

\subsubsection{See also:}\label{see-also-56}

\href{OperatorsNR\#row_at}{row\_at},
\href{OperatorsNR\#rows_list}{rows\_list},

\begin{center}\rule{0.5\linewidth}{\linethickness}\end{center}

\subsection{\texorpdfstring{\texttt{columns\_list}}{columns\_list}}\label{columns_list}

\subsubsection{Possible use:}\label{possible-use-92}

\begin{itemize}
\tightlist
\item
  \textbf{\texttt{columns\_list}} (\texttt{matrix}) ---\textgreater{}
  \texttt{list\textless{}list\textgreater{}}
\end{itemize}

\subsubsection{Result:}\label{result-90}

returns a list of the columns of the matrix, with each column as a list
of elements

\subsubsection{Examples:}\label{examples-71}

\begin{verbatim}
 
list<list> var0 <- columns_list(matrix([["el11","el12","el13"],["el21","el22","el23"],["el31","el32","el33"]])); // var0 equals [["el11","el12","el13"],["el21","el22","el23"],["el31","el32","el33"]]
\end{verbatim}

\subsubsection{See also:}\label{see-also-57}

\href{OperatorsNR\#rows_list}{rows\_list},

\begin{center}\rule{0.5\linewidth}{\linethickness}\end{center}

\subsection{\texorpdfstring{\texttt{command}}{command}}\label{command}

\subsubsection{Possible use:}\label{possible-use-93}

\begin{itemize}
\tightlist
\item
  \textbf{\texttt{command}} (\texttt{string}) ---\textgreater{}
  \texttt{string}
\item
  \texttt{string} \textbf{\texttt{command}} \texttt{string}
  ---\textgreater{} \texttt{string}
\item
  \textbf{\texttt{command}} (\texttt{string} , \texttt{string})
  ---\textgreater{} \texttt{string}
\item
  \textbf{\texttt{command}} (\texttt{string}, \texttt{string},
  \texttt{msi.gama.util.GamaMap\textless{}java.lang.String,java.lang.String\textgreater{}})
  ---\textgreater{} \texttt{string}
\end{itemize}

\subsubsection{Result:}\label{result-91}

command allows GAMA to issue a system command using the system terminal
or shell and to receive a string containing the outcome of the command
or script executed. By default, commands are blocking the agent calling
them, unless the sequence `\&' is used at the end. In this case, the
result of the operator is an empty string. The basic form with only one
string in argument uses the directory of the model and does not set any
environment variables. Two other forms (with a directory and a map of
environment variables) are available. command allows GAMA to issue a
system command using the system terminal or shell and to receive a
string containing the outcome of the command or script executed. By
default, commands are blocking the agent calling them, unless the
sequence `\&' is used at the end. In this case, the result of the
operator is an empty string command allows GAMA to issue a system
command using the system terminal or shell and to receive a string
containing the outcome of the command or script executed. By default,
commands are blocking the agent calling them, unless the sequence `\&'
is used at the end. In this case, the result of the operator is an empty
string. The basic form with only one string in argument uses the
directory of the model and does not set any environment variables. Two
other forms (with a directory and a map of environment variables) are
available.

\begin{center}\rule{0.5\linewidth}{\linethickness}\end{center}

\subsection{\texorpdfstring{\texttt{cone}}{cone}}\label{cone}

\subsubsection{Possible use:}\label{possible-use-94}

\begin{itemize}
\tightlist
\item
  \textbf{\texttt{cone}} (\texttt{point}) ---\textgreater{}
  \texttt{geometry}
\item
  \texttt{int} \textbf{\texttt{cone}} \texttt{int} ---\textgreater{}
  \texttt{geometry}
\item
  \textbf{\texttt{cone}} (\texttt{int} , \texttt{int}) ---\textgreater{}
  \texttt{geometry}
\end{itemize}

\subsubsection{Result:}\label{result-92}

A cone geometry which min and max angles are given by the operands. A
cone geometry which min and max angles are given by the operands.

\subsubsection{Comment:}\label{comment-22}

the center of the cone is by default the location of the current agent
in which has been called this operator.the center of the cone is by
default the location of the current agent in which has been called this
operator.

\subsubsection{Special cases:}\label{special-cases-30}

\begin{itemize}
\tightlist
\item
  returns nil if the operand is nil.\\
\item
  returns nil if the operand is nil.
\end{itemize}

\subsubsection{Examples:}\label{examples-72}

\begin{verbatim}
 
geometry var0 <- cone({0, 45}); // var0 equals a geometry as a cone with min angle is 0 and max angle is 45. 
geometry var1 <- cone(0, 45); // var1 equals a geometry as a cone with min angle is 0 and max angle is 45.
\end{verbatim}

\subsubsection{See also:}\label{see-also-58}

\href{OperatorsAA\#around}{around}, \href{OperatorsBC\#circle}{circle},
\href{OperatorsIM\#line}{line}, \href{OperatorsIM\#link}{link},
\href{OperatorsNR\#norm}{norm}, \href{OperatorsNR\#point}{point},
\href{OperatorsNR\#polygon}{polygon},
\href{OperatorsNR\#polyline}{polyline},
\href{OperatorsNR\#rectangle}{rectangle},
\href{OperatorsSZ\#square}{square},
\href{OperatorsSZ\#triangle}{triangle},

\begin{center}\rule{0.5\linewidth}{\linethickness}\end{center}

\subsection{\texorpdfstring{\texttt{cone3D}}{cone3D}}\label{cone3d}

\subsubsection{Possible use:}\label{possible-use-95}

\begin{itemize}
\tightlist
\item
  \texttt{float} \textbf{\texttt{cone3D}} \texttt{float}
  ---\textgreater{} \texttt{geometry}
\item
  \textbf{\texttt{cone3D}} (\texttt{float} , \texttt{float})
  ---\textgreater{} \texttt{geometry}
\end{itemize}

\subsubsection{Result:}\label{result-93}

A cone geometry which base radius size is equal to the first operand,
and which the height is equal to the second operand.

\subsubsection{Comment:}\label{comment-23}

the center of the cone is by default the location of the current agent
in which has been called this operator.

\subsubsection{Special cases:}\label{special-cases-31}

\begin{itemize}
\tightlist
\item
  returns a point if the operand is lower or equal to 0.
\end{itemize}

\subsubsection{Examples:}\label{examples-73}

\begin{verbatim}
 
geometry var0 <- cone3D(10.0,5.0); // var0 equals a geometry as a cone with a base circle of radius 10 and a height of 5.
\end{verbatim}

\subsubsection{See also:}\label{see-also-59}

\href{OperatorsAA\#around}{around}, \href{OperatorsBC\#cone}{cone},
\href{OperatorsIM\#line}{line}, \href{OperatorsIM\#link}{link},
\href{OperatorsNR\#norm}{norm}, \href{OperatorsNR\#point}{point},
\href{OperatorsNR\#polygon}{polygon},
\href{OperatorsNR\#polyline}{polyline},
\href{OperatorsNR\#rectangle}{rectangle},
\href{OperatorsSZ\#square}{square},
\href{OperatorsSZ\#triangle}{triangle},

\begin{center}\rule{0.5\linewidth}{\linethickness}\end{center}

\subsection{\texorpdfstring{\texttt{connected\_components\_of}}{connected\_components\_of}}\label{connected_components_of}

\subsubsection{Possible use:}\label{possible-use-96}

\begin{itemize}
\tightlist
\item
  \textbf{\texttt{connected\_components\_of}} (\texttt{graph})
  ---\textgreater{} \texttt{list\textless{}list\textgreater{}}
\item
  \texttt{graph} \textbf{\texttt{connected\_components\_of}}
  \texttt{bool} ---\textgreater{}
  \texttt{list\textless{}list\textgreater{}}
\item
  \textbf{\texttt{connected\_components\_of}} (\texttt{graph} ,
  \texttt{bool}) ---\textgreater{}
  \texttt{list\textless{}list\textgreater{}}
\end{itemize}

\subsubsection{Result:}\label{result-94}

returns the connected components of a graph, i.e.~the list of all
vertices that are in the maximally connected component together with the
specified vertex. returns the connected components of a graph, i.e.~the
list of all edges (if the boolean is true) or vertices (if the boolean
is false) that are in the connected components.

\subsubsection{Examples:}\label{examples-74}

\begin{verbatim}
graph my_graph <- graph([]);  
list<list> var1 <- connected_components_of (my_graph); // var1 equals the list of all the components as listgraph my_graph2 <- graph([]);  
list<list> var3 <- connected_components_of (my_graph2, true); // var3 equals the list of all the components as list
\end{verbatim}

\subsubsection{See also:}\label{see-also-60}

\href{OperatorsAA\#alpha_index}{alpha\_index},
\href{OperatorsBC\#connectivity_index}{connectivity\_index},
\href{OperatorsNR\#nb_cycles}{nb\_cycles},

\begin{center}\rule{0.5\linewidth}{\linethickness}\end{center}

\subsection{\texorpdfstring{\texttt{connectivity\_index}}{connectivity\_index}}\label{connectivity_index}

\subsubsection{Possible use:}\label{possible-use-97}

\begin{itemize}
\tightlist
\item
  \textbf{\texttt{connectivity\_index}} (\texttt{graph})
  ---\textgreater{} \texttt{float}
\end{itemize}

\subsubsection{Result:}\label{result-95}

returns a simple connectivity index. This number is estimated through
the number of nodes (v) and of sub-graphs (p) : IC = (v - p) /(v - 1).

\subsubsection{Examples:}\label{examples-75}

\begin{verbatim}
graph graphEpidemio <- graph([]);  
float var1 <- connectivity_index(graphEpidemio); // var1 equals the connectivity index of the graph
\end{verbatim}

\subsubsection{See also:}\label{see-also-61}

\href{OperatorsAA\#alpha_index}{alpha\_index},
\href{OperatorsBC\#beta_index}{beta\_index},
\href{OperatorsDH\#gamma_index}{gamma\_index},
\href{OperatorsNR\#nb_cycles}{nb\_cycles},

\begin{center}\rule{0.5\linewidth}{\linethickness}\end{center}

\subsection{\texorpdfstring{\texttt{container}}{container}}\label{container}

\subsubsection{Possible use:}\label{possible-use-98}

\begin{itemize}
\tightlist
\item
  \textbf{\texttt{container}} (\texttt{any}) ---\textgreater{}
  \texttt{container}
\end{itemize}

\subsubsection{Result:}\label{result-96}

Casts the operand into the type container

\begin{center}\rule{0.5\linewidth}{\linethickness}\end{center}

\subsection{\texorpdfstring{\texttt{contains}}{contains}}\label{contains}

\subsubsection{Possible use:}\label{possible-use-99}

\begin{itemize}
\tightlist
\item
  \texttt{container\textless{}KeyType,ValueType\textgreater{}}
  \textbf{\texttt{contains}} \texttt{unknown} ---\textgreater{}
  \texttt{bool}
\item
  \textbf{\texttt{contains}}
  (\texttt{container\textless{}KeyType,ValueType\textgreater{}} ,
  \texttt{unknown}) ---\textgreater{} \texttt{bool}
\item
  \texttt{string} \textbf{\texttt{contains}} \texttt{string}
  ---\textgreater{} \texttt{bool}
\item
  \textbf{\texttt{contains}} (\texttt{string} , \texttt{string})
  ---\textgreater{} \texttt{bool}
\end{itemize}

\subsubsection{Result:}\label{result-97}

true, if the container contains the right operand, false otherwise

\subsubsection{Comment:}\label{comment-24}

the contains operator behavior depends on the nature of the operand

\subsubsection{Special cases:}\label{special-cases-32}

\begin{itemize}
\tightlist
\item
  if it is a map, contains returns true if the operand is a key of the
  map\\
\item
  if it is a file, contains returns true it the operand is contained in
  the file content\\
\item
  if it is a population, contains returns true if the operand is an
  agent of the population, false otherwise\\
\item
  if it is a graph, contains returns true if the operand is a node or an
  edge of the graph, false otherwise\\
\item
  if both operands are strings, returns true if the right-hand operand
  contains the right-hand pattern;\\
\item
  if it is a list or a matrix, contains returns true if the list or
  matrix contains the right operand
\end{itemize}

\begin{verbatim}
 
bool var0 <- [1, 2, 3] contains 2; // var0 equals true 
bool var1 <- [{1,2}, {3,4}, {5,6}] contains {3,4}; // var1 equals true
\end{verbatim}

\subsubsection{Examples:}\label{examples-76}

\begin{verbatim}
 
bool var2 <- 'abcded' contains 'bc'; // var2 equals true
\end{verbatim}

\subsubsection{See also:}\label{see-also-62}

\href{OperatorsBC\#contains_all}{contains\_all},
\href{OperatorsBC\#contains_any}{contains\_any},

\begin{center}\rule{0.5\linewidth}{\linethickness}\end{center}

\subsection{\texorpdfstring{\texttt{contains\_all}}{contains\_all}}\label{contains_all}

\subsubsection{Possible use:}\label{possible-use-100}

\begin{itemize}
\tightlist
\item
  \texttt{string} \textbf{\texttt{contains\_all}} \texttt{list}
  ---\textgreater{} \texttt{bool}
\item
  \textbf{\texttt{contains\_all}} (\texttt{string} , \texttt{list})
  ---\textgreater{} \texttt{bool}
\item
  \texttt{container} \textbf{\texttt{contains\_all}} \texttt{container}
  ---\textgreater{} \texttt{bool}
\item
  \textbf{\texttt{contains\_all}} (\texttt{container} ,
  \texttt{container}) ---\textgreater{} \texttt{bool}
\end{itemize}

\subsubsection{Result:}\label{result-98}

true if the left operand contains all the elements of the right operand,
false otherwise

\subsubsection{Comment:}\label{comment-25}

the definition of contains depends on the container

\subsubsection{Special cases:}\label{special-cases-33}

\begin{itemize}
\tightlist
\item
  if the right operand is nil or empty, contains\_all returns true\\
\item
  if the left-operand is a string, test whether the string contains all
  the element of the list;
\end{itemize}

\begin{verbatim}
 
bool var0 <- "abcabcabc" contains_all ["ca","xy"]; // var0 equals false
\end{verbatim}

\subsubsection{Examples:}\label{examples-77}

\begin{verbatim}
 
bool var1 <- [1,2,3,4,5,6] contains_all [2,4]; // var1 equals true  
bool var2 <- [1,2,3,4,5,6] contains_all [2,8]; // var2 equals false 
bool var3 <- [1::2, 3::4, 5::6] contains_all [1,3]; // var3 equals false  
bool var4 <- [1::2, 3::4, 5::6] contains_all [2,4]; // var4 equals true
\end{verbatim}

\subsubsection{See also:}\label{see-also-63}

\href{OperatorsBC\#contains}{contains},
\href{OperatorsBC\#contains_any}{contains\_any},

\begin{center}\rule{0.5\linewidth}{\linethickness}\end{center}

\subsection{\texorpdfstring{\texttt{contains\_any}}{contains\_any}}\label{contains_any}

\subsubsection{Possible use:}\label{possible-use-101}

\begin{itemize}
\tightlist
\item
  \texttt{string} \textbf{\texttt{contains\_any}} \texttt{list}
  ---\textgreater{} \texttt{bool}
\item
  \textbf{\texttt{contains\_any}} (\texttt{string} , \texttt{list})
  ---\textgreater{} \texttt{bool}
\item
  \texttt{container} \textbf{\texttt{contains\_any}} \texttt{container}
  ---\textgreater{} \texttt{bool}
\item
  \textbf{\texttt{contains\_any}} (\texttt{container} ,
  \texttt{container}) ---\textgreater{} \texttt{bool}
\end{itemize}

\subsubsection{Result:}\label{result-99}

true if the left operand contains one of the elements of the right
operand, false otherwise

\subsubsection{Comment:}\label{comment-26}

the definition of contains depends on the container

\subsubsection{Special cases:}\label{special-cases-34}

\begin{itemize}
\tightlist
\item
  if the right operand is nil or empty, contains\_any returns false
\end{itemize}

\subsubsection{Examples:}\label{examples-78}

\begin{verbatim}
 
bool var0 <- "abcabcabc" contains_any ["ca","xy"]; // var0 equals true 
bool var1 <- [1,2,3,4,5,6] contains_any [2,4]; // var1 equals true  
bool var2 <- [1,2,3,4,5,6] contains_any [2,8]; // var2 equals true 
bool var3 <- [1::2, 3::4, 5::6] contains_any [1,3]; // var3 equals false 
bool var4 <- [1::2, 3::4, 5::6] contains_any [2,4]; // var4 equals true
\end{verbatim}

\subsubsection{See also:}\label{see-also-64}

\href{OperatorsBC\#contains}{contains},
\href{OperatorsBC\#contains_all}{contains\_all},

\begin{center}\rule{0.5\linewidth}{\linethickness}\end{center}

\subsection{\texorpdfstring{\texttt{contains\_edge}}{contains\_edge}}\label{contains_edge}

\subsubsection{Possible use:}\label{possible-use-102}

\begin{itemize}
\tightlist
\item
  \texttt{graph} \textbf{\texttt{contains\_edge}} \texttt{pair}
  ---\textgreater{} \texttt{bool}
\item
  \textbf{\texttt{contains\_edge}} (\texttt{graph} , \texttt{pair})
  ---\textgreater{} \texttt{bool}
\item
  \texttt{graph} \textbf{\texttt{contains\_edge}} \texttt{unknown}
  ---\textgreater{} \texttt{bool}
\item
  \textbf{\texttt{contains\_edge}} (\texttt{graph} , \texttt{unknown})
  ---\textgreater{} \texttt{bool}
\end{itemize}

\subsubsection{Result:}\label{result-100}

returns true if the graph(left-hand operand) contains the given edge
(righ-hand operand), false otherwise

\subsubsection{Special cases:}\label{special-cases-35}

\begin{itemize}
\tightlist
\item
  if the left-hand operand is nil, returns false\\
\item
  if the right-hand operand is a pair, returns true if it exists an edge
  between the two elements of the pair in the graph
\end{itemize}

\begin{verbatim}
 
bool var0 <- graphEpidemio contains_edge (node(0)::node(3)); // var0 equals true
\end{verbatim}

\subsubsection{Examples:}\label{examples-79}

\begin{verbatim}
graph graphFromMap <-  as_edge_graph([{1,5}::{12,45},{12,45}::{34,56}]);  
bool var2 <- graphFromMap contains_edge link({1,5},{12,45}); // var2 equals true
\end{verbatim}

\subsubsection{See also:}\label{see-also-65}

\href{OperatorsBC\#contains_vertex}{contains\_vertex},

\begin{center}\rule{0.5\linewidth}{\linethickness}\end{center}

\subsection{\texorpdfstring{\texttt{contains\_vertex}}{contains\_vertex}}\label{contains_vertex}

\subsubsection{Possible use:}\label{possible-use-103}

\begin{itemize}
\tightlist
\item
  \texttt{graph} \textbf{\texttt{contains\_vertex}} \texttt{unknown}
  ---\textgreater{} \texttt{bool}
\item
  \textbf{\texttt{contains\_vertex}} (\texttt{graph} , \texttt{unknown})
  ---\textgreater{} \texttt{bool}
\end{itemize}

\subsubsection{Result:}\label{result-101}

returns true if the graph(left-hand operand) contains the given vertex
(righ-hand operand), false otherwise

\subsubsection{Special cases:}\label{special-cases-36}

\begin{itemize}
\tightlist
\item
  if the left-hand operand is nil, returns false
\end{itemize}

\subsubsection{Examples:}\label{examples-80}

\begin{verbatim}
graph graphFromMap<-  as_edge_graph([{1,5}::{12,45},{12,45}::{34,56}]);  
bool var1 <- graphFromMap contains_vertex {1,5}; // var1 equals true
\end{verbatim}

\subsubsection{See also:}\label{see-also-66}

\href{OperatorsBC\#contains_edge}{contains\_edge},

\begin{center}\rule{0.5\linewidth}{\linethickness}\end{center}

\subsection{\texorpdfstring{\texttt{conversation}}{conversation}}\label{conversation}

\subsubsection{Possible use:}\label{possible-use-104}

\begin{itemize}
\tightlist
\item
  \textbf{\texttt{conversation}} (\texttt{unknown}) ---\textgreater{}
  \texttt{conversation}
\end{itemize}

\begin{center}\rule{0.5\linewidth}{\linethickness}\end{center}

\subsection{\texorpdfstring{\texttt{convex\_hull}}{convex\_hull}}\label{convex_hull}

\subsubsection{Possible use:}\label{possible-use-105}

\begin{itemize}
\tightlist
\item
  \textbf{\texttt{convex\_hull}} (\texttt{geometry}) ---\textgreater{}
  \texttt{geometry}
\end{itemize}

\subsubsection{Result:}\label{result-102}

A geometry corresponding to the convex hull of the operand.

\subsubsection{Examples:}\label{examples-81}

\begin{verbatim}
 
geometry var0 <- convex_hull(self); // var0 equals the convex hull of the geometry of the agent applying the operator
\end{verbatim}

\begin{center}\rule{0.5\linewidth}{\linethickness}\end{center}

\subsection{\texorpdfstring{\texttt{copy}}{copy}}\label{copy}

\subsubsection{Possible use:}\label{possible-use-106}

\begin{itemize}
\tightlist
\item
  \textbf{\texttt{copy}} (\texttt{unknown}) ---\textgreater{}
  \texttt{unknown}
\end{itemize}

\subsubsection{Result:}\label{result-103}

returns a copy of the operand.

\begin{center}\rule{0.5\linewidth}{\linethickness}\end{center}

\subsection{\texorpdfstring{\texttt{copy\_between}}{copy\_between}}\label{copy_between}

\subsubsection{Possible use:}\label{possible-use-107}

\begin{itemize}
\tightlist
\item
  \textbf{\texttt{copy\_between}} (\texttt{list}, \texttt{int},
  \texttt{int}) ---\textgreater{} \texttt{list}
\item
  \textbf{\texttt{copy\_between}} (\texttt{string}, \texttt{int},
  \texttt{int}) ---\textgreater{} \texttt{string}
\end{itemize}

\subsubsection{Result:}\label{result-104}

Returns a copy of the first operand between the indexes determined by
the second (inclusive) and third operands (exclusive)

\subsubsection{Special cases:}\label{special-cases-37}

\begin{itemize}
\tightlist
\item
  If the first operand is empty, returns an empty object of the same
  type\\
\item
  If the second operand is greater than or equal to the third operand,
  return an empty object of the same type\\
\item
  If the first operand is nil, raises an error
\end{itemize}

\subsubsection{Examples:}\label{examples-82}

\begin{verbatim}
 
list var0 <-  copy_between ([4, 1, 6, 9 ,7], 1, 3); // var0 equals [1, 6] 
string var1 <- copy_between("abcabcabc", 2,6); // var1 equals "cabc"
\end{verbatim}

\begin{center}\rule{0.5\linewidth}{\linethickness}\end{center}

\subsection{\texorpdfstring{\texttt{corR}}{corR}}\label{corr}

\subsubsection{Possible use:}\label{possible-use-108}

\begin{itemize}
\tightlist
\item
  \texttt{container} \textbf{\texttt{corR}} \texttt{container}
  ---\textgreater{} \texttt{unknown}
\item
  \textbf{\texttt{corR}} (\texttt{container} , \texttt{container})
  ---\textgreater{} \texttt{unknown}
\end{itemize}

\subsubsection{Result:}\label{result-105}

returns the Pearson correlation coefficient of two given vectors
(right-hand operands) in given variable (left-hand operand).

\subsubsection{Special cases:}\label{special-cases-38}

\begin{itemize}
\tightlist
\item
  if the lengths of two vectors in the right-hand aren't equal, returns
  0
\end{itemize}

\subsubsection{Examples:}\label{examples-83}

\begin{verbatim}
list X <- [1, 2, 3]; list Y <- [1, 2, 4];  
unknown var2 <- corR(X, Y); // var2 equals 0.981980506061966
\end{verbatim}

\begin{center}\rule{0.5\linewidth}{\linethickness}\end{center}

\subsection{\texorpdfstring{\texttt{correlation}}{correlation}}\label{correlation}

\subsubsection{Possible use:}\label{possible-use-109}

\begin{itemize}
\tightlist
\item
  \texttt{container} \textbf{\texttt{correlation}} \texttt{container}
  ---\textgreater{} \texttt{float}
\item
  \textbf{\texttt{correlation}} (\texttt{container} ,
  \texttt{container}) ---\textgreater{} \texttt{float}
\end{itemize}

\subsubsection{Result:}\label{result-106}

Returns the correlation of two data sequences

\begin{center}\rule{0.5\linewidth}{\linethickness}\end{center}

\subsection{\texorpdfstring{\texttt{cos}}{cos}}\label{cos}

\subsubsection{Possible use:}\label{possible-use-110}

\begin{itemize}
\tightlist
\item
  \textbf{\texttt{cos}} (\texttt{float}) ---\textgreater{}
  \texttt{float}
\item
  \textbf{\texttt{cos}} (\texttt{int}) ---\textgreater{} \texttt{float}
\end{itemize}

\subsubsection{Result:}\label{result-107}

Returns the value (in {[}-1,1{]}) of the cosinus of the operand (in
decimal degrees). The argument is casted to an int before being
evaluated.

\subsubsection{Special cases:}\label{special-cases-39}

\begin{itemize}
\tightlist
\item
  Operand values out of the range {[}0-359{]} are normalized.
\end{itemize}

\subsubsection{Examples:}\label{examples-84}

\begin{verbatim}
 
float var0 <- cos (0); // var0 equals 1.0 
float var1 <- cos(360); // var1 equals 1.0 
float var2 <- cos(-720); // var2 equals 1.0
\end{verbatim}

\subsubsection{See also:}\label{see-also-67}

\href{OperatorsSZ\#sin}{sin}, \href{OperatorsSZ\#tan}{tan},

\begin{center}\rule{0.5\linewidth}{\linethickness}\end{center}

\subsection{\texorpdfstring{\texttt{cos\_rad}}{cos\_rad}}\label{cos_rad}

\subsubsection{Possible use:}\label{possible-use-111}

\begin{itemize}
\tightlist
\item
  \textbf{\texttt{cos\_rad}} (\texttt{float}) ---\textgreater{}
  \texttt{float}
\end{itemize}

\subsubsection{Result:}\label{result-108}

Returns the value (in {[}-1,1{]}) of the cosinus of the operand (in
radians).

\subsubsection{Special cases:}\label{special-cases-40}

\begin{itemize}
\tightlist
\item
  Operand values out of the range {[}0-359{]} are normalized.
\end{itemize}

\subsubsection{See also:}\label{see-also-68}

\href{OperatorsSZ\#sin}{sin}, \href{OperatorsSZ\#tan}{tan},

\begin{center}\rule{0.5\linewidth}{\linethickness}\end{center}

\subsection{\texorpdfstring{\texttt{count}}{count}}\label{count}

\subsubsection{Possible use:}\label{possible-use-112}

\begin{itemize}
\tightlist
\item
  \texttt{container} \textbf{\texttt{count}} \texttt{any\ expression}
  ---\textgreater{} \texttt{int}
\item
  \textbf{\texttt{count}} (\texttt{container} ,
  \texttt{any\ expression}) ---\textgreater{} \texttt{int}
\end{itemize}

\subsubsection{Result:}\label{result-109}

returns an int, equal to the number of elements of the left-hand operand
that make the right-hand operand evaluate to true.

\subsubsection{Comment:}\label{comment-27}

in the right-hand operand, the keyword each can be used to represent, in
turn, each of the elements.

\subsubsection{Special cases:}\label{special-cases-41}

\begin{itemize}
\tightlist
\item
  if the left-hand operand is nil, count throws an error
\end{itemize}

\subsubsection{Examples:}\label{examples-85}

\begin{verbatim}
 
int var0 <- [1,2,3,4,5,6,7,8] count (each > 3); // var0 equals 5// Number of nodes of graph g2 without any out edge graph g2 <- graph([]);  
int var3 <- g2 count (length(g2 out_edges_of each) = 0  ) ; // var3 equals the total number of out edges// Number of agents node with x > 32 int n <- (list(node) count (round(node(each).location.x) > 32);  
int var6 <- [1::2, 3::4, 5::6] count (each > 4); // var6 equals 1
\end{verbatim}

\subsubsection{See also:}\label{see-also-69}

\href{OperatorsDH\#group_by}{group\_by},

\begin{center}\rule{0.5\linewidth}{\linethickness}\end{center}

\subsection{\texorpdfstring{\texttt{covariance}}{covariance}}\label{covariance}

\subsubsection{Possible use:}\label{possible-use-113}

\begin{itemize}
\tightlist
\item
  \texttt{container} \textbf{\texttt{covariance}} \texttt{container}
  ---\textgreater{} \texttt{float}
\item
  \textbf{\texttt{covariance}} (\texttt{container} , \texttt{container})
  ---\textgreater{} \texttt{float}
\end{itemize}

\subsubsection{Result:}\label{result-110}

Returns the covariance of two data sequences

\begin{center}\rule{0.5\linewidth}{\linethickness}\end{center}

\subsection{\texorpdfstring{\texttt{covers}}{covers}}\label{covers}

\subsubsection{Possible use:}\label{possible-use-114}

\begin{itemize}
\tightlist
\item
  \texttt{geometry} \textbf{\texttt{covers}} \texttt{geometry}
  ---\textgreater{} \texttt{bool}
\item
  \textbf{\texttt{covers}} (\texttt{geometry} , \texttt{geometry})
  ---\textgreater{} \texttt{bool}
\end{itemize}

\subsubsection{Result:}\label{result-111}

A boolean, equal to true if the left-geometry (or agent/point) covers
the right-geometry (or agent/point).

\subsubsection{Special cases:}\label{special-cases-42}

\begin{itemize}
\tightlist
\item
  if one of the operand is null, returns false.
\end{itemize}

\subsubsection{Examples:}\label{examples-86}

\begin{verbatim}
 
bool var0 <- square(5) covers square(2); // var0 equals true
\end{verbatim}

\subsubsection{See also:}\label{see-also-70}

\href{OperatorsDH\#disjoint_from}{disjoint\_from},
\href{OperatorsBC\#crosses}{crosses},
\href{OperatorsNR\#overlaps}{overlaps},
\href{OperatorsNR\#partially_overlaps}{partially\_overlaps},
\href{OperatorsSZ\#touches}{touches},

\begin{center}\rule{0.5\linewidth}{\linethickness}\end{center}

\subsection{\texorpdfstring{\texttt{create\_map}}{create\_map}}\label{create_map}

\subsubsection{Possible use:}\label{possible-use-115}

\begin{itemize}
\tightlist
\item
  \texttt{list} \textbf{\texttt{create\_map}} \texttt{list}
  ---\textgreater{} \texttt{map}
\item
  \textbf{\texttt{create\_map}} (\texttt{list} , \texttt{list})
  ---\textgreater{} \texttt{map}
\end{itemize}

\subsubsection{Result:}\label{result-112}

returns a new map using the left operand as keys for the right operand

\subsubsection{Special cases:}\label{special-cases-43}

\begin{itemize}
\tightlist
\item
  if the left operand contains duplicates, create\_map throws an
  error.\\
\item
  if both operands have different lengths, choose the minimum length
  between the two operandsfor the size of the map
\end{itemize}

\subsubsection{Examples:}\label{examples-87}

\begin{verbatim}
 
map<int,string> var0 <- create_map([0,1,2],['a','b','c']); // var0 equals [0::'a',1::'b',2::'c'] 
map<int,float> var1 <- create_map([0,1],[0.1,0.2,0.3]); // var1 equals [0::0.1,1::0.2] 
map<string,float> var2 <- create_map(['a','b','c','d'],[1.0,2.0,3.0]); // var2 equals ['a'::1.0,'b'::2.0,'c'::3.0]
\end{verbatim}

\begin{center}\rule{0.5\linewidth}{\linethickness}\end{center}

\subsection{\texorpdfstring{\texttt{cross}}{cross}}\label{cross}

\subsubsection{Possible use:}\label{possible-use-116}

\begin{itemize}
\tightlist
\item
  \textbf{\texttt{cross}} (\texttt{float}) ---\textgreater{}
  \texttt{geometry}
\item
  \texttt{float} \textbf{\texttt{cross}} \texttt{float}
  ---\textgreater{} \texttt{geometry}
\item
  \textbf{\texttt{cross}} (\texttt{float} , \texttt{float})
  ---\textgreater{} \texttt{geometry}
\end{itemize}

\subsubsection{Result:}\label{result-113}

A cross, which radius is equal to the first operand and the width of the
lines for the second A cross, which radius is equal to the first operand

\subsubsection{Examples:}\label{examples-88}

\begin{verbatim}
 
geometry var0 <- cross(10,2); // var0 equals a geometry as a cross of radius 10, and with a width of 2 for the lines  
geometry var1 <- cross(10); // var1 equals a geometry as a cross of radius 10
\end{verbatim}

\subsubsection{See also:}\label{see-also-71}

\href{OperatorsAA\#around}{around}, \href{OperatorsBC\#cone}{cone},
\href{OperatorsIM\#line}{line}, \href{OperatorsIM\#link}{link},
\href{OperatorsNR\#norm}{norm}, \href{OperatorsNR\#point}{point},
\href{OperatorsNR\#polygon}{polygon},
\href{OperatorsNR\#polyline}{polyline},
\href{OperatorsSZ\#super_ellipse}{super\_ellipse},
\href{OperatorsNR\#rectangle}{rectangle},
\href{OperatorsSZ\#square}{square}, \href{OperatorsBC\#circle}{circle},
\href{OperatorsDH\#ellipse}{ellipse},
\href{OperatorsSZ\#triangle}{triangle},

\begin{center}\rule{0.5\linewidth}{\linethickness}\end{center}

\subsection{\texorpdfstring{\texttt{crosses}}{crosses}}\label{crosses}

\subsubsection{Possible use:}\label{possible-use-117}

\begin{itemize}
\tightlist
\item
  \texttt{geometry} \textbf{\texttt{crosses}} \texttt{geometry}
  ---\textgreater{} \texttt{bool}
\item
  \textbf{\texttt{crosses}} (\texttt{geometry} , \texttt{geometry})
  ---\textgreater{} \texttt{bool}
\end{itemize}

\subsubsection{Result:}\label{result-114}

A boolean, equal to true if the left-geometry (or agent/point) crosses
the right-geometry (or agent/point).

\subsubsection{Special cases:}\label{special-cases-44}

\begin{itemize}
\tightlist
\item
  if one of the operand is null, returns false.\\
\item
  if one operand is a point, returns false.
\end{itemize}

\subsubsection{Examples:}\label{examples-89}

\begin{verbatim}
 
bool var0 <- polyline([{10,10},{20,20}]) crosses polyline([{10,20},{20,10}]); // var0 equals true 
bool var1 <- polyline([{10,10},{20,20}]) crosses {15,15}; // var1 equals true 
bool var2 <- polyline([{0,0},{25,25}]) crosses polygon([{10,10},{10,20},{20,20},{20,10}]); // var2 equals true
\end{verbatim}

\subsubsection{See also:}\label{see-also-72}

\href{OperatorsDH\#disjoint_from}{disjoint\_from},
\href{OperatorsIM\#intersects}{intersects},
\href{OperatorsNR\#overlaps}{overlaps},
\href{OperatorsNR\#partially_overlaps}{partially\_overlaps},
\href{OperatorsSZ\#touches}{touches},

\begin{center}\rule{0.5\linewidth}{\linethickness}\end{center}

\subsection{\texorpdfstring{\texttt{crs}}{crs}}\label{crs}

\subsubsection{Possible use:}\label{possible-use-118}

\begin{itemize}
\tightlist
\item
  \textbf{\texttt{crs}} (\texttt{file}) ---\textgreater{}
  \texttt{string}
\end{itemize}

\subsubsection{Result:}\label{result-115}

the Coordinate Reference System (CRS) of the GIS file

\subsubsection{Examples:}\label{examples-90}

\begin{verbatim}
 
string var0 <- crs(my_shapefile); // var0 equals the crs of the shapefile
\end{verbatim}

\begin{center}\rule{0.5\linewidth}{\linethickness}\end{center}

\subsection{\texorpdfstring{\texttt{CRS\_transform}}{CRS\_transform}}\label{crs_transform}

\subsubsection{Possible use:}\label{possible-use-119}

\begin{itemize}
\tightlist
\item
  \textbf{\texttt{CRS\_transform}} (\texttt{geometry}) ---\textgreater{}
  \texttt{geometry}
\item
  \texttt{geometry} \textbf{\texttt{CRS\_transform}} \texttt{string}
  ---\textgreater{} \texttt{geometry}
\item
  \textbf{\texttt{CRS\_transform}} (\texttt{geometry} , \texttt{string})
  ---\textgreater{} \texttt{geometry}
\end{itemize}

\subsubsection{Special cases:}\label{special-cases-45}

\begin{itemize}
\tightlist
\item
  returns the geometry corresponding to the transformation of the given
  geometry by the current CRS (Coordinate Reference System), the one
  corresponding to the world's agent one
\end{itemize}

\begin{verbatim}
 
geometry var0 <- CRS_transform(shape); // var0 equals a geometry corresponding to the agent geometry transformed into the current CRS
\end{verbatim}

\begin{itemize}
\tightlist
\item
  returns the geometry corresponding to the transformation of the given
  geometry by the left operand CRS (Coordinate Reference System)
\end{itemize}

\begin{verbatim}
 
geometry var1 <- shape CRS_transform("EPSG:4326"); // var1 equals a geometry corresponding to the agent geometry transformed into the EPSG:4326 CRS
\end{verbatim}

\begin{center}\rule{0.5\linewidth}{\linethickness}\end{center}

\subsection{\texorpdfstring{\texttt{csv\_file}}{csv\_file}}\label{csv_file}

\subsubsection{Possible use:}\label{possible-use-120}

\begin{itemize}
\tightlist
\item
  \textbf{\texttt{csv\_file}} (\texttt{string}) ---\textgreater{}
  \texttt{file}
\end{itemize}

\subsubsection{Result:}\label{result-116}

Constructs a file of type csv. Allowed extensions are limited to csv,
tsv

\begin{center}\rule{0.5\linewidth}{\linethickness}\end{center}

\subsection{\texorpdfstring{\texttt{cube}}{cube}}\label{cube}

\subsubsection{Possible use:}\label{possible-use-121}

\begin{itemize}
\tightlist
\item
  \textbf{\texttt{cube}} (\texttt{float}) ---\textgreater{}
  \texttt{geometry}
\end{itemize}

\subsubsection{Result:}\label{result-117}

A cube geometry which side size is equal to the operand.

\subsubsection{Comment:}\label{comment-28}

the center of the cube is by default the location of the current agent
in which has been called this operator.

\subsubsection{Special cases:}\label{special-cases-46}

\begin{itemize}
\tightlist
\item
  returns nil if the operand is nil.
\end{itemize}

\subsubsection{Examples:}\label{examples-91}

\begin{verbatim}
 
geometry var0 <- cube(10); // var0 equals a geometry as a square of side size 10.
\end{verbatim}

\subsubsection{See also:}\label{see-also-73}

\href{OperatorsAA\#around}{around}, \href{OperatorsBC\#circle}{circle},
\href{OperatorsBC\#cone}{cone}, \href{OperatorsIM\#line}{line},
\href{OperatorsIM\#link}{link}, \href{OperatorsNR\#norm}{norm},
\href{OperatorsNR\#point}{point}, \href{OperatorsNR\#polygon}{polygon},
\href{OperatorsNR\#polyline}{polyline},
\href{OperatorsNR\#rectangle}{rectangle},
\href{OperatorsSZ\#triangle}{triangle},

\begin{center}\rule{0.5\linewidth}{\linethickness}\end{center}

\subsection{\texorpdfstring{\texttt{curve}}{curve}}\label{curve}

\subsubsection{Possible use:}\label{possible-use-122}

\begin{itemize}
\tightlist
\item
  \textbf{\texttt{curve}} (\texttt{point}, \texttt{point},
  \texttt{float}) ---\textgreater{} \texttt{geometry}
\item
  \textbf{\texttt{curve}} (\texttt{point}, \texttt{point},
  \texttt{point}) ---\textgreater{} \texttt{geometry}
\item
  \textbf{\texttt{curve}} (\texttt{point}, \texttt{point},
  \texttt{point}, \texttt{int}) ---\textgreater{} \texttt{geometry}
\item
  \textbf{\texttt{curve}} (\texttt{point}, \texttt{point},
  \texttt{float}, \texttt{bool}) ---\textgreater{} \texttt{geometry}
\item
  \textbf{\texttt{curve}} (\texttt{point}, \texttt{point},
  \texttt{float}, \texttt{float}) ---\textgreater{} \texttt{geometry}
\item
  \textbf{\texttt{curve}} (\texttt{point}, \texttt{point},
  \texttt{point}, \texttt{point}) ---\textgreater{} \texttt{geometry}
\item
  \textbf{\texttt{curve}} (\texttt{point}, \texttt{point},
  \texttt{float}, \texttt{int}, \texttt{float}) ---\textgreater{}
  \texttt{geometry}
\item
  \textbf{\texttt{curve}} (\texttt{point}, \texttt{point},
  \texttt{point}, \texttt{point}, \texttt{int}) ---\textgreater{}
  \texttt{geometry}
\item
  \textbf{\texttt{curve}} (\texttt{point}, \texttt{point},
  \texttt{float}, \texttt{bool}, \texttt{int}) ---\textgreater{}
  \texttt{geometry}
\item
  \textbf{\texttt{curve}} (\texttt{point}, \texttt{point},
  \texttt{float}, \texttt{bool}, \texttt{int}, \texttt{float})
  ---\textgreater{} \texttt{geometry}
\item
  \textbf{\texttt{curve}} (\texttt{point}, \texttt{point},
  \texttt{float}, \texttt{int}, \texttt{float}, \texttt{float})
  ---\textgreater{} \texttt{geometry}
\end{itemize}

\subsubsection{Result:}\label{result-118}

A cubic Bezier curve geometry built from the two given points with the
given coefficient for the radius and composed of the given number of
points, considering the given rotation angle (90 = along the z axis). A
cubic Bezier curve geometry built from the four given points composed of
a given number of points. A cubic Bezier curve geometry built from the
two given points with the given coefficient for the radius and composed
of 10 points. A quadratic Bezier curve geometry built from the three
given points composed of a given numnber of points. A cubic Bezier curve
geometry built from the two given points with the given coefficient for
the radius and composed of 10 points - the last boolean is used to
specified if it is the right side. A cubic Bezier curve geometry built
from the two given points with the given coefficient for the radius
considering the given rotation angle (90 = along the z axis). A
quadratic Bezier curve geometry built from the three given points
composed of 10 points. A cubic Bezier curve geometry built from the four
given points composed of 10 points. A cubic Bezier curve geometry built
from the two given points with the given coefficient for the radius and
composed of the given number of points - the boolean is used to
specified if it is the right side and the last value to indicate where
is the inflection point (between 0.0 and 1.0 - default 0.5). A cubic
Bezier curve geometry built from the two given points with the given
coefficient for the radius and composed of the given number of points,
considering the given inflection point (between 0.0 and 1.0 - default
0.5), and the given rotation angle (90 = along the z axis). A cubic
Bezier curve geometry built from the two given points with the given
coefficient for the radius and composed of the given number of points -
the boolean is used to specified if it is the right side.

\subsubsection{Special cases:}\label{special-cases-47}

\begin{itemize}
\tightlist
\item
  if the operand is nil, returns nil\\
\item
  if the operand is nil, returns nil\\
\item
  if the last operand (number of points) is inferior to 2, returns nil\\
\item
  if the operand is nil, returns nil\\
\item
  if the operand is nil, returns nil\\
\item
  if the last operand (number of points) is inferior to 2, returns nil\\
\item
  if the operand is nil, returns nil\\
\item
  if the operand is nil, returns nil\\
\item
  if the operand is nil, returns nil\\
\item
  if the operand is nil, returns nil\\
\item
  if the operand is nil, returns nil\\
\item
  if the operand is nil, returns nil\\
\item
  if the operand is nil, returns nil
\end{itemize}

\subsubsection{Examples:}\label{examples-92}

\begin{verbatim}
 
geometry var0 <- curve({0,0},{10,10}, 0.5, 100, 90); // var0 equals a cubic Bezier curve geometry composed of 100 points from p0 to p1 at the right side. 
geometry var1 <- curve({0,0}, {0,10}, {10,10}); // var1 equals a cubic Bezier curve geometry composed of 10 points from p0 to p3. 
geometry var2 <- curve({0,0},{10,10}, 0.5); // var2 equals a cubic Bezier curve geometry composed of 10 points from p0 to p1. 
geometry var3 <- curve({0,0}, {0,10}, {10,10}, 20); // var3 equals a quadratic Bezier curve geometry composed of 20 points from p0 to p2. 
geometry var4 <- curve({0,0},{10,10}, 0.5, false); // var4 equals a cubic Bezier curve geometry composed of 10 points from p0 to p1 at the left side. 
geometry var5 <- curve({0,0},{10,10}, 0.5, 90); // var5 equals a cubic Bezier curve geometry composed of 100 points from p0 to p1 at the right side. 
geometry var6 <- curve({0,0}, {0,10}, {10,10}); // var6 equals a quadratic Bezier curve geometry composed of 10 points from p0 to p2. 
geometry var7 <- curve({0,0}, {0,10}, {10,10}); // var7 equals a cubic Bezier curve geometry composed of 10 points from p0 to p3. 
geometry var8 <- curve({0,0},{10,10}, 0.5, false, 100, 0.8); // var8 equals a cubic Bezier curve geometry composed of 100 points from p0 to p1 at the right side. 
geometry var9 <- curve({0,0},{10,10}, 0.5, 100, 0.8, 90); // var9 equals a cubic Bezier curve geometry composed of 100 points from p0 to p1 at the right side. 
geometry var10 <- curve({0,0},{10,10}, 0.5, false, 100); // var10 equals a cubic Bezier curve geometry composed of 100 points from p0 to p1 at the right side.
\end{verbatim}

\subsubsection{See also:}\label{see-also-74}

\href{OperatorsAA\#around}{around}, \href{OperatorsBC\#circle}{circle},
\href{OperatorsBC\#cone}{cone}, \href{OperatorsIM\#link}{link},
\href{OperatorsNR\#norm}{norm}, \href{OperatorsNR\#point}{point},
\href{OperatorsSZ\#polygone}{polygone},
\href{OperatorsNR\#rectangle}{rectangle},
\href{OperatorsSZ\#square}{square},
\href{OperatorsSZ\#triangle}{triangle}, \href{OperatorsIM\#line}{line},

\begin{center}\rule{0.5\linewidth}{\linethickness}\end{center}

\subsection{\texorpdfstring{\texttt{cylinder}}{cylinder}}\label{cylinder}

\subsubsection{Possible use:}\label{possible-use-123}

\begin{itemize}
\tightlist
\item
  \texttt{float} \textbf{\texttt{cylinder}} \texttt{float}
  ---\textgreater{} \texttt{geometry}
\item
  \textbf{\texttt{cylinder}} (\texttt{float} , \texttt{float})
  ---\textgreater{} \texttt{geometry}
\end{itemize}

\subsubsection{Result:}\label{result-119}

A cylinder geometry which radius is equal to the operand.

\subsubsection{Comment:}\label{comment-29}

the center of the cylinder is by default the location of the current
agent in which has been called this operator.

\subsubsection{Special cases:}\label{special-cases-48}

\begin{itemize}
\tightlist
\item
  returns a point if the operand is lower or equal to 0.
\end{itemize}

\subsubsection{Examples:}\label{examples-93}

\begin{verbatim}
 
geometry var0 <- cylinder(10,10); // var0 equals a geometry as a circle of radius 10.
\end{verbatim}

\subsubsection{See also:}\label{see-also-75}

\href{OperatorsAA\#around}{around}, \href{OperatorsBC\#cone}{cone},
\href{OperatorsIM\#line}{line}, \href{OperatorsIM\#link}{link},
\href{OperatorsNR\#norm}{norm}, \href{OperatorsNR\#point}{point},
\href{OperatorsNR\#polygon}{polygon},
\href{OperatorsNR\#polyline}{polyline},
\href{OperatorsNR\#rectangle}{rectangle},
\href{OperatorsSZ\#square}{square},
\href{OperatorsSZ\#triangle}{triangle},

\chapter{Operators (D to H)}\label{operators-d-to-h}

\begin{center}\rule{0.5\linewidth}{\linethickness}\end{center}

\textbf{This file is automatically generated from java files. Do Not
Edit It.}

\begin{center}\rule{0.5\linewidth}{\linethickness}\end{center}

\section{Definition}\label{definition-2}

Operators in the GAML language are used to compose complex expressions.
An operator performs a function on one, two, or n operands (which are
other expressions and thus may be themselves composed of operators) and
returns the result of this function.

Most of them use a classical prefixed functional syntax (i.e.
\texttt{operator\_name(operand1,\ operand2,\ operand3)}, see below),
with the exception of arithmetic (e.g. \texttt{+}, \texttt{/}), logical
(\texttt{and}, \texttt{or}), comparison (e.g. \texttt{\textgreater{}},
\texttt{\textless{}}), access (\texttt{.}, \texttt{{[}..{]}}) and pair
(\texttt{::}) operators, which require an infixed notation (i.e.
\texttt{operand1\ operator\_symbol\ operand1}).

The ternary functional if-else operator, \texttt{?\ :}, uses a special
infixed syntax composed with two symbols (e.g.
\texttt{operand1\ ?\ operand2\ :\ operand3}). Two unary operators
(\texttt{-} and \texttt{!}) use a traditional prefixed syntax that does
not require parentheses unless the operand is itself a complex
expression (e.g. \texttt{-\ 10}, \texttt{!\ (operand1\ or\ operand2)}).

Finally, special constructor operators (\texttt{\{...\}} for
constructing points, \texttt{{[}...{]}} for constructing lists and maps)
will require their operands to be placed between their two symbols (e.g.
\texttt{\{1,2,3\}}, \texttt{{[}operand1,\ operand2,\ ...,\ operandn{]}}
or \texttt{{[}key1::value1,\ key2::value2...\ keyn::valuen{]}}).

With the exception of these special cases above, the following rules
apply to the syntax of operators: * if they only have one operand, the
functional prefixed syntax is mandatory (e.g.
\texttt{operator\_name(operand1)}) * if they have two arguments, either
the functional prefixed syntax (e.g.
\texttt{operator\_name(operand1,\ operand2)}) or the infixed syntax
(e.g. \texttt{operand1\ operator\_name\ operand2}) can be used. * if
they have more than two arguments, either the functional prefixed syntax
(e.g. \texttt{operator\_name(operand1,\ operand2,\ ...,\ operand)}) or a
special infixed syntax with the first operand on the left-hand side of
the operator name (e.g.
\texttt{operand1\ operator\_name(operand2,\ ...,\ operand)}) can be
used.

All of these alternative syntaxes are completely equivalent.

Operators in GAML are purely functional, i.e.~they are guaranteed to not
have any side effects on their operands. For instance, the
\texttt{shuffle} operator, which randomizes the positions of elements in
a list, does not modify its list operand but returns a new shuffled
list.

\section{\texorpdfstring{}{ }}\label{section-20}

\section{Priority between operators}\label{priority-between-operators-2}

The priority of operators determines, in the case of complex expressions
composed of several operators, which one(s) will be evaluated first.

GAML follows in general the traditional priorities attributed to
arithmetic, boolean, comparison operators, with some twists. Namely: *
the constructor operators, like \texttt{::}, used to compose pairs of
operands, have the lowest priority of all operators (e.g.
\texttt{a\ \textgreater{}\ b\ ::\ b\ \textgreater{}\ c} will return a
pair of boolean values, which means that the two comparisons are
evaluated before the operator applies. Similarly,
\texttt{{[}a\ \textgreater{}\ 10,\ b\ \textgreater{}\ 5{]}} will return
a list of boolean values. * it is followed by the \texttt{?:} operator,
the functional if-else (e.g.
\texttt{a\ \textgreater{}\ b\ ?\ a\ +\ 10\ :\ a\ -\ 10} will return the
result of the if-else). * next are the logical operators, \texttt{and}
and \texttt{or} (e.g.
\texttt{a\ \textgreater{}\ b\ or\ b\ \textgreater{}\ c} will return the
value of the test) * next are the comparison operators (i.e.
\texttt{\textgreater{}}, \texttt{\textless{}}, \texttt{\textless{}=},
\texttt{\textgreater{}=}, \texttt{=}, \texttt{!=}) * next the arithmetic
operators in their logical order (multiplicative operators have a higher
priority than additive operators) * next the unary operators \texttt{-}
and \texttt{!} * next the access operators \texttt{.} and
\texttt{{[}{]}} (e.g.
\texttt{\{1,2,3\}.x\ \textgreater{}\ 20\ +\ \{4,5,6\}.y} will return the
result of the comparison between the x and y ordinates of the two
points) * and finally the functional operators, which have the highest
priority of all.

\begin{center}\rule{0.5\linewidth}{\linethickness}\end{center}

\section{Using actions as operators}\label{using-actions-as-operators-2}

Actions defined in species can be used as operators, provided they are
called on the correct agent. The syntax is that of normal functional
operators, but the agent that will perform the action must be added as
the first operand.

For instance, if the following species is defined:

\begin{verbatim}
species spec1 {
        int min(int x, int y) {
                return x > y ? x : y;
        }
}
\end{verbatim}

Any agent instance of spec1 can use \texttt{min} as an operator (if the
action conflicts with an existing operator, a warning will be emitted).
For instance, in the same model, the following line is perfectly
acceptable:

\begin{verbatim}
global {
        init {
                create spec1;
                spec1 my_agent <- spec1[0];
                int the_min <- my_agent min(10,20); // or min(my_agent, 10, 20);
        }
}
\end{verbatim}

If the action doesn't have any operands, the syntax to use is
\texttt{my\_agent\ the\_action()}. Finally, if it does not return a
value, it might still be used but is considering as returning a value of
type \texttt{unknown} (e.g.
\texttt{unknown\ result\ \textless{}-\ my\_agent\ the\_action(op1,\ op2);}).

Note that due to the fact that actions are written by modelers, the
general functional contract is not respected in that case: actions might
perfectly have side effects on their operands (including the agent).

\begin{center}\rule{0.5\linewidth}{\linethickness}\end{center}

\section{Table of Contents}\label{table-of-contents-2}

\begin{center}\rule{0.5\linewidth}{\linethickness}\end{center}

\section{Operators by categories}\label{operators-by-categories-2}

\begin{center}\rule{0.5\linewidth}{\linethickness}\end{center}

\subsection{3D}\label{d-2}

\href{OperatorsBC\#box}{box}, \href{OperatorsBC\#cone3d}{cone3D},
\href{OperatorsBC\#cube}{cube}, \href{OperatorsBC\#cylinder}{cylinder},
\href{OperatorsDH\#dem}{dem}, \href{OperatorsDH\#hexagon}{hexagon},
\href{OperatorsNR\#pyramid}{pyramid},
\href{OperatorsNR\#rgb_to_xyz}{rgb\_to\_xyz},
\href{OperatorsSZ\#set_z}{set\_z}, \href{OperatorsSZ\#sphere}{sphere},
\href{OperatorsSZ\#teapot}{teapot},

\begin{center}\rule{0.5\linewidth}{\linethickness}\end{center}

\subsection{Arithmetic operators}\label{arithmetic-operators-2}

\href{OperatorsAA\#-}{-}, \href{OperatorsAA\#/}{/},
{[}\textsuperscript{{]}(OperatorsAA\#}), \href{OperatorsAA\#*}{*},
\href{OperatorsAA\#+}{+}, \href{OperatorsAA\#abs}{abs},
\href{OperatorsAA\#acos}{acos}, \href{OperatorsAA\#asin}{asin},
\href{OperatorsAA\#atan}{atan}, \href{OperatorsAA\#atan2}{atan2},
\href{OperatorsBC\#ceil}{ceil}, \href{OperatorsBC\#cos}{cos},
\href{OperatorsBC\#cos_rad}{cos\_rad}, \href{OperatorsDH\#div}{div},
\href{OperatorsDH\#even}{even}, \href{OperatorsDH\#exp}{exp},
\href{OperatorsDH\#fact}{fact}, \href{OperatorsDH\#floor}{floor},
\href{OperatorsDH\#hypot}{hypot},
\href{OperatorsIM\#is_finite}{is\_finite},
\href{OperatorsIM\#is_number}{is\_number}, \href{OperatorsIM\#ln}{ln},
\href{OperatorsIM\#log}{log}, \href{OperatorsIM\#mod}{mod},
\href{OperatorsNR\#round}{round}, \href{OperatorsSZ\#signum}{signum},
\href{OperatorsSZ\#sin}{sin}, \href{OperatorsSZ\#sin_rad}{sin\_rad},
\href{OperatorsSZ\#sqrt}{sqrt}, \href{OperatorsSZ\#tan}{tan},
\href{OperatorsSZ\#tan_rad}{tan\_rad}, \href{OperatorsSZ\#tanh}{tanh},
\href{OperatorsSZ\#with_precision}{with\_precision},

\begin{center}\rule{0.5\linewidth}{\linethickness}\end{center}

\subsection{BDI}\label{bdi-2}

\href{OperatorsAA\#and}{and}, \href{OperatorsDH\#eval_when}{eval\_when},
\href{OperatorsDH\#get_about}{get\_about},
\href{OperatorsDH\#get_agent}{get\_agent},
\href{OperatorsDH\#get_agent_cause}{get\_agent\_cause},
\href{OperatorsDH\#get_belief_op}{get\_belief\_op},
\href{OperatorsDH\#get_belief_with_name_op}{get\_belief\_with\_name\_op},
\href{OperatorsDH\#get_beliefs_op}{get\_beliefs\_op},
\href{OperatorsDH\#get_beliefs_with_name_op}{get\_beliefs\_with\_name\_op},
\href{OperatorsDH\#get_current_intention_op}{get\_current\_intention\_op},
\href{OperatorsDH\#get_decay}{get\_decay},
\href{OperatorsDH\#get_desire_op}{get\_desire\_op},
\href{OperatorsDH\#get_desire_with_name_op}{get\_desire\_with\_name\_op},
\href{OperatorsDH\#get_desires_op}{get\_desires\_op},
\href{OperatorsDH\#get_desires_with_name_op}{get\_desires\_with\_name\_op},
\href{OperatorsDH\#get_dominance}{get\_dominance},
\href{OperatorsDH\#get_familiarity}{get\_familiarity},
\href{OperatorsDH\#get_ideal_op}{get\_ideal\_op},
\href{OperatorsDH\#get_ideal_with_name_op}{get\_ideal\_with\_name\_op},
\href{OperatorsDH\#get_ideals_op}{get\_ideals\_op},
\href{OperatorsDH\#get_ideals_with_name_op}{get\_ideals\_with\_name\_op},
\href{OperatorsDH\#get_intensity}{get\_intensity},
\href{OperatorsDH\#get_intention_op}{get\_intention\_op},
\href{OperatorsDH\#get_intention_with_name_op}{get\_intention\_with\_name\_op},
\href{OperatorsDH\#get_intentions_op}{get\_intentions\_op},
\href{OperatorsDH\#get_intentions_with_name_op}{get\_intentions\_with\_name\_op},
\href{OperatorsDH\#get_lifetime}{get\_lifetime},
\href{OperatorsDH\#get_liking}{get\_liking},
\href{OperatorsDH\#get_modality}{get\_modality},
\href{OperatorsDH\#get_obligation_op}{get\_obligation\_op},
\href{OperatorsDH\#get_obligation_with_name_op}{get\_obligation\_with\_name\_op},
\href{OperatorsDH\#get_obligations_op}{get\_obligations\_op},
\href{OperatorsDH\#get_obligations_with_name_op}{get\_obligations\_with\_name\_op},
\href{OperatorsDH\#get_plan_name}{get\_plan\_name},
\href{OperatorsDH\#get_predicate}{get\_predicate},
\href{OperatorsDH\#get_solidarity}{get\_solidarity},
\href{OperatorsDH\#get_strength}{get\_strength},
\href{OperatorsDH\#get_super_intention}{get\_super\_intention},
\href{OperatorsDH\#get_trust}{get\_trust},
\href{OperatorsDH\#get_truth}{get\_truth},
\href{OperatorsDH\#get_uncertainties_op}{get\_uncertainties\_op},
\href{OperatorsDH\#get_uncertainties_with_name_op}{get\_uncertainties\_with\_name\_op},
\href{OperatorsDH\#get_uncertainty_op}{get\_uncertainty\_op},
\href{OperatorsDH\#get_uncertainty_with_name_op}{get\_uncertainty\_with\_name\_op},
\href{OperatorsDH\#has_belief_op}{has\_belief\_op},
\href{OperatorsDH\#has_belief_with_name_op}{has\_belief\_with\_name\_op},
\href{OperatorsDH\#has_desire_op}{has\_desire\_op},
\href{OperatorsDH\#has_desire_with_name_op}{has\_desire\_with\_name\_op},
\href{OperatorsDH\#has_ideal_op}{has\_ideal\_op},
\href{OperatorsDH\#has_ideal_with_name_op}{has\_ideal\_with\_name\_op},
\href{OperatorsDH\#has_intention_op}{has\_intention\_op},
\href{OperatorsDH\#has_intention_with_name_op}{has\_intention\_with\_name\_op},
\href{OperatorsDH\#has_obligation_op}{has\_obligation\_op},
\href{OperatorsDH\#has_obligation_with_name_op}{has\_obligation\_with\_name\_op},
\href{OperatorsDH\#has_uncertainty_op}{has\_uncertainty\_op},
\href{OperatorsDH\#has_uncertainty_with_name_op}{has\_uncertainty\_with\_name\_op},
\href{OperatorsNR\#new_emotion}{new\_emotion},
\href{OperatorsNR\#new_mental_state}{new\_mental\_state},
\href{OperatorsNR\#new_predicate}{new\_predicate},
\href{OperatorsNR\#new_social_link}{new\_social\_link},
\href{OperatorsNR\#or}{or}, \href{OperatorsSZ\#set_about}{set\_about},
\href{OperatorsSZ\#set_agent}{set\_agent},
\href{OperatorsSZ\#set_agent_cause}{set\_agent\_cause},
\href{OperatorsSZ\#set_decay}{set\_decay},
\href{OperatorsSZ\#set_dominance}{set\_dominance},
\href{OperatorsSZ\#set_familiarity}{set\_familiarity},
\href{OperatorsSZ\#set_intensity}{set\_intensity},
\href{OperatorsSZ\#set_lifetime}{set\_lifetime},
\href{OperatorsSZ\#set_liking}{set\_liking},
\href{OperatorsSZ\#set_modality}{set\_modality},
\href{OperatorsSZ\#set_predicate}{set\_predicate},
\href{OperatorsSZ\#set_solidarity}{set\_solidarity},
\href{OperatorsSZ\#set_strength}{set\_strength},
\href{OperatorsSZ\#set_trust}{set\_trust},
\href{OperatorsSZ\#set_truth}{set\_truth},
\href{OperatorsSZ\#with_lifetime}{with\_lifetime},
\href{OperatorsSZ\#with_values}{with\_values},

\begin{center}\rule{0.5\linewidth}{\linethickness}\end{center}

\subsection{Casting operators}\label{casting-operators-2}

\href{OperatorsAA\#as}{as}, \href{OperatorsAA\#as_int}{as\_int},
\href{OperatorsAA\#as_matrix}{as\_matrix},
\href{OperatorsDH\#font}{font}, \href{OperatorsIM\#is}{is},
\href{OperatorsIM\#is_skill}{is\_skill},
\href{OperatorsIM\#list_with}{list\_with},
\href{OperatorsIM\#matrix_with}{matrix\_with},
\href{OperatorsSZ\#species}{species},
\href{OperatorsSZ\#to_gaml}{to\_gaml},
\href{OperatorsSZ\#topology}{topology},

\begin{center}\rule{0.5\linewidth}{\linethickness}\end{center}

\subsection{Color-related operators}\label{color-related-operators-2}

\href{OperatorsAA\#-}{-}, \href{OperatorsAA\#/}{/},
\href{OperatorsAA\#*}{*}, \href{OperatorsAA\#+}{+},
\href{OperatorsBC\#blend}{blend},
\href{OperatorsBC\#brewer_colors}{brewer\_colors},
\href{OperatorsBC\#brewer_palettes}{brewer\_palettes},
\href{OperatorsDH\#grayscale}{grayscale}, \href{OperatorsDH\#hsb}{hsb},
\href{OperatorsIM\#mean}{mean}, \href{OperatorsIM\#median}{median},
\href{OperatorsNR\#rgb}{rgb}, \href{OperatorsNR\#rnd_color}{rnd\_color},
\href{OperatorsSZ\#sum}{sum},

\begin{center}\rule{0.5\linewidth}{\linethickness}\end{center}

\subsection{Comparison operators}\label{comparison-operators-2}

\href{OperatorsAA\#!=}{!=}, \href{OperatorsAA\#\%3C}{\textless{}},
\href{OperatorsAA\#\%3C=}{\textless{}=}, \href{OperatorsAA\#=}{=},
\href{OperatorsAA\#\%3E}{\textgreater{}},
\href{OperatorsAA\#\%3E=}{\textgreater{}=},
\href{OperatorsBC\#between}{between},

\begin{center}\rule{0.5\linewidth}{\linethickness}\end{center}

\subsection{Containers-related
operators}\label{containers-related-operators-2}

\href{OperatorsAA\#-}{-}, \href{OperatorsAA\#::}{::},
\href{OperatorsAA\#+}{+}, \href{OperatorsAA\#accumulate}{accumulate},
\href{OperatorsAA\#among}{among}, \href{OperatorsAA\#at}{at},
\href{OperatorsBC\#collect}{collect},
\href{OperatorsBC\#contains}{contains},
\href{OperatorsBC\#contains_all}{contains\_all},
\href{OperatorsBC\#contains_any}{contains\_any},
\href{OperatorsBC\#count}{count},
\href{OperatorsDH\#distinct}{distinct},
\href{OperatorsDH\#empty}{empty}, \href{OperatorsDH\#every}{every},
\href{OperatorsDH\#first}{first},
\href{OperatorsDH\#first_with}{first\_with},
\href{OperatorsDH\#get}{get}, \href{OperatorsDH\#group_by}{group\_by},
\href{OperatorsIM\#in}{in}, \href{OperatorsIM\#index_by}{index\_by},
\href{OperatorsIM\#inter}{inter},
\href{OperatorsIM\#interleave}{interleave},
\href{OperatorsIM\#internal_at}{internal\_at},
\href{OperatorsIM\#internal_integrated_value}{internal\_integrated\_value},
\href{OperatorsIM\#last}{last},
\href{OperatorsIM\#last_with}{last\_with},
\href{OperatorsIM\#length}{length}, \href{OperatorsIM\#max}{max},
\href{OperatorsIM\#max_of}{max\_of}, \href{OperatorsIM\#mean}{mean},
\href{OperatorsIM\#mean_of}{mean\_of},
\href{OperatorsIM\#median}{median}, \href{OperatorsIM\#min}{min},
\href{OperatorsIM\#min_of}{min\_of}, \href{OperatorsIM\#mul}{mul},
\href{OperatorsNR\#one_of}{one\_of},
\href{OperatorsNR\#product_of}{product\_of},
\href{OperatorsNR\#range}{range}, \href{OperatorsNR\#reverse}{reverse},
\href{OperatorsSZ\#shuffle}{shuffle},
\href{OperatorsSZ\#sort_by}{sort\_by}, \href{OperatorsSZ\#split}{split},
\href{OperatorsSZ\#split_in}{split\_in},
\href{OperatorsSZ\#split_using}{split\_using},
\href{OperatorsSZ\#sum}{sum}, \href{OperatorsSZ\#sum_of}{sum\_of},
\href{OperatorsSZ\#union}{union},
\href{OperatorsSZ\#variance_of}{variance\_of},
\href{OperatorsSZ\#where}{where},
\href{OperatorsSZ\#with_max_of}{with\_max\_of},
\href{OperatorsSZ\#with_min_of}{with\_min\_of},

\begin{center}\rule{0.5\linewidth}{\linethickness}\end{center}

\subsection{Date-related operators}\label{date-related-operators-2}

\href{OperatorsAA\#-}{-}, \href{OperatorsAA\#!=}{!=},
\href{OperatorsAA\#+}{+}, \href{OperatorsAA\#\%3C}{\textless{}},
\href{OperatorsAA\#\%3C=}{\textless{}=}, \href{OperatorsAA\#=}{=},
\href{OperatorsAA\#\%3E}{\textgreater{}},
\href{OperatorsAA\#\%3E=}{\textgreater{}=},
\href{OperatorsAA\#after}{after}, \href{OperatorsBC\#before}{before},
\href{OperatorsBC\#between}{between}, \href{OperatorsDH\#every}{every},
\href{OperatorsIM\#milliseconds_between}{milliseconds\_between},
\href{OperatorsIM\#minus_days}{minus\_days},
\href{OperatorsIM\#minus_hours}{minus\_hours},
\href{OperatorsIM\#minus_minutes}{minus\_minutes},
\href{OperatorsIM\#minus_months}{minus\_months},
\href{OperatorsIM\#minus_ms}{minus\_ms},
\href{OperatorsIM\#minus_weeks}{minus\_weeks},
\href{OperatorsIM\#minus_years}{minus\_years},
\href{OperatorsIM\#months_between}{months\_between},
\href{OperatorsNR\#plus_days}{plus\_days},
\href{OperatorsNR\#plus_hours}{plus\_hours},
\href{OperatorsNR\#plus_minutes}{plus\_minutes},
\href{OperatorsNR\#plus_months}{plus\_months},
\href{OperatorsNR\#plus_ms}{plus\_ms},
\href{OperatorsNR\#plus_weeks}{plus\_weeks},
\href{OperatorsNR\#plus_years}{plus\_years},
\href{OperatorsSZ\#since}{since}, \href{OperatorsSZ\#to}{to},
\href{OperatorsSZ\#until}{until},
\href{OperatorsSZ\#years_between}{years\_between},

\begin{center}\rule{0.5\linewidth}{\linethickness}\end{center}

\subsection{Dates}\label{dates-2}

\begin{center}\rule{0.5\linewidth}{\linethickness}\end{center}

\subsection{DescriptiveStatistics}\label{descriptivestatistics-2}

\href{OperatorsAA\#auto_correlation}{auto\_correlation},
\href{OperatorsBC\#correlation}{correlation},
\href{OperatorsBC\#covariance}{covariance},
\href{OperatorsDH\#durbin_watson}{durbin\_watson},
\href{OperatorsIM\#kurtosis}{kurtosis},
\href{OperatorsIM\#moment}{moment},
\href{OperatorsNR\#quantile}{quantile},
\href{OperatorsNR\#quantile_inverse}{quantile\_inverse},
\href{OperatorsNR\#rank_interpolated}{rank\_interpolated},
\href{OperatorsNR\#rms}{rms}, \href{OperatorsSZ\#skew}{skew},
\href{OperatorsSZ\#variance}{variance},

\begin{center}\rule{0.5\linewidth}{\linethickness}\end{center}

\subsection{Displays}\label{displays-2}

\href{OperatorsDH\#horizontal}{horizontal},
\href{OperatorsSZ\#stack}{stack},
\href{OperatorsSZ\#vertical}{vertical},

\begin{center}\rule{0.5\linewidth}{\linethickness}\end{center}

\subsection{Distributions}\label{distributions-2}

\href{OperatorsBC\#binomial_coeff}{binomial\_coeff},
\href{OperatorsBC\#binomial_complemented}{binomial\_complemented},
\href{OperatorsBC\#binomial_sum}{binomial\_sum},
\href{OperatorsBC\#chi_square}{chi\_square},
\href{OperatorsBC\#chi_square_complemented}{chi\_square\_complemented},
\href{OperatorsDH\#gamma_distribution}{gamma\_distribution},
\href{OperatorsDH\#gamma_distribution_complemented}{gamma\_distribution\_complemented},
\href{OperatorsNR\#normal_area}{normal\_area},
\href{OperatorsNR\#normal_density}{normal\_density},
\href{OperatorsNR\#normal_inverse}{normal\_inverse},
\href{OperatorsNR\#pvalue_for_fstat}{pValue\_for\_fStat},
\href{OperatorsNR\#pvalue_for_tstat}{pValue\_for\_tStat},
\href{OperatorsSZ\#student_area}{student\_area},
\href{OperatorsSZ\#student_t_inverse}{student\_t\_inverse},

\begin{center}\rule{0.5\linewidth}{\linethickness}\end{center}

\subsection{Driving operators}\label{driving-operators-2}

\href{OperatorsAA\#as_driving_graph}{as\_driving\_graph},

\begin{center}\rule{0.5\linewidth}{\linethickness}\end{center}

\subsection{edge}\label{edge-2}

\href{OperatorsDH\#edge_between}{edge\_between},
\href{OperatorsSZ\#strahler}{strahler},

\begin{center}\rule{0.5\linewidth}{\linethickness}\end{center}

\subsection{EDP-related operators}\label{edp-related-operators-2}

\href{OperatorsDH\#diff}{diff}, \href{OperatorsDH\#diff2}{diff2},
\href{OperatorsIM\#internal_zero_order_equation}{internal\_zero\_order\_equation},

\begin{center}\rule{0.5\linewidth}{\linethickness}\end{center}

\subsection{Files-related operators}\label{files-related-operators-2}

\href{OperatorsBC\#crs}{crs},
\href{OperatorsDH\#evaluate_sub_model}{evaluate\_sub\_model},
\href{OperatorsDH\#file}{file},
\href{OperatorsDH\#file_exists}{file\_exists},
\href{OperatorsDH\#folder}{folder}, \href{OperatorsDH\#get}{get},
\href{OperatorsIM\#load_sub_model}{load\_sub\_model},
\href{OperatorsNR\#new_folder}{new\_folder},
\href{OperatorsNR\#osm_file}{osm\_file}, \href{OperatorsNR\#read}{read},
\href{OperatorsSZ\#step_sub_model}{step\_sub\_model},
\href{OperatorsSZ\#writable}{writable},

\begin{center}\rule{0.5\linewidth}{\linethickness}\end{center}

\subsection{FIPA-related operators}\label{fipa-related-operators-2}

\href{OperatorsBC\#conversation}{conversation},
\href{OperatorsIM\#message}{message},

\begin{center}\rule{0.5\linewidth}{\linethickness}\end{center}

\subsection{GamaMetaType}\label{gamametatype-2}

\href{OperatorsSZ\#type_of}{type\_of},

\begin{center}\rule{0.5\linewidth}{\linethickness}\end{center}

\subsection{GammaFunction}\label{gammafunction-2}

\href{OperatorsBC\#beta}{beta}, \href{OperatorsDH\#gamma}{gamma},
\href{OperatorsIM\#incomplete_beta}{incomplete\_beta},
\href{OperatorsIM\#incomplete_gamma}{incomplete\_gamma},
\href{OperatorsIM\#incomplete_gamma_complement}{incomplete\_gamma\_complement},
\href{OperatorsIM\#log_gamma}{log\_gamma},

\begin{center}\rule{0.5\linewidth}{\linethickness}\end{center}

\subsection{Graphs-related operators}\label{graphs-related-operators-2}

\href{OperatorsAA\#add_edge}{add\_edge},
\href{OperatorsAA\#add_node}{add\_node},
\href{OperatorsAA\#adjacency}{adjacency},
\href{OperatorsAA\#agent_from_geometry}{agent\_from\_geometry},
\href{OperatorsAA\#all_pairs_shortest_path}{all\_pairs\_shortest\_path},
\href{OperatorsAA\#alpha_index}{alpha\_index},
\href{OperatorsAA\#as_distance_graph}{as\_distance\_graph},
\href{OperatorsAA\#as_edge_graph}{as\_edge\_graph},
\href{OperatorsAA\#as_intersection_graph}{as\_intersection\_graph},
\href{OperatorsAA\#as_path}{as\_path},
\href{OperatorsBC\#beta_index}{beta\_index},
\href{OperatorsBC\#betweenness_centrality}{betweenness\_centrality},
\href{OperatorsBC\#biggest_cliques_of}{biggest\_cliques\_of},
\href{OperatorsBC\#connected_components_of}{connected\_components\_of},
\href{OperatorsBC\#connectivity_index}{connectivity\_index},
\href{OperatorsBC\#contains_edge}{contains\_edge},
\href{OperatorsBC\#contains_vertex}{contains\_vertex},
\href{OperatorsDH\#degree_of}{degree\_of},
\href{OperatorsDH\#directed}{directed}, \href{OperatorsDH\#edge}{edge},
\href{OperatorsDH\#edge_between}{edge\_between},
\href{OperatorsDH\#edge_betweenness}{edge\_betweenness},
\href{OperatorsDH\#edges}{edges},
\href{OperatorsDH\#gamma_index}{gamma\_index},
\href{OperatorsDH\#generate_barabasi_albert}{generate\_barabasi\_albert},
\href{OperatorsDH\#generate_complete_graph}{generate\_complete\_graph},
\href{OperatorsDH\#generate_watts_strogatz}{generate\_watts\_strogatz},
\href{OperatorsDH\#grid_cells_to_graph}{grid\_cells\_to\_graph},
\href{OperatorsIM\#in_degree_of}{in\_degree\_of},
\href{OperatorsIM\#in_edges_of}{in\_edges\_of},
\href{OperatorsIM\#layout}{layout},
\href{OperatorsIM\#load_graph_from_file}{load\_graph\_from\_file},
\href{OperatorsIM\#load_shortest_paths}{load\_shortest\_paths},
\href{OperatorsIM\#main_connected_component}{main\_connected\_component},
\href{OperatorsIM\#max_flow_between}{max\_flow\_between},
\href{OperatorsIM\#maximal_cliques_of}{maximal\_cliques\_of},
\href{OperatorsNR\#nb_cycles}{nb\_cycles},
\href{OperatorsNR\#neighbors_of}{neighbors\_of},
\href{OperatorsNR\#node}{node}, \href{OperatorsNR\#nodes}{nodes},
\href{OperatorsNR\#out_degree_of}{out\_degree\_of},
\href{OperatorsNR\#out_edges_of}{out\_edges\_of},
\href{OperatorsNR\#path_between}{path\_between},
\href{OperatorsNR\#paths_between}{paths\_between},
\href{OperatorsNR\#predecessors_of}{predecessors\_of},
\href{OperatorsNR\#remove_node_from}{remove\_node\_from},
\href{OperatorsNR\#rewire_n}{rewire\_n},
\href{OperatorsSZ\#source_of}{source\_of},
\href{OperatorsSZ\#spatial_graph}{spatial\_graph},
\href{OperatorsSZ\#strahler}{strahler},
\href{OperatorsSZ\#successors_of}{successors\_of},
\href{OperatorsSZ\#sum}{sum}, \href{OperatorsSZ\#target_of}{target\_of},
\href{OperatorsSZ\#undirected}{undirected},
\href{OperatorsSZ\#use_cache}{use\_cache},
\href{OperatorsSZ\#weight_of}{weight\_of},
\href{OperatorsSZ\#with_optimizer_type}{with\_optimizer\_type},
\href{OperatorsSZ\#with_weights}{with\_weights},

\begin{center}\rule{0.5\linewidth}{\linethickness}\end{center}

\subsection{Grid-related operators}\label{grid-related-operators-2}

\href{OperatorsAA\#as_4_grid}{as\_4\_grid},
\href{OperatorsAA\#as_grid}{as\_grid},
\href{OperatorsAA\#as_hexagonal_grid}{as\_hexagonal\_grid},
\href{OperatorsDH\#grid_at}{grid\_at},
\href{OperatorsNR\#path_between}{path\_between},

\begin{center}\rule{0.5\linewidth}{\linethickness}\end{center}

\subsection{Iterator operators}\label{iterator-operators-2}

\href{OperatorsAA\#accumulate}{accumulate},
\href{OperatorsAA\#as_map}{as\_map},
\href{OperatorsBC\#collect}{collect}, \href{OperatorsBC\#count}{count},
\href{OperatorsBC\#create_map}{create\_map},
\href{OperatorsDH\#distribution_of}{distribution\_of},
\href{OperatorsDH\#distribution_of}{distribution\_of},
\href{OperatorsDH\#distribution_of}{distribution\_of},
\href{OperatorsDH\#distribution2d_of}{distribution2d\_of},
\href{OperatorsDH\#distribution2d_of}{distribution2d\_of},
\href{OperatorsDH\#distribution2d_of}{distribution2d\_of},
\href{OperatorsDH\#first_with}{first\_with},
\href{OperatorsDH\#frequency_of}{frequency\_of},
\href{OperatorsDH\#group_by}{group\_by},
\href{OperatorsIM\#index_by}{index\_by},
\href{OperatorsIM\#last_with}{last\_with},
\href{OperatorsIM\#max_of}{max\_of},
\href{OperatorsIM\#mean_of}{mean\_of},
\href{OperatorsIM\#min_of}{min\_of},
\href{OperatorsNR\#product_of}{product\_of},
\href{OperatorsSZ\#sort_by}{sort\_by},
\href{OperatorsSZ\#sum_of}{sum\_of},
\href{OperatorsSZ\#variance_of}{variance\_of},
\href{OperatorsSZ\#where}{where},
\href{OperatorsSZ\#with_max_of}{with\_max\_of},
\href{OperatorsSZ\#with_min_of}{with\_min\_of},

\begin{center}\rule{0.5\linewidth}{\linethickness}\end{center}

\subsection{List-related operators}\label{list-related-operators-2}

\href{OperatorsBC\#copy_between}{copy\_between},
\href{OperatorsIM\#index_of}{index\_of},
\href{OperatorsIM\#last_index_of}{last\_index\_of},

\begin{center}\rule{0.5\linewidth}{\linethickness}\end{center}

\subsection{Logical operators}\label{logical-operators-2}

\href{OperatorsAA\#:}{:}, \href{OperatorsAA\#!}{!},
\href{OperatorsAA\#?}{?}, \href{OperatorsAA\#and}{and},
\href{OperatorsNR\#or}{or}, \href{OperatorsSZ\#xor}{xor},

\begin{center}\rule{0.5\linewidth}{\linethickness}\end{center}

\subsection{Map comparaison
operators}\label{map-comparaison-operators-2}

\href{OperatorsDH\#fuzzy_kappa}{fuzzy\_kappa},
\href{OperatorsDH\#fuzzy_kappa_sim}{fuzzy\_kappa\_sim},
\href{OperatorsIM\#kappa}{kappa},
\href{OperatorsIM\#kappa_sim}{kappa\_sim},
\href{OperatorsNR\#percent_absolute_deviation}{percent\_absolute\_deviation},

\begin{center}\rule{0.5\linewidth}{\linethickness}\end{center}

\subsection{Map-related operators}\label{map-related-operators-2}

\href{OperatorsAA\#as_map}{as\_map},
\href{OperatorsBC\#create_map}{create\_map},
\href{OperatorsIM\#index_of}{index\_of},
\href{OperatorsIM\#last_index_of}{last\_index\_of},

\begin{center}\rule{0.5\linewidth}{\linethickness}\end{center}

\subsection{Material}\label{material-2}

\href{OperatorsIM\#material}{material},

\begin{center}\rule{0.5\linewidth}{\linethickness}\end{center}

\subsection{Matrix-related operators}\label{matrix-related-operators-2}

\href{OperatorsAA\#-}{-}, \href{OperatorsAA\#/}{/},
\href{OperatorsAA\#.}{.}, \href{OperatorsAA\#*}{*},
\href{OperatorsAA\#+}{+},
\href{OperatorsAA\#append_horizontally}{append\_horizontally},
\href{OperatorsAA\#append_vertically}{append\_vertically},
\href{OperatorsBC\#column_at}{column\_at},
\href{OperatorsBC\#columns_list}{columns\_list},
\href{OperatorsDH\#determinant}{determinant},
\href{OperatorsDH\#eigenvalues}{eigenvalues},
\href{OperatorsIM\#index_of}{index\_of},
\href{OperatorsIM\#inverse}{inverse},
\href{OperatorsIM\#last_index_of}{last\_index\_of},
\href{OperatorsNR\#row_at}{row\_at},
\href{OperatorsNR\#rows_list}{rows\_list},
\href{OperatorsSZ\#shuffle}{shuffle}, \href{OperatorsSZ\#trace}{trace},
\href{OperatorsSZ\#transpose}{transpose},

\begin{center}\rule{0.5\linewidth}{\linethickness}\end{center}

\subsection{multicriteria operators}\label{multicriteria-operators-2}

\href{OperatorsDH\#electre_dm}{electre\_DM},
\href{OperatorsDH\#evidence_theory_dm}{evidence\_theory\_DM},
\href{OperatorsDH\#fuzzy_choquet_dm}{fuzzy\_choquet\_DM},
\href{OperatorsNR\#promethee_dm}{promethee\_DM},
\href{OperatorsSZ\#weighted_means_dm}{weighted\_means\_DM},

\begin{center}\rule{0.5\linewidth}{\linethickness}\end{center}

\subsection{Path-related operators}\label{path-related-operators-2}

\href{OperatorsAA\#agent_from_geometry}{agent\_from\_geometry},
\href{OperatorsAA\#all_pairs_shortest_path}{all\_pairs\_shortest\_path},
\href{OperatorsAA\#as_path}{as\_path},
\href{OperatorsIM\#load_shortest_paths}{load\_shortest\_paths},
\href{OperatorsIM\#max_flow_between}{max\_flow\_between},
\href{OperatorsNR\#path_between}{path\_between},
\href{OperatorsNR\#path_to}{path\_to},
\href{OperatorsNR\#paths_between}{paths\_between},
\href{OperatorsSZ\#use_cache}{use\_cache},

\begin{center}\rule{0.5\linewidth}{\linethickness}\end{center}

\subsection{Points-related operators}\label{points-related-operators-2}

\href{OperatorsAA\#-}{-}, \href{OperatorsAA\#/}{/},
\href{OperatorsAA\#*}{*}, \href{OperatorsAA\#+}{+},
\href{OperatorsAA\#\%3C}{\textless{}},
\href{OperatorsAA\#\%3C=}{\textless{}=},
\href{OperatorsAA\#\%3E}{\textgreater{}},
\href{OperatorsAA\#\%3E=}{\textgreater{}=},
\href{OperatorsAA\#add_point}{add\_point},
\href{OperatorsAA\#angle_between}{angle\_between},
\href{OperatorsAA\#any_location_in}{any\_location\_in},
\href{OperatorsBC\#centroid}{centroid},
\href{OperatorsBC\#closest_points_with}{closest\_points\_with},
\href{OperatorsDH\#farthest_point_to}{farthest\_point\_to},
\href{OperatorsDH\#grid_at}{grid\_at}, \href{OperatorsNR\#norm}{norm},
\href{OperatorsNR\#point}{point},
\href{OperatorsNR\#points_along}{points\_along},
\href{OperatorsNR\#points_at}{points\_at},
\href{OperatorsNR\#points_on}{points\_on},

\begin{center}\rule{0.5\linewidth}{\linethickness}\end{center}

\subsection{Random operators}\label{random-operators-2}

\href{OperatorsBC\#binomial}{binomial}, \href{OperatorsDH\#flip}{flip},
\href{OperatorsDH\#gauss}{gauss},
\href{OperatorsIM\#improved_generator}{improved\_generator},
\href{OperatorsNR\#open_simplex_generator}{open\_simplex\_generator},
\href{OperatorsNR\#poisson}{poisson}, \href{OperatorsNR\#rnd}{rnd},
\href{OperatorsNR\#rnd_choice}{rnd\_choice},
\href{OperatorsSZ\#sample}{sample},
\href{OperatorsSZ\#shuffle}{shuffle},
\href{OperatorsSZ\#simplex_generator}{simplex\_generator},
\href{OperatorsSZ\#skew_gauss}{skew\_gauss},
\href{OperatorsSZ\#truncated_gauss}{truncated\_gauss},

\begin{center}\rule{0.5\linewidth}{\linethickness}\end{center}

\subsection{ReverseOperators}\label{reverseoperators-2}

\href{OperatorsSZ\#savesimulation}{saveSimulation},
\href{OperatorsSZ\#serialize}{serialize},
\href{OperatorsSZ\#serializeagent}{serializeAgent},
\href{OperatorsSZ\#unserializesimulation}{unSerializeSimulation},
\href{OperatorsSZ\#unserializesimulationfromfile}{unSerializeSimulationFromFile},

\begin{center}\rule{0.5\linewidth}{\linethickness}\end{center}

\subsection{Shape}\label{shape-2}

\href{OperatorsAA\#arc}{arc}, \href{OperatorsBC\#box}{box},
\href{OperatorsBC\#circle}{circle}, \href{OperatorsBC\#cone}{cone},
\href{OperatorsBC\#cone3d}{cone3D}, \href{OperatorsBC\#cross}{cross},
\href{OperatorsBC\#cube}{cube}, \href{OperatorsBC\#curve}{curve},
\href{OperatorsBC\#cylinder}{cylinder},
\href{OperatorsDH\#ellipse}{ellipse},
\href{OperatorsDH\#envelope}{envelope},
\href{OperatorsDH\#geometry_collection}{geometry\_collection},
\href{OperatorsDH\#hexagon}{hexagon}, \href{OperatorsIM\#line}{line},
\href{OperatorsIM\#link}{link}, \href{OperatorsNR\#plan}{plan},
\href{OperatorsNR\#polygon}{polygon},
\href{OperatorsNR\#polyhedron}{polyhedron},
\href{OperatorsNR\#pyramid}{pyramid},
\href{OperatorsNR\#rectangle}{rectangle},
\href{OperatorsSZ\#sphere}{sphere}, \href{OperatorsSZ\#square}{square},
\href{OperatorsSZ\#squircle}{squircle},
\href{OperatorsSZ\#teapot}{teapot},
\href{OperatorsSZ\#triangle}{triangle},

\begin{center}\rule{0.5\linewidth}{\linethickness}\end{center}

\subsection{Spatial operators}\label{spatial-operators-2}

\href{OperatorsAA\#-}{-}, \href{OperatorsAA\#*}{*},
\href{OperatorsAA\#+}{+}, \href{OperatorsAA\#add_point}{add\_point},
\href{OperatorsAA\#agent_closest_to}{agent\_closest\_to},
\href{OperatorsAA\#agent_farthest_to}{agent\_farthest\_to},
\href{OperatorsAA\#agents_at_distance}{agents\_at\_distance},
\href{OperatorsAA\#agents_inside}{agents\_inside},
\href{OperatorsAA\#agents_overlapping}{agents\_overlapping},
\href{OperatorsAA\#angle_between}{angle\_between},
\href{OperatorsAA\#any_location_in}{any\_location\_in},
\href{OperatorsAA\#arc}{arc}, \href{OperatorsAA\#around}{around},
\href{OperatorsAA\#as_4_grid}{as\_4\_grid},
\href{OperatorsAA\#as_grid}{as\_grid},
\href{OperatorsAA\#as_hexagonal_grid}{as\_hexagonal\_grid},
\href{OperatorsAA\#at_distance}{at\_distance},
\href{OperatorsAA\#at_location}{at\_location},
\href{OperatorsBC\#box}{box}, \href{OperatorsBC\#centroid}{centroid},
\href{OperatorsBC\#circle}{circle}, \href{OperatorsBC\#clean}{clean},
\href{OperatorsBC\#clean_network}{clean\_network},
\href{OperatorsBC\#closest_points_with}{closest\_points\_with},
\href{OperatorsBC\#closest_to}{closest\_to},
\href{OperatorsBC\#cone}{cone}, \href{OperatorsBC\#cone3d}{cone3D},
\href{OperatorsBC\#convex_hull}{convex\_hull},
\href{OperatorsBC\#covers}{covers}, \href{OperatorsBC\#cross}{cross},
\href{OperatorsBC\#crosses}{crosses}, \href{OperatorsBC\#crs}{crs},
\href{OperatorsBC\#crs_transform}{CRS\_transform},
\href{OperatorsBC\#cube}{cube}, \href{OperatorsBC\#curve}{curve},
\href{OperatorsBC\#cylinder}{cylinder}, \href{OperatorsDH\#dem}{dem},
\href{OperatorsDH\#direction_between}{direction\_between},
\href{OperatorsDH\#disjoint_from}{disjoint\_from},
\href{OperatorsDH\#distance_between}{distance\_between},
\href{OperatorsDH\#distance_to}{distance\_to},
\href{OperatorsDH\#ellipse}{ellipse},
\href{OperatorsDH\#envelope}{envelope},
\href{OperatorsDH\#farthest_point_to}{farthest\_point\_to},
\href{OperatorsDH\#farthest_to}{farthest\_to},
\href{OperatorsDH\#geometry_collection}{geometry\_collection},
\href{OperatorsDH\#gini}{gini}, \href{OperatorsDH\#hexagon}{hexagon},
\href{OperatorsDH\#hierarchical_clustering}{hierarchical\_clustering},
\href{OperatorsIM\#idw}{IDW}, \href{OperatorsIM\#inside}{inside},
\href{OperatorsIM\#inter}{inter},
\href{OperatorsIM\#intersects}{intersects},
\href{OperatorsIM\#line}{line}, \href{OperatorsIM\#link}{link},
\href{OperatorsIM\#masked_by}{masked\_by},
\href{OperatorsIM\#moran}{moran},
\href{OperatorsNR\#neighbors_at}{neighbors\_at},
\href{OperatorsNR\#neighbors_of}{neighbors\_of},
\href{OperatorsNR\#overlapping}{overlapping},
\href{OperatorsNR\#overlaps}{overlaps},
\href{OperatorsNR\#partially_overlaps}{partially\_overlaps},
\href{OperatorsNR\#path_between}{path\_between},
\href{OperatorsNR\#path_to}{path\_to}, \href{OperatorsNR\#plan}{plan},
\href{OperatorsNR\#points_along}{points\_along},
\href{OperatorsNR\#points_at}{points\_at},
\href{OperatorsNR\#points_on}{points\_on},
\href{OperatorsNR\#polygon}{polygon},
\href{OperatorsNR\#polyhedron}{polyhedron},
\href{OperatorsNR\#pyramid}{pyramid},
\href{OperatorsNR\#rectangle}{rectangle},
\href{OperatorsNR\#rgb_to_xyz}{rgb\_to\_xyz},
\href{OperatorsNR\#rotated_by}{rotated\_by},
\href{OperatorsNR\#round}{round},
\href{OperatorsSZ\#scaled_to}{scaled\_to},
\href{OperatorsSZ\#set_z}{set\_z},
\href{OperatorsSZ\#simple_clustering_by_distance}{simple\_clustering\_by\_distance},
\href{OperatorsSZ\#simplification}{simplification},
\href{OperatorsSZ\#skeletonize}{skeletonize},
\href{OperatorsSZ\#smooth}{smooth}, \href{OperatorsSZ\#sphere}{sphere},
\href{OperatorsSZ\#split_at}{split\_at},
\href{OperatorsSZ\#split_geometry}{split\_geometry},
\href{OperatorsSZ\#split_lines}{split\_lines},
\href{OperatorsSZ\#square}{square},
\href{OperatorsSZ\#squircle}{squircle},
\href{OperatorsSZ\#teapot}{teapot},
\href{OperatorsSZ\#to_gama_crs}{to\_GAMA\_CRS},
\href{OperatorsSZ\#to_rectangles}{to\_rectangles},
\href{OperatorsSZ\#to_squares}{to\_squares},
\href{OperatorsSZ\#to_sub_geometries}{to\_sub\_geometries},
\href{OperatorsSZ\#touches}{touches},
\href{OperatorsSZ\#towards}{towards},
\href{OperatorsSZ\#transformed_by}{transformed\_by},
\href{OperatorsSZ\#translated_by}{translated\_by},
\href{OperatorsSZ\#triangle}{triangle},
\href{OperatorsSZ\#triangulate}{triangulate},
\href{OperatorsSZ\#union}{union}, \href{OperatorsSZ\#using}{using},
\href{OperatorsSZ\#voronoi}{voronoi},
\href{OperatorsSZ\#with_precision}{with\_precision},
\href{OperatorsSZ\#without_holes}{without\_holes},

\begin{center}\rule{0.5\linewidth}{\linethickness}\end{center}

\subsection{Spatial properties
operators}\label{spatial-properties-operators-2}

\href{OperatorsBC\#covers}{covers},
\href{OperatorsBC\#crosses}{crosses},
\href{OperatorsIM\#intersects}{intersects},
\href{OperatorsNR\#partially_overlaps}{partially\_overlaps},
\href{OperatorsSZ\#touches}{touches},

\begin{center}\rule{0.5\linewidth}{\linethickness}\end{center}

\subsection{Spatial queries
operators}\label{spatial-queries-operators-2}

\href{OperatorsAA\#agent_closest_to}{agent\_closest\_to},
\href{OperatorsAA\#agent_farthest_to}{agent\_farthest\_to},
\href{OperatorsAA\#agents_at_distance}{agents\_at\_distance},
\href{OperatorsAA\#agents_inside}{agents\_inside},
\href{OperatorsAA\#agents_overlapping}{agents\_overlapping},
\href{OperatorsAA\#at_distance}{at\_distance},
\href{OperatorsBC\#closest_to}{closest\_to},
\href{OperatorsDH\#farthest_to}{farthest\_to},
\href{OperatorsIM\#inside}{inside},
\href{OperatorsNR\#neighbors_at}{neighbors\_at},
\href{OperatorsNR\#neighbors_of}{neighbors\_of},
\href{OperatorsNR\#overlapping}{overlapping},

\begin{center}\rule{0.5\linewidth}{\linethickness}\end{center}

\subsection{Spatial relations
operators}\label{spatial-relations-operators-2}

\href{OperatorsDH\#direction_between}{direction\_between},
\href{OperatorsDH\#distance_between}{distance\_between},
\href{OperatorsDH\#distance_to}{distance\_to},
\href{OperatorsNR\#path_between}{path\_between},
\href{OperatorsNR\#path_to}{path\_to},
\href{OperatorsSZ\#towards}{towards},

\begin{center}\rule{0.5\linewidth}{\linethickness}\end{center}

\subsection{Spatial statistical
operators}\label{spatial-statistical-operators-2}

\href{OperatorsDH\#hierarchical_clustering}{hierarchical\_clustering},
\href{OperatorsSZ\#simple_clustering_by_distance}{simple\_clustering\_by\_distance},

\begin{center}\rule{0.5\linewidth}{\linethickness}\end{center}

\subsection{Spatial transformations
operators}\label{spatial-transformations-operators-2}

\href{OperatorsAA\#-}{-}, \href{OperatorsAA\#*}{*},
\href{OperatorsAA\#+}{+}, \href{OperatorsAA\#as_4_grid}{as\_4\_grid},
\href{OperatorsAA\#as_grid}{as\_grid},
\href{OperatorsAA\#as_hexagonal_grid}{as\_hexagonal\_grid},
\href{OperatorsAA\#at_location}{at\_location},
\href{OperatorsBC\#clean}{clean},
\href{OperatorsBC\#clean_network}{clean\_network},
\href{OperatorsBC\#convex_hull}{convex\_hull},
\href{OperatorsBC\#crs_transform}{CRS\_transform},
\href{OperatorsNR\#rotated_by}{rotated\_by},
\href{OperatorsSZ\#scaled_to}{scaled\_to},
\href{OperatorsSZ\#simplification}{simplification},
\href{OperatorsSZ\#skeletonize}{skeletonize},
\href{OperatorsSZ\#smooth}{smooth},
\href{OperatorsSZ\#split_geometry}{split\_geometry},
\href{OperatorsSZ\#split_lines}{split\_lines},
\href{OperatorsSZ\#to_gama_crs}{to\_GAMA\_CRS},
\href{OperatorsSZ\#to_rectangles}{to\_rectangles},
\href{OperatorsSZ\#to_squares}{to\_squares},
\href{OperatorsSZ\#to_sub_geometries}{to\_sub\_geometries},
\href{OperatorsSZ\#transformed_by}{transformed\_by},
\href{OperatorsSZ\#translated_by}{translated\_by},
\href{OperatorsSZ\#triangulate}{triangulate},
\href{OperatorsSZ\#voronoi}{voronoi},
\href{OperatorsSZ\#with_precision}{with\_precision},
\href{OperatorsSZ\#without_holes}{without\_holes},

\begin{center}\rule{0.5\linewidth}{\linethickness}\end{center}

\subsection{Species-related
operators}\label{species-related-operators-2}

\href{OperatorsIM\#index_of}{index\_of},
\href{OperatorsIM\#last_index_of}{last\_index\_of},
\href{OperatorsNR\#of_generic_species}{of\_generic\_species},
\href{OperatorsNR\#of_species}{of\_species},

\begin{center}\rule{0.5\linewidth}{\linethickness}\end{center}

\subsection{Statistical operators}\label{statistical-operators-2}

\href{OperatorsBC\#build}{build}, \href{OperatorsBC\#corr}{corR},
\href{OperatorsDH\#dbscan}{dbscan},
\href{OperatorsDH\#distribution_of}{distribution\_of},
\href{OperatorsDH\#distribution2d_of}{distribution2d\_of},
\href{OperatorsDH\#dtw}{dtw},
\href{OperatorsDH\#frequency_of}{frequency\_of},
\href{OperatorsDH\#gamma_rnd}{gamma\_rnd},
\href{OperatorsDH\#geometric_mean}{geometric\_mean},
\href{OperatorsDH\#gini}{gini},
\href{OperatorsDH\#harmonic_mean}{harmonic\_mean},
\href{OperatorsDH\#hierarchical_clustering}{hierarchical\_clustering},
\href{OperatorsIM\#kmeans}{kmeans},
\href{OperatorsIM\#kurtosis}{kurtosis}, \href{OperatorsIM\#max}{max},
\href{OperatorsIM\#mean}{mean},
\href{OperatorsIM\#mean_deviation}{mean\_deviation},
\href{OperatorsIM\#meanr}{meanR}, \href{OperatorsIM\#median}{median},
\href{OperatorsIM\#min}{min}, \href{OperatorsIM\#moran}{moran},
\href{OperatorsIM\#mul}{mul}, \href{OperatorsNR\#predict}{predict},
\href{OperatorsSZ\#simple_clustering_by_distance}{simple\_clustering\_by\_distance},
\href{OperatorsSZ\#skewness}{skewness},
\href{OperatorsSZ\#split}{split},
\href{OperatorsSZ\#split_in}{split\_in},
\href{OperatorsSZ\#split_using}{split\_using},
\href{OperatorsSZ\#standard_deviation}{standard\_deviation},
\href{OperatorsSZ\#sum}{sum}, \href{OperatorsSZ\#variance}{variance},

\begin{center}\rule{0.5\linewidth}{\linethickness}\end{center}

\subsection{Strings-related
operators}\label{strings-related-operators-2}

\href{OperatorsAA\#+}{+}, \href{OperatorsAA\#\%3C}{\textless{}},
\href{OperatorsAA\#\%3C=}{\textless{}=},
\href{OperatorsAA\#\%3E}{\textgreater{}},
\href{OperatorsAA\#\%3E=}{\textgreater{}=}, \href{OperatorsAA\#at}{at},
\href{OperatorsBC\#char}{char}, \href{OperatorsBC\#contains}{contains},
\href{OperatorsBC\#contains_all}{contains\_all},
\href{OperatorsBC\#contains_any}{contains\_any},
\href{OperatorsBC\#copy_between}{copy\_between},
\href{OperatorsDH\#date}{date}, \href{OperatorsDH\#empty}{empty},
\href{OperatorsDH\#first}{first}, \href{OperatorsIM\#in}{in},
\href{OperatorsIM\#indented_by}{indented\_by},
\href{OperatorsIM\#index_of}{index\_of},
\href{OperatorsIM\#is_number}{is\_number},
\href{OperatorsIM\#last}{last},
\href{OperatorsIM\#last_index_of}{last\_index\_of},
\href{OperatorsIM\#length}{length},
\href{OperatorsIM\#lower_case}{lower\_case},
\href{OperatorsNR\#replace}{replace},
\href{OperatorsNR\#replace_regex}{replace\_regex},
\href{OperatorsNR\#reverse}{reverse},
\href{OperatorsSZ\#sample}{sample},
\href{OperatorsSZ\#shuffle}{shuffle},
\href{OperatorsSZ\#split_with}{split\_with},
\href{OperatorsSZ\#string}{string},
\href{OperatorsSZ\#upper_case}{upper\_case},

\begin{center}\rule{0.5\linewidth}{\linethickness}\end{center}

\subsection{System}\label{system-2}

\href{OperatorsAA\#.}{.}, \href{OperatorsBC\#command}{command},
\href{OperatorsBC\#copy}{copy}, \href{OperatorsDH\#dead}{dead},
\href{OperatorsDH\#eval_gaml}{eval\_gaml},
\href{OperatorsDH\#every}{every},
\href{OperatorsIM\#is_error}{is\_error},
\href{OperatorsIM\#is_warning}{is\_warning},
\href{OperatorsSZ\#user_input}{user\_input},

\begin{center}\rule{0.5\linewidth}{\linethickness}\end{center}

\subsection{Time-related operators}\label{time-related-operators-2}

\href{OperatorsDH\#date}{date}, \href{OperatorsSZ\#string}{string},

\begin{center}\rule{0.5\linewidth}{\linethickness}\end{center}

\subsection{Types-related operators}\label{types-related-operators-2}

\begin{center}\rule{0.5\linewidth}{\linethickness}\end{center}

\subsection{User control operators}\label{user-control-operators-2}

\href{OperatorsSZ\#user_input}{user\_input},

\begin{center}\rule{0.5\linewidth}{\linethickness}\end{center}

\section{Operators}\label{operators-2}

\begin{center}\rule{0.5\linewidth}{\linethickness}\end{center}

\subsection{\texorpdfstring{\texttt{date}}{date}}\label{date}

\subsubsection{Possible use:}\label{possible-use-124}

\begin{itemize}
\tightlist
\item
  \texttt{string} \textbf{\texttt{date}} \texttt{string}
  ---\textgreater{} \texttt{date}
\item
  \textbf{\texttt{date}} (\texttt{string} , \texttt{string})
  ---\textgreater{} \texttt{date}
\item
  \textbf{\texttt{date}} (\texttt{string}, \texttt{string},
  \texttt{string}) ---\textgreater{} \texttt{date}
\end{itemize}

\subsubsection{Result:}\label{result-120}

converts a string to a date following a custom pattern and a specific
locale (e.g. `fr', `en'\ldots{}). The pattern can use ``\%Y \%M \%N \%D
\%E \%h \%m \%s \%z'' for parsing years, months, name of month, days,
name of days, hours, minutes, seconds and the time-zone. A null or empty
pattern will parse the date using one of the ISO date \& time formats
(similar to date(`\ldots{}') in that case). The pattern can also follow
the pattern definition found here, which gives much more control over
what will be parsed:
\url{https://docs.oracle.com/javase/8/docs/api/java/time/format/DateTimeFormatter.html\#patterns}.
Different patterns are available by default as constant: \#iso\_local,
\#iso\_simple, \#iso\_offset, \#iso\_zoned and \#custom, which can be
changed in the preferences converts a string to a date following a
custom pattern. The pattern can use ``\%Y \%M \%N \%D \%E \%h \%m \%s
\%z'' for outputting years, months, name of month, days, name of days,
hours, minutes, seconds and the time-zone. A null or empty pattern will
parse the date using one of the ISO date \& time formats (similar to
date(`\ldots{}') in that case). The pattern can also follow the pattern
definition found here, which gives much more control over what will be
parsed:
\url{https://docs.oracle.com/javase/8/docs/api/java/time/format/DateTimeFormatter.html\#patterns}.
Different patterns are available by default as constant: \#iso\_local,
\#iso\_simple, \#iso\_offset, \#iso\_zoned and \#custom, which can be
changed in the preferences

\subsubsection{Examples:}\label{examples-94}

\begin{verbatim}
date d <- date("1999-january-30", 'yyyy-MMMM-dd', 'en'); date den <- date("1999-12-30", 'yyyy-MM-dd'); 
\end{verbatim}

\begin{center}\rule{0.5\linewidth}{\linethickness}\end{center}

\subsection{\texorpdfstring{\texttt{dbscan}}{dbscan}}\label{dbscan}

\subsubsection{Possible use:}\label{possible-use-125}

\begin{itemize}
\tightlist
\item
  \textbf{\texttt{dbscan}} (\texttt{list}, \texttt{float}, \texttt{int})
  ---\textgreater{} \texttt{list\textless{}list\textgreater{}}
\end{itemize}

\subsubsection{Result:}\label{result-121}

returns the list of clusters (list of instance indices) computed with
the dbscan (density-based spatial clustering of applications with noise)
algorithm from the first operand data according to the maximum radius of
the neighborhood to be considered (eps) and the minimum number of points
needed for a cluster (minPts). Usage: dbscan(data,eps,minPoints)

\subsubsection{Special cases:}\label{special-cases-49}

\begin{itemize}
\tightlist
\item
  if the lengths of two vectors in the right-hand aren't equal, returns
  0
\end{itemize}

\subsubsection{Examples:}\label{examples-95}

\begin{verbatim}
 
list<list> var0 <- dbscan ([[2,4,5], [3,8,2], [1,1,3], [4,3,4]],10,2); // var0 equals []
\end{verbatim}

\begin{center}\rule{0.5\linewidth}{\linethickness}\end{center}

\subsection{\texorpdfstring{\texttt{dead}}{dead}}\label{dead}

\subsubsection{Possible use:}\label{possible-use-126}

\begin{itemize}
\tightlist
\item
  \textbf{\texttt{dead}} (\texttt{agent}) ---\textgreater{}
  \texttt{bool}
\end{itemize}

\subsubsection{Result:}\label{result-122}

true if the agent is dead (or null), false otherwise.

\subsubsection{Examples:}\label{examples-96}

\begin{verbatim}
 
bool var0 <- dead(agent_A); // var0 equals true or false
\end{verbatim}

\begin{center}\rule{0.5\linewidth}{\linethickness}\end{center}

\subsection{\texorpdfstring{\texttt{degree\_of}}{degree\_of}}\label{degree_of}

\subsubsection{Possible use:}\label{possible-use-127}

\begin{itemize}
\tightlist
\item
  \texttt{graph} \textbf{\texttt{degree\_of}} \texttt{unknown}
  ---\textgreater{} \texttt{int}
\item
  \textbf{\texttt{degree\_of}} (\texttt{graph} , \texttt{unknown})
  ---\textgreater{} \texttt{int}
\end{itemize}

\subsubsection{Result:}\label{result-123}

returns the degree (in+out) of a vertex (right-hand operand) in the
graph given as left-hand operand.

\subsubsection{Examples:}\label{examples-97}

\begin{verbatim}
 
int var1 <- graphFromMap degree_of (node(3)); // var1 equals 3
\end{verbatim}

\subsubsection{See also:}\label{see-also-76}

\href{OperatorsIM\#in_degree_of}{in\_degree\_of},
\href{OperatorsNR\#out_degree_of}{out\_degree\_of},

\begin{center}\rule{0.5\linewidth}{\linethickness}\end{center}

\subsection{\texorpdfstring{\texttt{dem}}{dem}}\label{dem}

\subsubsection{Possible use:}\label{possible-use-128}

\begin{itemize}
\tightlist
\item
  \textbf{\texttt{dem}} (\texttt{file}) ---\textgreater{}
  \texttt{geometry}
\item
  \texttt{file} \textbf{\texttt{dem}} \texttt{file} ---\textgreater{}
  \texttt{geometry}
\item
  \textbf{\texttt{dem}} (\texttt{file} , \texttt{file})
  ---\textgreater{} \texttt{geometry}
\item
  \texttt{file} \textbf{\texttt{dem}} \texttt{float} ---\textgreater{}
  \texttt{geometry}
\item
  \textbf{\texttt{dem}} (\texttt{file} , \texttt{float})
  ---\textgreater{} \texttt{geometry}
\item
  \textbf{\texttt{dem}} (\texttt{file}, \texttt{file}, \texttt{float})
  ---\textgreater{} \texttt{geometry}
\end{itemize}

\subsubsection{Result:}\label{result-124}

A polygon that is equivalent to the surface of the texture

\subsubsection{Examples:}\label{examples-98}

\begin{verbatim}
 
geometry var0 <- dem(dem,texture); // var0 equals a geometry as a rectangle of weight and height equal to the texture. 
geometry var1 <- dem(dem); // var1 equals returns a geometry as a rectangle of width and height equal to the texture. 
geometry var2 <- dem(dem,texture,z_factor); // var2 equals a geometry as a rectangle of width and height equal to the texture. 
geometry var3 <- dem(dem,z_factor); // var3 equals a geometry as a rectangle of weight and height equal to the texture.
\end{verbatim}

\begin{center}\rule{0.5\linewidth}{\linethickness}\end{center}

\subsection{\texorpdfstring{\texttt{det}}{det}}\label{det}

Same signification as \href{OperatorsDH\#determinant}{determinant}

\begin{center}\rule{0.5\linewidth}{\linethickness}\end{center}

\subsection{\texorpdfstring{\texttt{determinant}}{determinant}}\label{determinant}

\subsubsection{Possible use:}\label{possible-use-129}

\begin{itemize}
\tightlist
\item
  \textbf{\texttt{determinant}} (\texttt{matrix}) ---\textgreater{}
  \texttt{float}
\end{itemize}

\subsubsection{Result:}\label{result-125}

The determinant of the given matrix

\subsubsection{Examples:}\label{examples-99}

\begin{verbatim}
 
float var0 <- determinant(matrix([[1,2],[3,4]])); // var0 equals -2
\end{verbatim}

\begin{center}\rule{0.5\linewidth}{\linethickness}\end{center}

\subsection{\texorpdfstring{\texttt{diff}}{diff}}\label{diff}

\subsubsection{Possible use:}\label{possible-use-130}

\begin{itemize}
\tightlist
\item
  \texttt{float} \textbf{\texttt{diff}} \texttt{float} ---\textgreater{}
  \texttt{float}
\item
  \textbf{\texttt{diff}} (\texttt{float} , \texttt{float})
  ---\textgreater{} \texttt{float}
\end{itemize}

\subsubsection{Result:}\label{result-126}

A placeholder function for expressing equations

\begin{center}\rule{0.5\linewidth}{\linethickness}\end{center}

\subsection{\texorpdfstring{\texttt{diff2}}{diff2}}\label{diff2}

\subsubsection{Possible use:}\label{possible-use-131}

\begin{itemize}
\tightlist
\item
  \texttt{float} \textbf{\texttt{diff2}} \texttt{float}
  ---\textgreater{} \texttt{float}
\item
  \textbf{\texttt{diff2}} (\texttt{float} , \texttt{float})
  ---\textgreater{} \texttt{float}
\end{itemize}

\subsubsection{Result:}\label{result-127}

A placeholder function for expressing equations

\begin{center}\rule{0.5\linewidth}{\linethickness}\end{center}

\subsection{\texorpdfstring{\texttt{directed}}{directed}}\label{directed}

\subsubsection{Possible use:}\label{possible-use-132}

\begin{itemize}
\tightlist
\item
  \textbf{\texttt{directed}} (\texttt{graph}) ---\textgreater{}
  \texttt{graph}
\end{itemize}

\subsubsection{Result:}\label{result-128}

the operand graph becomes a directed graph.

\subsubsection{Comment:}\label{comment-30}

the operator alters the operand graph, it does not create a new one.

\subsubsection{See also:}\label{see-also-77}

\href{OperatorsSZ\#undirected}{undirected},

\begin{center}\rule{0.5\linewidth}{\linethickness}\end{center}

\subsection{\texorpdfstring{\texttt{direction\_between}}{direction\_between}}\label{direction_between}

\subsubsection{Possible use:}\label{possible-use-133}

\begin{itemize}
\tightlist
\item
  \texttt{topology} \textbf{\texttt{direction\_between}}
  \texttt{container\textless{}geometry\textgreater{}} ---\textgreater{}
  \texttt{float}
\item
  \textbf{\texttt{direction\_between}} (\texttt{topology} ,
  \texttt{container\textless{}geometry\textgreater{}}) ---\textgreater{}
  \texttt{float}
\end{itemize}

\subsubsection{Result:}\label{result-129}

A direction (in degree) between a list of two geometries (geometries,
agents, points) considering a topology.

\subsubsection{Examples:}\label{examples-100}

\begin{verbatim}
 
float var0 <- my_topology direction_between [ag1, ag2]; // var0 equals the direction between ag1 and ag2 considering the topology my_topology
\end{verbatim}

\subsubsection{See also:}\label{see-also-78}

\href{OperatorsSZ\#towards}{towards},
\href{OperatorsDH\#direction_to}{direction\_to},
\href{OperatorsDH\#distance_to}{distance\_to},
\href{OperatorsDH\#distance_between}{distance\_between},
\href{OperatorsNR\#path_between}{path\_between},
\href{OperatorsNR\#path_to}{path\_to},

\begin{center}\rule{0.5\linewidth}{\linethickness}\end{center}

\subsection{\texorpdfstring{\texttt{direction\_to}}{direction\_to}}\label{direction_to}

Same signification as \href{OperatorsSZ\#towards}{towards}

\begin{center}\rule{0.5\linewidth}{\linethickness}\end{center}

\subsection{\texorpdfstring{\texttt{disjoint\_from}}{disjoint\_from}}\label{disjoint_from}

\subsubsection{Possible use:}\label{possible-use-134}

\begin{itemize}
\tightlist
\item
  \texttt{geometry} \textbf{\texttt{disjoint\_from}} \texttt{geometry}
  ---\textgreater{} \texttt{bool}
\item
  \textbf{\texttt{disjoint\_from}} (\texttt{geometry} ,
  \texttt{geometry}) ---\textgreater{} \texttt{bool}
\end{itemize}

\subsubsection{Result:}\label{result-130}

A boolean, equal to true if the left-geometry (or agent/point) is
disjoints from the right-geometry (or agent/point).

\subsubsection{Special cases:}\label{special-cases-50}

\begin{itemize}
\tightlist
\item
  if one of the operand is null, returns true.\\
\item
  if one operand is a point, returns false if the point is included in
  the geometry.
\end{itemize}

\subsubsection{Examples:}\label{examples-101}

\begin{verbatim}
 
bool var0 <- polyline([{10,10},{20,20}]) disjoint_from polyline([{15,15},{25,25}]); // var0 equals false 
bool var1 <- polygon([{10,10},{10,20},{20,20},{20,10}]) disjoint_from polygon([{15,15},{15,25},{25,25},{25,15}]); // var1 equals false 
bool var2 <- polygon([{10,10},{10,20},{20,20},{20,10}]) disjoint_from {15,15}; // var2 equals false 
bool var3 <- polygon([{10,10},{10,20},{20,20},{20,10}]) disjoint_from {25,25}; // var3 equals true 
bool var4 <- polygon([{10,10},{10,20},{20,20},{20,10}]) disjoint_from polygon([{35,35},{35,45},{45,45},{45,35}]); // var4 equals true
\end{verbatim}

\subsubsection{See also:}\label{see-also-79}

\href{OperatorsIM\#intersects}{intersects},
\href{OperatorsBC\#crosses}{crosses},
\href{OperatorsNR\#overlaps}{overlaps},
\href{OperatorsNR\#partially_overlaps}{partially\_overlaps},
\href{OperatorsSZ\#touches}{touches},

\begin{center}\rule{0.5\linewidth}{\linethickness}\end{center}

\subsection{\texorpdfstring{\texttt{distance\_between}}{distance\_between}}\label{distance_between}

\subsubsection{Possible use:}\label{possible-use-135}

\begin{itemize}
\tightlist
\item
  \texttt{topology} \textbf{\texttt{distance\_between}}
  \texttt{container\textless{}geometry\textgreater{}} ---\textgreater{}
  \texttt{float}
\item
  \textbf{\texttt{distance\_between}} (\texttt{topology} ,
  \texttt{container\textless{}geometry\textgreater{}}) ---\textgreater{}
  \texttt{float}
\end{itemize}

\subsubsection{Result:}\label{result-131}

A distance between a list of geometries (geometries, agents, points)
considering a topology.

\subsubsection{Examples:}\label{examples-102}

\begin{verbatim}
 
float var0 <- my_topology distance_between [ag1, ag2, ag3]; // var0 equals the distance between ag1, ag2 and ag3 considering the topology my_topology
\end{verbatim}

\subsubsection{See also:}\label{see-also-80}

\href{OperatorsSZ\#towards}{towards},
\href{OperatorsDH\#direction_to}{direction\_to},
\href{OperatorsDH\#distance_to}{distance\_to},
\href{OperatorsDH\#direction_between}{direction\_between},
\href{OperatorsNR\#path_between}{path\_between},
\href{OperatorsNR\#path_to}{path\_to},

\begin{center}\rule{0.5\linewidth}{\linethickness}\end{center}

\subsection{\texorpdfstring{\texttt{distance\_to}}{distance\_to}}\label{distance_to}

\subsubsection{Possible use:}\label{possible-use-136}

\begin{itemize}
\tightlist
\item
  \texttt{geometry} \textbf{\texttt{distance\_to}} \texttt{geometry}
  ---\textgreater{} \texttt{float}
\item
  \textbf{\texttt{distance\_to}} (\texttt{geometry} , \texttt{geometry})
  ---\textgreater{} \texttt{float}
\item
  \texttt{point} \textbf{\texttt{distance\_to}} \texttt{point}
  ---\textgreater{} \texttt{float}
\item
  \textbf{\texttt{distance\_to}} (\texttt{point} , \texttt{point})
  ---\textgreater{} \texttt{float}
\end{itemize}

\subsubsection{Result:}\label{result-132}

A distance between two geometries (geometries, agents or points)
considering the topology of the agent applying the operator.

\subsubsection{Examples:}\label{examples-103}

\begin{verbatim}
 
float var0 <- ag1 distance_to ag2; // var0 equals the distance between ag1 and ag2 considering the topology of the agent applying the operator
\end{verbatim}

\subsubsection{See also:}\label{see-also-81}

\href{OperatorsSZ\#towards}{towards},
\href{OperatorsDH\#direction_to}{direction\_to},
\href{OperatorsDH\#distance_between}{distance\_between},
\href{OperatorsDH\#direction_between}{direction\_between},
\href{OperatorsNR\#path_between}{path\_between},
\href{OperatorsNR\#path_to}{path\_to},

\begin{center}\rule{0.5\linewidth}{\linethickness}\end{center}

\subsection{\texorpdfstring{\texttt{distinct}}{distinct}}\label{distinct}

\subsubsection{Possible use:}\label{possible-use-137}

\begin{itemize}
\tightlist
\item
  \textbf{\texttt{distinct}} (\texttt{container}) ---\textgreater{}
  \texttt{list}
\end{itemize}

\subsubsection{Result:}\label{result-133}

produces a set from the elements of the operand (i.e.~a list without
duplicated elements)

\subsubsection{Special cases:}\label{special-cases-51}

\begin{itemize}
\tightlist
\item
  if the operand is nil, remove\_duplicates returns nil\\
\item
  if the operand is a graph, remove\_duplicates returns the set of
  nodes\\
\item
  if the operand is a matrix, remove\_duplicates returns a matrix
  without duplicated row\\
\item
  if the operand is a map, remove\_duplicates returns the set of values
  without duplicate
\end{itemize}

\begin{verbatim}
 
list var1 <- remove_duplicates([1::3,2::4,3::3,5::7]); // var1 equals [3,4,7]
\end{verbatim}

\subsubsection{Examples:}\label{examples-104}

\begin{verbatim}
 
list var0 <- remove_duplicates([3,2,5,1,2,3,5,5,5]); // var0 equals [3,2,5,1]
\end{verbatim}

\begin{center}\rule{0.5\linewidth}{\linethickness}\end{center}

\subsection{\texorpdfstring{\texttt{distribution\_of}}{distribution\_of}}\label{distribution_of}

\subsubsection{Possible use:}\label{possible-use-138}

\begin{itemize}
\tightlist
\item
  \textbf{\texttt{distribution\_of}} (\texttt{container})
  ---\textgreater{} \texttt{map}
\item
  \texttt{container} \textbf{\texttt{distribution\_of}} \texttt{int}
  ---\textgreater{} \texttt{map}
\item
  \textbf{\texttt{distribution\_of}} (\texttt{container} , \texttt{int})
  ---\textgreater{} \texttt{map}
\item
  \textbf{\texttt{distribution\_of}} (\texttt{container}, \texttt{int},
  \texttt{float}, \texttt{float}) ---\textgreater{} \texttt{map}
\end{itemize}

\subsubsection{Result:}\label{result-134}

Discretize a list of values into n bins (computes the bins from a
numerical variable into n (default 10) bins. Returns a distribution map
with the values (values key), the interval legends (legend key), the
distribution parameters (params keys, for cumulative charts). Parameters
can be (list), (list, nbbins) or (list,nbbins,valmin,valmax)

\subsubsection{Examples:}\label{examples-105}

\begin{verbatim}
 
map var0 <- distribution_of([1,1,2,12.5]); // var0 equals map(['values'::[2,1,0,0,0,0,1,0,0,0],'legend'::['[0.0:2.0]','[2.0:4.0]','[4.0:6.0]','[6.0:8.0]','[8.0:10.0]','[10.0:12.0]','[12.0:14.0]','[14.0:16.0]','[16.0:18.0]','[18.0:20.0]'],'parlist'::[1,0]]) 
map var1 <- distribution_of([1,1,2,12.5]); // var1 equals map(['values'::[2,1,0,0,0,0,1,0,0,0],'legend'::['[0.0:2.0]','[2.0:4.0]','[4.0:6.0]','[6.0:8.0]','[8.0:10.0]','[10.0:12.0]','[12.0:14.0]','[14.0:16.0]','[16.0:18.0]','[18.0:20.0]'],'parlist'::[1,0]]) 
map var2 <- distribution_of([1,1,2,12.5],10); // var2 equals map(['values'::[2,1,0,0,0,0,1,0,0,0],'legend'::['[0.0:2.0]','[2.0:4.0]','[4.0:6.0]','[6.0:8.0]','[8.0:10.0]','[10.0:12.0]','[12.0:14.0]','[14.0:16.0]','[16.0:18.0]','[18.0:20.0]'],'parlist'::[1,0]])
\end{verbatim}

\subsubsection{See also:}\label{see-also-82}

\href{OperatorsAA\#as_map}{as\_map},

\begin{center}\rule{0.5\linewidth}{\linethickness}\end{center}

\subsection{\texorpdfstring{\texttt{distribution2d\_of}}{distribution2d\_of}}\label{distribution2d_of}

\subsubsection{Possible use:}\label{possible-use-139}

\begin{itemize}
\tightlist
\item
  \texttt{container} \textbf{\texttt{distribution2d\_of}}
  \texttt{container} ---\textgreater{} \texttt{map}
\item
  \textbf{\texttt{distribution2d\_of}} (\texttt{container} ,
  \texttt{container}) ---\textgreater{} \texttt{map}
\item
  \textbf{\texttt{distribution2d\_of}} (\texttt{container},
  \texttt{container}, \texttt{int}, \texttt{int}) ---\textgreater{}
  \texttt{map}
\item
  \textbf{\texttt{distribution2d\_of}} (\texttt{container},
  \texttt{container}, \texttt{int}, \texttt{float}, \texttt{float},
  \texttt{int}, \texttt{float}, \texttt{float}) ---\textgreater{}
  \texttt{map}
\end{itemize}

\subsubsection{Result:}\label{result-135}

Discretize two lists of values into n bins (computes the bins from a
numerical variable into n (default 10) bins. Returns a distribution map
with the values (values key), the interval legends (legend key), the
distribution parameters (params keys, for cumulative charts). Parameters
can be (list), (list, nbbins) or (list,nbbins,valmin,valmax)

\subsubsection{Examples:}\label{examples-106}

\begin{verbatim}
 
map var0 <- distribution_of([1,1,2,12.5],10); // var0 equals map(['values'::[2,1,0,0,0,0,1,0,0,0],'legend'::['[0.0:2.0]','[2.0:4.0]','[4.0:6.0]','[6.0:8.0]','[8.0:10.0]','[10.0:12.0]','[12.0:14.0]','[14.0:16.0]','[16.0:18.0]','[18.0:20.0]'],'parlist'::[1,0]]) 
map var1 <- distribution2d_of([1,1,2,12.5]); // var1 equals map(['values'::[2,1,0,0,0,0,1,0,0,0],'legend'::['[0.0:2.0]','[2.0:4.0]','[4.0:6.0]','[6.0:8.0]','[8.0:10.0]','[10.0:12.0]','[12.0:14.0]','[14.0:16.0]','[16.0:18.0]','[18.0:20.0]'],'parlist'::[1,0]]) 
map var2 <- distribution_of([1,1,2,12.5],10); // var2 equals map(['values'::[2,1,0,0,0,0,1,0,0,0],'legend'::['[0.0:2.0]','[2.0:4.0]','[4.0:6.0]','[6.0:8.0]','[8.0:10.0]','[10.0:12.0]','[12.0:14.0]','[14.0:16.0]','[16.0:18.0]','[18.0:20.0]'],'parlist'::[1,0]])
\end{verbatim}

\subsubsection{See also:}\label{see-also-83}

\href{OperatorsAA\#as_map}{as\_map},

\begin{center}\rule{0.5\linewidth}{\linethickness}\end{center}

\subsection{\texorpdfstring{\texttt{div}}{div}}\label{div}

\subsubsection{Possible use:}\label{possible-use-140}

\begin{itemize}
\tightlist
\item
  \texttt{int} \textbf{\texttt{div}} \texttt{float} ---\textgreater{}
  \texttt{int}
\item
  \textbf{\texttt{div}} (\texttt{int} , \texttt{float})
  ---\textgreater{} \texttt{int}
\item
  \texttt{float} \textbf{\texttt{div}} \texttt{float} ---\textgreater{}
  \texttt{int}
\item
  \textbf{\texttt{div}} (\texttt{float} , \texttt{float})
  ---\textgreater{} \texttt{int}
\item
  \texttt{float} \textbf{\texttt{div}} \texttt{int} ---\textgreater{}
  \texttt{int}
\item
  \textbf{\texttt{div}} (\texttt{float} , \texttt{int})
  ---\textgreater{} \texttt{int}
\item
  \texttt{int} \textbf{\texttt{div}} \texttt{int} ---\textgreater{}
  \texttt{int}
\item
  \textbf{\texttt{div}} (\texttt{int} , \texttt{int}) ---\textgreater{}
  \texttt{int}
\end{itemize}

\subsubsection{Result:}\label{result-136}

Returns the truncation of the division of the left-hand operand by the
right-hand operand.

\subsubsection{Special cases:}\label{special-cases-52}

\begin{itemize}
\tightlist
\item
  if the right-hand operand is equal to zero, raises an exception.\\
\item
  if the right-hand operand is equal to zero, raises an exception.\\
\item
  if the right-hand operand is equal to zero, raises an exception.
\end{itemize}

\subsubsection{Examples:}\label{examples-107}

\begin{verbatim}
 
int var0 <- 40 div 4.1; // var0 equals 9 
int var1 <- 40.1 div 4.5; // var1 equals 8 
int var2 <- 40.5 div 3; // var2 equals 13 
int var3 <- 40 div 3; // var3 equals 13
\end{verbatim}

\subsubsection{See also:}\label{see-also-84}

\href{OperatorsIM\#mod}{mod},

\begin{center}\rule{0.5\linewidth}{\linethickness}\end{center}

\subsection{\texorpdfstring{\texttt{dnorm}}{dnorm}}\label{dnorm}

Same signification as
\href{OperatorsNR\#normal_density}{normal\_density}

\begin{center}\rule{0.5\linewidth}{\linethickness}\end{center}

\subsection{\texorpdfstring{\texttt{dtw}}{dtw}}\label{dtw}

\subsubsection{Possible use:}\label{possible-use-141}

\begin{itemize}
\tightlist
\item
  \texttt{list} \textbf{\texttt{dtw}} \texttt{list} ---\textgreater{}
  \texttt{float}
\item
  \textbf{\texttt{dtw}} (\texttt{list} , \texttt{list})
  ---\textgreater{} \texttt{float}
\item
  \textbf{\texttt{dtw}} (\texttt{list}, \texttt{list}, \texttt{int})
  ---\textgreater{} \texttt{float}
\end{itemize}

\subsubsection{Result:}\label{result-137}

returns the dynamic time warping between the two series of value with
Sakoe-Chiba band (radius: the window width of Sakoe-Chiba band) returns
the dynamic time warping between the two series of value

\subsubsection{Examples:}\label{examples-108}

\begin{verbatim}
 
float var0 <- dtw([10.0,5.0,1.0, 3.0],[1.0,10.0,5.0,1.0], 2); // var0 equals 2.0 
float var1 <- dtw([10.0,5.0,1.0, 3.0],[1.0,10.0,5.0,1.0]); // var1 equals 2
\end{verbatim}

\begin{center}\rule{0.5\linewidth}{\linethickness}\end{center}

\subsection{\texorpdfstring{\texttt{durbin\_watson}}{durbin\_watson}}\label{durbin_watson}

\subsubsection{Possible use:}\label{possible-use-142}

\begin{itemize}
\tightlist
\item
  \textbf{\texttt{durbin\_watson}} (\texttt{container})
  ---\textgreater{} \texttt{float}
\end{itemize}

\subsubsection{Result:}\label{result-138}

Durbin-Watson computation

\begin{center}\rule{0.5\linewidth}{\linethickness}\end{center}

\subsection{\texorpdfstring{\texttt{dxf\_file}}{dxf\_file}}\label{dxf_file}

\subsubsection{Possible use:}\label{possible-use-143}

\begin{itemize}
\tightlist
\item
  \textbf{\texttt{dxf\_file}} (\texttt{string}) ---\textgreater{}
  \texttt{file}
\end{itemize}

\subsubsection{Result:}\label{result-139}

Constructs a file of type dxf. Allowed extensions are limited to dxf

\begin{center}\rule{0.5\linewidth}{\linethickness}\end{center}

\subsection{\texorpdfstring{\texttt{edge}}{edge}}\label{edge-3}

\subsubsection{Possible use:}\label{possible-use-144}

\begin{itemize}
\tightlist
\item
  \textbf{\texttt{edge}} (\texttt{unknown}) ---\textgreater{}
  \texttt{unknown}
\item
  \textbf{\texttt{edge}} (\texttt{pair}) ---\textgreater{}
  \texttt{unknown}
\item
  \texttt{pair} \textbf{\texttt{edge}} \texttt{float} ---\textgreater{}
  \texttt{unknown}
\item
  \textbf{\texttt{edge}} (\texttt{pair} , \texttt{float})
  ---\textgreater{} \texttt{unknown}
\item
  \texttt{unknown} \textbf{\texttt{edge}} \texttt{unknown}
  ---\textgreater{} \texttt{unknown}
\item
  \textbf{\texttt{edge}} (\texttt{unknown} , \texttt{unknown})
  ---\textgreater{} \texttt{unknown}
\item
  \texttt{unknown} \textbf{\texttt{edge}} \texttt{float}
  ---\textgreater{} \texttt{unknown}
\item
  \textbf{\texttt{edge}} (\texttt{unknown} , \texttt{float})
  ---\textgreater{} \texttt{unknown}
\item
  \textbf{\texttt{edge}} (\texttt{unknown}, \texttt{unknown},
  \texttt{unknown}) ---\textgreater{} \texttt{unknown}
\item
  \textbf{\texttt{edge}} (\texttt{pair}, \texttt{unknown},
  \texttt{float}) ---\textgreater{} \texttt{unknown}
\item
  \textbf{\texttt{edge}} (\texttt{unknown}, \texttt{unknown},
  \texttt{float}) ---\textgreater{} \texttt{unknown}
\item
  \textbf{\texttt{edge}} (\texttt{unknown}, \texttt{unknown},
  \texttt{unknown}, \texttt{float}) ---\textgreater{} \texttt{unknown}
\end{itemize}

\begin{center}\rule{0.5\linewidth}{\linethickness}\end{center}

\subsection{\texorpdfstring{\texttt{edge\_between}}{edge\_between}}\label{edge_between}

\subsubsection{Possible use:}\label{possible-use-145}

\begin{itemize}
\tightlist
\item
  \texttt{graph} \textbf{\texttt{edge\_between}} \texttt{pair}
  ---\textgreater{} \texttt{unknown}
\item
  \textbf{\texttt{edge\_between}} (\texttt{graph} , \texttt{pair})
  ---\textgreater{} \texttt{unknown}
\end{itemize}

\subsubsection{Result:}\label{result-140}

returns the edge linking two nodes

\subsubsection{Examples:}\label{examples-109}

\begin{verbatim}
 
unknown var0 <- graphFromMap edge_between node1::node2; // var0 equals edge1
\end{verbatim}

\subsubsection{See also:}\label{see-also-85}

\href{OperatorsNR\#out_edges_of}{out\_edges\_of},
\href{OperatorsIM\#in_edges_of}{in\_edges\_of},

\begin{center}\rule{0.5\linewidth}{\linethickness}\end{center}

\subsection{\texorpdfstring{\texttt{edge\_betweenness}}{edge\_betweenness}}\label{edge_betweenness}

\subsubsection{Possible use:}\label{possible-use-146}

\begin{itemize}
\tightlist
\item
  \textbf{\texttt{edge\_betweenness}} (\texttt{graph}) ---\textgreater{}
  \texttt{map}
\end{itemize}

\subsubsection{Result:}\label{result-141}

returns a map containing for each edge (key), its betweenness centrality
(value): number of shortest paths passing through each edge

\subsubsection{Examples:}\label{examples-110}

\begin{verbatim}
graph graphEpidemio <- graph([]);  
map var1 <- edge_betweenness(graphEpidemio); // var1 equals the edge betweenness index of the graph
\end{verbatim}

\begin{center}\rule{0.5\linewidth}{\linethickness}\end{center}

\subsection{\texorpdfstring{\texttt{edges}}{edges}}\label{edges}

\subsubsection{Possible use:}\label{possible-use-147}

\begin{itemize}
\tightlist
\item
  \textbf{\texttt{edges}} (\texttt{container}) ---\textgreater{}
  \texttt{container}
\end{itemize}

\begin{center}\rule{0.5\linewidth}{\linethickness}\end{center}

\subsection{\texorpdfstring{\texttt{eigenvalues}}{eigenvalues}}\label{eigenvalues}

\subsubsection{Possible use:}\label{possible-use-148}

\begin{itemize}
\tightlist
\item
  \textbf{\texttt{eigenvalues}} (\texttt{matrix}) ---\textgreater{}
  \texttt{list\textless{}float\textgreater{}}
\end{itemize}

\subsubsection{Result:}\label{result-142}

The eigen values (matrix) of the given matrix

\subsubsection{Examples:}\label{examples-111}

\begin{verbatim}
 
list<float> var0 <- eigenvalues(matrix([[5,-3],[6,-4]])); // var0 equals [2.0000000000000004,-0.9999999999999998]
\end{verbatim}

\begin{center}\rule{0.5\linewidth}{\linethickness}\end{center}

\subsection{\texorpdfstring{\texttt{electre\_DM}}{electre\_DM}}\label{electre_dm}

\subsubsection{Possible use:}\label{possible-use-149}

\begin{itemize}
\tightlist
\item
  \textbf{\texttt{electre\_DM}}
  (\texttt{msi.gama.util.IList\textless{}java.util.List\textgreater{}},
  \texttt{msi.gama.util.IList\textless{}java.util.Map\textless{}java.lang.String,java.lang.Object\textgreater{}\textgreater{}},
  \texttt{float}) ---\textgreater{} \texttt{int}
\end{itemize}

\subsubsection{Result:}\label{result-143}

The index of the best candidate according to a method based on the
ELECTRE methods. The principle of the ELECTRE methods is to compare the
possible candidates by pair. These methods analyses the possible
outranking relation existing between two candidates. An candidate
outranks another if this one is at least as good as the other one. The
ELECTRE methods are based on two concepts: the concordance and the
discordance. The concordance characterizes the fact that, for an
outranking relation to be validated, a sufficient majority of criteria
should be in favor of this assertion. The discordance characterizes the
fact that, for an outranking relation to be validated, none of the
criteria in the minority should oppose too strongly this assertion.
These two conditions must be true for validating the outranking
assertion. More information about the ELECTRE methods can be found in
{[}\url{http://www.springerlink.com/content/g367r44322876223/} Figueira,
J., Mousseau, V., Roy, B.: ELECTRE Methods. In: Figueira, J., Greco, S.,
and Ehrgott, M., (Eds.), Multiple Criteria Decision Analysis: State of
the Art Surveys, Springer, New York, 133--162 (2005){]}. The first
operand is the list of candidates (a candidate is a list of criterion
values); the second operand the list of criterion: A criterion is a map
that contains fives elements: a name, a weight, a preference value (p),
an indifference value (q) and a veto value (v). The preference value
represents the threshold from which the difference between two criterion
values allows to prefer one vector of values over another. The
indifference value represents the threshold from which the difference
between two criterion values is considered significant. The veto value
represents the threshold from which the difference between two criterion
values disqualifies the candidate that obtained the smaller value; the
last operand is the fuzzy cut.

\subsubsection{Special cases:}\label{special-cases-53}

\begin{itemize}
\tightlist
\item
  returns -1 is the list of candidates is nil or empty
\end{itemize}

\subsubsection{Examples:}\label{examples-112}

\begin{verbatim}
 
int var0 <- electre_DM([[1.0, 7.0],[4.0,2.0],[3.0, 3.0]], [["name"::"utility", "weight" :: 2.0,"p"::0.5, "q"::0.0, "s"::1.0, "maximize" :: true],["name"::"price", "weight" :: 1.0,"p"::0.5, "q"::0.0, "s"::1.0, "maximize" :: false]],0.7); // var0 equals 0
\end{verbatim}

\subsubsection{See also:}\label{see-also-86}

\href{OperatorsSZ\#weighted_means_dm}{weighted\_means\_DM},
\href{OperatorsNR\#promethee_dm}{promethee\_DM},
\href{OperatorsDH\#evidence_theory_dm}{evidence\_theory\_DM},

\begin{center}\rule{0.5\linewidth}{\linethickness}\end{center}

\subsection{\texorpdfstring{\texttt{ellipse}}{ellipse}}\label{ellipse}

\subsubsection{Possible use:}\label{possible-use-150}

\begin{itemize}
\tightlist
\item
  \texttt{float} \textbf{\texttt{ellipse}} \texttt{float}
  ---\textgreater{} \texttt{geometry}
\item
  \textbf{\texttt{ellipse}} (\texttt{float} , \texttt{float})
  ---\textgreater{} \texttt{geometry}
\end{itemize}

\subsubsection{Result:}\label{result-144}

An ellipse geometry which x-radius is equal to the first operand and
y-radius is equal to the second operand

\subsubsection{Comment:}\label{comment-31}

the center of the ellipse is by default the location of the current
agent in which has been called this operator.

\subsubsection{Special cases:}\label{special-cases-54}

\begin{itemize}
\tightlist
\item
  returns a point if both operands are lower or equal to 0, a line if
  only one is.
\end{itemize}

\subsubsection{Examples:}\label{examples-113}

\begin{verbatim}
 
geometry var0 <- ellipse(10, 10); // var0 equals a geometry as an ellipse of width 10 and height 10.
\end{verbatim}

\subsubsection{See also:}\label{see-also-87}

\href{OperatorsAA\#around}{around}, \href{OperatorsBC\#cone}{cone},
\href{OperatorsIM\#line}{line}, \href{OperatorsIM\#link}{link},
\href{OperatorsNR\#norm}{norm}, \href{OperatorsNR\#point}{point},
\href{OperatorsNR\#polygon}{polygon},
\href{OperatorsNR\#polyline}{polyline},
\href{OperatorsNR\#rectangle}{rectangle},
\href{OperatorsSZ\#square}{square}, \href{OperatorsBC\#circle}{circle},
\href{OperatorsSZ\#squircle}{squircle},
\href{OperatorsSZ\#triangle}{triangle},

\begin{center}\rule{0.5\linewidth}{\linethickness}\end{center}

\subsection{\texorpdfstring{\texttt{emotion}}{emotion}}\label{emotion}

\subsubsection{Possible use:}\label{possible-use-151}

\begin{itemize}
\tightlist
\item
  \textbf{\texttt{emotion}} (\texttt{any}) ---\textgreater{}
  \texttt{emotion}
\end{itemize}

\subsubsection{Result:}\label{result-145}

Casts the operand into the type emotion

\begin{center}\rule{0.5\linewidth}{\linethickness}\end{center}

\subsection{\texorpdfstring{\texttt{empty}}{empty}}\label{empty}

\subsubsection{Possible use:}\label{possible-use-152}

\begin{itemize}
\tightlist
\item
  \textbf{\texttt{empty}} (\texttt{string}) ---\textgreater{}
  \texttt{bool}
\item
  \textbf{\texttt{empty}}
  (\texttt{container\textless{}KeyType,ValueType\textgreater{}})
  ---\textgreater{} \texttt{bool}
\end{itemize}

\subsubsection{Result:}\label{result-146}

true if the operand is empty, false otherwise.

\subsubsection{Comment:}\label{comment-32}

the empty operator behavior depends on the nature of the operand

\subsubsection{Special cases:}\label{special-cases-55}

\begin{itemize}
\tightlist
\item
  if it is a map, empty returns true if the map contains no key-value
  mappings, and false otherwise\\
\item
  if it is a file, empty returns true if the content of the file (that
  is also a container) is empty, and false otherwise\\
\item
  if it is a population, empty returns true if there is no agent in the
  population, and false otherwise\\
\item
  if it is a graph, empty returns true if it contains no vertex and no
  edge, and false otherwise\\
\item
  if it is a matrix of int, float or object, it will return true if all
  elements are respectively 0, 0.0 or null, and false otherwise\\
\item
  if it is a matrix of geometry, it will return true if the matrix
  contains no cell, and false otherwise\\
\item
  if it is a string, empty returns true if the string does not contain
  any character, and false otherwise
\end{itemize}

\begin{verbatim}
 
bool var0 <- empty ('abced'); // var0 equals false
\end{verbatim}

\begin{itemize}
\tightlist
\item
  if it is a list, empty returns true if there is no element in the
  list, and false otherwise
\end{itemize}

\begin{verbatim}
 
bool var1 <- empty([]); // var1 equals true
\end{verbatim}

\begin{center}\rule{0.5\linewidth}{\linethickness}\end{center}

\subsection{\texorpdfstring{\texttt{enlarged\_by}}{enlarged\_by}}\label{enlarged_by}

Same signification as \href{OperatorsAA\#+}{+}

\begin{center}\rule{0.5\linewidth}{\linethickness}\end{center}

\subsection{\texorpdfstring{\texttt{envelope}}{envelope}}\label{envelope}

\subsubsection{Possible use:}\label{possible-use-153}

\begin{itemize}
\tightlist
\item
  \textbf{\texttt{envelope}} (\texttt{unknown}) ---\textgreater{}
  \texttt{geometry}
\end{itemize}

\subsubsection{Result:}\label{result-147}

A 3D geometry that represents the box that surrounds the geometries or
the surface described by the arguments. More general than
geometry(arguments).envelope, as it allows to pass int, double, point,
image files, shape files, asc files, or any list combining these
arguments, in which case the envelope will be correctly expanded. If an
envelope cannot be determined from the arguments, a default one of
dimensions (0,100, 0, 100, 0, 100) is returned

\begin{center}\rule{0.5\linewidth}{\linethickness}\end{center}

\subsection{\texorpdfstring{\texttt{eval\_gaml}}{eval\_gaml}}\label{eval_gaml}

\subsubsection{Possible use:}\label{possible-use-154}

\begin{itemize}
\tightlist
\item
  \textbf{\texttt{eval\_gaml}} (\texttt{string}) ---\textgreater{}
  \texttt{unknown}
\end{itemize}

\subsubsection{Result:}\label{result-148}

evaluates the given GAML string.

\subsubsection{Examples:}\label{examples-114}

\begin{verbatim}
 
unknown var0 <- eval_gaml("2+3"); // var0 equals 5
\end{verbatim}

\begin{center}\rule{0.5\linewidth}{\linethickness}\end{center}

\subsection{\texorpdfstring{\texttt{eval\_when}}{eval\_when}}\label{eval_when}

\subsubsection{Possible use:}\label{possible-use-155}

\begin{itemize}
\tightlist
\item
  \textbf{\texttt{eval\_when}} (\texttt{BDIPlan}) ---\textgreater{}
  \texttt{bool}
\end{itemize}

\subsubsection{Result:}\label{result-149}

evaluate the facet when of a given plan

\subsubsection{Examples:}\label{examples-115}

\begin{verbatim}
eval_when(plan1) 
\end{verbatim}

\begin{center}\rule{0.5\linewidth}{\linethickness}\end{center}

\subsection{\texorpdfstring{\texttt{evaluate\_sub\_model}}{evaluate\_sub\_model}}\label{evaluate_sub_model}

\subsubsection{Possible use:}\label{possible-use-156}

\begin{itemize}
\tightlist
\item
  \texttt{msi.gama.kernel.experiment.IExperimentAgent}
  \textbf{\texttt{evaluate\_sub\_model}} \texttt{string}
  ---\textgreater{} \texttt{unknown}
\item
  \textbf{\texttt{evaluate\_sub\_model}}
  (\texttt{msi.gama.kernel.experiment.IExperimentAgent} ,
  \texttt{string}) ---\textgreater{} \texttt{unknown}
\end{itemize}

\subsubsection{Result:}\label{result-150}

Load a submodel

\subsubsection{Comment:}\label{comment-33}

loaded submodel

\begin{center}\rule{0.5\linewidth}{\linethickness}\end{center}

\subsection{\texorpdfstring{\texttt{even}}{even}}\label{even}

\subsubsection{Possible use:}\label{possible-use-157}

\begin{itemize}
\tightlist
\item
  \textbf{\texttt{even}} (\texttt{int}) ---\textgreater{} \texttt{bool}
\end{itemize}

\subsubsection{Result:}\label{result-151}

Returns true if the operand is even and false if it is odd.

\subsubsection{Special cases:}\label{special-cases-56}

\begin{itemize}
\tightlist
\item
  if the operand is equal to 0, it returns true.\\
\item
  if the operand is a float, it is truncated before
\end{itemize}

\subsubsection{Examples:}\label{examples-116}

\begin{verbatim}
 
bool var0 <- even (3); // var0 equals false 
bool var1 <- even(-12); // var1 equals true
\end{verbatim}

\begin{center}\rule{0.5\linewidth}{\linethickness}\end{center}

\subsection{\texorpdfstring{\texttt{every}}{every}}\label{every}

\subsubsection{Possible use:}\label{possible-use-158}

\begin{itemize}
\tightlist
\item
  \textbf{\texttt{every}} (\texttt{int}) ---\textgreater{} \texttt{bool}
\item
  \textbf{\texttt{every}} (\texttt{any\ expression}) ---\textgreater{}
  \texttt{bool}
\item
  \texttt{list} \textbf{\texttt{every}} \texttt{int} ---\textgreater{}
  \texttt{list}
\item
  \textbf{\texttt{every}} (\texttt{list} , \texttt{int})
  ---\textgreater{} \texttt{list}
\item
  \texttt{msi.gama.util.GamaDateInterval} \textbf{\texttt{every}}
  \texttt{any\ expression} ---\textgreater{}
  \texttt{msi.gama.util.IList\textless{}msi.gama.util.GamaDate\textgreater{}}
\item
  \textbf{\texttt{every}} (\texttt{msi.gama.util.GamaDateInterval} ,
  \texttt{any\ expression}) ---\textgreater{}
  \texttt{msi.gama.util.IList\textless{}msi.gama.util.GamaDate\textgreater{}}
\end{itemize}

\subsubsection{Result:}\label{result-152}

true every operand * cycle, false otherwise Retrieves elements from the
first argument every \texttt{step} (second argument) elements. Raises an
error if the step is negative or equal to zero expects a frequency
(expressed in seconds of simulated time) as argument. Will return true
every time the current\_date matches with this frequency applies a step
to an interval of dates defined by `date1 to date2'

\subsubsection{Comment:}\label{comment-34}

the value of the every operator depends on the cycle. It can be used to
do something every x cycle.Used to do something at regular intervals of
time. Can be used in conjunction with `since', `after', `before',
`until' or `between', so that this computation only takes place in the
temporal segment defined by these operators. In all cases, the
starting\_date of the model is used as a reference starting point

\subsubsection{Examples:}\label{examples-117}

\begin{verbatim}
if every(2#cycle) {write "the cycle number is even";}        else {write "the cycle number is odd";} reflex when: every(2#days) since date('2000-01-01') { .. } state a { transition to: b when: every(2#mn);} state b { transition to: a when: every(30#s);} // This oscillatory behavior will use the starting_date of the model as its starting point in time (date('2000-01-01') to date('2010-01-01')) every (#month) // builds an interval between these two dates which contains all the monthly dates starting from the beginning of the interval 
\end{verbatim}

\subsubsection{See also:}\label{see-also-88}

\href{OperatorsSZ\#since}{since}, \href{OperatorsAA\#after}{after},
\href{OperatorsSZ\#to}{to},

\begin{center}\rule{0.5\linewidth}{\linethickness}\end{center}

\subsection{\texorpdfstring{\texttt{every\_cycle}}{every\_cycle}}\label{every_cycle}

Same signification as \href{OperatorsDH\#every}{every}

\begin{center}\rule{0.5\linewidth}{\linethickness}\end{center}

\subsection{\texorpdfstring{\texttt{evidence\_theory\_DM}}{evidence\_theory\_DM}}\label{evidence_theory_dm}

\subsubsection{Possible use:}\label{possible-use-159}

\begin{itemize}
\tightlist
\item
  \texttt{msi.gama.util.IList\textless{}java.util.List\textgreater{}}
  \textbf{\texttt{evidence\_theory\_DM}}
  \texttt{msi.gama.util.IList\textless{}java.util.Map\textless{}java.lang.String,java.lang.Object\textgreater{}\textgreater{}}
  ---\textgreater{} \texttt{int}
\item
  \textbf{\texttt{evidence\_theory\_DM}}
  (\texttt{msi.gama.util.IList\textless{}java.util.List\textgreater{}} ,
  \texttt{msi.gama.util.IList\textless{}java.util.Map\textless{}java.lang.String,java.lang.Object\textgreater{}\textgreater{}})
  ---\textgreater{} \texttt{int}
\item
  \textbf{\texttt{evidence\_theory\_DM}}
  (\texttt{msi.gama.util.IList\textless{}java.util.List\textgreater{}},
  \texttt{msi.gama.util.IList\textless{}java.util.Map\textless{}java.lang.String,java.lang.Object\textgreater{}\textgreater{}},
  \texttt{bool}) ---\textgreater{} \texttt{int}
\end{itemize}

\subsubsection{Result:}\label{result-153}

The index of the best candidate according to a method based on the
Evidence theory. This theory, which was proposed by Shafer
({[}\url{http://www.glennshafer.com/books/amte.html} Shafer G (1976) A
mathematical theory of evidence, Princeton University Press{]}), is
based on the work of Dempster
({[}\url{http://projecteuclid.org/DPubS?service=UI\&version=1.0\&verb=Display\&handle=euclid.aoms/1177698950}
Dempster A (1967) Upper and lower probabilities induced by multivalued
mapping. Annals of Mathematical Statistics, vol. 38, pp.~325--339{]}) on
lower and upper probability distributions. The first operand is the list
of candidates (a candidate is a list of criterion values); the second
operand the list of criterion: A criterion is a map that contains seven
elements: a name, a first threshold s1, a second threshold s2, a value
for the assertion ``this candidate is the best'' at threshold s1 (v1p),
a value for the assertion ``this candidate is the best'' at threshold s2
(v2p), a value for the assertion ``this candidate is not the best'' at
threshold s1 (v1c), a value for the assertion ``this candidate is not
the best'' at threshold s2 (v2c). v1p, v2p, v1c and v2c have to been
defined in order that: v1p + v1c \textless{}= 1.0; v2p + v2c
\textless{}= 1.0.; the last operand allows to use a simple version of
this multi-criteria decision making method (simple if true)

\subsubsection{Special cases:}\label{special-cases-57}

\begin{itemize}
\tightlist
\item
  if the operator is used with only 2 operands (the candidates and the
  criteria), the last parameter (use simple method) is set to true\\
\item
  returns -1 is the list of candidates is nil or empty
\end{itemize}

\subsubsection{Examples:}\label{examples-118}

\begin{verbatim}
 
int var0 <- evidence_theory_DM([[1.0, 7.0],[4.0,2.0],[3.0, 3.0]], [["name"::"utility", "s1" :: 0.0,"s2"::1.0, "v1p"::0.0, "v2p"::1.0, "v1c"::0.0, "v2c"::0.0, "maximize" :: true],["name"::"price",  "s1" :: 0.0,"s2"::1.0, "v1p"::0.0, "v2p"::1.0, "v1c"::0.0, "v2c"::0.0, "maximize" :: true]], true); // var0 equals 0
\end{verbatim}

\subsubsection{See also:}\label{see-also-89}

\href{OperatorsSZ\#weighted_means_dm}{weighted\_means\_DM},
\href{OperatorsDH\#electre_dm}{electre\_DM},

\begin{center}\rule{0.5\linewidth}{\linethickness}\end{center}

\subsection{\texorpdfstring{\texttt{exp}}{exp}}\label{exp}

\subsubsection{Possible use:}\label{possible-use-160}

\begin{itemize}
\tightlist
\item
  \textbf{\texttt{exp}} (\texttt{float}) ---\textgreater{}
  \texttt{float}
\item
  \textbf{\texttt{exp}} (\texttt{int}) ---\textgreater{} \texttt{float}
\end{itemize}

\subsubsection{Result:}\label{result-154}

Returns Euler's number e raised to the power of the operand.

\subsubsection{Special cases:}\label{special-cases-58}

\begin{itemize}
\tightlist
\item
  the operand is casted to a float before being evaluated.\\
\item
  the operand is casted to a float before being evaluated.
\end{itemize}

\subsubsection{Examples:}\label{examples-119}

\begin{verbatim}
 
float var0 <- exp (0); // var0 equals 1.0
\end{verbatim}

\subsubsection{See also:}\label{see-also-90}

\href{OperatorsIM\#ln}{ln},

\begin{center}\rule{0.5\linewidth}{\linethickness}\end{center}

\subsection{\texorpdfstring{\texttt{fact}}{fact}}\label{fact}

\subsubsection{Possible use:}\label{possible-use-161}

\begin{itemize}
\tightlist
\item
  \textbf{\texttt{fact}} (\texttt{int}) ---\textgreater{} \texttt{float}
\end{itemize}

\subsubsection{Result:}\label{result-155}

Returns the factorial of the operand.

\subsubsection{Special cases:}\label{special-cases-59}

\begin{itemize}
\tightlist
\item
  if the operand is less than 0, fact returns 0.
\end{itemize}

\subsubsection{Examples:}\label{examples-120}

\begin{verbatim}
 
float var0 <- fact(4); // var0 equals 24
\end{verbatim}

\begin{center}\rule{0.5\linewidth}{\linethickness}\end{center}

\subsection{\texorpdfstring{\texttt{farthest\_point\_to}}{farthest\_point\_to}}\label{farthest_point_to}

\subsubsection{Possible use:}\label{possible-use-162}

\begin{itemize}
\tightlist
\item
  \texttt{geometry} \textbf{\texttt{farthest\_point\_to}} \texttt{point}
  ---\textgreater{} \texttt{point}
\item
  \textbf{\texttt{farthest\_point\_to}} (\texttt{geometry} ,
  \texttt{point}) ---\textgreater{} \texttt{point}
\end{itemize}

\subsubsection{Result:}\label{result-156}

the farthest point of the left-operand to the left-point.

\subsubsection{Examples:}\label{examples-121}

\begin{verbatim}
 
point var0 <- geom farthest_point_to(pt); // var0 equals the farthest point of geom to pt
\end{verbatim}

\subsubsection{See also:}\label{see-also-91}

\href{OperatorsAA\#any_location_in}{any\_location\_in},
\href{OperatorsAA\#any_point_in}{any\_point\_in},
\href{OperatorsBC\#closest_points_with}{closest\_points\_with},
\href{OperatorsNR\#points_at}{points\_at},

\begin{center}\rule{0.5\linewidth}{\linethickness}\end{center}

\subsection{\texorpdfstring{\texttt{farthest\_to}}{farthest\_to}}\label{farthest_to}

\subsubsection{Possible use:}\label{possible-use-163}

\begin{itemize}
\tightlist
\item
  \texttt{container\textless{}agent\textgreater{}}
  \textbf{\texttt{farthest\_to}} \texttt{geometry} ---\textgreater{}
  \texttt{geometry}
\item
  \textbf{\texttt{farthest\_to}}
  (\texttt{container\textless{}agent\textgreater{}} , \texttt{geometry})
  ---\textgreater{} \texttt{geometry}
\end{itemize}

\subsubsection{Result:}\label{result-157}

An agent or a geometry among the left-operand list of agents, species or
meta-population (addition of species), the farthest to the operand
(casted as a geometry).

\subsubsection{Comment:}\label{comment-35}

the distance is computed in the topology of the calling agent (the agent
in which this operator is used), with the distance algorithm specific to
the topology.

\subsubsection{Examples:}\label{examples-122}

\begin{verbatim}
 
geometry var0 <- [ag1, ag2, ag3] closest_to(self); // var0 equals return the farthest agent among ag1, ag2 and ag3 to the agent applying the operator.(species1 + species2) closest_to self 
\end{verbatim}

\subsubsection{See also:}\label{see-also-92}

\href{OperatorsNR\#neighbors_at}{neighbors\_at},
\href{OperatorsNR\#neighbors_of}{neighbors\_of},
\href{OperatorsIM\#inside}{inside},
\href{OperatorsNR\#overlapping}{overlapping},
\href{OperatorsAA\#agents_overlapping}{agents\_overlapping},
\href{OperatorsAA\#agents_inside}{agents\_inside},
\href{OperatorsAA\#agent_closest_to}{agent\_closest\_to},
\href{OperatorsBC\#closest_to}{closest\_to},
\href{OperatorsAA\#agent_farthest_to}{agent\_farthest\_to},

\begin{center}\rule{0.5\linewidth}{\linethickness}\end{center}

\subsection{\texorpdfstring{\texttt{file}}{file}}\label{file}

\subsubsection{Possible use:}\label{possible-use-164}

\begin{itemize}
\tightlist
\item
  \textbf{\texttt{file}} (\texttt{string}) ---\textgreater{}
  \texttt{file}
\item
  \texttt{string} \textbf{\texttt{file}} \texttt{container}
  ---\textgreater{} \texttt{file}
\item
  \textbf{\texttt{file}} (\texttt{string} , \texttt{container})
  ---\textgreater{} \texttt{file}
\end{itemize}

\subsubsection{Result:}\label{result-158}

Creates a file in read/write mode, setting its contents to the container
passed in parameter opens a file in read only mode, creates a GAML file
object, and tries to determine and store the file content in the
contents attribute.

\subsubsection{Comment:}\label{comment-36}

The type of container to pass will depend on the type of file (see the
management of files in the documentation). Can be used to copy files
since files are considered as containers. For example: save
file(`image\_copy.png', file(`image.png')); will copy image.png to
image\_copy.pngThe file should have a supported extension, see file type
definition for supported file extensions.

\subsubsection{Special cases:}\label{special-cases-60}

\begin{itemize}
\tightlist
\item
  If the specified string does not refer to an existing file, an
  exception is risen when the variable is used.
\end{itemize}

\subsubsection{Examples:}\label{examples-123}

\begin{verbatim}
let fileT type: file value: file("../includes/Stupid_Cell.Data");           // fileT represents the file "../includes/Stupid_Cell.Data"             // fileT.contents here contains a matrix storing all the data of the text file 
\end{verbatim}

\subsubsection{See also:}\label{see-also-93}

\href{OperatorsDH\#folder}{folder},
\href{OperatorsNR\#new_folder}{new\_folder},

\begin{center}\rule{0.5\linewidth}{\linethickness}\end{center}

\subsection{\texorpdfstring{\texttt{file}}{file}}\label{file-1}

\subsubsection{Possible use:}\label{possible-use-165}

\begin{itemize}
\tightlist
\item
  \textbf{\texttt{file}} (\texttt{any}) ---\textgreater{} \texttt{file}
\end{itemize}

\subsubsection{Result:}\label{result-159}

Casts the operand into the type file

\begin{center}\rule{0.5\linewidth}{\linethickness}\end{center}

\subsection{\texorpdfstring{\texttt{file\_exists}}{file\_exists}}\label{file_exists}

\subsubsection{Possible use:}\label{possible-use-166}

\begin{itemize}
\tightlist
\item
  \textbf{\texttt{file\_exists}} (\texttt{string}) ---\textgreater{}
  \texttt{bool}
\end{itemize}

\subsubsection{Result:}\label{result-160}

Test whether the parameter is the path to an existing file.

\begin{center}\rule{0.5\linewidth}{\linethickness}\end{center}

\subsection{\texorpdfstring{\texttt{first}}{first}}\label{first}

\subsubsection{Possible use:}\label{possible-use-167}

\begin{itemize}
\tightlist
\item
  \textbf{\texttt{first}}
  (\texttt{container\textless{}KeyType,ValueType\textgreater{}})
  ---\textgreater{} \texttt{ValueType}
\item
  \textbf{\texttt{first}} (\texttt{string}) ---\textgreater{}
  \texttt{string}
\item
  \texttt{int} \textbf{\texttt{first}} \texttt{container}
  ---\textgreater{} \texttt{list}
\item
  \textbf{\texttt{first}} (\texttt{int} , \texttt{container})
  ---\textgreater{} \texttt{list}
\end{itemize}

\subsubsection{Result:}\label{result-161}

the first value of the operand

\subsubsection{Comment:}\label{comment-37}

the first operator behavior depends on the nature of the operand

\subsubsection{Special cases:}\label{special-cases-61}

\begin{itemize}
\tightlist
\item
  if it is a map, first returns the first value of the first pair (in
  insertion order)\\
\item
  if it is a file, first returns the first element of the content of the
  file (that is also a container)\\
\item
  if it is a population, first returns the first agent of the
  population\\
\item
  if it is a graph, first returns the first edge (in creation order)\\
\item
  if it is a matrix, first returns the element at \{0,0\} in the
  matrix\\
\item
  for a matrix of int or float, it will return 0 if the matrix is
  empty\\
\item
  for a matrix of object or geometry, it will return nil if the matrix
  is empty\\
\item
  if it is a list, first returns the first element of the list, or nil
  if the list is empty
\end{itemize}

\begin{verbatim}
 
int var0 <- first ([1, 2, 3]); // var0 equals 1
\end{verbatim}

\begin{itemize}
\tightlist
\item
  if it is a string, first returns a string composed of its first
  character
\end{itemize}

\begin{verbatim}
 
string var1 <- first ('abce'); // var1 equals 'a'
\end{verbatim}

\subsubsection{See also:}\label{see-also-94}

\href{OperatorsIM\#last}{last},

\begin{center}\rule{0.5\linewidth}{\linethickness}\end{center}

\subsection{\texorpdfstring{\texttt{first\_of}}{first\_of}}\label{first_of}

Same signification as \href{OperatorsDH\#first}{first}

\begin{center}\rule{0.5\linewidth}{\linethickness}\end{center}

\subsection{\texorpdfstring{\texttt{first\_with}}{first\_with}}\label{first_with}

\subsubsection{Possible use:}\label{possible-use-168}

\begin{itemize}
\tightlist
\item
  \texttt{container} \textbf{\texttt{first\_with}}
  \texttt{any\ expression} ---\textgreater{} \texttt{unknown}
\item
  \textbf{\texttt{first\_with}} (\texttt{container} ,
  \texttt{any\ expression}) ---\textgreater{} \texttt{unknown}
\end{itemize}

\subsubsection{Result:}\label{result-162}

the first element of the left-hand operand that makes the right-hand
operand evaluate to true.

\subsubsection{Comment:}\label{comment-38}

in the right-hand operand, the keyword each can be used to represent, in
turn, each of the right-hand operand elements.

\subsubsection{Special cases:}\label{special-cases-62}

\begin{itemize}
\tightlist
\item
  if the left-hand operand is nil, first\_with throws an error. If there
  is no element that satisfies the condition, it returns nil\\
\item
  if the left-operand is a map, the keyword each will contain each value
\end{itemize}

\begin{verbatim}
 
unknown var4 <- [1::2, 3::4, 5::6] first_with (each >= 4); // var4 equals 4 
unknown var5 <- [1::2, 3::4, 5::6].pairs first_with (each.value >= 4); // var5 equals (3::4)
\end{verbatim}

\subsubsection{Examples:}\label{examples-124}

\begin{verbatim}
 
unknown var0 <- [1,2,3,4,5,6,7,8] first_with (each > 3); // var0 equals 4 
unknown var2 <- g2 first_with (length(g2 out_edges_of each) = 0); // var2 equals node9 
unknown var3 <- (list(node) first_with (round(node(each).location.x) > 32); // var3 equals node2
\end{verbatim}

\subsubsection{See also:}\label{see-also-95}

\href{OperatorsDH\#group_by}{group\_by},
\href{OperatorsIM\#last_with}{last\_with},
\href{OperatorsSZ\#where}{where},

\begin{center}\rule{0.5\linewidth}{\linethickness}\end{center}

\subsection{\texorpdfstring{\texttt{flip}}{flip}}\label{flip}

\subsubsection{Possible use:}\label{possible-use-169}

\begin{itemize}
\tightlist
\item
  \textbf{\texttt{flip}} (\texttt{float}) ---\textgreater{}
  \texttt{bool}
\end{itemize}

\subsubsection{Result:}\label{result-163}

true or false given the probability represented by the operand

\subsubsection{Special cases:}\label{special-cases-63}

\begin{itemize}
\tightlist
\item
  flip 0 always returns false, flip 1 true
\end{itemize}

\subsubsection{Examples:}\label{examples-125}

\begin{verbatim}
 
bool var0 <- flip (0.66666); // var0 equals 2/3 chances to return true.
\end{verbatim}

\subsubsection{See also:}\label{see-also-96}

\href{OperatorsNR\#rnd}{rnd},

\begin{center}\rule{0.5\linewidth}{\linethickness}\end{center}

\subsection{\texorpdfstring{\texttt{float}}{float}}\label{float}

\subsubsection{Possible use:}\label{possible-use-170}

\begin{itemize}
\tightlist
\item
  \textbf{\texttt{float}} (\texttt{any}) ---\textgreater{}
  \texttt{float}
\end{itemize}

\subsubsection{Result:}\label{result-164}

Casts the operand into the type float

\begin{center}\rule{0.5\linewidth}{\linethickness}\end{center}

\subsection{\texorpdfstring{\texttt{floor}}{floor}}\label{floor}

\subsubsection{Possible use:}\label{possible-use-171}

\begin{itemize}
\tightlist
\item
  \textbf{\texttt{floor}} (\texttt{float}) ---\textgreater{}
  \texttt{float}
\end{itemize}

\subsubsection{Result:}\label{result-165}

Maps the operand to the largest previous following integer, i.e.~the
largest integer not greater than x.

\subsubsection{Examples:}\label{examples-126}

\begin{verbatim}
 
float var0 <- floor(3); // var0 equals 3.0 
float var1 <- floor(3.5); // var1 equals 3.0 
float var2 <- floor(-4.7); // var2 equals -5.0
\end{verbatim}

\subsubsection{See also:}\label{see-also-97}

\href{OperatorsBC\#ceil}{ceil}, \href{OperatorsNR\#round}{round},

\begin{center}\rule{0.5\linewidth}{\linethickness}\end{center}

\subsection{\texorpdfstring{\texttt{folder}}{folder}}\label{folder}

\subsubsection{Possible use:}\label{possible-use-172}

\begin{itemize}
\tightlist
\item
  \textbf{\texttt{folder}} (\texttt{string}) ---\textgreater{}
  \texttt{file}
\end{itemize}

\subsubsection{Result:}\label{result-166}

opens an existing repository

\subsubsection{Special cases:}\label{special-cases-64}

\begin{itemize}
\tightlist
\item
  If the specified string does not refer to an existing repository, an
  exception is risen.
\end{itemize}

\subsubsection{Examples:}\label{examples-127}

\begin{verbatim}
file dirT <- folder("../includes/");                // dirT represents the repository "../includes/"                // dirT.contents here contains the list of the names of included files 
\end{verbatim}

\subsubsection{See also:}\label{see-also-98}

\href{OperatorsDH\#file}{file},
\href{OperatorsNR\#new_folder}{new\_folder},

\begin{center}\rule{0.5\linewidth}{\linethickness}\end{center}

\subsection{\texorpdfstring{\texttt{font}}{font}}\label{font}

\subsubsection{Possible use:}\label{possible-use-173}

\begin{itemize}
\tightlist
\item
  \textbf{\texttt{font}} (\texttt{string}, \texttt{int}, \texttt{int})
  ---\textgreater{} \texttt{font}
\end{itemize}

\subsubsection{Result:}\label{result-167}

Creates a new font, by specifying its name (either a font face name like
`Lucida Grande Bold' or `Helvetica', or a logical name like `Dialog',
`SansSerif', `Serif', etc.), a size in points and a style, either
\#bold, \#italic or \#plain or a combination (addition) of them.

\subsubsection{Examples:}\label{examples-128}

\begin{verbatim}
 
font var0 <- font ('Helvetica Neue',12, #bold + #italic); // var0 equals a bold and italic face of the Helvetica Neue family
\end{verbatim}

\begin{center}\rule{0.5\linewidth}{\linethickness}\end{center}

\subsection{\texorpdfstring{\texttt{frequency\_of}}{frequency\_of}}\label{frequency_of}

\subsubsection{Possible use:}\label{possible-use-174}

\begin{itemize}
\tightlist
\item
  \texttt{container} \textbf{\texttt{frequency\_of}}
  \texttt{any\ expression} ---\textgreater{} \texttt{map}
\item
  \textbf{\texttt{frequency\_of}} (\texttt{container} ,
  \texttt{any\ expression}) ---\textgreater{} \texttt{map}
\end{itemize}

\subsubsection{Result:}\label{result-168}

Returns a map with keys equal to the application of the right-hand
argument (like collect) and values equal to the frequency of this key
(i.e.~how many times it has been obtained)

\subsubsection{Examples:}\label{examples-129}

\begin{verbatim}
 
map var0 <- [ag1, ag2, ag3, ag4] frequency_of each.size; // var0 equals the different sizes as keys and the number of agents of this size as values
\end{verbatim}

\subsubsection{See also:}\label{see-also-99}

\href{OperatorsAA\#as_map}{as\_map},

\begin{center}\rule{0.5\linewidth}{\linethickness}\end{center}

\subsection{\texorpdfstring{\texttt{from}}{from}}\label{from}

Same signification as \href{OperatorsSZ\#since}{since}

\begin{center}\rule{0.5\linewidth}{\linethickness}\end{center}

\subsection{\texorpdfstring{\texttt{fuzzy\_choquet\_DM}}{fuzzy\_choquet\_DM}}\label{fuzzy_choquet_dm}

\subsubsection{Possible use:}\label{possible-use-175}

\begin{itemize}
\tightlist
\item
  \textbf{\texttt{fuzzy\_choquet\_DM}}
  (\texttt{msi.gama.util.IList\textless{}java.util.List\textgreater{}},
  \texttt{list\textless{}string\textgreater{}}, \texttt{map})
  ---\textgreater{} \texttt{int}
\end{itemize}

\subsubsection{Result:}\label{result-169}

The index of the candidate that maximizes the Fuzzy Choquet Integral
value. The first operand is the list of candidates (a candidate is a
list of criterion values); the second operand the list of criterion
(list of string); the third operand the weights of each sub-set of
criteria (map with list for key and float for value)

\subsubsection{Special cases:}\label{special-cases-65}

\begin{itemize}
\tightlist
\item
  returns -1 is the list of candidates is nil or empty
\end{itemize}

\subsubsection{Examples:}\label{examples-130}

\begin{verbatim}
 
int var0 <- fuzzy_choquet_DM([[1.0, 7.0],[4.0,2.0],[3.0, 3.0]], ["utility", "price", "size"],[["utility"]::0.5,["size"]::0.1,["price"]::0.4,["utility", "price"]::0.55]); // var0 equals 0
\end{verbatim}

\subsubsection{See also:}\label{see-also-100}

\href{OperatorsNR\#promethee_dm}{promethee\_DM},
\href{OperatorsDH\#electre_dm}{electre\_DM},
\href{OperatorsDH\#evidence_theory_dm}{evidence\_theory\_DM},

\begin{center}\rule{0.5\linewidth}{\linethickness}\end{center}

\subsection{\texorpdfstring{\texttt{fuzzy\_kappa}}{fuzzy\_kappa}}\label{fuzzy_kappa}

\subsubsection{Possible use:}\label{possible-use-176}

\begin{itemize}
\tightlist
\item
  \textbf{\texttt{fuzzy\_kappa}}
  (\texttt{list\textless{}agent\textgreater{}}, \texttt{list},
  \texttt{list}, \texttt{list\textless{}float\textgreater{}},
  \texttt{list}, \texttt{matrix\textless{}float\textgreater{}},
  \texttt{float}) ---\textgreater{} \texttt{float}
\item
  \textbf{\texttt{fuzzy\_kappa}}
  (\texttt{list\textless{}agent\textgreater{}}, \texttt{list},
  \texttt{list}, \texttt{list\textless{}float\textgreater{}},
  \texttt{list}, \texttt{matrix\textless{}float\textgreater{}},
  \texttt{float}, \texttt{list}) ---\textgreater{} \texttt{float}
\end{itemize}

\subsubsection{Result:}\label{result-170}

fuzzy kappa indicator for 2 map comparisons:
fuzzy\_kappa(agents\_list,list\_vals1,list\_vals2,
output\_similarity\_per\_agents,categories,fuzzy\_categories\_matrix,
fuzzy\_distance). Reference: Visser, H., and T. de Nijs, 2006. The map
comparison kit, Environmental Modelling \& Software, 21 fuzzy kappa
indicator for 2 map comparisons:
fuzzy\_kappa(agents\_list,list\_vals1,list\_vals2,
output\_similarity\_per\_agents,categories,fuzzy\_categories\_matrix,
fuzzy\_distance, weights). Reference: Visser, H., and T. de Nijs, 2006.
The map comparison kit, Environmental Modelling \& Software, 21

\subsubsection{Examples:}\label{examples-131}

\begin{verbatim}
fuzzy_kappa([ag1, ag2, ag3, ag4, ag5],[cat1,cat1,cat2,cat3,cat2],[cat2,cat1,cat2,cat1,cat2], similarity_per_agents,[cat1,cat2,cat3],[[1,0,0],[0,1,0],[0,0,1]], 2) fuzzy_kappa([ag1, ag2, ag3, ag4, ag5],[cat1,cat1,cat2,cat3,cat2],[cat2,cat1,cat2,cat1,cat2], similarity_per_agents,[cat1,cat2,cat3],[[1,0,0],[0,1,0],[0,0,1]], 2, [1.0,3.0,2.0,2.0,4.0]) 
\end{verbatim}

\begin{center}\rule{0.5\linewidth}{\linethickness}\end{center}

\subsection{\texorpdfstring{\texttt{fuzzy\_kappa\_sim}}{fuzzy\_kappa\_sim}}\label{fuzzy_kappa_sim}

\subsubsection{Possible use:}\label{possible-use-177}

\begin{itemize}
\tightlist
\item
  \textbf{\texttt{fuzzy\_kappa\_sim}}
  (\texttt{list\textless{}agent\textgreater{}}, \texttt{list},
  \texttt{list}, \texttt{list},
  \texttt{list\textless{}float\textgreater{}}, \texttt{list},
  \texttt{matrix\textless{}float\textgreater{}}, \texttt{float})
  ---\textgreater{} \texttt{float}
\item
  \textbf{\texttt{fuzzy\_kappa\_sim}}
  (\texttt{list\textless{}agent\textgreater{}}, \texttt{list},
  \texttt{list}, \texttt{list},
  \texttt{list\textless{}float\textgreater{}}, \texttt{list},
  \texttt{matrix\textless{}float\textgreater{}}, \texttt{float},
  \texttt{list}) ---\textgreater{} \texttt{float}
\end{itemize}

\subsubsection{Result:}\label{result-171}

fuzzy kappa simulation indicator for 2 map comparisons:
fuzzy\_kappa\_sim(agents\_list,list\_vals1,list\_vals2,
output\_similarity\_per\_agents,fuzzy\_transitions\_matrix,
fuzzy\_distance). Reference: Jasper van Vliet, Alex Hagen-Zanker, Jelle
Hurkens, Hedwig van Delden, A fuzzy set approach to assess the
predictive accuracy of land use simulations, Ecological Modelling, 24
July 2013, Pages 32-42, ISSN 0304-3800, fuzzy kappa simulation indicator
for 2 map comparisons:
fuzzy\_kappa\_sim(agents\_list,list\_vals1,list\_vals2,
output\_similarity\_per\_agents,fuzzy\_transitions\_matrix,
fuzzy\_distance, weights). Reference: Jasper van Vliet, Alex
Hagen-Zanker, Jelle Hurkens, Hedwig van Delden, A fuzzy set approach to
assess the predictive accuracy of land use simulations, Ecological
Modelling, 24 July 2013, Pages 32-42, ISSN 0304-3800,

\subsubsection{Examples:}\label{examples-132}

\begin{verbatim}
fuzzy_kappa_sim([ag1, ag2, ag3, ag4, ag5], [cat1,cat1,cat2,cat3,cat2],[cat2,cat1,cat2,cat1,cat2], similarity_per_agents,[cat1,cat2,cat3],[[1,0,0,0,0,0,0,0,0],[0,1,0,0,0,0,0,0,0],[0,0,1,0,0,0,0,0,0],[0,0,0,1,0,0,0,0,0],[0,0,0,0,1,0,0,0,0],[0,0,0,0,0,1,0,0,0],[0,0,0,0,0,0,1,0,0],[0,0,0,0,0,0,0,1,0],[0,0,0,0,0,0,0,0,1]], 2) fuzzy_kappa_sim([ag1, ag2, ag3, ag4, ag5], [cat1,cat1,cat2,cat3,cat2],[cat2,cat1,cat2,cat1,cat2], similarity_per_agents,[cat1,cat2,cat3],[[1,0,0,0,0,0,0,0,0],[0,1,0,0,0,0,0,0,0],[0,0,1,0,0,0,0,0,0],[0,0,0,1,0,0,0,0,0],[0,0,0,0,1,0,0,0,0],[0,0,0,0,0,1,0,0,0],[0,0,0,0,0,0,1,0,0],[0,0,0,0,0,0,0,1,0],[0,0,0,0,0,0,0,0,1]], 2,[1.0,3.0,2.0,2.0,4.0]) 
\end{verbatim}

\begin{center}\rule{0.5\linewidth}{\linethickness}\end{center}

\subsection{\texorpdfstring{\texttt{gaml\_file}}{gaml\_file}}\label{gaml_file}

\subsubsection{Possible use:}\label{possible-use-178}

\begin{itemize}
\tightlist
\item
  \textbf{\texttt{gaml\_file}} (\texttt{string}) ---\textgreater{}
  \texttt{file}
\end{itemize}

\subsubsection{Result:}\label{result-172}

Constructs a file of type gaml. Allowed extensions are limited to gaml,
experiment

\begin{center}\rule{0.5\linewidth}{\linethickness}\end{center}

\subsection{\texorpdfstring{\texttt{gaml\_type}}{gaml\_type}}\label{gaml_type}

\subsubsection{Possible use:}\label{possible-use-179}

\begin{itemize}
\tightlist
\item
  \textbf{\texttt{gaml\_type}} (\texttt{any}) ---\textgreater{}
  \texttt{gaml\_type}
\end{itemize}

\subsubsection{Result:}\label{result-173}

Casts the operand into the type gaml\_type

\begin{center}\rule{0.5\linewidth}{\linethickness}\end{center}

\subsection{\texorpdfstring{\texttt{gamma}}{gamma}}\label{gamma}

\subsubsection{Possible use:}\label{possible-use-180}

\begin{itemize}
\tightlist
\item
  \textbf{\texttt{gamma}} (\texttt{float}) ---\textgreater{}
  \texttt{float}
\end{itemize}

\subsubsection{Result:}\label{result-174}

Returns the value of the Gamma function at x.

\begin{center}\rule{0.5\linewidth}{\linethickness}\end{center}

\subsection{\texorpdfstring{\texttt{gamma\_distribution}}{gamma\_distribution}}\label{gamma_distribution}

\subsubsection{Possible use:}\label{possible-use-181}

\begin{itemize}
\tightlist
\item
  \textbf{\texttt{gamma\_distribution}} (\texttt{float}, \texttt{float},
  \texttt{float}) ---\textgreater{} \texttt{float}
\end{itemize}

\subsubsection{Result:}\label{result-175}

Returns the integral from zero to x of the gamma probability density
function.

\subsubsection{Comment:}\label{comment-39}

incomplete\_gamma(a,x) is equal to pgamma(a,1,x).

\begin{center}\rule{0.5\linewidth}{\linethickness}\end{center}

\subsection{\texorpdfstring{\texttt{gamma\_distribution\_complemented}}{gamma\_distribution\_complemented}}\label{gamma_distribution_complemented}

\subsubsection{Possible use:}\label{possible-use-182}

\begin{itemize}
\tightlist
\item
  \textbf{\texttt{gamma\_distribution\_complemented}} (\texttt{float},
  \texttt{float}, \texttt{float}) ---\textgreater{} \texttt{float}
\end{itemize}

\subsubsection{Result:}\label{result-176}

Returns the integral from x to infinity of the gamma probability density
function.

\begin{center}\rule{0.5\linewidth}{\linethickness}\end{center}

\subsection{\texorpdfstring{\texttt{gamma\_index}}{gamma\_index}}\label{gamma_index}

\subsubsection{Possible use:}\label{possible-use-183}

\begin{itemize}
\tightlist
\item
  \textbf{\texttt{gamma\_index}} (\texttt{graph}) ---\textgreater{}
  \texttt{float}
\end{itemize}

\subsubsection{Result:}\label{result-177}

returns the gamma index of the graph (A measure of connectivity that
considers the relationship between the number of observed links and the
number of possible links: gamma = e/(3 \texttt{*} (v - 2)) - for planar
graph.

\subsubsection{Examples:}\label{examples-133}

\begin{verbatim}
graph graphEpidemio <- graph([]);  
float var1 <- gamma_index(graphEpidemio); // var1 equals the gamma index of the graph
\end{verbatim}

\subsubsection{See also:}\label{see-also-101}

\href{OperatorsAA\#alpha_index}{alpha\_index},
\href{OperatorsBC\#beta_index}{beta\_index},
\href{OperatorsNR\#nb_cycles}{nb\_cycles},
\href{OperatorsBC\#connectivity_index}{connectivity\_index},

\begin{center}\rule{0.5\linewidth}{\linethickness}\end{center}

\subsection{\texorpdfstring{\texttt{gamma\_rnd}}{gamma\_rnd}}\label{gamma_rnd}

\subsubsection{Possible use:}\label{possible-use-184}

\begin{itemize}
\tightlist
\item
  \texttt{float} \textbf{\texttt{gamma\_rnd}} \texttt{float}
  ---\textgreater{} \texttt{float}
\item
  \textbf{\texttt{gamma\_rnd}} (\texttt{float} , \texttt{float})
  ---\textgreater{} \texttt{float}
\end{itemize}

\subsubsection{Result:}\label{result-178}

returns a random value from a gamma distribution with specified values
of the shape and scale parameters

\subsubsection{Examples:}\label{examples-134}

\begin{verbatim}
gamma_rnd(10.0,5.0) 
\end{verbatim}

\begin{center}\rule{0.5\linewidth}{\linethickness}\end{center}

\subsection{\texorpdfstring{\texttt{gauss}}{gauss}}\label{gauss}

\subsubsection{Possible use:}\label{possible-use-185}

\begin{itemize}
\tightlist
\item
  \textbf{\texttt{gauss}} (\texttt{point}) ---\textgreater{}
  \texttt{float}
\item
  \texttt{float} \textbf{\texttt{gauss}} \texttt{float}
  ---\textgreater{} \texttt{float}
\item
  \textbf{\texttt{gauss}} (\texttt{float} , \texttt{float})
  ---\textgreater{} \texttt{float}
\end{itemize}

\subsubsection{Result:}\label{result-179}

A value from a normally distributed random variable with expected value
(mean) and variance (standardDeviation). The probability density
function of such a variable is a Gaussian. A value from a normally
distributed random variable with expected value (mean) and variance
(standardDeviation). The probability density function of such a variable
is a Gaussian.

\subsubsection{Special cases:}\label{special-cases-66}

\begin{itemize}
\tightlist
\item
  when the operand is a point, it is read as \{mean,
  standardDeviation\}\\
\item
  when standardDeviation value is 0.0, it always returns the mean
  value\\
\item
  when the operand is a point, it is read as \{mean,
  standardDeviation\}\\
\item
  when standardDeviation value is 0.0, it always returns the mean value
\end{itemize}

\subsubsection{Examples:}\label{examples-135}

\begin{verbatim}
 
float var0 <- gauss({0,0.3}); // var0 equals 0.22354 
float var1 <- gauss({0,0.3}); // var1 equals -0.1357 
float var2 <- gauss(0,0.3); // var2 equals 0.22354 
float var3 <- gauss(0,0.3); // var3 equals -0.1357
\end{verbatim}

\subsubsection{See also:}\label{see-also-102}

\href{OperatorsSZ\#truncated_gauss}{truncated\_gauss},
\href{OperatorsNR\#poisson}{poisson},
\href{OperatorsSZ\#skew_gauss}{skew\_gauss},

\begin{center}\rule{0.5\linewidth}{\linethickness}\end{center}

\subsection{\texorpdfstring{\texttt{generate\_barabasi\_albert}}{generate\_barabasi\_albert}}\label{generate_barabasi_albert}

\subsubsection{Possible use:}\label{possible-use-186}

\begin{itemize}
\tightlist
\item
  \textbf{\texttt{generate\_barabasi\_albert}}
  (\texttt{container\textless{}agent\textgreater{}}, \texttt{species},
  \texttt{int}, \texttt{bool}) ---\textgreater{} \texttt{graph}
\item
  \textbf{\texttt{generate\_barabasi\_albert}} (\texttt{species},
  \texttt{species}, \texttt{int}, \texttt{int}, \texttt{bool})
  ---\textgreater{} \texttt{graph}
\end{itemize}

\subsubsection{Result:}\label{result-180}

returns a random scale-free network (following Barabasi-Albert (BA)
model). returns a random scale-free network (following Barabasi-Albert
(BA) model).

\subsubsection{Comment:}\label{comment-40}

The Barabasi-Albert (BA) model is an algorithm for generating random
scale-free networks using a preferential attachment mechanism. A
scale-free network is a network whose degree distribution follows a
power law, at least asymptotically.Such networks are widely observed in
natural and human-made systems, including the Internet, the world wide
web, citation networks, and some social networks. {[}From Wikipedia
article{]}The map operand should includes following elements:The
Barabasi-Albert (BA) model is an algorithm for generating random
scale-free networks using a preferential attachment mechanism. A
scale-free network is a network whose degree distribution follows a
power law, at least asymptotically.Such networks are widely observed in
natural and human-made systems, including the Internet, the world wide
web, citation networks, and some social networks. {[}From Wikipedia
article{]}The map operand should includes following elements:

\subsubsection{Special cases:}\label{special-cases-67}

\begin{itemize}
\tightlist
\item
  ``agents'': list of existing node agents\\
\item
  ``edges\_species'': the species of edges\\
\item
  ``size'': the graph will contain (size + 1) nodes\\
\item
  ``m'': the number of edges added per novel node\\
\item
  ``synchronized'': is the graph and the species of vertices and edges
  synchronized?\\
\item
  ``vertices\_specy'': the species of vertices\\
\item
  ``edges\_species'': the species of edges\\
\item
  ``size'': the graph will contain (size + 1) nodes\\
\item
  ``m'': the number of edges added per novel node\\
\item
  ``synchronized'': is the graph and the species of vertices and edges
  synchronized?
\end{itemize}

\subsubsection{Examples:}\label{examples-136}

\begin{verbatim}
graph<yourNodeSpecy,yourEdgeSpecy> graphEpidemio <- generate_barabasi_albert(       yourListOfNodes,        yourEdgeSpecy,      3,      5,      true); graph<yourNodeSpecy,yourEdgeSpecy> graphEpidemio <- generate_barabasi_albert(        yourNodeSpecy,      yourEdgeSpecy,      3,      5,      true); 
\end{verbatim}

\subsubsection{See also:}\label{see-also-103}

\href{OperatorsDH\#generate_watts_strogatz}{generate\_watts\_strogatz},

\begin{center}\rule{0.5\linewidth}{\linethickness}\end{center}

\subsection{\texorpdfstring{\texttt{generate\_complete\_graph}}{generate\_complete\_graph}}\label{generate_complete_graph}

\subsubsection{Possible use:}\label{possible-use-187}

\begin{itemize}
\tightlist
\item
  \textbf{\texttt{generate\_complete\_graph}}
  (\texttt{container\textless{}agent\textgreater{}}, \texttt{species},
  \texttt{bool}) ---\textgreater{} \texttt{graph}
\item
  \textbf{\texttt{generate\_complete\_graph}} (\texttt{species},
  \texttt{species}, \texttt{int}, \texttt{bool}) ---\textgreater{}
  \texttt{graph}
\item
  \textbf{\texttt{generate\_complete\_graph}}
  (\texttt{container\textless{}agent\textgreater{}}, \texttt{species},
  \texttt{float}, \texttt{bool}) ---\textgreater{} \texttt{graph}
\item
  \textbf{\texttt{generate\_complete\_graph}} (\texttt{species},
  \texttt{species}, \texttt{int}, \texttt{float}, \texttt{bool})
  ---\textgreater{} \texttt{graph}
\end{itemize}

\subsubsection{Result:}\label{result-181}

returns a fully connected graph. returns a fully connected graph.
returns a fully connected graph. returns a fully connected graph.

\subsubsection{Comment:}\label{comment-41}

Arguments should include following elements:Arguments should include
following elements:Arguments should include following elements:Arguments
should include following elements:

\subsubsection{Special cases:}\label{special-cases-68}

\begin{itemize}
\tightlist
\item
  ``vertices\_specy'': the species of vertices\\
\item
  ``edges\_species'': the species of edges\\
\item
  ``size'': the graph will contain size nodes.\\
\item
  ``synchronized'': is the graph and the species of vertices and edges
  synchronized?\\
\item
  ``vertices\_specy'': the species of vertices\\
\item
  ``edges\_species'': the species of edges\\
\item
  ``size'': the graph will contain size nodes.\\
\item
  ``layoutRadius'': nodes of the graph will be located on a circle with
  radius layoutRadius and centered in the environment.\\
\item
  ``synchronized'': is the graph and the species of vertices and edges
  synchronized?\\
\item
  ``agents'': list of existing node agents\\
\item
  ``edges\_species'': the species of edges\\
\item
  ``synchronized'': is the graph and the species of vertices and edges
  synchronized?\\
\item
  ``agents'': list of existing node agents\\
\item
  ``edges\_species'': the species of edges\\
\item
  ``layoutRadius'': nodes of the graph will be located on a circle with
  radius layoutRadius and centered in the environment.\\
\item
  ``synchronized'': is the graph and the species of vertices and edges
  synchronized?
\end{itemize}

\subsubsection{Examples:}\label{examples-137}

\begin{verbatim}
graph<myVertexSpecy,myEdgeSpecy> myGraph <- generate_complete_graph(            myVertexSpecy,          myEdgeSpecy,            10,         true); graph<myVertexSpecy,myEdgeSpecy> myGraph <- generate_complete_graph(             myVertexSpecy,          myEdgeSpecy,            10, 25,         true); graph<myVertexSpecy,myEdgeSpecy> myGraph <- generate_complete_graph(             myListOfNodes,          myEdgeSpecy,        true); graph<myVertexSpecy,myEdgeSpecy> myGraph <- generate_complete_graph(             myListOfNodes,          myEdgeSpecy,            25,         true); 
\end{verbatim}

\subsubsection{See also:}\label{see-also-104}

\href{OperatorsDH\#generate_barabasi_albert}{generate\_barabasi\_albert},
\href{OperatorsDH\#generate_watts_strogatz}{generate\_watts\_strogatz},

\begin{center}\rule{0.5\linewidth}{\linethickness}\end{center}

\subsection{\texorpdfstring{\texttt{generate\_watts\_strogatz}}{generate\_watts\_strogatz}}\label{generate_watts_strogatz}

\subsubsection{Possible use:}\label{possible-use-188}

\begin{itemize}
\tightlist
\item
  \textbf{\texttt{generate\_watts\_strogatz}}
  (\texttt{container\textless{}agent\textgreater{}}, \texttt{species},
  \texttt{float}, \texttt{int}, \texttt{bool}) ---\textgreater{}
  \texttt{graph}
\item
  \textbf{\texttt{generate\_watts\_strogatz}} (\texttt{species},
  \texttt{species}, \texttt{int}, \texttt{float}, \texttt{int},
  \texttt{bool}) ---\textgreater{} \texttt{graph}
\end{itemize}

\subsubsection{Result:}\label{result-182}

returns a random small-world network (following Watts-Strogatz model).
returns a random small-world network (following Watts-Strogatz model).

\subsubsection{Comment:}\label{comment-42}

The Watts-Strogatz model is a random graph generation model that
produces graphs with small-world properties, including short average
path lengths and high clustering.A small-world network is a type of
graph in which most nodes are not neighbors of one another, but most
nodes can be reached from every other by a small number of hops or
steps. {[}From Wikipedia article{]}The map operand should includes
following elements:The Watts-Strogatz model is a random graph generation
model that produces graphs with small-world properties, including short
average path lengths and high clustering.A small-world network is a type
of graph in which most nodes are not neighbors of one another, but most
nodes can be reached from every other by a small number of hops or
steps. {[}From Wikipedia article{]}The map operand should includes
following elements:

\subsubsection{Special cases:}\label{special-cases-69}

\begin{itemize}
\tightlist
\item
  ``agents'': list of existing node agents\\
\item
  ``edges\_species'': the species of edges\\
\item
  ``p'': probability to ``rewire'' an edge. So it must be between 0 and
  1. The parameter is often called beta in the literature.\\
\item
  ``k'': the base degree of each node. k must be greater than 2 and
  even.\\
\item
  ``synchronized'': is the graph and the species of vertices and edges
  synchronized?\\
\item
  ``vertices\_specy'': the species of vertices\\
\item
  ``edges\_species'': the species of edges\\
\item
  ``size'': the graph will contain (size + 1) nodes. Size must be
  greater than k.\\
\item
  ``p'': probability to ``rewire'' an edge. So it must be between 0 and
  1. The parameter is often called beta in the literature.\\
\item
  ``k'': the base degree of each node. k must be greater than 2 and
  even.\\
\item
  ``synchronized'': is the graph and the species of vertices and edges
  synchronized?
\end{itemize}

\subsubsection{Examples:}\label{examples-138}

\begin{verbatim}
graph<myVertexSpecy,myEdgeSpecy> myGraph <- generate_watts_strogatz(            myListOfNodes,          myEdgeSpecy,            0.3,            2,      true); graph<myVertexSpecy,myEdgeSpecy> myGraph <- generate_watts_strogatz(             myVertexSpecy,          myEdgeSpecy,            2,          0.3,            2,      true); 
\end{verbatim}

\subsubsection{See also:}\label{see-also-105}

\href{OperatorsDH\#generate_barabasi_albert}{generate\_barabasi\_albert},

\begin{center}\rule{0.5\linewidth}{\linethickness}\end{center}

\subsection{\texorpdfstring{\texttt{geojson\_file}}{geojson\_file}}\label{geojson_file}

\subsubsection{Possible use:}\label{possible-use-189}

\begin{itemize}
\tightlist
\item
  \textbf{\texttt{geojson\_file}} (\texttt{string}) ---\textgreater{}
  \texttt{file}
\end{itemize}

\subsubsection{Result:}\label{result-183}

Constructs a file of type geojson. Allowed extensions are limited to
json, geojson, geo.json

\begin{center}\rule{0.5\linewidth}{\linethickness}\end{center}

\subsection{\texorpdfstring{\texttt{geometric\_mean}}{geometric\_mean}}\label{geometric_mean}

\subsubsection{Possible use:}\label{possible-use-190}

\begin{itemize}
\tightlist
\item
  \textbf{\texttt{geometric\_mean}} (\texttt{container})
  ---\textgreater{} \texttt{float}
\end{itemize}

\subsubsection{Result:}\label{result-184}

the geometric mean of the elements of the operand. See Geometric\_mean
for more details.

\subsubsection{Comment:}\label{comment-43}

The operator casts all the numerical element of the list into float. The
elements that are not numerical are discarded.

\subsubsection{Examples:}\label{examples-139}

\begin{verbatim}
 
float var0 <- geometric_mean ([4.5, 3.5, 5.5, 7.0]); // var0 equals 4.962326343467649
\end{verbatim}

\subsubsection{See also:}\label{see-also-106}

\href{OperatorsIM\#mean}{mean}, \href{OperatorsIM\#median}{median},
\href{OperatorsDH\#harmonic_mean}{harmonic\_mean},

\begin{center}\rule{0.5\linewidth}{\linethickness}\end{center}

\subsection{\texorpdfstring{\texttt{geometry}}{geometry}}\label{geometry}

\subsubsection{Possible use:}\label{possible-use-191}

\begin{itemize}
\tightlist
\item
  \textbf{\texttt{geometry}} (\texttt{any}) ---\textgreater{}
  \texttt{geometry}
\end{itemize}

\subsubsection{Result:}\label{result-185}

Casts the operand into the type geometry

\begin{center}\rule{0.5\linewidth}{\linethickness}\end{center}

\subsection{\texorpdfstring{\texttt{geometry\_collection}}{geometry\_collection}}\label{geometry_collection}

\subsubsection{Possible use:}\label{possible-use-192}

\begin{itemize}
\tightlist
\item
  \textbf{\texttt{geometry\_collection}}
  (\texttt{container\textless{}geometry\textgreater{}})
  ---\textgreater{} \texttt{geometry}
\end{itemize}

\subsubsection{Result:}\label{result-186}

A geometry collection (multi-geometry) composed of the given list of
geometries.

\subsubsection{Special cases:}\label{special-cases-70}

\begin{itemize}
\tightlist
\item
  if the operand is nil, returns the point geometry \{0,0\}\\
\item
  if the operand is composed of a single geometry, returns a copy of the
  geometry.
\end{itemize}

\subsubsection{Examples:}\label{examples-140}

\begin{verbatim}
 
geometry var0 <- geometry_collection([{0,0}, {0,10}, {10,10}, {10,0}]); // var0 equals a geometry composed of the 4 points (multi-point).
\end{verbatim}

\subsubsection{See also:}\label{see-also-107}

\href{OperatorsAA\#around}{around}, \href{OperatorsBC\#circle}{circle},
\href{OperatorsBC\#cone}{cone}, \href{OperatorsIM\#link}{link},
\href{OperatorsNR\#norm}{norm}, \href{OperatorsNR\#point}{point},
\href{OperatorsSZ\#polygone}{polygone},
\href{OperatorsNR\#rectangle}{rectangle},
\href{OperatorsSZ\#square}{square},
\href{OperatorsSZ\#triangle}{triangle}, \href{OperatorsIM\#line}{line},

\begin{center}\rule{0.5\linewidth}{\linethickness}\end{center}

\subsection{\texorpdfstring{\texttt{get}}{get}}\label{get}

Same signification as \href{OperatorsNR\#read}{read}

\subsubsection{Possible use:}\label{possible-use-193}

\begin{itemize}
\tightlist
\item
  \texttt{agent} \textbf{\texttt{get}} \texttt{string} ---\textgreater{}
  \texttt{unknown}
\item
  \textbf{\texttt{get}} (\texttt{agent} , \texttt{string})
  ---\textgreater{} \texttt{unknown}
\item
  \texttt{geometry} \textbf{\texttt{get}} \texttt{string}
  ---\textgreater{} \texttt{unknown}
\item
  \textbf{\texttt{get}} (\texttt{geometry} , \texttt{string})
  ---\textgreater{} \texttt{unknown}
\end{itemize}

\subsubsection{Result:}\label{result-187}

Reads an attribute of the specified agent (left operand). The attribute
name is specified by the right operand. Reads an attribute of the
specified geometry (left operand). The attribute name is specified by
the right operand.

\subsubsection{Special cases:}\label{special-cases-71}

\begin{itemize}
\tightlist
\item
  Reading the attribute of another agent
\end{itemize}

\begin{verbatim}
string agent_name <- an_agent get('name');     // reads then 'name' attribute of an_agent then assigns the returned value to the agent_name variable 
\end{verbatim}

\begin{itemize}
\tightlist
\item
  Reading the attribute of a geometry
\end{itemize}

\begin{verbatim}
string geom_area <- a_geometry get('area');     // reads then 'area' attribute of 'a_geometry' variable then assigns the returned value to the geom_area variable 
\end{verbatim}

\begin{center}\rule{0.5\linewidth}{\linethickness}\end{center}

\subsection{\texorpdfstring{\texttt{get\_about}}{get\_about}}\label{get_about}

\subsubsection{Possible use:}\label{possible-use-194}

\begin{itemize}
\tightlist
\item
  \textbf{\texttt{get\_about}} (\texttt{emotion}) ---\textgreater{}
  \texttt{predicate}
\end{itemize}

\subsubsection{Result:}\label{result-188}

get the about value of the given emotion

\subsubsection{Examples:}\label{examples-141}

\begin{verbatim}
get_about(emotion) 
\end{verbatim}

\begin{center}\rule{0.5\linewidth}{\linethickness}\end{center}

\subsection{\texorpdfstring{\texttt{get\_agent}}{get\_agent}}\label{get_agent}

\subsubsection{Possible use:}\label{possible-use-195}

\begin{itemize}
\tightlist
\item
  \textbf{\texttt{get\_agent}}
  (\texttt{msi.gaml.architecture.simplebdi.SocialLink})
  ---\textgreater{} \texttt{agent}
\end{itemize}

\subsubsection{Result:}\label{result-189}

get the agent value of the given social link

\subsubsection{Examples:}\label{examples-142}

\begin{verbatim}
get_agent(social_link1) 
\end{verbatim}

\begin{center}\rule{0.5\linewidth}{\linethickness}\end{center}

\subsection{\texorpdfstring{\texttt{get\_agent\_cause}}{get\_agent\_cause}}\label{get_agent_cause}

\subsubsection{Possible use:}\label{possible-use-196}

\begin{itemize}
\tightlist
\item
  \textbf{\texttt{get\_agent\_cause}} (\texttt{predicate})
  ---\textgreater{} \texttt{agent}
\item
  \textbf{\texttt{get\_agent\_cause}} (\texttt{emotion})
  ---\textgreater{} \texttt{agent}
\end{itemize}

\subsubsection{Result:}\label{result-190}

get the agent cause value of the given emotion

\subsubsection{Examples:}\label{examples-143}

\begin{verbatim}
get_agent_cause(emotion) 
\end{verbatim}

\begin{center}\rule{0.5\linewidth}{\linethickness}\end{center}

\subsection{\texorpdfstring{\texttt{get\_belief\_op}}{get\_belief\_op}}\label{get_belief_op}

\subsubsection{Possible use:}\label{possible-use-197}

\begin{itemize}
\tightlist
\item
  \texttt{agent} \textbf{\texttt{get\_belief\_op}} \texttt{predicate}
  ---\textgreater{} \texttt{mental\_state}
\item
  \textbf{\texttt{get\_belief\_op}} (\texttt{agent} ,
  \texttt{predicate}) ---\textgreater{} \texttt{mental\_state}
\end{itemize}

\subsubsection{Result:}\label{result-191}

get the belief in the belief base with the given predicate.

\subsubsection{Examples:}\label{examples-144}

\begin{verbatim}
get_belief_op(self,has_water) 
\end{verbatim}

\begin{center}\rule{0.5\linewidth}{\linethickness}\end{center}

\subsection{\texorpdfstring{\texttt{get\_belief\_with\_name\_op}}{get\_belief\_with\_name\_op}}\label{get_belief_with_name_op}

\subsubsection{Possible use:}\label{possible-use-198}

\begin{itemize}
\tightlist
\item
  \texttt{agent} \textbf{\texttt{get\_belief\_with\_name\_op}}
  \texttt{string} ---\textgreater{} \texttt{mental\_state}
\item
  \textbf{\texttt{get\_belief\_with\_name\_op}} (\texttt{agent} ,
  \texttt{string}) ---\textgreater{} \texttt{mental\_state}
\end{itemize}

\subsubsection{Result:}\label{result-192}

get the belief in the belief base with the given name.

\subsubsection{Examples:}\label{examples-145}

\begin{verbatim}
get_belief_with_name_op(self,"has_water") 
\end{verbatim}

\begin{center}\rule{0.5\linewidth}{\linethickness}\end{center}

\subsection{\texorpdfstring{\texttt{get\_beliefs\_op}}{get\_beliefs\_op}}\label{get_beliefs_op}

\subsubsection{Possible use:}\label{possible-use-199}

\begin{itemize}
\tightlist
\item
  \texttt{agent} \textbf{\texttt{get\_beliefs\_op}} \texttt{predicate}
  ---\textgreater{}
  \texttt{msi.gama.util.IList\textless{}msi.gaml.architecture.simplebdi.MentalState\textgreater{}}
\item
  \textbf{\texttt{get\_beliefs\_op}} (\texttt{agent} ,
  \texttt{predicate}) ---\textgreater{}
  \texttt{msi.gama.util.IList\textless{}msi.gaml.architecture.simplebdi.MentalState\textgreater{}}
\end{itemize}

\subsubsection{Result:}\label{result-193}

get the beliefs in the belief base with the given predicate.

\subsubsection{Examples:}\label{examples-146}

\begin{verbatim}
get_beliefs_op(self,has_water) 
\end{verbatim}

\begin{center}\rule{0.5\linewidth}{\linethickness}\end{center}

\subsection{\texorpdfstring{\texttt{get\_beliefs\_with\_name\_op}}{get\_beliefs\_with\_name\_op}}\label{get_beliefs_with_name_op}

\subsubsection{Possible use:}\label{possible-use-200}

\begin{itemize}
\tightlist
\item
  \texttt{agent} \textbf{\texttt{get\_beliefs\_with\_name\_op}}
  \texttt{string} ---\textgreater{}
  \texttt{msi.gama.util.IList\textless{}msi.gaml.architecture.simplebdi.MentalState\textgreater{}}
\item
  \textbf{\texttt{get\_beliefs\_with\_name\_op}} (\texttt{agent} ,
  \texttt{string}) ---\textgreater{}
  \texttt{msi.gama.util.IList\textless{}msi.gaml.architecture.simplebdi.MentalState\textgreater{}}
\end{itemize}

\subsubsection{Result:}\label{result-194}

get the list of beliefs in the belief base which predicate has the given
name.

\subsubsection{Examples:}\label{examples-147}

\begin{verbatim}
get_beliefs_with_name_op(self,"has_water") 
\end{verbatim}

\begin{center}\rule{0.5\linewidth}{\linethickness}\end{center}

\subsection{\texorpdfstring{\texttt{get\_current\_intention\_op}}{get\_current\_intention\_op}}\label{get_current_intention_op}

\subsubsection{Possible use:}\label{possible-use-201}

\begin{itemize}
\tightlist
\item
  \textbf{\texttt{get\_current\_intention\_op}} (\texttt{agent})
  ---\textgreater{} \texttt{mental\_state}
\end{itemize}

\subsubsection{Result:}\label{result-195}

get the current intention.

\subsubsection{Examples:}\label{examples-148}

\begin{verbatim}
get_current_intention_op(self,has_water) 
\end{verbatim}

\begin{center}\rule{0.5\linewidth}{\linethickness}\end{center}

\subsection{\texorpdfstring{\texttt{get\_decay}}{get\_decay}}\label{get_decay}

\subsubsection{Possible use:}\label{possible-use-202}

\begin{itemize}
\tightlist
\item
  \textbf{\texttt{get\_decay}} (\texttt{emotion}) ---\textgreater{}
  \texttt{float}
\end{itemize}

\subsubsection{Result:}\label{result-196}

get the decay value of the given emotion

\subsubsection{Examples:}\label{examples-149}

\begin{verbatim}
get_decay(emotion) 
\end{verbatim}

\begin{center}\rule{0.5\linewidth}{\linethickness}\end{center}

\subsection{\texorpdfstring{\texttt{get\_desire\_op}}{get\_desire\_op}}\label{get_desire_op}

\subsubsection{Possible use:}\label{possible-use-203}

\begin{itemize}
\tightlist
\item
  \texttt{agent} \textbf{\texttt{get\_desire\_op}} \texttt{predicate}
  ---\textgreater{} \texttt{mental\_state}
\item
  \textbf{\texttt{get\_desire\_op}} (\texttt{agent} ,
  \texttt{predicate}) ---\textgreater{} \texttt{mental\_state}
\end{itemize}

\subsubsection{Result:}\label{result-197}

get the desire in the desire base with the given predicate.

\subsubsection{Examples:}\label{examples-150}

\begin{verbatim}
get_belief_op(self,has_water) 
\end{verbatim}

\begin{center}\rule{0.5\linewidth}{\linethickness}\end{center}

\subsection{\texorpdfstring{\texttt{get\_desire\_with\_name\_op}}{get\_desire\_with\_name\_op}}\label{get_desire_with_name_op}

\subsubsection{Possible use:}\label{possible-use-204}

\begin{itemize}
\tightlist
\item
  \texttt{agent} \textbf{\texttt{get\_desire\_with\_name\_op}}
  \texttt{string} ---\textgreater{} \texttt{mental\_state}
\item
  \textbf{\texttt{get\_desire\_with\_name\_op}} (\texttt{agent} ,
  \texttt{string}) ---\textgreater{} \texttt{mental\_state}
\end{itemize}

\subsubsection{Result:}\label{result-198}

get the desire in the desire base with the given name.

\subsubsection{Examples:}\label{examples-151}

\begin{verbatim}
 
mental_state var0 <- get_desire_with_name_op(self,"has_water"); // var0 equals nil
\end{verbatim}

\begin{center}\rule{0.5\linewidth}{\linethickness}\end{center}

\subsection{\texorpdfstring{\texttt{get\_desires\_op}}{get\_desires\_op}}\label{get_desires_op}

\subsubsection{Possible use:}\label{possible-use-205}

\begin{itemize}
\tightlist
\item
  \texttt{agent} \textbf{\texttt{get\_desires\_op}} \texttt{predicate}
  ---\textgreater{}
  \texttt{msi.gama.util.IList\textless{}msi.gaml.architecture.simplebdi.MentalState\textgreater{}}
\item
  \textbf{\texttt{get\_desires\_op}} (\texttt{agent} ,
  \texttt{predicate}) ---\textgreater{}
  \texttt{msi.gama.util.IList\textless{}msi.gaml.architecture.simplebdi.MentalState\textgreater{}}
\end{itemize}

\subsubsection{Result:}\label{result-199}

get the desires in the desire base with the given predicate.

\subsubsection{Examples:}\label{examples-152}

\begin{verbatim}
get_desires_op(self,has_water) 
\end{verbatim}

\begin{center}\rule{0.5\linewidth}{\linethickness}\end{center}

\subsection{\texorpdfstring{\texttt{get\_desires\_with\_name\_op}}{get\_desires\_with\_name\_op}}\label{get_desires_with_name_op}

\subsubsection{Possible use:}\label{possible-use-206}

\begin{itemize}
\tightlist
\item
  \texttt{agent} \textbf{\texttt{get\_desires\_with\_name\_op}}
  \texttt{string} ---\textgreater{}
  \texttt{msi.gama.util.IList\textless{}msi.gaml.architecture.simplebdi.MentalState\textgreater{}}
\item
  \textbf{\texttt{get\_desires\_with\_name\_op}} (\texttt{agent} ,
  \texttt{string}) ---\textgreater{}
  \texttt{msi.gama.util.IList\textless{}msi.gaml.architecture.simplebdi.MentalState\textgreater{}}
\end{itemize}

\subsubsection{Result:}\label{result-200}

get the list of desires in the desire base which predicate has the given
name.

\subsubsection{Examples:}\label{examples-153}

\begin{verbatim}
get_desires_with_name_op(self,"has_water") 
\end{verbatim}

\begin{center}\rule{0.5\linewidth}{\linethickness}\end{center}

\subsection{\texorpdfstring{\texttt{get\_dominance}}{get\_dominance}}\label{get_dominance}

\subsubsection{Possible use:}\label{possible-use-207}

\begin{itemize}
\tightlist
\item
  \textbf{\texttt{get\_dominance}}
  (\texttt{msi.gaml.architecture.simplebdi.SocialLink})
  ---\textgreater{} \texttt{float}
\end{itemize}

\subsubsection{Result:}\label{result-201}

get the dominance value of the given social link

\subsubsection{Examples:}\label{examples-154}

\begin{verbatim}
get_dominance(social_link1) 
\end{verbatim}

\begin{center}\rule{0.5\linewidth}{\linethickness}\end{center}

\subsection{\texorpdfstring{\texttt{get\_familiarity}}{get\_familiarity}}\label{get_familiarity}

\subsubsection{Possible use:}\label{possible-use-208}

\begin{itemize}
\tightlist
\item
  \textbf{\texttt{get\_familiarity}}
  (\texttt{msi.gaml.architecture.simplebdi.SocialLink})
  ---\textgreater{} \texttt{float}
\end{itemize}

\subsubsection{Result:}\label{result-202}

get the familiarity value of the given social link

\subsubsection{Examples:}\label{examples-155}

\begin{verbatim}
get_familiarity(social_link1) 
\end{verbatim}

\begin{center}\rule{0.5\linewidth}{\linethickness}\end{center}

\subsection{\texorpdfstring{\texttt{get\_ideal\_op}}{get\_ideal\_op}}\label{get_ideal_op}

\subsubsection{Possible use:}\label{possible-use-209}

\begin{itemize}
\tightlist
\item
  \texttt{agent} \textbf{\texttt{get\_ideal\_op}} \texttt{predicate}
  ---\textgreater{} \texttt{mental\_state}
\item
  \textbf{\texttt{get\_ideal\_op}} (\texttt{agent} , \texttt{predicate})
  ---\textgreater{} \texttt{mental\_state}
\end{itemize}

\subsubsection{Result:}\label{result-203}

get the ideal in the ideal base with the given name.

\subsubsection{Examples:}\label{examples-156}

\begin{verbatim}
get_ideal_op(self,has_water) 
\end{verbatim}

\begin{center}\rule{0.5\linewidth}{\linethickness}\end{center}

\subsection{\texorpdfstring{\texttt{get\_ideal\_with\_name\_op}}{get\_ideal\_with\_name\_op}}\label{get_ideal_with_name_op}

\subsubsection{Possible use:}\label{possible-use-210}

\begin{itemize}
\tightlist
\item
  \texttt{agent} \textbf{\texttt{get\_ideal\_with\_name\_op}}
  \texttt{string} ---\textgreater{} \texttt{mental\_state}
\item
  \textbf{\texttt{get\_ideal\_with\_name\_op}} (\texttt{agent} ,
  \texttt{string}) ---\textgreater{} \texttt{mental\_state}
\end{itemize}

\subsubsection{Result:}\label{result-204}

get the ideal in the ideal base with the given name.

\subsubsection{Examples:}\label{examples-157}

\begin{verbatim}
get_ideal_with_name_op(self,"has_water") 
\end{verbatim}

\begin{center}\rule{0.5\linewidth}{\linethickness}\end{center}

\subsection{\texorpdfstring{\texttt{get\_ideals\_op}}{get\_ideals\_op}}\label{get_ideals_op}

\subsubsection{Possible use:}\label{possible-use-211}

\begin{itemize}
\tightlist
\item
  \texttt{agent} \textbf{\texttt{get\_ideals\_op}} \texttt{predicate}
  ---\textgreater{}
  \texttt{msi.gama.util.IList\textless{}msi.gaml.architecture.simplebdi.MentalState\textgreater{}}
\item
  \textbf{\texttt{get\_ideals\_op}} (\texttt{agent} ,
  \texttt{predicate}) ---\textgreater{}
  \texttt{msi.gama.util.IList\textless{}msi.gaml.architecture.simplebdi.MentalState\textgreater{}}
\end{itemize}

\subsubsection{Result:}\label{result-205}

get the ideal in the ideal base with the given name.

\subsubsection{Examples:}\label{examples-158}

\begin{verbatim}
get_ideals_op(self,has_water) 
\end{verbatim}

\begin{center}\rule{0.5\linewidth}{\linethickness}\end{center}

\subsection{\texorpdfstring{\texttt{get\_ideals\_with\_name\_op}}{get\_ideals\_with\_name\_op}}\label{get_ideals_with_name_op}

\subsubsection{Possible use:}\label{possible-use-212}

\begin{itemize}
\tightlist
\item
  \texttt{agent} \textbf{\texttt{get\_ideals\_with\_name\_op}}
  \texttt{string} ---\textgreater{}
  \texttt{msi.gama.util.IList\textless{}msi.gaml.architecture.simplebdi.MentalState\textgreater{}}
\item
  \textbf{\texttt{get\_ideals\_with\_name\_op}} (\texttt{agent} ,
  \texttt{string}) ---\textgreater{}
  \texttt{msi.gama.util.IList\textless{}msi.gaml.architecture.simplebdi.MentalState\textgreater{}}
\end{itemize}

\subsubsection{Result:}\label{result-206}

get the list of ideals in the ideal base which predicate has the given
name.

\subsubsection{Examples:}\label{examples-159}

\begin{verbatim}
get_ideals_with_name_op(self,"has_water") 
\end{verbatim}

\begin{center}\rule{0.5\linewidth}{\linethickness}\end{center}

\subsection{\texorpdfstring{\texttt{get\_intensity}}{get\_intensity}}\label{get_intensity}

\subsubsection{Possible use:}\label{possible-use-213}

\begin{itemize}
\tightlist
\item
  \textbf{\texttt{get\_intensity}} (\texttt{emotion}) ---\textgreater{}
  \texttt{float}
\end{itemize}

\subsubsection{Result:}\label{result-207}

get the intensity value of the given emotion

\subsubsection{Examples:}\label{examples-160}

\begin{verbatim}
emotion set_intensity 12 
\end{verbatim}

\begin{center}\rule{0.5\linewidth}{\linethickness}\end{center}

\subsection{\texorpdfstring{\texttt{get\_intention\_op}}{get\_intention\_op}}\label{get_intention_op}

\subsubsection{Possible use:}\label{possible-use-214}

\begin{itemize}
\tightlist
\item
  \texttt{agent} \textbf{\texttt{get\_intention\_op}} \texttt{predicate}
  ---\textgreater{} \texttt{mental\_state}
\item
  \textbf{\texttt{get\_intention\_op}} (\texttt{agent} ,
  \texttt{predicate}) ---\textgreater{} \texttt{mental\_state}
\end{itemize}

\subsubsection{Result:}\label{result-208}

get the intention in the intention base with the given predicate.

\subsubsection{Examples:}\label{examples-161}

\begin{verbatim}
get_intention_op(self,has_water) 
\end{verbatim}

\begin{center}\rule{0.5\linewidth}{\linethickness}\end{center}

\subsection{\texorpdfstring{\texttt{get\_intention\_with\_name\_op}}{get\_intention\_with\_name\_op}}\label{get_intention_with_name_op}

\subsubsection{Possible use:}\label{possible-use-215}

\begin{itemize}
\tightlist
\item
  \texttt{agent} \textbf{\texttt{get\_intention\_with\_name\_op}}
  \texttt{string} ---\textgreater{} \texttt{mental\_state}
\item
  \textbf{\texttt{get\_intention\_with\_name\_op}} (\texttt{agent} ,
  \texttt{string}) ---\textgreater{} \texttt{mental\_state}
\end{itemize}

\subsubsection{Result:}\label{result-209}

get the intention in the intention base with the given name.

\subsubsection{Examples:}\label{examples-162}

\begin{verbatim}
get_intention_with_name_op(self,"has_water") 
\end{verbatim}

\begin{center}\rule{0.5\linewidth}{\linethickness}\end{center}

\subsection{\texorpdfstring{\texttt{get\_intentions\_op}}{get\_intentions\_op}}\label{get_intentions_op}

\subsubsection{Possible use:}\label{possible-use-216}

\begin{itemize}
\tightlist
\item
  \texttt{agent} \textbf{\texttt{get\_intentions\_op}}
  \texttt{predicate} ---\textgreater{}
  \texttt{msi.gama.util.IList\textless{}msi.gaml.architecture.simplebdi.MentalState\textgreater{}}
\item
  \textbf{\texttt{get\_intentions\_op}} (\texttt{agent} ,
  \texttt{predicate}) ---\textgreater{}
  \texttt{msi.gama.util.IList\textless{}msi.gaml.architecture.simplebdi.MentalState\textgreater{}}
\end{itemize}

\subsubsection{Result:}\label{result-210}

get the intentions in the intention base with the given predicate.

\subsubsection{Examples:}\label{examples-163}

\begin{verbatim}
get_intentions_op(self,has_water) 
\end{verbatim}

\begin{center}\rule{0.5\linewidth}{\linethickness}\end{center}

\subsection{\texorpdfstring{\texttt{get\_intentions\_with\_name\_op}}{get\_intentions\_with\_name\_op}}\label{get_intentions_with_name_op}

\subsubsection{Possible use:}\label{possible-use-217}

\begin{itemize}
\tightlist
\item
  \texttt{agent} \textbf{\texttt{get\_intentions\_with\_name\_op}}
  \texttt{string} ---\textgreater{}
  \texttt{msi.gama.util.IList\textless{}msi.gaml.architecture.simplebdi.MentalState\textgreater{}}
\item
  \textbf{\texttt{get\_intentions\_with\_name\_op}} (\texttt{agent} ,
  \texttt{string}) ---\textgreater{}
  \texttt{msi.gama.util.IList\textless{}msi.gaml.architecture.simplebdi.MentalState\textgreater{}}
\end{itemize}

\subsubsection{Result:}\label{result-211}

get the list of intentions in the intention base which predicate has the
given name.

\subsubsection{Examples:}\label{examples-164}

\begin{verbatim}
get_intentions_with_name_op(self,"has_water") 
\end{verbatim}

\begin{center}\rule{0.5\linewidth}{\linethickness}\end{center}

\subsection{\texorpdfstring{\texttt{get\_lifetime}}{get\_lifetime}}\label{get_lifetime}

\subsubsection{Possible use:}\label{possible-use-218}

\begin{itemize}
\tightlist
\item
  \textbf{\texttt{get\_lifetime}} (\texttt{predicate}) ---\textgreater{}
  \texttt{int}
\item
  \textbf{\texttt{get\_lifetime}} (\texttt{mental\_state})
  ---\textgreater{} \texttt{int}
\end{itemize}

\subsubsection{Result:}\label{result-212}

get the lifetime value of the given mental state

\subsubsection{Examples:}\label{examples-165}

\begin{verbatim}
get_lifetime(mental_state1) 
\end{verbatim}

\begin{center}\rule{0.5\linewidth}{\linethickness}\end{center}

\subsection{\texorpdfstring{\texttt{get\_liking}}{get\_liking}}\label{get_liking}

\subsubsection{Possible use:}\label{possible-use-219}

\begin{itemize}
\tightlist
\item
  \textbf{\texttt{get\_liking}}
  (\texttt{msi.gaml.architecture.simplebdi.SocialLink})
  ---\textgreater{} \texttt{float}
\end{itemize}

\subsubsection{Result:}\label{result-213}

get the liking value of the given social link

\subsubsection{Examples:}\label{examples-166}

\begin{verbatim}
get_liking(social_link1) 
\end{verbatim}

\begin{center}\rule{0.5\linewidth}{\linethickness}\end{center}

\subsection{\texorpdfstring{\texttt{get\_modality}}{get\_modality}}\label{get_modality}

\subsubsection{Possible use:}\label{possible-use-220}

\begin{itemize}
\tightlist
\item
  \textbf{\texttt{get\_modality}} (\texttt{mental\_state})
  ---\textgreater{} \texttt{string}
\end{itemize}

\subsubsection{Result:}\label{result-214}

get the modality value of the given mental state

\subsubsection{Examples:}\label{examples-167}

\begin{verbatim}
get_modality(mental_state1) 
\end{verbatim}

\begin{center}\rule{0.5\linewidth}{\linethickness}\end{center}

\subsection{\texorpdfstring{\texttt{get\_obligation\_op}}{get\_obligation\_op}}\label{get_obligation_op}

\subsubsection{Possible use:}\label{possible-use-221}

\begin{itemize}
\tightlist
\item
  \texttt{agent} \textbf{\texttt{get\_obligation\_op}}
  \texttt{predicate} ---\textgreater{} \texttt{mental\_state}
\item
  \textbf{\texttt{get\_obligation\_op}} (\texttt{agent} ,
  \texttt{predicate}) ---\textgreater{} \texttt{mental\_state}
\end{itemize}

\subsubsection{Result:}\label{result-215}

get the obligation in the obligation base with the given predicate.

\subsubsection{Examples:}\label{examples-168}

\begin{verbatim}
get_obligation_op(self,has_water) 
\end{verbatim}

\begin{center}\rule{0.5\linewidth}{\linethickness}\end{center}

\subsection{\texorpdfstring{\texttt{get\_obligation\_with\_name\_op}}{get\_obligation\_with\_name\_op}}\label{get_obligation_with_name_op}

\subsubsection{Possible use:}\label{possible-use-222}

\begin{itemize}
\tightlist
\item
  \texttt{agent} \textbf{\texttt{get\_obligation\_with\_name\_op}}
  \texttt{string} ---\textgreater{} \texttt{mental\_state}
\item
  \textbf{\texttt{get\_obligation\_with\_name\_op}} (\texttt{agent} ,
  \texttt{string}) ---\textgreater{} \texttt{mental\_state}
\end{itemize}

\subsubsection{Result:}\label{result-216}

get the obligation in the obligation base with the given name.

\subsubsection{Examples:}\label{examples-169}

\begin{verbatim}
get_obligation_with_name_op(self,"has_water") 
\end{verbatim}

\begin{center}\rule{0.5\linewidth}{\linethickness}\end{center}

\subsection{\texorpdfstring{\texttt{get\_obligations\_op}}{get\_obligations\_op}}\label{get_obligations_op}

\subsubsection{Possible use:}\label{possible-use-223}

\begin{itemize}
\tightlist
\item
  \texttt{agent} \textbf{\texttt{get\_obligations\_op}}
  \texttt{predicate} ---\textgreater{}
  \texttt{msi.gama.util.IList\textless{}msi.gaml.architecture.simplebdi.MentalState\textgreater{}}
\item
  \textbf{\texttt{get\_obligations\_op}} (\texttt{agent} ,
  \texttt{predicate}) ---\textgreater{}
  \texttt{msi.gama.util.IList\textless{}msi.gaml.architecture.simplebdi.MentalState\textgreater{}}
\end{itemize}

\subsubsection{Result:}\label{result-217}

get the obligations in the obligation base with the given predicate.

\subsubsection{Examples:}\label{examples-170}

\begin{verbatim}
get_obligations_op(self,has_water) 
\end{verbatim}

\begin{center}\rule{0.5\linewidth}{\linethickness}\end{center}

\subsection{\texorpdfstring{\texttt{get\_obligations\_with\_name\_op}}{get\_obligations\_with\_name\_op}}\label{get_obligations_with_name_op}

\subsubsection{Possible use:}\label{possible-use-224}

\begin{itemize}
\tightlist
\item
  \texttt{agent} \textbf{\texttt{get\_obligations\_with\_name\_op}}
  \texttt{string} ---\textgreater{}
  \texttt{msi.gama.util.IList\textless{}msi.gaml.architecture.simplebdi.MentalState\textgreater{}}
\item
  \textbf{\texttt{get\_obligations\_with\_name\_op}} (\texttt{agent} ,
  \texttt{string}) ---\textgreater{}
  \texttt{msi.gama.util.IList\textless{}msi.gaml.architecture.simplebdi.MentalState\textgreater{}}
\end{itemize}

\subsubsection{Result:}\label{result-218}

get the list of obligations in the obligation base which predicate has
the given name.

\subsubsection{Examples:}\label{examples-171}

\begin{verbatim}
get_obligations_with_name_op(self,"has_water") 
\end{verbatim}

\begin{center}\rule{0.5\linewidth}{\linethickness}\end{center}

\subsection{\texorpdfstring{\texttt{get\_plan\_name}}{get\_plan\_name}}\label{get_plan_name}

\subsubsection{Possible use:}\label{possible-use-225}

\begin{itemize}
\tightlist
\item
  \textbf{\texttt{get\_plan\_name}} (\texttt{BDIPlan}) ---\textgreater{}
  \texttt{string}
\end{itemize}

\subsubsection{Result:}\label{result-219}

get the name of a given plan

\subsubsection{Examples:}\label{examples-172}

\begin{verbatim}
get_plan_name(agent.current_plan) 
\end{verbatim}

\begin{center}\rule{0.5\linewidth}{\linethickness}\end{center}

\subsection{\texorpdfstring{\texttt{get\_predicate}}{get\_predicate}}\label{get_predicate}

\subsubsection{Possible use:}\label{possible-use-226}

\begin{itemize}
\tightlist
\item
  \textbf{\texttt{get\_predicate}} (\texttt{mental\_state})
  ---\textgreater{} \texttt{predicate}
\end{itemize}

\subsubsection{Result:}\label{result-220}

get the predicate value of the given mental state

\subsubsection{Examples:}\label{examples-173}

\begin{verbatim}
get_predicate(mental_state1) 
\end{verbatim}

\begin{center}\rule{0.5\linewidth}{\linethickness}\end{center}

\subsection{\texorpdfstring{\texttt{get\_solidarity}}{get\_solidarity}}\label{get_solidarity}

\subsubsection{Possible use:}\label{possible-use-227}

\begin{itemize}
\tightlist
\item
  \textbf{\texttt{get\_solidarity}}
  (\texttt{msi.gaml.architecture.simplebdi.SocialLink})
  ---\textgreater{} \texttt{float}
\end{itemize}

\subsubsection{Result:}\label{result-221}

get the solidarity value of the given social link

\subsubsection{Examples:}\label{examples-174}

\begin{verbatim}
get_solidarity(social_link1) 
\end{verbatim}

\begin{center}\rule{0.5\linewidth}{\linethickness}\end{center}

\subsection{\texorpdfstring{\texttt{get\_strength}}{get\_strength}}\label{get_strength}

\subsubsection{Possible use:}\label{possible-use-228}

\begin{itemize}
\tightlist
\item
  \textbf{\texttt{get\_strength}} (\texttt{mental\_state})
  ---\textgreater{} \texttt{float}
\end{itemize}

\subsubsection{Result:}\label{result-222}

get the strength value of the given mental state

\subsubsection{Examples:}\label{examples-175}

\begin{verbatim}
get_strength(mental_state1) 
\end{verbatim}

\begin{center}\rule{0.5\linewidth}{\linethickness}\end{center}

\subsection{\texorpdfstring{\texttt{get\_super\_intention}}{get\_super\_intention}}\label{get_super_intention}

\subsubsection{Possible use:}\label{possible-use-229}

\begin{itemize}
\tightlist
\item
  \textbf{\texttt{get\_super\_intention}} (\texttt{predicate})
  ---\textgreater{} \texttt{mental\_state}
\end{itemize}

\begin{center}\rule{0.5\linewidth}{\linethickness}\end{center}

\subsection{\texorpdfstring{\texttt{get\_trust}}{get\_trust}}\label{get_trust}

\subsubsection{Possible use:}\label{possible-use-230}

\begin{itemize}
\tightlist
\item
  \textbf{\texttt{get\_trust}}
  (\texttt{msi.gaml.architecture.simplebdi.SocialLink})
  ---\textgreater{} \texttt{float}
\end{itemize}

\subsubsection{Result:}\label{result-223}

get the familiarity value of the given social link

\subsubsection{Examples:}\label{examples-176}

\begin{verbatim}
get_familiarity(social_link1) 
\end{verbatim}

\begin{center}\rule{0.5\linewidth}{\linethickness}\end{center}

\subsection{\texorpdfstring{\texttt{get\_truth}}{get\_truth}}\label{get_truth}

\subsubsection{Possible use:}\label{possible-use-231}

\begin{itemize}
\tightlist
\item
  \textbf{\texttt{get\_truth}} (\texttt{predicate}) ---\textgreater{}
  \texttt{bool}
\end{itemize}

\begin{center}\rule{0.5\linewidth}{\linethickness}\end{center}

\subsection{\texorpdfstring{\texttt{get\_uncertainties\_op}}{get\_uncertainties\_op}}\label{get_uncertainties_op}

\subsubsection{Possible use:}\label{possible-use-232}

\begin{itemize}
\tightlist
\item
  \texttt{agent} \textbf{\texttt{get\_uncertainties\_op}}
  \texttt{predicate} ---\textgreater{}
  \texttt{msi.gama.util.IList\textless{}msi.gaml.architecture.simplebdi.MentalState\textgreater{}}
\item
  \textbf{\texttt{get\_uncertainties\_op}} (\texttt{agent} ,
  \texttt{predicate}) ---\textgreater{}
  \texttt{msi.gama.util.IList\textless{}msi.gaml.architecture.simplebdi.MentalState\textgreater{}}
\end{itemize}

\subsubsection{Result:}\label{result-224}

get the uncertainties in the uncertainty base with the given predicate.

\subsubsection{Examples:}\label{examples-177}

\begin{verbatim}
get_uncertinties_op(self,has_water) 
\end{verbatim}

\begin{center}\rule{0.5\linewidth}{\linethickness}\end{center}

\subsection{\texorpdfstring{\texttt{get\_uncertainties\_with\_name\_op}}{get\_uncertainties\_with\_name\_op}}\label{get_uncertainties_with_name_op}

\subsubsection{Possible use:}\label{possible-use-233}

\begin{itemize}
\tightlist
\item
  \texttt{agent} \textbf{\texttt{get\_uncertainties\_with\_name\_op}}
  \texttt{string} ---\textgreater{}
  \texttt{msi.gama.util.IList\textless{}msi.gaml.architecture.simplebdi.MentalState\textgreater{}}
\item
  \textbf{\texttt{get\_uncertainties\_with\_name\_op}} (\texttt{agent} ,
  \texttt{string}) ---\textgreater{}
  \texttt{msi.gama.util.IList\textless{}msi.gaml.architecture.simplebdi.MentalState\textgreater{}}
\end{itemize}

\subsubsection{Result:}\label{result-225}

get the list of uncertainties in the uncertainty base which predicate
has the given name.

\subsubsection{Examples:}\label{examples-178}

\begin{verbatim}
get_uncertainties_with_name_op(self,"has_water") 
\end{verbatim}

\begin{center}\rule{0.5\linewidth}{\linethickness}\end{center}

\subsection{\texorpdfstring{\texttt{get\_uncertainty\_op}}{get\_uncertainty\_op}}\label{get_uncertainty_op}

\subsubsection{Possible use:}\label{possible-use-234}

\begin{itemize}
\tightlist
\item
  \texttt{agent} \textbf{\texttt{get\_uncertainty\_op}}
  \texttt{predicate} ---\textgreater{} \texttt{mental\_state}
\item
  \textbf{\texttt{get\_uncertainty\_op}} (\texttt{agent} ,
  \texttt{predicate}) ---\textgreater{} \texttt{mental\_state}
\end{itemize}

\subsubsection{Result:}\label{result-226}

get the uncertainty in the uncertainty base with the given predicate.

\subsubsection{Examples:}\label{examples-179}

\begin{verbatim}
get_uncertainty_op(self,has_water) 
\end{verbatim}

\begin{center}\rule{0.5\linewidth}{\linethickness}\end{center}

\subsection{\texorpdfstring{\texttt{get\_uncertainty\_with\_name\_op}}{get\_uncertainty\_with\_name\_op}}\label{get_uncertainty_with_name_op}

\subsubsection{Possible use:}\label{possible-use-235}

\begin{itemize}
\tightlist
\item
  \texttt{agent} \textbf{\texttt{get\_uncertainty\_with\_name\_op}}
  \texttt{string} ---\textgreater{} \texttt{mental\_state}
\item
  \textbf{\texttt{get\_uncertainty\_with\_name\_op}} (\texttt{agent} ,
  \texttt{string}) ---\textgreater{} \texttt{mental\_state}
\end{itemize}

\subsubsection{Result:}\label{result-227}

get the uncertainty in the uncertainty base with the given name.

\subsubsection{Examples:}\label{examples-180}

\begin{verbatim}
get_uncertainty_with_name_op(self,"has_water") 
\end{verbatim}

\begin{center}\rule{0.5\linewidth}{\linethickness}\end{center}

\subsection{\texorpdfstring{\texttt{gif\_file}}{gif\_file}}\label{gif_file}

\subsubsection{Possible use:}\label{possible-use-236}

\begin{itemize}
\tightlist
\item
  \textbf{\texttt{gif\_file}} (\texttt{string}) ---\textgreater{}
  \texttt{file}
\end{itemize}

\subsubsection{Result:}\label{result-228}

Constructs a file of type gif. Allowed extensions are limited to gif

\begin{center}\rule{0.5\linewidth}{\linethickness}\end{center}

\subsection{\texorpdfstring{\texttt{gini}}{gini}}\label{gini}

\subsubsection{Possible use:}\label{possible-use-237}

\begin{itemize}
\tightlist
\item
  \textbf{\texttt{gini}} (\texttt{list\textless{}float\textgreater{}})
  ---\textgreater{} \texttt{float}
\end{itemize}

\subsubsection{Special cases:}\label{special-cases-72}

\begin{itemize}
\tightlist
\item
  return the Gini Index of the given list of values (list of floats)
\end{itemize}

\begin{verbatim}
 
float var0 <- gini([1.0, 0.5, 2.0]); // var0 equals the gini index computed
\end{verbatim}

\begin{center}\rule{0.5\linewidth}{\linethickness}\end{center}

\subsection{\texorpdfstring{\texttt{gml\_file}}{gml\_file}}\label{gml_file}

\subsubsection{Possible use:}\label{possible-use-238}

\begin{itemize}
\tightlist
\item
  \textbf{\texttt{gml\_file}} (\texttt{string}) ---\textgreater{}
  \texttt{file}
\end{itemize}

\subsubsection{Result:}\label{result-229}

Constructs a file of type gml. Allowed extensions are limited to gml

\begin{center}\rule{0.5\linewidth}{\linethickness}\end{center}

\subsection{\texorpdfstring{\texttt{graph}}{graph}}\label{graph}

\subsubsection{Possible use:}\label{possible-use-239}

\begin{itemize}
\tightlist
\item
  \textbf{\texttt{graph}} (\texttt{any}) ---\textgreater{}
  \texttt{graph}
\end{itemize}

\subsubsection{Result:}\label{result-230}

Casts the operand into the type graph

\begin{center}\rule{0.5\linewidth}{\linethickness}\end{center}

\subsection{\texorpdfstring{\texttt{grayscale}}{grayscale}}\label{grayscale}

\subsubsection{Possible use:}\label{possible-use-240}

\begin{itemize}
\tightlist
\item
  \textbf{\texttt{grayscale}} (\texttt{rgb}) ---\textgreater{}
  \texttt{rgb}
\end{itemize}

\subsubsection{Result:}\label{result-231}

Converts rgb color to grayscale value

\subsubsection{Comment:}\label{comment-44}

r=red, g=green, b=blue. Between 0 and 255 and gray = 0.299 \texttt{*}
red + 0.587 \texttt{*} green + 0.114 \texttt{*} blue (Photoshop value)

\subsubsection{Examples:}\label{examples-181}

\begin{verbatim}
 
rgb var0 <- grayscale (rgb(255,0,0)); // var0 equals to a dark grey
\end{verbatim}

\subsubsection{See also:}\label{see-also-108}

\href{OperatorsNR\#rgb}{rgb}, \href{OperatorsDH\#hsb}{hsb},

\begin{center}\rule{0.5\linewidth}{\linethickness}\end{center}

\subsection{\texorpdfstring{\texttt{grid\_at}}{grid\_at}}\label{grid_at}

\subsubsection{Possible use:}\label{possible-use-241}

\begin{itemize}
\tightlist
\item
  \texttt{species} \textbf{\texttt{grid\_at}} \texttt{point}
  ---\textgreater{} \texttt{agent}
\item
  \textbf{\texttt{grid\_at}} (\texttt{species} , \texttt{point})
  ---\textgreater{} \texttt{agent}
\end{itemize}

\subsubsection{Result:}\label{result-232}

returns the cell of the grid (right-hand operand) at the position given
by the right-hand operand

\subsubsection{Comment:}\label{comment-45}

If the left-hand operand is a point of floats, it is used as a point of
ints.

\subsubsection{Special cases:}\label{special-cases-73}

\begin{itemize}
\tightlist
\item
  if the left-hand operand is not a grid cell species, returns nil
\end{itemize}

\subsubsection{Examples:}\label{examples-182}

\begin{verbatim}
 
agent var0 <- grid_cell grid_at {1,2}; // var0 equals the agent grid_cell with grid_x=1 and grid_y = 2
\end{verbatim}

\begin{center}\rule{0.5\linewidth}{\linethickness}\end{center}

\subsection{\texorpdfstring{\texttt{grid\_cells\_to\_graph}}{grid\_cells\_to\_graph}}\label{grid_cells_to_graph}

\subsubsection{Possible use:}\label{possible-use-242}

\begin{itemize}
\tightlist
\item
  \textbf{\texttt{grid\_cells\_to\_graph}} (\texttt{container})
  ---\textgreater{} \texttt{graph}
\end{itemize}

\subsubsection{Result:}\label{result-233}

creates a graph from a list of cells (operand). An edge is created
between neighbors.

\subsubsection{Examples:}\label{examples-183}

\begin{verbatim}
my_cell_graph<-grid_cells_to_graph(cells_list) 
\end{verbatim}

\begin{center}\rule{0.5\linewidth}{\linethickness}\end{center}

\subsection{\texorpdfstring{\texttt{grid\_file}}{grid\_file}}\label{grid_file}

\subsubsection{Possible use:}\label{possible-use-243}

\begin{itemize}
\tightlist
\item
  \textbf{\texttt{grid\_file}} (\texttt{string}) ---\textgreater{}
  \texttt{file}
\end{itemize}

\subsubsection{Result:}\label{result-234}

Constructs a file of type grid. Allowed extensions are limited to asc,
tif

\begin{center}\rule{0.5\linewidth}{\linethickness}\end{center}

\subsection{\texorpdfstring{\texttt{group\_by}}{group\_by}}\label{group_by}

\subsubsection{Possible use:}\label{possible-use-244}

\begin{itemize}
\tightlist
\item
  \texttt{container} \textbf{\texttt{group\_by}}
  \texttt{any\ expression} ---\textgreater{} \texttt{map}
\item
  \textbf{\texttt{group\_by}} (\texttt{container} ,
  \texttt{any\ expression}) ---\textgreater{} \texttt{map}
\end{itemize}

\subsubsection{Result:}\label{result-235}

Returns a map, where the keys take the possible values of the right-hand
operand and the map values are the list of elements of the left-hand
operand associated to the key value

\subsubsection{Comment:}\label{comment-46}

in the right-hand operand, the keyword each can be used to represent, in
turn, each of the right-hand operand elements.

\subsubsection{Special cases:}\label{special-cases-74}

\begin{itemize}
\tightlist
\item
  if the left-hand operand is nil, group\_by throws an error
\end{itemize}

\subsubsection{Examples:}\label{examples-184}

\begin{verbatim}
 
map var0 <- [1,2,3,4,5,6,7,8] group_by (each > 3); // var0 equals [false::[1, 2, 3], true::[4, 5, 6, 7, 8]] 
map var1 <- g2 group_by (length(g2 out_edges_of each) ); // var1 equals [ 0::[node9, node7, node10, node8, node11], 1::[node6], 2::[node5], 3::[node4]] 
map var2 <- (list(node) group_by (round(node(each).location.x)); // var2 equals [32::[node5], 21::[node1], 4::[node0], 66::[node2], 96::[node3]] 
map<bool,list> var3 <- [1::2, 3::4, 5::6] group_by (each > 4); // var3 equals [false::[2, 4], true::[6]]
\end{verbatim}

\subsubsection{See also:}\label{see-also-109}

\href{OperatorsDH\#first_with}{first\_with},
\href{OperatorsIM\#last_with}{last\_with},
\href{OperatorsSZ\#where}{where},

\begin{center}\rule{0.5\linewidth}{\linethickness}\end{center}

\subsection{\texorpdfstring{\texttt{harmonic\_mean}}{harmonic\_mean}}\label{harmonic_mean}

\subsubsection{Possible use:}\label{possible-use-245}

\begin{itemize}
\tightlist
\item
  \textbf{\texttt{harmonic\_mean}} (\texttt{container})
  ---\textgreater{} \texttt{float}
\end{itemize}

\subsubsection{Result:}\label{result-236}

the harmonic mean of the elements of the operand. See Harmonic\_mean for
more details.

\subsubsection{Comment:}\label{comment-47}

The operator casts all the numerical element of the list into float. The
elements that are not numerical are discarded.

\subsubsection{Examples:}\label{examples-185}

\begin{verbatim}
 
float var0 <- harmonic_mean ([4.5, 3.5, 5.5, 7.0]); // var0 equals 4.804159445407279
\end{verbatim}

\subsubsection{See also:}\label{see-also-110}

\href{OperatorsIM\#mean}{mean}, \href{OperatorsIM\#median}{median},
\href{OperatorsDH\#geometric_mean}{geometric\_mean},

\begin{center}\rule{0.5\linewidth}{\linethickness}\end{center}

\subsection{\texorpdfstring{\texttt{has\_belief\_op}}{has\_belief\_op}}\label{has_belief_op}

\subsubsection{Possible use:}\label{possible-use-246}

\begin{itemize}
\tightlist
\item
  \texttt{agent} \textbf{\texttt{has\_belief\_op}} \texttt{predicate}
  ---\textgreater{} \texttt{bool}
\item
  \textbf{\texttt{has\_belief\_op}} (\texttt{agent} ,
  \texttt{predicate}) ---\textgreater{} \texttt{bool}
\end{itemize}

\subsubsection{Result:}\label{result-237}

indicates if there already is a belief about the given predicate.

\subsubsection{Examples:}\label{examples-186}

\begin{verbatim}
has_belief_op(self,has_water) 
\end{verbatim}

\begin{center}\rule{0.5\linewidth}{\linethickness}\end{center}

\subsection{\texorpdfstring{\texttt{has\_belief\_with\_name\_op}}{has\_belief\_with\_name\_op}}\label{has_belief_with_name_op}

\subsubsection{Possible use:}\label{possible-use-247}

\begin{itemize}
\tightlist
\item
  \texttt{agent} \textbf{\texttt{has\_belief\_with\_name\_op}}
  \texttt{string} ---\textgreater{} \texttt{bool}
\item
  \textbf{\texttt{has\_belief\_with\_name\_op}} (\texttt{agent} ,
  \texttt{string}) ---\textgreater{} \texttt{bool}
\end{itemize}

\subsubsection{Result:}\label{result-238}

indicates if there already is a belief about the given name.

\subsubsection{Examples:}\label{examples-187}

\begin{verbatim}
has_belief_with_name_op(self,"has_water") 
\end{verbatim}

\begin{center}\rule{0.5\linewidth}{\linethickness}\end{center}

\subsection{\texorpdfstring{\texttt{has\_desire\_op}}{has\_desire\_op}}\label{has_desire_op}

\subsubsection{Possible use:}\label{possible-use-248}

\begin{itemize}
\tightlist
\item
  \texttt{agent} \textbf{\texttt{has\_desire\_op}} \texttt{predicate}
  ---\textgreater{} \texttt{bool}
\item
  \textbf{\texttt{has\_desire\_op}} (\texttt{agent} ,
  \texttt{predicate}) ---\textgreater{} \texttt{bool}
\end{itemize}

\subsubsection{Result:}\label{result-239}

indicates if there already is a desire about the given predicate.

\subsubsection{Examples:}\label{examples-188}

\begin{verbatim}
has_desire_op(self,has_water) 
\end{verbatim}

\begin{center}\rule{0.5\linewidth}{\linethickness}\end{center}

\subsection{\texorpdfstring{\texttt{has\_desire\_with\_name\_op}}{has\_desire\_with\_name\_op}}\label{has_desire_with_name_op}

\subsubsection{Possible use:}\label{possible-use-249}

\begin{itemize}
\tightlist
\item
  \texttt{agent} \textbf{\texttt{has\_desire\_with\_name\_op}}
  \texttt{string} ---\textgreater{} \texttt{bool}
\item
  \textbf{\texttt{has\_desire\_with\_name\_op}} (\texttt{agent} ,
  \texttt{string}) ---\textgreater{} \texttt{bool}
\end{itemize}

\subsubsection{Result:}\label{result-240}

indicates if there already is a desire about the given name.

\subsubsection{Examples:}\label{examples-189}

\begin{verbatim}
has_desire_with_name_op(self,"has_water") 
\end{verbatim}

\begin{center}\rule{0.5\linewidth}{\linethickness}\end{center}

\subsection{\texorpdfstring{\texttt{has\_ideal\_op}}{has\_ideal\_op}}\label{has_ideal_op}

\subsubsection{Possible use:}\label{possible-use-250}

\begin{itemize}
\tightlist
\item
  \texttt{agent} \textbf{\texttt{has\_ideal\_op}} \texttt{predicate}
  ---\textgreater{} \texttt{bool}
\item
  \textbf{\texttt{has\_ideal\_op}} (\texttt{agent} , \texttt{predicate})
  ---\textgreater{} \texttt{bool}
\end{itemize}

\subsubsection{Result:}\label{result-241}

indicates if there already is an ideal about the given predicate.

\subsubsection{Examples:}\label{examples-190}

\begin{verbatim}
has_ideal_op(self,has_water) 
\end{verbatim}

\begin{center}\rule{0.5\linewidth}{\linethickness}\end{center}

\subsection{\texorpdfstring{\texttt{has\_ideal\_with\_name\_op}}{has\_ideal\_with\_name\_op}}\label{has_ideal_with_name_op}

\subsubsection{Possible use:}\label{possible-use-251}

\begin{itemize}
\tightlist
\item
  \texttt{agent} \textbf{\texttt{has\_ideal\_with\_name\_op}}
  \texttt{string} ---\textgreater{} \texttt{bool}
\item
  \textbf{\texttt{has\_ideal\_with\_name\_op}} (\texttt{agent} ,
  \texttt{string}) ---\textgreater{} \texttt{bool}
\end{itemize}

\subsubsection{Result:}\label{result-242}

indicates if there already is an ideal about the given name.

\subsubsection{Examples:}\label{examples-191}

\begin{verbatim}
has_ideal_with_name_op(self,"has_water") 
\end{verbatim}

\begin{center}\rule{0.5\linewidth}{\linethickness}\end{center}

\subsection{\texorpdfstring{\texttt{has\_intention\_op}}{has\_intention\_op}}\label{has_intention_op}

\subsubsection{Possible use:}\label{possible-use-252}

\begin{itemize}
\tightlist
\item
  \texttt{agent} \textbf{\texttt{has\_intention\_op}} \texttt{predicate}
  ---\textgreater{} \texttt{bool}
\item
  \textbf{\texttt{has\_intention\_op}} (\texttt{agent} ,
  \texttt{predicate}) ---\textgreater{} \texttt{bool}
\end{itemize}

\subsubsection{Result:}\label{result-243}

indicates if there already is an intention about the given predicate.

\subsubsection{Examples:}\label{examples-192}

\begin{verbatim}
has_intention_op(self,has_water) 
\end{verbatim}

\begin{center}\rule{0.5\linewidth}{\linethickness}\end{center}

\subsection{\texorpdfstring{\texttt{has\_intention\_with\_name\_op}}{has\_intention\_with\_name\_op}}\label{has_intention_with_name_op}

\subsubsection{Possible use:}\label{possible-use-253}

\begin{itemize}
\tightlist
\item
  \texttt{agent} \textbf{\texttt{has\_intention\_with\_name\_op}}
  \texttt{string} ---\textgreater{} \texttt{bool}
\item
  \textbf{\texttt{has\_intention\_with\_name\_op}} (\texttt{agent} ,
  \texttt{string}) ---\textgreater{} \texttt{bool}
\end{itemize}

\subsubsection{Result:}\label{result-244}

indicates if there already is an intention about the given name.

\subsubsection{Examples:}\label{examples-193}

\begin{verbatim}
has_intention_with_name_op(self,"has_water") 
\end{verbatim}

\begin{center}\rule{0.5\linewidth}{\linethickness}\end{center}

\subsection{\texorpdfstring{\texttt{has\_obligation\_op}}{has\_obligation\_op}}\label{has_obligation_op}

\subsubsection{Possible use:}\label{possible-use-254}

\begin{itemize}
\tightlist
\item
  \texttt{agent} \textbf{\texttt{has\_obligation\_op}}
  \texttt{predicate} ---\textgreater{} \texttt{bool}
\item
  \textbf{\texttt{has\_obligation\_op}} (\texttt{agent} ,
  \texttt{predicate}) ---\textgreater{} \texttt{bool}
\end{itemize}

\subsubsection{Result:}\label{result-245}

indicates if there already is an obligation about the given predicate.

\subsubsection{Examples:}\label{examples-194}

\begin{verbatim}
has_obligation_op(self,has_water) 
\end{verbatim}

\begin{center}\rule{0.5\linewidth}{\linethickness}\end{center}

\subsection{\texorpdfstring{\texttt{has\_obligation\_with\_name\_op}}{has\_obligation\_with\_name\_op}}\label{has_obligation_with_name_op}

\subsubsection{Possible use:}\label{possible-use-255}

\begin{itemize}
\tightlist
\item
  \texttt{agent} \textbf{\texttt{has\_obligation\_with\_name\_op}}
  \texttt{string} ---\textgreater{} \texttt{bool}
\item
  \textbf{\texttt{has\_obligation\_with\_name\_op}} (\texttt{agent} ,
  \texttt{string}) ---\textgreater{} \texttt{bool}
\end{itemize}

\subsubsection{Result:}\label{result-246}

indicates if there already is an obligation about the given name.

\subsubsection{Examples:}\label{examples-195}

\begin{verbatim}
has_obligation_with_name_op(self,"has_water") 
\end{verbatim}

\begin{center}\rule{0.5\linewidth}{\linethickness}\end{center}

\subsection{\texorpdfstring{\texttt{has\_uncertainty\_op}}{has\_uncertainty\_op}}\label{has_uncertainty_op}

\subsubsection{Possible use:}\label{possible-use-256}

\begin{itemize}
\tightlist
\item
  \texttt{agent} \textbf{\texttt{has\_uncertainty\_op}}
  \texttt{predicate} ---\textgreater{} \texttt{bool}
\item
  \textbf{\texttt{has\_uncertainty\_op}} (\texttt{agent} ,
  \texttt{predicate}) ---\textgreater{} \texttt{bool}
\end{itemize}

\subsubsection{Result:}\label{result-247}

indicates if there already is an uncertainty about the given predicate.

\subsubsection{Examples:}\label{examples-196}

\begin{verbatim}
has_uncertainty_op(self,has_water) 
\end{verbatim}

\begin{center}\rule{0.5\linewidth}{\linethickness}\end{center}

\subsection{\texorpdfstring{\texttt{has\_uncertainty\_with\_name\_op}}{has\_uncertainty\_with\_name\_op}}\label{has_uncertainty_with_name_op}

\subsubsection{Possible use:}\label{possible-use-257}

\begin{itemize}
\tightlist
\item
  \texttt{agent} \textbf{\texttt{has\_uncertainty\_with\_name\_op}}
  \texttt{string} ---\textgreater{} \texttt{bool}
\item
  \textbf{\texttt{has\_uncertainty\_with\_name\_op}} (\texttt{agent} ,
  \texttt{string}) ---\textgreater{} \texttt{bool}
\end{itemize}

\subsubsection{Result:}\label{result-248}

indicates if there already is an uncertainty about the given name.

\subsubsection{Examples:}\label{examples-197}

\begin{verbatim}
has_uncertainty_with_name_op(self,"has_water") 
\end{verbatim}

\begin{center}\rule{0.5\linewidth}{\linethickness}\end{center}

\subsection{\texorpdfstring{\texttt{hexagon}}{hexagon}}\label{hexagon}

\subsubsection{Possible use:}\label{possible-use-258}

\begin{itemize}
\tightlist
\item
  \textbf{\texttt{hexagon}} (\texttt{point}) ---\textgreater{}
  \texttt{geometry}
\item
  \textbf{\texttt{hexagon}} (\texttt{float}) ---\textgreater{}
  \texttt{geometry}
\item
  \texttt{float} \textbf{\texttt{hexagon}} \texttt{float}
  ---\textgreater{} \texttt{geometry}
\item
  \textbf{\texttt{hexagon}} (\texttt{float} , \texttt{float})
  ---\textgreater{} \texttt{geometry}
\end{itemize}

\subsubsection{Result:}\label{result-249}

A hexagon geometry which the given with and height

\subsubsection{Comment:}\label{comment-48}

the center of the hexagon is by default the location of the current
agent in which has been called this operator.the center of the hexagon
is by default the location of the current agent in which has been called
this operator.the center of the hexagon is by default the location of
the current agent in which has been called this operator.

\subsubsection{Special cases:}\label{special-cases-75}

\begin{itemize}
\tightlist
\item
  returns nil if the operand is nil.\\
\item
  returns nil if the operand is nil.\\
\item
  returns nil if the operand is nil.
\end{itemize}

\subsubsection{Examples:}\label{examples-198}

\begin{verbatim}
 
geometry var0 <- hexagon({10,5}); // var0 equals a geometry as a hexagon of width of 10 and height of 5. 
geometry var1 <- hexagon(10); // var1 equals a geometry as a hexagon of width of 10 and height of 10. 
geometry var2 <- hexagon(10,5); // var2 equals a geometry as a hexagon of width of 10 and height of 5.
\end{verbatim}

\subsubsection{See also:}\label{see-also-111}

\href{OperatorsAA\#around}{around}, \href{OperatorsBC\#circle}{circle},
\href{OperatorsBC\#cone}{cone}, \href{OperatorsIM\#line}{line},
\href{OperatorsIM\#link}{link}, \href{OperatorsNR\#norm}{norm},
\href{OperatorsNR\#point}{point}, \href{OperatorsNR\#polygon}{polygon},
\href{OperatorsNR\#polyline}{polyline},
\href{OperatorsNR\#rectangle}{rectangle},
\href{OperatorsSZ\#triangle}{triangle},

\begin{center}\rule{0.5\linewidth}{\linethickness}\end{center}

\subsection{\texorpdfstring{\texttt{hierarchical\_clustering}}{hierarchical\_clustering}}\label{hierarchical_clustering}

\subsubsection{Possible use:}\label{possible-use-259}

\begin{itemize}
\tightlist
\item
  \texttt{container\textless{}agent\textgreater{}}
  \textbf{\texttt{hierarchical\_clustering}} \texttt{float}
  ---\textgreater{} \texttt{list}
\item
  \textbf{\texttt{hierarchical\_clustering}}
  (\texttt{container\textless{}agent\textgreater{}} , \texttt{float})
  ---\textgreater{} \texttt{list}
\end{itemize}

\subsubsection{Result:}\label{result-250}

A tree (list of list) contained groups of agents clustered by distance
considering a distance min between two groups.

\subsubsection{Comment:}\label{comment-49}

use of hierarchical clustering with Minimum for linkage criterion
between two groups of agents.

\subsubsection{Examples:}\label{examples-199}

\begin{verbatim}
 
list var0 <- [ag1, ag2, ag3, ag4, ag5] hierarchical_clustering 20.0; // var0 equals for example, can return [[[ag1],[ag3]], [ag2], [[[ag4],[ag5]],[ag6]]
\end{verbatim}

\subsubsection{See also:}\label{see-also-112}

\href{OperatorsSZ\#simple_clustering_by_distance}{simple\_clustering\_by\_distance},

\begin{center}\rule{0.5\linewidth}{\linethickness}\end{center}

\subsection{\texorpdfstring{\texttt{horizontal}}{horizontal}}\label{horizontal}

\subsubsection{Possible use:}\label{possible-use-260}

\begin{itemize}
\tightlist
\item
  \textbf{\texttt{horizontal}}
  (\texttt{msi.gama.util.GamaMap\textless{}java.lang.Object,java.lang.Integer\textgreater{}})
  ---\textgreater{}
  \texttt{msi.gama.util.tree.GamaNode\textless{}java.lang.String\textgreater{}}
\end{itemize}

\begin{center}\rule{0.5\linewidth}{\linethickness}\end{center}

\subsection{\texorpdfstring{\texttt{hsb}}{hsb}}\label{hsb}

\subsubsection{Possible use:}\label{possible-use-261}

\begin{itemize}
\tightlist
\item
  \textbf{\texttt{hsb}} (\texttt{float}, \texttt{float}, \texttt{float})
  ---\textgreater{} \texttt{rgb}
\item
  \textbf{\texttt{hsb}} (\texttt{float}, \texttt{float}, \texttt{float},
  \texttt{int}) ---\textgreater{} \texttt{rgb}
\item
  \textbf{\texttt{hsb}} (\texttt{float}, \texttt{float}, \texttt{float},
  \texttt{float}) ---\textgreater{} \texttt{rgb}
\end{itemize}

\subsubsection{Result:}\label{result-251}

Converts hsb (h=hue, s=saturation, b=brightness) value to Gama color

\subsubsection{Comment:}\label{comment-50}

h,s and b components should be floating-point values between 0.0 and 1.0
and when used alpha should be an integer (between 0 and 255) or a float
(between 0 and 1) . Examples: Red=(0.0,1.0,1.0), Yellow=(0.16,1.0,1.0),
Green=(0.33,1.0,1.0), Cyan=(0.5,1.0,1.0), Blue=(0.66,1.0,1.0),
Magenta=(0.83,1.0,1.0)

\subsubsection{Examples:}\label{examples-200}

\begin{verbatim}
 
rgb var0 <- hsb (0.5,1.0,1.0,0.0); // var0 equals rgb("cyan",0) 
rgb var1 <- hsb (0.0,1.0,1.0); // var1 equals rgb("red")
\end{verbatim}

\subsubsection{See also:}\label{see-also-113}

\href{OperatorsNR\#rgb}{rgb},

\begin{center}\rule{0.5\linewidth}{\linethickness}\end{center}

\subsection{\texorpdfstring{\texttt{hypot}}{hypot}}\label{hypot}

\subsubsection{Possible use:}\label{possible-use-262}

\begin{itemize}
\tightlist
\item
  \textbf{\texttt{hypot}} (\texttt{float}, \texttt{float},
  \texttt{float}, \texttt{float}) ---\textgreater{} \texttt{float}
\end{itemize}

\subsubsection{Result:}\label{result-252}

Returns sqrt(x2 +y2) without intermediate overflow or underflow.

\subsubsection{Special cases:}\label{special-cases-76}

\begin{itemize}
\tightlist
\item
  If either argument is infinite, then the result is positive infinity.
  If either argument is NaN and neither argument is infinite, then the
  result is NaN.
\end{itemize}

\subsubsection{Examples:}\label{examples-201}

\begin{verbatim}
 
float var0 <- hypot(0,1,0,1); // var0 equals sqrt(2)
\end{verbatim}

\chapter{Operators (I to M)}\label{operators-i-to-m}

\begin{center}\rule{0.5\linewidth}{\linethickness}\end{center}

\textbf{This file is automatically generated from java files. Do Not
Edit It.}

\begin{center}\rule{0.5\linewidth}{\linethickness}\end{center}

\section{Definition}\label{definition-3}

Operators in the GAML language are used to compose complex expressions.
An operator performs a function on one, two, or n operands (which are
other expressions and thus may be themselves composed of operators) and
returns the result of this function.

Most of them use a classical prefixed functional syntax (i.e.
\texttt{operator\_name(operand1,\ operand2,\ operand3)}, see below),
with the exception of arithmetic (e.g. \texttt{+}, \texttt{/}), logical
(\texttt{and}, \texttt{or}), comparison (e.g. \texttt{\textgreater{}},
\texttt{\textless{}}), access (\texttt{.}, \texttt{{[}..{]}}) and pair
(\texttt{::}) operators, which require an infixed notation (i.e.
\texttt{operand1\ operator\_symbol\ operand1}).

The ternary functional if-else operator, \texttt{?\ :}, uses a special
infixed syntax composed with two symbols (e.g.
\texttt{operand1\ ?\ operand2\ :\ operand3}). Two unary operators
(\texttt{-} and \texttt{!}) use a traditional prefixed syntax that does
not require parentheses unless the operand is itself a complex
expression (e.g. \texttt{-\ 10}, \texttt{!\ (operand1\ or\ operand2)}).

Finally, special constructor operators (\texttt{\{...\}} for
constructing points, \texttt{{[}...{]}} for constructing lists and maps)
will require their operands to be placed between their two symbols (e.g.
\texttt{\{1,2,3\}}, \texttt{{[}operand1,\ operand2,\ ...,\ operandn{]}}
or \texttt{{[}key1::value1,\ key2::value2...\ keyn::valuen{]}}).

With the exception of these special cases above, the following rules
apply to the syntax of operators: * if they only have one operand, the
functional prefixed syntax is mandatory (e.g.
\texttt{operator\_name(operand1)}) * if they have two arguments, either
the functional prefixed syntax (e.g.
\texttt{operator\_name(operand1,\ operand2)}) or the infixed syntax
(e.g. \texttt{operand1\ operator\_name\ operand2}) can be used. * if
they have more than two arguments, either the functional prefixed syntax
(e.g. \texttt{operator\_name(operand1,\ operand2,\ ...,\ operand)}) or a
special infixed syntax with the first operand on the left-hand side of
the operator name (e.g.
\texttt{operand1\ operator\_name(operand2,\ ...,\ operand)}) can be
used.

All of these alternative syntaxes are completely equivalent.

Operators in GAML are purely functional, i.e.~they are guaranteed to not
have any side effects on their operands. For instance, the
\texttt{shuffle} operator, which randomizes the positions of elements in
a list, does not modify its list operand but returns a new shuffled
list.

\section{\texorpdfstring{}{ }}\label{section-21}

\section{Priority between operators}\label{priority-between-operators-3}

The priority of operators determines, in the case of complex expressions
composed of several operators, which one(s) will be evaluated first.

GAML follows in general the traditional priorities attributed to
arithmetic, boolean, comparison operators, with some twists. Namely: *
the constructor operators, like \texttt{::}, used to compose pairs of
operands, have the lowest priority of all operators (e.g.
\texttt{a\ \textgreater{}\ b\ ::\ b\ \textgreater{}\ c} will return a
pair of boolean values, which means that the two comparisons are
evaluated before the operator applies. Similarly,
\texttt{{[}a\ \textgreater{}\ 10,\ b\ \textgreater{}\ 5{]}} will return
a list of boolean values. * it is followed by the \texttt{?:} operator,
the functional if-else (e.g.
\texttt{a\ \textgreater{}\ b\ ?\ a\ +\ 10\ :\ a\ -\ 10} will return the
result of the if-else). * next are the logical operators, \texttt{and}
and \texttt{or} (e.g.
\texttt{a\ \textgreater{}\ b\ or\ b\ \textgreater{}\ c} will return the
value of the test) * next are the comparison operators (i.e.
\texttt{\textgreater{}}, \texttt{\textless{}}, \texttt{\textless{}=},
\texttt{\textgreater{}=}, \texttt{=}, \texttt{!=}) * next the arithmetic
operators in their logical order (multiplicative operators have a higher
priority than additive operators) * next the unary operators \texttt{-}
and \texttt{!} * next the access operators \texttt{.} and
\texttt{{[}{]}} (e.g.
\texttt{\{1,2,3\}.x\ \textgreater{}\ 20\ +\ \{4,5,6\}.y} will return the
result of the comparison between the x and y ordinates of the two
points) * and finally the functional operators, which have the highest
priority of all.

\begin{center}\rule{0.5\linewidth}{\linethickness}\end{center}

\section{Using actions as operators}\label{using-actions-as-operators-3}

Actions defined in species can be used as operators, provided they are
called on the correct agent. The syntax is that of normal functional
operators, but the agent that will perform the action must be added as
the first operand.

For instance, if the following species is defined:

\begin{verbatim}
species spec1 {
        int min(int x, int y) {
                return x > y ? x : y;
        }
}
\end{verbatim}

Any agent instance of spec1 can use \texttt{min} as an operator (if the
action conflicts with an existing operator, a warning will be emitted).
For instance, in the same model, the following line is perfectly
acceptable:

\begin{verbatim}
global {
        init {
                create spec1;
                spec1 my_agent <- spec1[0];
                int the_min <- my_agent min(10,20); // or min(my_agent, 10, 20);
        }
}
\end{verbatim}

If the action doesn't have any operands, the syntax to use is
\texttt{my\_agent\ the\_action()}. Finally, if it does not return a
value, it might still be used but is considering as returning a value of
type \texttt{unknown} (e.g.
\texttt{unknown\ result\ \textless{}-\ my\_agent\ the\_action(op1,\ op2);}).

Note that due to the fact that actions are written by modelers, the
general functional contract is not respected in that case: actions might
perfectly have side effects on their operands (including the agent).

\begin{center}\rule{0.5\linewidth}{\linethickness}\end{center}

\section{Table of Contents}\label{table-of-contents-3}

\begin{center}\rule{0.5\linewidth}{\linethickness}\end{center}

\section{Operators by categories}\label{operators-by-categories-3}

\begin{center}\rule{0.5\linewidth}{\linethickness}\end{center}

\subsection{3D}\label{d-3}

\href{OperatorsBC\#box}{box}, \href{OperatorsBC\#cone3d}{cone3D},
\href{OperatorsBC\#cube}{cube}, \href{OperatorsBC\#cylinder}{cylinder},
\href{OperatorsDH\#dem}{dem}, \href{OperatorsDH\#hexagon}{hexagon},
\href{OperatorsNR\#pyramid}{pyramid},
\href{OperatorsNR\#rgb_to_xyz}{rgb\_to\_xyz},
\href{OperatorsSZ\#set_z}{set\_z}, \href{OperatorsSZ\#sphere}{sphere},
\href{OperatorsSZ\#teapot}{teapot},

\begin{center}\rule{0.5\linewidth}{\linethickness}\end{center}

\subsection{Arithmetic operators}\label{arithmetic-operators-3}

\href{OperatorsAA\#-}{-}, \href{OperatorsAA\#/}{/},
{[}\textsuperscript{{]}(OperatorsAA\#}), \href{OperatorsAA\#*}{*},
\href{OperatorsAA\#+}{+}, \href{OperatorsAA\#abs}{abs},
\href{OperatorsAA\#acos}{acos}, \href{OperatorsAA\#asin}{asin},
\href{OperatorsAA\#atan}{atan}, \href{OperatorsAA\#atan2}{atan2},
\href{OperatorsBC\#ceil}{ceil}, \href{OperatorsBC\#cos}{cos},
\href{OperatorsBC\#cos_rad}{cos\_rad}, \href{OperatorsDH\#div}{div},
\href{OperatorsDH\#even}{even}, \href{OperatorsDH\#exp}{exp},
\href{OperatorsDH\#fact}{fact}, \href{OperatorsDH\#floor}{floor},
\href{OperatorsDH\#hypot}{hypot},
\href{OperatorsIM\#is_finite}{is\_finite},
\href{OperatorsIM\#is_number}{is\_number}, \href{OperatorsIM\#ln}{ln},
\href{OperatorsIM\#log}{log}, \href{OperatorsIM\#mod}{mod},
\href{OperatorsNR\#round}{round}, \href{OperatorsSZ\#signum}{signum},
\href{OperatorsSZ\#sin}{sin}, \href{OperatorsSZ\#sin_rad}{sin\_rad},
\href{OperatorsSZ\#sqrt}{sqrt}, \href{OperatorsSZ\#tan}{tan},
\href{OperatorsSZ\#tan_rad}{tan\_rad}, \href{OperatorsSZ\#tanh}{tanh},
\href{OperatorsSZ\#with_precision}{with\_precision},

\begin{center}\rule{0.5\linewidth}{\linethickness}\end{center}

\subsection{BDI}\label{bdi-3}

\href{OperatorsAA\#and}{and}, \href{OperatorsDH\#eval_when}{eval\_when},
\href{OperatorsDH\#get_about}{get\_about},
\href{OperatorsDH\#get_agent}{get\_agent},
\href{OperatorsDH\#get_agent_cause}{get\_agent\_cause},
\href{OperatorsDH\#get_belief_op}{get\_belief\_op},
\href{OperatorsDH\#get_belief_with_name_op}{get\_belief\_with\_name\_op},
\href{OperatorsDH\#get_beliefs_op}{get\_beliefs\_op},
\href{OperatorsDH\#get_beliefs_with_name_op}{get\_beliefs\_with\_name\_op},
\href{OperatorsDH\#get_current_intention_op}{get\_current\_intention\_op},
\href{OperatorsDH\#get_decay}{get\_decay},
\href{OperatorsDH\#get_desire_op}{get\_desire\_op},
\href{OperatorsDH\#get_desire_with_name_op}{get\_desire\_with\_name\_op},
\href{OperatorsDH\#get_desires_op}{get\_desires\_op},
\href{OperatorsDH\#get_desires_with_name_op}{get\_desires\_with\_name\_op},
\href{OperatorsDH\#get_dominance}{get\_dominance},
\href{OperatorsDH\#get_familiarity}{get\_familiarity},
\href{OperatorsDH\#get_ideal_op}{get\_ideal\_op},
\href{OperatorsDH\#get_ideal_with_name_op}{get\_ideal\_with\_name\_op},
\href{OperatorsDH\#get_ideals_op}{get\_ideals\_op},
\href{OperatorsDH\#get_ideals_with_name_op}{get\_ideals\_with\_name\_op},
\href{OperatorsDH\#get_intensity}{get\_intensity},
\href{OperatorsDH\#get_intention_op}{get\_intention\_op},
\href{OperatorsDH\#get_intention_with_name_op}{get\_intention\_with\_name\_op},
\href{OperatorsDH\#get_intentions_op}{get\_intentions\_op},
\href{OperatorsDH\#get_intentions_with_name_op}{get\_intentions\_with\_name\_op},
\href{OperatorsDH\#get_lifetime}{get\_lifetime},
\href{OperatorsDH\#get_liking}{get\_liking},
\href{OperatorsDH\#get_modality}{get\_modality},
\href{OperatorsDH\#get_obligation_op}{get\_obligation\_op},
\href{OperatorsDH\#get_obligation_with_name_op}{get\_obligation\_with\_name\_op},
\href{OperatorsDH\#get_obligations_op}{get\_obligations\_op},
\href{OperatorsDH\#get_obligations_with_name_op}{get\_obligations\_with\_name\_op},
\href{OperatorsDH\#get_plan_name}{get\_plan\_name},
\href{OperatorsDH\#get_predicate}{get\_predicate},
\href{OperatorsDH\#get_solidarity}{get\_solidarity},
\href{OperatorsDH\#get_strength}{get\_strength},
\href{OperatorsDH\#get_super_intention}{get\_super\_intention},
\href{OperatorsDH\#get_trust}{get\_trust},
\href{OperatorsDH\#get_truth}{get\_truth},
\href{OperatorsDH\#get_uncertainties_op}{get\_uncertainties\_op},
\href{OperatorsDH\#get_uncertainties_with_name_op}{get\_uncertainties\_with\_name\_op},
\href{OperatorsDH\#get_uncertainty_op}{get\_uncertainty\_op},
\href{OperatorsDH\#get_uncertainty_with_name_op}{get\_uncertainty\_with\_name\_op},
\href{OperatorsDH\#has_belief_op}{has\_belief\_op},
\href{OperatorsDH\#has_belief_with_name_op}{has\_belief\_with\_name\_op},
\href{OperatorsDH\#has_desire_op}{has\_desire\_op},
\href{OperatorsDH\#has_desire_with_name_op}{has\_desire\_with\_name\_op},
\href{OperatorsDH\#has_ideal_op}{has\_ideal\_op},
\href{OperatorsDH\#has_ideal_with_name_op}{has\_ideal\_with\_name\_op},
\href{OperatorsDH\#has_intention_op}{has\_intention\_op},
\href{OperatorsDH\#has_intention_with_name_op}{has\_intention\_with\_name\_op},
\href{OperatorsDH\#has_obligation_op}{has\_obligation\_op},
\href{OperatorsDH\#has_obligation_with_name_op}{has\_obligation\_with\_name\_op},
\href{OperatorsDH\#has_uncertainty_op}{has\_uncertainty\_op},
\href{OperatorsDH\#has_uncertainty_with_name_op}{has\_uncertainty\_with\_name\_op},
\href{OperatorsNR\#new_emotion}{new\_emotion},
\href{OperatorsNR\#new_mental_state}{new\_mental\_state},
\href{OperatorsNR\#new_predicate}{new\_predicate},
\href{OperatorsNR\#new_social_link}{new\_social\_link},
\href{OperatorsNR\#or}{or}, \href{OperatorsSZ\#set_about}{set\_about},
\href{OperatorsSZ\#set_agent}{set\_agent},
\href{OperatorsSZ\#set_agent_cause}{set\_agent\_cause},
\href{OperatorsSZ\#set_decay}{set\_decay},
\href{OperatorsSZ\#set_dominance}{set\_dominance},
\href{OperatorsSZ\#set_familiarity}{set\_familiarity},
\href{OperatorsSZ\#set_intensity}{set\_intensity},
\href{OperatorsSZ\#set_lifetime}{set\_lifetime},
\href{OperatorsSZ\#set_liking}{set\_liking},
\href{OperatorsSZ\#set_modality}{set\_modality},
\href{OperatorsSZ\#set_predicate}{set\_predicate},
\href{OperatorsSZ\#set_solidarity}{set\_solidarity},
\href{OperatorsSZ\#set_strength}{set\_strength},
\href{OperatorsSZ\#set_trust}{set\_trust},
\href{OperatorsSZ\#set_truth}{set\_truth},
\href{OperatorsSZ\#with_lifetime}{with\_lifetime},
\href{OperatorsSZ\#with_values}{with\_values},

\begin{center}\rule{0.5\linewidth}{\linethickness}\end{center}

\subsection{Casting operators}\label{casting-operators-3}

\href{OperatorsAA\#as}{as}, \href{OperatorsAA\#as_int}{as\_int},
\href{OperatorsAA\#as_matrix}{as\_matrix},
\href{OperatorsDH\#font}{font}, \href{OperatorsIM\#is}{is},
\href{OperatorsIM\#is_skill}{is\_skill},
\href{OperatorsIM\#list_with}{list\_with},
\href{OperatorsIM\#matrix_with}{matrix\_with},
\href{OperatorsSZ\#species}{species},
\href{OperatorsSZ\#to_gaml}{to\_gaml},
\href{OperatorsSZ\#topology}{topology},

\begin{center}\rule{0.5\linewidth}{\linethickness}\end{center}

\subsection{Color-related operators}\label{color-related-operators-3}

\href{OperatorsAA\#-}{-}, \href{OperatorsAA\#/}{/},
\href{OperatorsAA\#*}{*}, \href{OperatorsAA\#+}{+},
\href{OperatorsBC\#blend}{blend},
\href{OperatorsBC\#brewer_colors}{brewer\_colors},
\href{OperatorsBC\#brewer_palettes}{brewer\_palettes},
\href{OperatorsDH\#grayscale}{grayscale}, \href{OperatorsDH\#hsb}{hsb},
\href{OperatorsIM\#mean}{mean}, \href{OperatorsIM\#median}{median},
\href{OperatorsNR\#rgb}{rgb}, \href{OperatorsNR\#rnd_color}{rnd\_color},
\href{OperatorsSZ\#sum}{sum},

\begin{center}\rule{0.5\linewidth}{\linethickness}\end{center}

\subsection{Comparison operators}\label{comparison-operators-3}

\href{OperatorsAA\#!=}{!=}, \href{OperatorsAA\#\%3C}{\textless{}},
\href{OperatorsAA\#\%3C=}{\textless{}=}, \href{OperatorsAA\#=}{=},
\href{OperatorsAA\#\%3E}{\textgreater{}},
\href{OperatorsAA\#\%3E=}{\textgreater{}=},
\href{OperatorsBC\#between}{between},

\begin{center}\rule{0.5\linewidth}{\linethickness}\end{center}

\subsection{Containers-related
operators}\label{containers-related-operators-3}

\href{OperatorsAA\#-}{-}, \href{OperatorsAA\#::}{::},
\href{OperatorsAA\#+}{+}, \href{OperatorsAA\#accumulate}{accumulate},
\href{OperatorsAA\#among}{among}, \href{OperatorsAA\#at}{at},
\href{OperatorsBC\#collect}{collect},
\href{OperatorsBC\#contains}{contains},
\href{OperatorsBC\#contains_all}{contains\_all},
\href{OperatorsBC\#contains_any}{contains\_any},
\href{OperatorsBC\#count}{count},
\href{OperatorsDH\#distinct}{distinct},
\href{OperatorsDH\#empty}{empty}, \href{OperatorsDH\#every}{every},
\href{OperatorsDH\#first}{first},
\href{OperatorsDH\#first_with}{first\_with},
\href{OperatorsDH\#get}{get}, \href{OperatorsDH\#group_by}{group\_by},
\href{OperatorsIM\#in}{in}, \href{OperatorsIM\#index_by}{index\_by},
\href{OperatorsIM\#inter}{inter},
\href{OperatorsIM\#interleave}{interleave},
\href{OperatorsIM\#internal_at}{internal\_at},
\href{OperatorsIM\#internal_integrated_value}{internal\_integrated\_value},
\href{OperatorsIM\#last}{last},
\href{OperatorsIM\#last_with}{last\_with},
\href{OperatorsIM\#length}{length}, \href{OperatorsIM\#max}{max},
\href{OperatorsIM\#max_of}{max\_of}, \href{OperatorsIM\#mean}{mean},
\href{OperatorsIM\#mean_of}{mean\_of},
\href{OperatorsIM\#median}{median}, \href{OperatorsIM\#min}{min},
\href{OperatorsIM\#min_of}{min\_of}, \href{OperatorsIM\#mul}{mul},
\href{OperatorsNR\#one_of}{one\_of},
\href{OperatorsNR\#product_of}{product\_of},
\href{OperatorsNR\#range}{range}, \href{OperatorsNR\#reverse}{reverse},
\href{OperatorsSZ\#shuffle}{shuffle},
\href{OperatorsSZ\#sort_by}{sort\_by}, \href{OperatorsSZ\#split}{split},
\href{OperatorsSZ\#split_in}{split\_in},
\href{OperatorsSZ\#split_using}{split\_using},
\href{OperatorsSZ\#sum}{sum}, \href{OperatorsSZ\#sum_of}{sum\_of},
\href{OperatorsSZ\#union}{union},
\href{OperatorsSZ\#variance_of}{variance\_of},
\href{OperatorsSZ\#where}{where},
\href{OperatorsSZ\#with_max_of}{with\_max\_of},
\href{OperatorsSZ\#with_min_of}{with\_min\_of},

\begin{center}\rule{0.5\linewidth}{\linethickness}\end{center}

\subsection{Date-related operators}\label{date-related-operators-3}

\href{OperatorsAA\#-}{-}, \href{OperatorsAA\#!=}{!=},
\href{OperatorsAA\#+}{+}, \href{OperatorsAA\#\%3C}{\textless{}},
\href{OperatorsAA\#\%3C=}{\textless{}=}, \href{OperatorsAA\#=}{=},
\href{OperatorsAA\#\%3E}{\textgreater{}},
\href{OperatorsAA\#\%3E=}{\textgreater{}=},
\href{OperatorsAA\#after}{after}, \href{OperatorsBC\#before}{before},
\href{OperatorsBC\#between}{between}, \href{OperatorsDH\#every}{every},
\href{OperatorsIM\#milliseconds_between}{milliseconds\_between},
\href{OperatorsIM\#minus_days}{minus\_days},
\href{OperatorsIM\#minus_hours}{minus\_hours},
\href{OperatorsIM\#minus_minutes}{minus\_minutes},
\href{OperatorsIM\#minus_months}{minus\_months},
\href{OperatorsIM\#minus_ms}{minus\_ms},
\href{OperatorsIM\#minus_weeks}{minus\_weeks},
\href{OperatorsIM\#minus_years}{minus\_years},
\href{OperatorsIM\#months_between}{months\_between},
\href{OperatorsNR\#plus_days}{plus\_days},
\href{OperatorsNR\#plus_hours}{plus\_hours},
\href{OperatorsNR\#plus_minutes}{plus\_minutes},
\href{OperatorsNR\#plus_months}{plus\_months},
\href{OperatorsNR\#plus_ms}{plus\_ms},
\href{OperatorsNR\#plus_weeks}{plus\_weeks},
\href{OperatorsNR\#plus_years}{plus\_years},
\href{OperatorsSZ\#since}{since}, \href{OperatorsSZ\#to}{to},
\href{OperatorsSZ\#until}{until},
\href{OperatorsSZ\#years_between}{years\_between},

\begin{center}\rule{0.5\linewidth}{\linethickness}\end{center}

\subsection{Dates}\label{dates-3}

\begin{center}\rule{0.5\linewidth}{\linethickness}\end{center}

\subsection{DescriptiveStatistics}\label{descriptivestatistics-3}

\href{OperatorsAA\#auto_correlation}{auto\_correlation},
\href{OperatorsBC\#correlation}{correlation},
\href{OperatorsBC\#covariance}{covariance},
\href{OperatorsDH\#durbin_watson}{durbin\_watson},
\href{OperatorsIM\#kurtosis}{kurtosis},
\href{OperatorsIM\#moment}{moment},
\href{OperatorsNR\#quantile}{quantile},
\href{OperatorsNR\#quantile_inverse}{quantile\_inverse},
\href{OperatorsNR\#rank_interpolated}{rank\_interpolated},
\href{OperatorsNR\#rms}{rms}, \href{OperatorsSZ\#skew}{skew},
\href{OperatorsSZ\#variance}{variance},

\begin{center}\rule{0.5\linewidth}{\linethickness}\end{center}

\subsection{Displays}\label{displays-3}

\href{OperatorsDH\#horizontal}{horizontal},
\href{OperatorsSZ\#stack}{stack},
\href{OperatorsSZ\#vertical}{vertical},

\begin{center}\rule{0.5\linewidth}{\linethickness}\end{center}

\subsection{Distributions}\label{distributions-3}

\href{OperatorsBC\#binomial_coeff}{binomial\_coeff},
\href{OperatorsBC\#binomial_complemented}{binomial\_complemented},
\href{OperatorsBC\#binomial_sum}{binomial\_sum},
\href{OperatorsBC\#chi_square}{chi\_square},
\href{OperatorsBC\#chi_square_complemented}{chi\_square\_complemented},
\href{OperatorsDH\#gamma_distribution}{gamma\_distribution},
\href{OperatorsDH\#gamma_distribution_complemented}{gamma\_distribution\_complemented},
\href{OperatorsNR\#normal_area}{normal\_area},
\href{OperatorsNR\#normal_density}{normal\_density},
\href{OperatorsNR\#normal_inverse}{normal\_inverse},
\href{OperatorsNR\#pvalue_for_fstat}{pValue\_for\_fStat},
\href{OperatorsNR\#pvalue_for_tstat}{pValue\_for\_tStat},
\href{OperatorsSZ\#student_area}{student\_area},
\href{OperatorsSZ\#student_t_inverse}{student\_t\_inverse},

\begin{center}\rule{0.5\linewidth}{\linethickness}\end{center}

\subsection{Driving operators}\label{driving-operators-3}

\href{OperatorsAA\#as_driving_graph}{as\_driving\_graph},

\begin{center}\rule{0.5\linewidth}{\linethickness}\end{center}

\subsection{edge}\label{edge-4}

\href{OperatorsDH\#edge_between}{edge\_between},
\href{OperatorsSZ\#strahler}{strahler},

\begin{center}\rule{0.5\linewidth}{\linethickness}\end{center}

\subsection{EDP-related operators}\label{edp-related-operators-3}

\href{OperatorsDH\#diff}{diff}, \href{OperatorsDH\#diff2}{diff2},
\href{OperatorsIM\#internal_zero_order_equation}{internal\_zero\_order\_equation},

\begin{center}\rule{0.5\linewidth}{\linethickness}\end{center}

\subsection{Files-related operators}\label{files-related-operators-3}

\href{OperatorsBC\#crs}{crs},
\href{OperatorsDH\#evaluate_sub_model}{evaluate\_sub\_model},
\href{OperatorsDH\#file}{file},
\href{OperatorsDH\#file_exists}{file\_exists},
\href{OperatorsDH\#folder}{folder}, \href{OperatorsDH\#get}{get},
\href{OperatorsIM\#load_sub_model}{load\_sub\_model},
\href{OperatorsNR\#new_folder}{new\_folder},
\href{OperatorsNR\#osm_file}{osm\_file}, \href{OperatorsNR\#read}{read},
\href{OperatorsSZ\#step_sub_model}{step\_sub\_model},
\href{OperatorsSZ\#writable}{writable},

\begin{center}\rule{0.5\linewidth}{\linethickness}\end{center}

\subsection{FIPA-related operators}\label{fipa-related-operators-3}

\href{OperatorsBC\#conversation}{conversation},
\href{OperatorsIM\#message}{message},

\begin{center}\rule{0.5\linewidth}{\linethickness}\end{center}

\subsection{GamaMetaType}\label{gamametatype-3}

\href{OperatorsSZ\#type_of}{type\_of},

\begin{center}\rule{0.5\linewidth}{\linethickness}\end{center}

\subsection{GammaFunction}\label{gammafunction-3}

\href{OperatorsBC\#beta}{beta}, \href{OperatorsDH\#gamma}{gamma},
\href{OperatorsIM\#incomplete_beta}{incomplete\_beta},
\href{OperatorsIM\#incomplete_gamma}{incomplete\_gamma},
\href{OperatorsIM\#incomplete_gamma_complement}{incomplete\_gamma\_complement},
\href{OperatorsIM\#log_gamma}{log\_gamma},

\begin{center}\rule{0.5\linewidth}{\linethickness}\end{center}

\subsection{Graphs-related operators}\label{graphs-related-operators-3}

\href{OperatorsAA\#add_edge}{add\_edge},
\href{OperatorsAA\#add_node}{add\_node},
\href{OperatorsAA\#adjacency}{adjacency},
\href{OperatorsAA\#agent_from_geometry}{agent\_from\_geometry},
\href{OperatorsAA\#all_pairs_shortest_path}{all\_pairs\_shortest\_path},
\href{OperatorsAA\#alpha_index}{alpha\_index},
\href{OperatorsAA\#as_distance_graph}{as\_distance\_graph},
\href{OperatorsAA\#as_edge_graph}{as\_edge\_graph},
\href{OperatorsAA\#as_intersection_graph}{as\_intersection\_graph},
\href{OperatorsAA\#as_path}{as\_path},
\href{OperatorsBC\#beta_index}{beta\_index},
\href{OperatorsBC\#betweenness_centrality}{betweenness\_centrality},
\href{OperatorsBC\#biggest_cliques_of}{biggest\_cliques\_of},
\href{OperatorsBC\#connected_components_of}{connected\_components\_of},
\href{OperatorsBC\#connectivity_index}{connectivity\_index},
\href{OperatorsBC\#contains_edge}{contains\_edge},
\href{OperatorsBC\#contains_vertex}{contains\_vertex},
\href{OperatorsDH\#degree_of}{degree\_of},
\href{OperatorsDH\#directed}{directed}, \href{OperatorsDH\#edge}{edge},
\href{OperatorsDH\#edge_between}{edge\_between},
\href{OperatorsDH\#edge_betweenness}{edge\_betweenness},
\href{OperatorsDH\#edges}{edges},
\href{OperatorsDH\#gamma_index}{gamma\_index},
\href{OperatorsDH\#generate_barabasi_albert}{generate\_barabasi\_albert},
\href{OperatorsDH\#generate_complete_graph}{generate\_complete\_graph},
\href{OperatorsDH\#generate_watts_strogatz}{generate\_watts\_strogatz},
\href{OperatorsDH\#grid_cells_to_graph}{grid\_cells\_to\_graph},
\href{OperatorsIM\#in_degree_of}{in\_degree\_of},
\href{OperatorsIM\#in_edges_of}{in\_edges\_of},
\href{OperatorsIM\#layout}{layout},
\href{OperatorsIM\#load_graph_from_file}{load\_graph\_from\_file},
\href{OperatorsIM\#load_shortest_paths}{load\_shortest\_paths},
\href{OperatorsIM\#main_connected_component}{main\_connected\_component},
\href{OperatorsIM\#max_flow_between}{max\_flow\_between},
\href{OperatorsIM\#maximal_cliques_of}{maximal\_cliques\_of},
\href{OperatorsNR\#nb_cycles}{nb\_cycles},
\href{OperatorsNR\#neighbors_of}{neighbors\_of},
\href{OperatorsNR\#node}{node}, \href{OperatorsNR\#nodes}{nodes},
\href{OperatorsNR\#out_degree_of}{out\_degree\_of},
\href{OperatorsNR\#out_edges_of}{out\_edges\_of},
\href{OperatorsNR\#path_between}{path\_between},
\href{OperatorsNR\#paths_between}{paths\_between},
\href{OperatorsNR\#predecessors_of}{predecessors\_of},
\href{OperatorsNR\#remove_node_from}{remove\_node\_from},
\href{OperatorsNR\#rewire_n}{rewire\_n},
\href{OperatorsSZ\#source_of}{source\_of},
\href{OperatorsSZ\#spatial_graph}{spatial\_graph},
\href{OperatorsSZ\#strahler}{strahler},
\href{OperatorsSZ\#successors_of}{successors\_of},
\href{OperatorsSZ\#sum}{sum}, \href{OperatorsSZ\#target_of}{target\_of},
\href{OperatorsSZ\#undirected}{undirected},
\href{OperatorsSZ\#use_cache}{use\_cache},
\href{OperatorsSZ\#weight_of}{weight\_of},
\href{OperatorsSZ\#with_optimizer_type}{with\_optimizer\_type},
\href{OperatorsSZ\#with_weights}{with\_weights},

\begin{center}\rule{0.5\linewidth}{\linethickness}\end{center}

\subsection{Grid-related operators}\label{grid-related-operators-3}

\href{OperatorsAA\#as_4_grid}{as\_4\_grid},
\href{OperatorsAA\#as_grid}{as\_grid},
\href{OperatorsAA\#as_hexagonal_grid}{as\_hexagonal\_grid},
\href{OperatorsDH\#grid_at}{grid\_at},
\href{OperatorsNR\#path_between}{path\_between},

\begin{center}\rule{0.5\linewidth}{\linethickness}\end{center}

\subsection{Iterator operators}\label{iterator-operators-3}

\href{OperatorsAA\#accumulate}{accumulate},
\href{OperatorsAA\#as_map}{as\_map},
\href{OperatorsBC\#collect}{collect}, \href{OperatorsBC\#count}{count},
\href{OperatorsBC\#create_map}{create\_map},
\href{OperatorsDH\#distribution_of}{distribution\_of},
\href{OperatorsDH\#distribution_of}{distribution\_of},
\href{OperatorsDH\#distribution_of}{distribution\_of},
\href{OperatorsDH\#distribution2d_of}{distribution2d\_of},
\href{OperatorsDH\#distribution2d_of}{distribution2d\_of},
\href{OperatorsDH\#distribution2d_of}{distribution2d\_of},
\href{OperatorsDH\#first_with}{first\_with},
\href{OperatorsDH\#frequency_of}{frequency\_of},
\href{OperatorsDH\#group_by}{group\_by},
\href{OperatorsIM\#index_by}{index\_by},
\href{OperatorsIM\#last_with}{last\_with},
\href{OperatorsIM\#max_of}{max\_of},
\href{OperatorsIM\#mean_of}{mean\_of},
\href{OperatorsIM\#min_of}{min\_of},
\href{OperatorsNR\#product_of}{product\_of},
\href{OperatorsSZ\#sort_by}{sort\_by},
\href{OperatorsSZ\#sum_of}{sum\_of},
\href{OperatorsSZ\#variance_of}{variance\_of},
\href{OperatorsSZ\#where}{where},
\href{OperatorsSZ\#with_max_of}{with\_max\_of},
\href{OperatorsSZ\#with_min_of}{with\_min\_of},

\begin{center}\rule{0.5\linewidth}{\linethickness}\end{center}

\subsection{List-related operators}\label{list-related-operators-3}

\href{OperatorsBC\#copy_between}{copy\_between},
\href{OperatorsIM\#index_of}{index\_of},
\href{OperatorsIM\#last_index_of}{last\_index\_of},

\begin{center}\rule{0.5\linewidth}{\linethickness}\end{center}

\subsection{Logical operators}\label{logical-operators-3}

\href{OperatorsAA\#:}{:}, \href{OperatorsAA\#!}{!},
\href{OperatorsAA\#?}{?}, \href{OperatorsAA\#and}{and},
\href{OperatorsNR\#or}{or}, \href{OperatorsSZ\#xor}{xor},

\begin{center}\rule{0.5\linewidth}{\linethickness}\end{center}

\subsection{Map comparaison
operators}\label{map-comparaison-operators-3}

\href{OperatorsDH\#fuzzy_kappa}{fuzzy\_kappa},
\href{OperatorsDH\#fuzzy_kappa_sim}{fuzzy\_kappa\_sim},
\href{OperatorsIM\#kappa}{kappa},
\href{OperatorsIM\#kappa_sim}{kappa\_sim},
\href{OperatorsNR\#percent_absolute_deviation}{percent\_absolute\_deviation},

\begin{center}\rule{0.5\linewidth}{\linethickness}\end{center}

\subsection{Map-related operators}\label{map-related-operators-3}

\href{OperatorsAA\#as_map}{as\_map},
\href{OperatorsBC\#create_map}{create\_map},
\href{OperatorsIM\#index_of}{index\_of},
\href{OperatorsIM\#last_index_of}{last\_index\_of},

\begin{center}\rule{0.5\linewidth}{\linethickness}\end{center}

\subsection{Material}\label{material-3}

\href{OperatorsIM\#material}{material},

\begin{center}\rule{0.5\linewidth}{\linethickness}\end{center}

\subsection{Matrix-related operators}\label{matrix-related-operators-3}

\href{OperatorsAA\#-}{-}, \href{OperatorsAA\#/}{/},
\href{OperatorsAA\#.}{.}, \href{OperatorsAA\#*}{*},
\href{OperatorsAA\#+}{+},
\href{OperatorsAA\#append_horizontally}{append\_horizontally},
\href{OperatorsAA\#append_vertically}{append\_vertically},
\href{OperatorsBC\#column_at}{column\_at},
\href{OperatorsBC\#columns_list}{columns\_list},
\href{OperatorsDH\#determinant}{determinant},
\href{OperatorsDH\#eigenvalues}{eigenvalues},
\href{OperatorsIM\#index_of}{index\_of},
\href{OperatorsIM\#inverse}{inverse},
\href{OperatorsIM\#last_index_of}{last\_index\_of},
\href{OperatorsNR\#row_at}{row\_at},
\href{OperatorsNR\#rows_list}{rows\_list},
\href{OperatorsSZ\#shuffle}{shuffle}, \href{OperatorsSZ\#trace}{trace},
\href{OperatorsSZ\#transpose}{transpose},

\begin{center}\rule{0.5\linewidth}{\linethickness}\end{center}

\subsection{multicriteria operators}\label{multicriteria-operators-3}

\href{OperatorsDH\#electre_dm}{electre\_DM},
\href{OperatorsDH\#evidence_theory_dm}{evidence\_theory\_DM},
\href{OperatorsDH\#fuzzy_choquet_dm}{fuzzy\_choquet\_DM},
\href{OperatorsNR\#promethee_dm}{promethee\_DM},
\href{OperatorsSZ\#weighted_means_dm}{weighted\_means\_DM},

\begin{center}\rule{0.5\linewidth}{\linethickness}\end{center}

\subsection{Path-related operators}\label{path-related-operators-3}

\href{OperatorsAA\#agent_from_geometry}{agent\_from\_geometry},
\href{OperatorsAA\#all_pairs_shortest_path}{all\_pairs\_shortest\_path},
\href{OperatorsAA\#as_path}{as\_path},
\href{OperatorsIM\#load_shortest_paths}{load\_shortest\_paths},
\href{OperatorsIM\#max_flow_between}{max\_flow\_between},
\href{OperatorsNR\#path_between}{path\_between},
\href{OperatorsNR\#path_to}{path\_to},
\href{OperatorsNR\#paths_between}{paths\_between},
\href{OperatorsSZ\#use_cache}{use\_cache},

\begin{center}\rule{0.5\linewidth}{\linethickness}\end{center}

\subsection{Points-related operators}\label{points-related-operators-3}

\href{OperatorsAA\#-}{-}, \href{OperatorsAA\#/}{/},
\href{OperatorsAA\#*}{*}, \href{OperatorsAA\#+}{+},
\href{OperatorsAA\#\%3C}{\textless{}},
\href{OperatorsAA\#\%3C=}{\textless{}=},
\href{OperatorsAA\#\%3E}{\textgreater{}},
\href{OperatorsAA\#\%3E=}{\textgreater{}=},
\href{OperatorsAA\#add_point}{add\_point},
\href{OperatorsAA\#angle_between}{angle\_between},
\href{OperatorsAA\#any_location_in}{any\_location\_in},
\href{OperatorsBC\#centroid}{centroid},
\href{OperatorsBC\#closest_points_with}{closest\_points\_with},
\href{OperatorsDH\#farthest_point_to}{farthest\_point\_to},
\href{OperatorsDH\#grid_at}{grid\_at}, \href{OperatorsNR\#norm}{norm},
\href{OperatorsNR\#point}{point},
\href{OperatorsNR\#points_along}{points\_along},
\href{OperatorsNR\#points_at}{points\_at},
\href{OperatorsNR\#points_on}{points\_on},

\begin{center}\rule{0.5\linewidth}{\linethickness}\end{center}

\subsection{Random operators}\label{random-operators-3}

\href{OperatorsBC\#binomial}{binomial}, \href{OperatorsDH\#flip}{flip},
\href{OperatorsDH\#gauss}{gauss},
\href{OperatorsIM\#improved_generator}{improved\_generator},
\href{OperatorsNR\#open_simplex_generator}{open\_simplex\_generator},
\href{OperatorsNR\#poisson}{poisson}, \href{OperatorsNR\#rnd}{rnd},
\href{OperatorsNR\#rnd_choice}{rnd\_choice},
\href{OperatorsSZ\#sample}{sample},
\href{OperatorsSZ\#shuffle}{shuffle},
\href{OperatorsSZ\#simplex_generator}{simplex\_generator},
\href{OperatorsSZ\#skew_gauss}{skew\_gauss},
\href{OperatorsSZ\#truncated_gauss}{truncated\_gauss},

\begin{center}\rule{0.5\linewidth}{\linethickness}\end{center}

\subsection{ReverseOperators}\label{reverseoperators-3}

\href{OperatorsSZ\#savesimulation}{saveSimulation},
\href{OperatorsSZ\#serialize}{serialize},
\href{OperatorsSZ\#serializeagent}{serializeAgent},
\href{OperatorsSZ\#unserializesimulation}{unSerializeSimulation},
\href{OperatorsSZ\#unserializesimulationfromfile}{unSerializeSimulationFromFile},

\begin{center}\rule{0.5\linewidth}{\linethickness}\end{center}

\subsection{Shape}\label{shape-3}

\href{OperatorsAA\#arc}{arc}, \href{OperatorsBC\#box}{box},
\href{OperatorsBC\#circle}{circle}, \href{OperatorsBC\#cone}{cone},
\href{OperatorsBC\#cone3d}{cone3D}, \href{OperatorsBC\#cross}{cross},
\href{OperatorsBC\#cube}{cube}, \href{OperatorsBC\#curve}{curve},
\href{OperatorsBC\#cylinder}{cylinder},
\href{OperatorsDH\#ellipse}{ellipse},
\href{OperatorsDH\#envelope}{envelope},
\href{OperatorsDH\#geometry_collection}{geometry\_collection},
\href{OperatorsDH\#hexagon}{hexagon}, \href{OperatorsIM\#line}{line},
\href{OperatorsIM\#link}{link}, \href{OperatorsNR\#plan}{plan},
\href{OperatorsNR\#polygon}{polygon},
\href{OperatorsNR\#polyhedron}{polyhedron},
\href{OperatorsNR\#pyramid}{pyramid},
\href{OperatorsNR\#rectangle}{rectangle},
\href{OperatorsSZ\#sphere}{sphere}, \href{OperatorsSZ\#square}{square},
\href{OperatorsSZ\#squircle}{squircle},
\href{OperatorsSZ\#teapot}{teapot},
\href{OperatorsSZ\#triangle}{triangle},

\begin{center}\rule{0.5\linewidth}{\linethickness}\end{center}

\subsection{Spatial operators}\label{spatial-operators-3}

\href{OperatorsAA\#-}{-}, \href{OperatorsAA\#*}{*},
\href{OperatorsAA\#+}{+}, \href{OperatorsAA\#add_point}{add\_point},
\href{OperatorsAA\#agent_closest_to}{agent\_closest\_to},
\href{OperatorsAA\#agent_farthest_to}{agent\_farthest\_to},
\href{OperatorsAA\#agents_at_distance}{agents\_at\_distance},
\href{OperatorsAA\#agents_inside}{agents\_inside},
\href{OperatorsAA\#agents_overlapping}{agents\_overlapping},
\href{OperatorsAA\#angle_between}{angle\_between},
\href{OperatorsAA\#any_location_in}{any\_location\_in},
\href{OperatorsAA\#arc}{arc}, \href{OperatorsAA\#around}{around},
\href{OperatorsAA\#as_4_grid}{as\_4\_grid},
\href{OperatorsAA\#as_grid}{as\_grid},
\href{OperatorsAA\#as_hexagonal_grid}{as\_hexagonal\_grid},
\href{OperatorsAA\#at_distance}{at\_distance},
\href{OperatorsAA\#at_location}{at\_location},
\href{OperatorsBC\#box}{box}, \href{OperatorsBC\#centroid}{centroid},
\href{OperatorsBC\#circle}{circle}, \href{OperatorsBC\#clean}{clean},
\href{OperatorsBC\#clean_network}{clean\_network},
\href{OperatorsBC\#closest_points_with}{closest\_points\_with},
\href{OperatorsBC\#closest_to}{closest\_to},
\href{OperatorsBC\#cone}{cone}, \href{OperatorsBC\#cone3d}{cone3D},
\href{OperatorsBC\#convex_hull}{convex\_hull},
\href{OperatorsBC\#covers}{covers}, \href{OperatorsBC\#cross}{cross},
\href{OperatorsBC\#crosses}{crosses}, \href{OperatorsBC\#crs}{crs},
\href{OperatorsBC\#crs_transform}{CRS\_transform},
\href{OperatorsBC\#cube}{cube}, \href{OperatorsBC\#curve}{curve},
\href{OperatorsBC\#cylinder}{cylinder}, \href{OperatorsDH\#dem}{dem},
\href{OperatorsDH\#direction_between}{direction\_between},
\href{OperatorsDH\#disjoint_from}{disjoint\_from},
\href{OperatorsDH\#distance_between}{distance\_between},
\href{OperatorsDH\#distance_to}{distance\_to},
\href{OperatorsDH\#ellipse}{ellipse},
\href{OperatorsDH\#envelope}{envelope},
\href{OperatorsDH\#farthest_point_to}{farthest\_point\_to},
\href{OperatorsDH\#farthest_to}{farthest\_to},
\href{OperatorsDH\#geometry_collection}{geometry\_collection},
\href{OperatorsDH\#gini}{gini}, \href{OperatorsDH\#hexagon}{hexagon},
\href{OperatorsDH\#hierarchical_clustering}{hierarchical\_clustering},
\href{OperatorsIM\#idw}{IDW}, \href{OperatorsIM\#inside}{inside},
\href{OperatorsIM\#inter}{inter},
\href{OperatorsIM\#intersects}{intersects},
\href{OperatorsIM\#line}{line}, \href{OperatorsIM\#link}{link},
\href{OperatorsIM\#masked_by}{masked\_by},
\href{OperatorsIM\#moran}{moran},
\href{OperatorsNR\#neighbors_at}{neighbors\_at},
\href{OperatorsNR\#neighbors_of}{neighbors\_of},
\href{OperatorsNR\#overlapping}{overlapping},
\href{OperatorsNR\#overlaps}{overlaps},
\href{OperatorsNR\#partially_overlaps}{partially\_overlaps},
\href{OperatorsNR\#path_between}{path\_between},
\href{OperatorsNR\#path_to}{path\_to}, \href{OperatorsNR\#plan}{plan},
\href{OperatorsNR\#points_along}{points\_along},
\href{OperatorsNR\#points_at}{points\_at},
\href{OperatorsNR\#points_on}{points\_on},
\href{OperatorsNR\#polygon}{polygon},
\href{OperatorsNR\#polyhedron}{polyhedron},
\href{OperatorsNR\#pyramid}{pyramid},
\href{OperatorsNR\#rectangle}{rectangle},
\href{OperatorsNR\#rgb_to_xyz}{rgb\_to\_xyz},
\href{OperatorsNR\#rotated_by}{rotated\_by},
\href{OperatorsNR\#round}{round},
\href{OperatorsSZ\#scaled_to}{scaled\_to},
\href{OperatorsSZ\#set_z}{set\_z},
\href{OperatorsSZ\#simple_clustering_by_distance}{simple\_clustering\_by\_distance},
\href{OperatorsSZ\#simplification}{simplification},
\href{OperatorsSZ\#skeletonize}{skeletonize},
\href{OperatorsSZ\#smooth}{smooth}, \href{OperatorsSZ\#sphere}{sphere},
\href{OperatorsSZ\#split_at}{split\_at},
\href{OperatorsSZ\#split_geometry}{split\_geometry},
\href{OperatorsSZ\#split_lines}{split\_lines},
\href{OperatorsSZ\#square}{square},
\href{OperatorsSZ\#squircle}{squircle},
\href{OperatorsSZ\#teapot}{teapot},
\href{OperatorsSZ\#to_gama_crs}{to\_GAMA\_CRS},
\href{OperatorsSZ\#to_rectangles}{to\_rectangles},
\href{OperatorsSZ\#to_squares}{to\_squares},
\href{OperatorsSZ\#to_sub_geometries}{to\_sub\_geometries},
\href{OperatorsSZ\#touches}{touches},
\href{OperatorsSZ\#towards}{towards},
\href{OperatorsSZ\#transformed_by}{transformed\_by},
\href{OperatorsSZ\#translated_by}{translated\_by},
\href{OperatorsSZ\#triangle}{triangle},
\href{OperatorsSZ\#triangulate}{triangulate},
\href{OperatorsSZ\#union}{union}, \href{OperatorsSZ\#using}{using},
\href{OperatorsSZ\#voronoi}{voronoi},
\href{OperatorsSZ\#with_precision}{with\_precision},
\href{OperatorsSZ\#without_holes}{without\_holes},

\begin{center}\rule{0.5\linewidth}{\linethickness}\end{center}

\subsection{Spatial properties
operators}\label{spatial-properties-operators-3}

\href{OperatorsBC\#covers}{covers},
\href{OperatorsBC\#crosses}{crosses},
\href{OperatorsIM\#intersects}{intersects},
\href{OperatorsNR\#partially_overlaps}{partially\_overlaps},
\href{OperatorsSZ\#touches}{touches},

\begin{center}\rule{0.5\linewidth}{\linethickness}\end{center}

\subsection{Spatial queries
operators}\label{spatial-queries-operators-3}

\href{OperatorsAA\#agent_closest_to}{agent\_closest\_to},
\href{OperatorsAA\#agent_farthest_to}{agent\_farthest\_to},
\href{OperatorsAA\#agents_at_distance}{agents\_at\_distance},
\href{OperatorsAA\#agents_inside}{agents\_inside},
\href{OperatorsAA\#agents_overlapping}{agents\_overlapping},
\href{OperatorsAA\#at_distance}{at\_distance},
\href{OperatorsBC\#closest_to}{closest\_to},
\href{OperatorsDH\#farthest_to}{farthest\_to},
\href{OperatorsIM\#inside}{inside},
\href{OperatorsNR\#neighbors_at}{neighbors\_at},
\href{OperatorsNR\#neighbors_of}{neighbors\_of},
\href{OperatorsNR\#overlapping}{overlapping},

\begin{center}\rule{0.5\linewidth}{\linethickness}\end{center}

\subsection{Spatial relations
operators}\label{spatial-relations-operators-3}

\href{OperatorsDH\#direction_between}{direction\_between},
\href{OperatorsDH\#distance_between}{distance\_between},
\href{OperatorsDH\#distance_to}{distance\_to},
\href{OperatorsNR\#path_between}{path\_between},
\href{OperatorsNR\#path_to}{path\_to},
\href{OperatorsSZ\#towards}{towards},

\begin{center}\rule{0.5\linewidth}{\linethickness}\end{center}

\subsection{Spatial statistical
operators}\label{spatial-statistical-operators-3}

\href{OperatorsDH\#hierarchical_clustering}{hierarchical\_clustering},
\href{OperatorsSZ\#simple_clustering_by_distance}{simple\_clustering\_by\_distance},

\begin{center}\rule{0.5\linewidth}{\linethickness}\end{center}

\subsection{Spatial transformations
operators}\label{spatial-transformations-operators-3}

\href{OperatorsAA\#-}{-}, \href{OperatorsAA\#*}{*},
\href{OperatorsAA\#+}{+}, \href{OperatorsAA\#as_4_grid}{as\_4\_grid},
\href{OperatorsAA\#as_grid}{as\_grid},
\href{OperatorsAA\#as_hexagonal_grid}{as\_hexagonal\_grid},
\href{OperatorsAA\#at_location}{at\_location},
\href{OperatorsBC\#clean}{clean},
\href{OperatorsBC\#clean_network}{clean\_network},
\href{OperatorsBC\#convex_hull}{convex\_hull},
\href{OperatorsBC\#crs_transform}{CRS\_transform},
\href{OperatorsNR\#rotated_by}{rotated\_by},
\href{OperatorsSZ\#scaled_to}{scaled\_to},
\href{OperatorsSZ\#simplification}{simplification},
\href{OperatorsSZ\#skeletonize}{skeletonize},
\href{OperatorsSZ\#smooth}{smooth},
\href{OperatorsSZ\#split_geometry}{split\_geometry},
\href{OperatorsSZ\#split_lines}{split\_lines},
\href{OperatorsSZ\#to_gama_crs}{to\_GAMA\_CRS},
\href{OperatorsSZ\#to_rectangles}{to\_rectangles},
\href{OperatorsSZ\#to_squares}{to\_squares},
\href{OperatorsSZ\#to_sub_geometries}{to\_sub\_geometries},
\href{OperatorsSZ\#transformed_by}{transformed\_by},
\href{OperatorsSZ\#translated_by}{translated\_by},
\href{OperatorsSZ\#triangulate}{triangulate},
\href{OperatorsSZ\#voronoi}{voronoi},
\href{OperatorsSZ\#with_precision}{with\_precision},
\href{OperatorsSZ\#without_holes}{without\_holes},

\begin{center}\rule{0.5\linewidth}{\linethickness}\end{center}

\subsection{Species-related
operators}\label{species-related-operators-3}

\href{OperatorsIM\#index_of}{index\_of},
\href{OperatorsIM\#last_index_of}{last\_index\_of},
\href{OperatorsNR\#of_generic_species}{of\_generic\_species},
\href{OperatorsNR\#of_species}{of\_species},

\begin{center}\rule{0.5\linewidth}{\linethickness}\end{center}

\subsection{Statistical operators}\label{statistical-operators-3}

\href{OperatorsBC\#build}{build}, \href{OperatorsBC\#corr}{corR},
\href{OperatorsDH\#dbscan}{dbscan},
\href{OperatorsDH\#distribution_of}{distribution\_of},
\href{OperatorsDH\#distribution2d_of}{distribution2d\_of},
\href{OperatorsDH\#dtw}{dtw},
\href{OperatorsDH\#frequency_of}{frequency\_of},
\href{OperatorsDH\#gamma_rnd}{gamma\_rnd},
\href{OperatorsDH\#geometric_mean}{geometric\_mean},
\href{OperatorsDH\#gini}{gini},
\href{OperatorsDH\#harmonic_mean}{harmonic\_mean},
\href{OperatorsDH\#hierarchical_clustering}{hierarchical\_clustering},
\href{OperatorsIM\#kmeans}{kmeans},
\href{OperatorsIM\#kurtosis}{kurtosis}, \href{OperatorsIM\#max}{max},
\href{OperatorsIM\#mean}{mean},
\href{OperatorsIM\#mean_deviation}{mean\_deviation},
\href{OperatorsIM\#meanr}{meanR}, \href{OperatorsIM\#median}{median},
\href{OperatorsIM\#min}{min}, \href{OperatorsIM\#moran}{moran},
\href{OperatorsIM\#mul}{mul}, \href{OperatorsNR\#predict}{predict},
\href{OperatorsSZ\#simple_clustering_by_distance}{simple\_clustering\_by\_distance},
\href{OperatorsSZ\#skewness}{skewness},
\href{OperatorsSZ\#split}{split},
\href{OperatorsSZ\#split_in}{split\_in},
\href{OperatorsSZ\#split_using}{split\_using},
\href{OperatorsSZ\#standard_deviation}{standard\_deviation},
\href{OperatorsSZ\#sum}{sum}, \href{OperatorsSZ\#variance}{variance},

\begin{center}\rule{0.5\linewidth}{\linethickness}\end{center}

\subsection{Strings-related
operators}\label{strings-related-operators-3}

\href{OperatorsAA\#+}{+}, \href{OperatorsAA\#\%3C}{\textless{}},
\href{OperatorsAA\#\%3C=}{\textless{}=},
\href{OperatorsAA\#\%3E}{\textgreater{}},
\href{OperatorsAA\#\%3E=}{\textgreater{}=}, \href{OperatorsAA\#at}{at},
\href{OperatorsBC\#char}{char}, \href{OperatorsBC\#contains}{contains},
\href{OperatorsBC\#contains_all}{contains\_all},
\href{OperatorsBC\#contains_any}{contains\_any},
\href{OperatorsBC\#copy_between}{copy\_between},
\href{OperatorsDH\#date}{date}, \href{OperatorsDH\#empty}{empty},
\href{OperatorsDH\#first}{first}, \href{OperatorsIM\#in}{in},
\href{OperatorsIM\#indented_by}{indented\_by},
\href{OperatorsIM\#index_of}{index\_of},
\href{OperatorsIM\#is_number}{is\_number},
\href{OperatorsIM\#last}{last},
\href{OperatorsIM\#last_index_of}{last\_index\_of},
\href{OperatorsIM\#length}{length},
\href{OperatorsIM\#lower_case}{lower\_case},
\href{OperatorsNR\#replace}{replace},
\href{OperatorsNR\#replace_regex}{replace\_regex},
\href{OperatorsNR\#reverse}{reverse},
\href{OperatorsSZ\#sample}{sample},
\href{OperatorsSZ\#shuffle}{shuffle},
\href{OperatorsSZ\#split_with}{split\_with},
\href{OperatorsSZ\#string}{string},
\href{OperatorsSZ\#upper_case}{upper\_case},

\begin{center}\rule{0.5\linewidth}{\linethickness}\end{center}

\subsection{System}\label{system-3}

\href{OperatorsAA\#.}{.}, \href{OperatorsBC\#command}{command},
\href{OperatorsBC\#copy}{copy}, \href{OperatorsDH\#dead}{dead},
\href{OperatorsDH\#eval_gaml}{eval\_gaml},
\href{OperatorsDH\#every}{every},
\href{OperatorsIM\#is_error}{is\_error},
\href{OperatorsIM\#is_warning}{is\_warning},
\href{OperatorsSZ\#user_input}{user\_input},

\begin{center}\rule{0.5\linewidth}{\linethickness}\end{center}

\subsection{Time-related operators}\label{time-related-operators-3}

\href{OperatorsDH\#date}{date}, \href{OperatorsSZ\#string}{string},

\begin{center}\rule{0.5\linewidth}{\linethickness}\end{center}

\subsection{Types-related operators}\label{types-related-operators-3}

\begin{center}\rule{0.5\linewidth}{\linethickness}\end{center}

\subsection{User control operators}\label{user-control-operators-3}

\href{OperatorsSZ\#user_input}{user\_input},

\begin{center}\rule{0.5\linewidth}{\linethickness}\end{center}

\section{Operators}\label{operators-3}

\begin{center}\rule{0.5\linewidth}{\linethickness}\end{center}

\subsection{\texorpdfstring{\texttt{IDW}}{IDW}}\label{idw}

\subsubsection{Possible use:}\label{possible-use-263}

\begin{itemize}
\tightlist
\item
  \textbf{\texttt{IDW}}
  (\texttt{container\textless{}agent\textgreater{}},
  \texttt{map\textless{}point,float\textgreater{}}, \texttt{int})
  ---\textgreater{} \texttt{map\textless{}agent,float\textgreater{}}
\end{itemize}

\subsubsection{Result:}\label{result-253}

Inverse Distance Weighting (IDW) is a type of deterministic method for
multivariate interpolation with a known scattered set of points. The
assigned values to each geometry are calculated with a weighted average
of the values available at the known points. See:
\url{http://en.wikipedia.org/wiki/Inverse_distance_weighting} Usage: IDW
(list of geometries, map of points (key: point, value: value), power
parameter)

\subsubsection{Examples:}\label{examples-202}

\begin{verbatim}
 
map<agent,float> var0 <- IDW([ag1, ag2, ag3, ag4, ag5],[{10,10}::25.0, {10,80}::10.0, {100,10}::15.0], 2); // var0 equals for example, can return [ag1::12.0, ag2::23.0,ag3::12.0,ag4::14.0,ag5::17.0]
\end{verbatim}

\begin{center}\rule{0.5\linewidth}{\linethickness}\end{center}

\subsection{\texorpdfstring{\texttt{image\_file}}{image\_file}}\label{image_file}

\subsubsection{Possible use:}\label{possible-use-264}

\begin{itemize}
\tightlist
\item
  \textbf{\texttt{image\_file}} (\texttt{string}) ---\textgreater{}
  \texttt{file}
\end{itemize}

\subsubsection{Result:}\label{result-254}

Constructs a file of type image. Allowed extensions are limited to tiff,
jpg, jpeg, png, pict, bmp

\begin{center}\rule{0.5\linewidth}{\linethickness}\end{center}

\subsection{\texorpdfstring{\texttt{improved\_generator}}{improved\_generator}}\label{improved_generator}

\subsubsection{Possible use:}\label{possible-use-265}

\begin{itemize}
\tightlist
\item
  \textbf{\texttt{improved\_generator}} (\texttt{float}, \texttt{float},
  \texttt{float}, \texttt{float}) ---\textgreater{} \texttt{float}
\end{itemize}

\subsubsection{Result:}\label{result-255}

take a x, y, z and a bias parameters and gives a value

\subsubsection{Examples:}\label{examples-203}

\begin{verbatim}
 
float var0 <- improved_generator(2,3,4,253); // var0 equals 10.2
\end{verbatim}

\begin{center}\rule{0.5\linewidth}{\linethickness}\end{center}

\subsection{\texorpdfstring{\texttt{in}}{in}}\label{in}

\subsubsection{Possible use:}\label{possible-use-266}

\begin{itemize}
\tightlist
\item
  \texttt{string} \textbf{\texttt{in}} \texttt{string} ---\textgreater{}
  \texttt{bool}
\item
  \textbf{\texttt{in}} (\texttt{string} , \texttt{string})
  ---\textgreater{} \texttt{bool}
\item
  \texttt{unknown} \textbf{\texttt{in}} \texttt{container}
  ---\textgreater{} \texttt{bool}
\item
  \textbf{\texttt{in}} (\texttt{unknown} , \texttt{container})
  ---\textgreater{} \texttt{bool}
\end{itemize}

\subsubsection{Result:}\label{result-256}

true if the right operand contains the left operand, false otherwise

\subsubsection{Comment:}\label{comment-51}

the definition of in depends on the container

\subsubsection{Special cases:}\label{special-cases-77}

\begin{itemize}
\tightlist
\item
  if both operands are strings, returns true if the left-hand operand
  patterns is included in to the right-hand string;\\
\item
  if the right operand is nil or empty, in returns false
\end{itemize}

\subsubsection{Examples:}\label{examples-204}

\begin{verbatim}
 
bool var0 <-  'bc' in 'abcded'; // var0 equals true 
bool var1 <- 2 in [1,2,3,4,5,6]; // var1 equals true 
bool var2 <- 7 in [1,2,3,4,5,6]; // var2 equals false 
bool var3 <- 3 in [1::2, 3::4, 5::6]; // var3 equals false 
bool var4 <- 6 in [1::2, 3::4, 5::6]; // var4 equals true
\end{verbatim}

\subsubsection{See also:}\label{see-also-114}

\href{OperatorsBC\#contains}{contains},

\begin{center}\rule{0.5\linewidth}{\linethickness}\end{center}

\subsection{\texorpdfstring{\texttt{in\_degree\_of}}{in\_degree\_of}}\label{in_degree_of}

\subsubsection{Possible use:}\label{possible-use-267}

\begin{itemize}
\tightlist
\item
  \texttt{graph} \textbf{\texttt{in\_degree\_of}} \texttt{unknown}
  ---\textgreater{} \texttt{int}
\item
  \textbf{\texttt{in\_degree\_of}} (\texttt{graph} , \texttt{unknown})
  ---\textgreater{} \texttt{int}
\end{itemize}

\subsubsection{Result:}\label{result-257}

returns the in degree of a vertex (right-hand operand) in the graph
given as left-hand operand.

\subsubsection{Examples:}\label{examples-205}

\begin{verbatim}
 
int var1 <- graphFromMap in_degree_of (node(3)); // var1 equals 2
\end{verbatim}

\subsubsection{See also:}\label{see-also-115}

\href{OperatorsNR\#out_degree_of}{out\_degree\_of},
\href{OperatorsDH\#degree_of}{degree\_of},

\begin{center}\rule{0.5\linewidth}{\linethickness}\end{center}

\subsection{\texorpdfstring{\texttt{in\_edges\_of}}{in\_edges\_of}}\label{in_edges_of}

\subsubsection{Possible use:}\label{possible-use-268}

\begin{itemize}
\tightlist
\item
  \texttt{graph} \textbf{\texttt{in\_edges\_of}} \texttt{unknown}
  ---\textgreater{} \texttt{list}
\item
  \textbf{\texttt{in\_edges\_of}} (\texttt{graph} , \texttt{unknown})
  ---\textgreater{} \texttt{list}
\end{itemize}

\subsubsection{Result:}\label{result-258}

returns the list of the in-edges of a vertex (right-hand operand) in the
graph given as left-hand operand.

\subsubsection{Examples:}\label{examples-206}

\begin{verbatim}
 
list var1 <- graphFromMap in_edges_of node({12,45}); // var1 equals [LineString]
\end{verbatim}

\subsubsection{See also:}\label{see-also-116}

\href{OperatorsNR\#out_edges_of}{out\_edges\_of},

\begin{center}\rule{0.5\linewidth}{\linethickness}\end{center}

\subsection{\texorpdfstring{\texttt{incomplete\_beta}}{incomplete\_beta}}\label{incomplete_beta}

\subsubsection{Possible use:}\label{possible-use-269}

\begin{itemize}
\tightlist
\item
  \textbf{\texttt{incomplete\_beta}} (\texttt{float}, \texttt{float},
  \texttt{float}) ---\textgreater{} \texttt{float}
\end{itemize}

\subsubsection{Result:}\label{result-259}

Returns the regularized integral of the beta function with arguments a
and b, from zero to x.

\begin{center}\rule{0.5\linewidth}{\linethickness}\end{center}

\subsection{\texorpdfstring{\texttt{incomplete\_gamma}}{incomplete\_gamma}}\label{incomplete_gamma}

\subsubsection{Possible use:}\label{possible-use-270}

\begin{itemize}
\tightlist
\item
  \texttt{float} \textbf{\texttt{incomplete\_gamma}} \texttt{float}
  ---\textgreater{} \texttt{float}
\item
  \textbf{\texttt{incomplete\_gamma}} (\texttt{float} , \texttt{float})
  ---\textgreater{} \texttt{float}
\end{itemize}

\subsubsection{Result:}\label{result-260}

Returns the regularized integral of the Gamma function with argument a
to the integration end point x.

\begin{center}\rule{0.5\linewidth}{\linethickness}\end{center}

\subsection{\texorpdfstring{\texttt{incomplete\_gamma\_complement}}{incomplete\_gamma\_complement}}\label{incomplete_gamma_complement}

\subsubsection{Possible use:}\label{possible-use-271}

\begin{itemize}
\tightlist
\item
  \texttt{float} \textbf{\texttt{incomplete\_gamma\_complement}}
  \texttt{float} ---\textgreater{} \texttt{float}
\item
  \textbf{\texttt{incomplete\_gamma\_complement}} (\texttt{float} ,
  \texttt{float}) ---\textgreater{} \texttt{float}
\end{itemize}

\subsubsection{Result:}\label{result-261}

Returns the complemented regularized incomplete Gamma function of the
argument a and integration start point x.

\begin{center}\rule{0.5\linewidth}{\linethickness}\end{center}

\subsection{\texorpdfstring{\texttt{indented\_by}}{indented\_by}}\label{indented_by}

\subsubsection{Possible use:}\label{possible-use-272}

\begin{itemize}
\tightlist
\item
  \texttt{string} \textbf{\texttt{indented\_by}} \texttt{int}
  ---\textgreater{} \texttt{string}
\item
  \textbf{\texttt{indented\_by}} (\texttt{string} , \texttt{int})
  ---\textgreater{} \texttt{string}
\end{itemize}

\subsubsection{Result:}\label{result-262}

Converts a (possibly multiline) string by indenting it by a number --
specified by the second operand -- of tabulations to the right

\begin{center}\rule{0.5\linewidth}{\linethickness}\end{center}

\subsection{\texorpdfstring{\texttt{index\_by}}{index\_by}}\label{index_by}

\subsubsection{Possible use:}\label{possible-use-273}

\begin{itemize}
\tightlist
\item
  \texttt{container} \textbf{\texttt{index\_by}}
  \texttt{any\ expression} ---\textgreater{} \texttt{map}
\item
  \textbf{\texttt{index\_by}} (\texttt{container} ,
  \texttt{any\ expression}) ---\textgreater{} \texttt{map}
\end{itemize}

\subsubsection{Result:}\label{result-263}

produces a new map from the evaluation of the right-hand operand for
each element of the left-hand operand

\subsubsection{Special cases:}\label{special-cases-78}

\begin{itemize}
\tightlist
\item
  if the left-hand operand is nil, index\_by throws an error. If the
  operation results in duplicate keys, only the first value
  corresponding to the key is kept
\end{itemize}

\subsubsection{Examples:}\label{examples-207}

\begin{verbatim}
 
map var0 <- [1,2,3,4,5,6,7,8] index_by (each - 1); // var0 equals [0::1, 1::2, 2::3, 3::4, 4::5, 5::6, 6::7, 7::8]
\end{verbatim}

\begin{center}\rule{0.5\linewidth}{\linethickness}\end{center}

\subsection{\texorpdfstring{\texttt{index\_of}}{index\_of}}\label{index_of}

\subsubsection{Possible use:}\label{possible-use-274}

\begin{itemize}
\tightlist
\item
  \texttt{map} \textbf{\texttt{index\_of}} \texttt{unknown}
  ---\textgreater{} \texttt{unknown}
\item
  \textbf{\texttt{index\_of}} (\texttt{map} , \texttt{unknown})
  ---\textgreater{} \texttt{unknown}
\item
  \texttt{matrix} \textbf{\texttt{index\_of}} \texttt{unknown}
  ---\textgreater{} \texttt{point}
\item
  \textbf{\texttt{index\_of}} (\texttt{matrix} , \texttt{unknown})
  ---\textgreater{} \texttt{point}
\item
  \texttt{list} \textbf{\texttt{index\_of}} \texttt{unknown}
  ---\textgreater{} \texttt{int}
\item
  \textbf{\texttt{index\_of}} (\texttt{list} , \texttt{unknown})
  ---\textgreater{} \texttt{int}
\item
  \texttt{string} \textbf{\texttt{index\_of}} \texttt{string}
  ---\textgreater{} \texttt{int}
\item
  \textbf{\texttt{index\_of}} (\texttt{string} , \texttt{string})
  ---\textgreater{} \texttt{int}
\item
  \texttt{species} \textbf{\texttt{index\_of}} \texttt{unknown}
  ---\textgreater{} \texttt{int}
\item
  \textbf{\texttt{index\_of}} (\texttt{species} , \texttt{unknown})
  ---\textgreater{} \texttt{int}
\end{itemize}

\subsubsection{Result:}\label{result-264}

the index of the first occurence of the right operand in the left
operand container the index of the first occurence of the right operand
in the left operand container

\subsubsection{Comment:}\label{comment-52}

The definition of index\_of and the type of the index depend on the
container

\subsubsection{Special cases:}\label{special-cases-79}

\begin{itemize}
\tightlist
\item
  if the left operand is a map, index\_of returns the index of a value
  or nil if the value is not mapped\\
\item
  if the left operator is a species, returns the index of an agent in a
  species. If the argument is not an agent of this species, returns -1.
  Use int(agent) instead\\
\item
  if the left operand is a matrix, index\_of returns the index as a
  point
\end{itemize}

\begin{verbatim}
 
point var1 <- matrix([[1,2,3],[4,5,6]]) index_of 4; // var1 equals {1.0,0.0}
\end{verbatim}

\begin{itemize}
\tightlist
\item
  if the left operand is a list, index\_of returns the index as an
  integer
\end{itemize}

\begin{verbatim}
 
int var2 <- [1,2,3,4,5,6] index_of 4; // var2 equals 3 
int var3 <- [4,2,3,4,5,4] index_of 4; // var3 equals 0
\end{verbatim}

\begin{itemize}
\tightlist
\item
  if both operands are strings, returns the index within the left-hand
  string of the first occurrence of the given right-hand string
\end{itemize}

\begin{verbatim}
 
int var4 <-  "abcabcabc" index_of "ca"; // var4 equals 2
\end{verbatim}

\subsubsection{Examples:}\label{examples-208}

\begin{verbatim}
 
unknown var0 <- [1::2, 3::4, 5::6] index_of 4; // var0 equals 3
\end{verbatim}

\subsubsection{See also:}\label{see-also-117}

\href{OperatorsAA\#at}{at},
\href{OperatorsIM\#last_index_of}{last\_index\_of},

\begin{center}\rule{0.5\linewidth}{\linethickness}\end{center}

\subsection{\texorpdfstring{\texttt{inside}}{inside}}\label{inside}

\subsubsection{Possible use:}\label{possible-use-275}

\begin{itemize}
\tightlist
\item
  \texttt{container\textless{}agent\textgreater{}}
  \textbf{\texttt{inside}} \texttt{geometry} ---\textgreater{}
  \texttt{list\textless{}geometry\textgreater{}}
\item
  \textbf{\texttt{inside}}
  (\texttt{container\textless{}agent\textgreater{}} , \texttt{geometry})
  ---\textgreater{} \texttt{list\textless{}geometry\textgreater{}}
\end{itemize}

\subsubsection{Result:}\label{result-265}

A list of agents or geometries among the left-operand list, species or
meta-population (addition of species), covered by the operand (casted as
a geometry).

\subsubsection{Examples:}\label{examples-209}

\begin{verbatim}
 
list<geometry> var0 <- [ag1, ag2, ag3] inside(self); // var0 equals the agents among ag1, ag2 and ag3 that are covered by the shape of the right-hand argument. 
list<geometry> var1 <- (species1 + species2) inside (self); // var1 equals the agents among species species1 and species2 that are covered by the shape of the right-hand argument.
\end{verbatim}

\subsubsection{See also:}\label{see-also-118}

\href{OperatorsNR\#neighbors_at}{neighbors\_at},
\href{OperatorsNR\#neighbors_of}{neighbors\_of},
\href{OperatorsBC\#closest_to}{closest\_to},
\href{OperatorsNR\#overlapping}{overlapping},
\href{OperatorsAA\#agents_overlapping}{agents\_overlapping},
\href{OperatorsAA\#agents_inside}{agents\_inside},
\href{OperatorsAA\#agent_closest_to}{agent\_closest\_to},

\begin{center}\rule{0.5\linewidth}{\linethickness}\end{center}

\subsection{\texorpdfstring{\texttt{int}}{int}}\label{int}

\subsubsection{Possible use:}\label{possible-use-276}

\begin{itemize}
\tightlist
\item
  \textbf{\texttt{int}} (\texttt{any}) ---\textgreater{} \texttt{int}
\end{itemize}

\subsubsection{Result:}\label{result-266}

Casts the operand into the type int

\begin{center}\rule{0.5\linewidth}{\linethickness}\end{center}

\subsection{\texorpdfstring{\texttt{inter}}{inter}}\label{inter}

\subsubsection{Possible use:}\label{possible-use-277}

\begin{itemize}
\tightlist
\item
  \texttt{container} \textbf{\texttt{inter}} \texttt{container}
  ---\textgreater{} \texttt{list}
\item
  \textbf{\texttt{inter}} (\texttt{container} , \texttt{container})
  ---\textgreater{} \texttt{list}
\item
  \texttt{geometry} \textbf{\texttt{inter}} \texttt{geometry}
  ---\textgreater{} \texttt{geometry}
\item
  \textbf{\texttt{inter}} (\texttt{geometry} , \texttt{geometry})
  ---\textgreater{} \texttt{geometry}
\end{itemize}

\subsubsection{Result:}\label{result-267}

the intersection of the two operands A geometry resulting from the
intersection between the two geometries

\subsubsection{Comment:}\label{comment-53}

both containers are transformed into sets (so without duplicated
element, cf.~remove\_deplicates operator) before the set intersection is
computed.

\subsubsection{Special cases:}\label{special-cases-80}

\begin{itemize}
\tightlist
\item
  if an operand is a graph, it will be transformed into the set of its
  nodes\\
\item
  returns nil if one of the operands is nil\\
\item
  if an operand is a map, it will be transformed into the set of its
  values
\end{itemize}

\begin{verbatim}
 
list var0 <- [1::2, 3::4, 5::6] inter [2,4]; // var0 equals [2,4] 
list var1 <- [1::2, 3::4, 5::6] inter [1,3]; // var1 equals []
\end{verbatim}

\begin{itemize}
\tightlist
\item
  if an operand is a matrix, it will be transformed into the set of the
  lines
\end{itemize}

\begin{verbatim}
 
list var2 <- matrix([[3,2,1],[4,5,4]]) inter [3,4]; // var2 equals [3,4]
\end{verbatim}

\subsubsection{Examples:}\label{examples-210}

\begin{verbatim}
 
list var3 <- [1,2,3,4,5,6] inter [2,4]; // var3 equals [2,4] 
list var4 <- [1,2,3,4,5,6] inter [0,8]; // var4 equals [] 
geometry var5 <- square(10) inter circle(5); // var5 equals circle(5)
\end{verbatim}

\subsubsection{See also:}\label{see-also-119}

\href{OperatorsNR\#remove_duplicates}{remove\_duplicates},
\href{OperatorsSZ\#union}{union}, \href{OperatorsAA\#+}{+},
\href{OperatorsAA\#-}{-},

\begin{center}\rule{0.5\linewidth}{\linethickness}\end{center}

\subsection{\texorpdfstring{\texttt{interleave}}{interleave}}\label{interleave}

\subsubsection{Possible use:}\label{possible-use-278}

\begin{itemize}
\tightlist
\item
  \textbf{\texttt{interleave}} (\texttt{container}) ---\textgreater{}
  \texttt{list}
\end{itemize}

\subsubsection{Result:}\label{result-268}

a new list containing the interleaved elements of the containers
contained in the operand

\subsubsection{Comment:}\label{comment-54}

the operand should be a list of lists of elements. The result is a list
of elements.

\subsubsection{Examples:}\label{examples-211}

\begin{verbatim}
 
list var0 <- interleave([1,2,4,3,5,7,6,8]); // var0 equals [1,2,4,3,5,7,6,8] 
list var1 <- interleave([['e11','e12','e13'],['e21','e22','e23'],['e31','e32','e33']]); // var1 equals ['e11','e21','e31','e12','e22','e32','e13','e23','e33']
\end{verbatim}

\begin{center}\rule{0.5\linewidth}{\linethickness}\end{center}

\subsection{\texorpdfstring{\texttt{internal\_at}}{internal\_at}}\label{internal_at}

\subsubsection{Possible use:}\label{possible-use-279}

\begin{itemize}
\tightlist
\item
  \texttt{container\textless{}KeyType,ValueType\textgreater{}}
  \textbf{\texttt{internal\_at}}
  \texttt{list\textless{}KeyType\textgreater{}} ---\textgreater{}
  \texttt{ValueType}
\item
  \textbf{\texttt{internal\_at}}
  (\texttt{container\textless{}KeyType,ValueType\textgreater{}} ,
  \texttt{list\textless{}KeyType\textgreater{}}) ---\textgreater{}
  \texttt{ValueType}
\item
  \texttt{agent} \textbf{\texttt{internal\_at}} \texttt{list}
  ---\textgreater{} \texttt{unknown}
\item
  \textbf{\texttt{internal\_at}} (\texttt{agent} , \texttt{list})
  ---\textgreater{} \texttt{unknown}
\item
  \texttt{geometry} \textbf{\texttt{internal\_at}} \texttt{list}
  ---\textgreater{} \texttt{unknown}
\item
  \textbf{\texttt{internal\_at}} (\texttt{geometry} , \texttt{list})
  ---\textgreater{} \texttt{unknown}
\end{itemize}

\subsubsection{Result:}\label{result-269}

For internal use only. Corresponds to the implementation of the access
to containers with {[}index{]} For internal use only. Corresponds to the
implementation, for agents, of the access to containers with {[}index{]}
For internal use only. Corresponds to the implementation, for
geometries, of the access to containers with {[}index{]}

\subsubsection{See also:}\label{see-also-120}

\href{OperatorsAA\#at}{at},

\begin{center}\rule{0.5\linewidth}{\linethickness}\end{center}

\subsection{\texorpdfstring{\texttt{internal\_integrated\_value}}{internal\_integrated\_value}}\label{internal_integrated_value}

\subsubsection{Possible use:}\label{possible-use-280}

\begin{itemize}
\tightlist
\item
  \texttt{any\ expression} \textbf{\texttt{internal\_integrated\_value}}
  \texttt{any\ expression} ---\textgreater{} \texttt{list}
\item
  \textbf{\texttt{internal\_integrated\_value}}
  (\texttt{any\ expression} , \texttt{any\ expression})
  ---\textgreater{} \texttt{list}
\end{itemize}

\subsubsection{Result:}\label{result-270}

For internal use only. Corresponds to the implementation, for agents, of
the access to containers with {[}index{]}

\begin{center}\rule{0.5\linewidth}{\linethickness}\end{center}

\subsection{\texorpdfstring{\texttt{internal\_zero\_order\_equation}}{internal\_zero\_order\_equation}}\label{internal_zero_order_equation}

\subsubsection{Possible use:}\label{possible-use-281}

\begin{itemize}
\tightlist
\item
  \textbf{\texttt{internal\_zero\_order\_equation}}
  (\texttt{any\ expression}) ---\textgreater{} \texttt{float}
\end{itemize}

\subsubsection{Result:}\label{result-271}

An internal placeholder function

\begin{center}\rule{0.5\linewidth}{\linethickness}\end{center}

\subsection{\texorpdfstring{\texttt{intersection}}{intersection}}\label{intersection}

Same signification as \href{OperatorsIM\#inter}{inter}

\begin{center}\rule{0.5\linewidth}{\linethickness}\end{center}

\subsection{\texorpdfstring{\texttt{intersects}}{intersects}}\label{intersects}

\subsubsection{Possible use:}\label{possible-use-282}

\begin{itemize}
\tightlist
\item
  \texttt{geometry} \textbf{\texttt{intersects}} \texttt{geometry}
  ---\textgreater{} \texttt{bool}
\item
  \textbf{\texttt{intersects}} (\texttt{geometry} , \texttt{geometry})
  ---\textgreater{} \texttt{bool}
\end{itemize}

\subsubsection{Result:}\label{result-272}

A boolean, equal to true if the left-geometry (or agent/point)
intersects the right-geometry (or agent/point).

\subsubsection{Special cases:}\label{special-cases-81}

\begin{itemize}
\tightlist
\item
  if one of the operand is null, returns false.
\end{itemize}

\subsubsection{Examples:}\label{examples-212}

\begin{verbatim}
 
bool var0 <- square(5) intersects {10,10}; // var0 equals false
\end{verbatim}

\subsubsection{See also:}\label{see-also-121}

\href{OperatorsDH\#disjoint_from}{disjoint\_from},
\href{OperatorsBC\#crosses}{crosses},
\href{OperatorsNR\#overlaps}{overlaps},
\href{OperatorsNR\#partially_overlaps}{partially\_overlaps},
\href{OperatorsSZ\#touches}{touches},

\begin{center}\rule{0.5\linewidth}{\linethickness}\end{center}

\subsection{\texorpdfstring{\texttt{inverse}}{inverse}}\label{inverse}

\subsubsection{Possible use:}\label{possible-use-283}

\begin{itemize}
\tightlist
\item
  \textbf{\texttt{inverse}} (\texttt{matrix}) ---\textgreater{}
  \texttt{matrix\textless{}float\textgreater{}}
\end{itemize}

\subsubsection{Result:}\label{result-273}

The inverse matrix of the given matrix. If no inverse exists, returns a
matrix that has properties that resemble that of an inverse.

\subsubsection{Examples:}\label{examples-213}

\begin{verbatim}
 
matrix<float> var0 <- inverse(matrix([[4,3],[3,2]])); // var0 equals matrix([[-2.0,3.0],[3.0,-4.0]])
\end{verbatim}

\begin{center}\rule{0.5\linewidth}{\linethickness}\end{center}

\subsection{\texorpdfstring{\texttt{inverse\_distance\_weighting}}{inverse\_distance\_weighting}}\label{inverse_distance_weighting}

Same signification as \href{OperatorsIM\#IDW}{IDW}

\begin{center}\rule{0.5\linewidth}{\linethickness}\end{center}

\subsection{\texorpdfstring{\texttt{is}}{is}}\label{is}

\subsubsection{Possible use:}\label{possible-use-284}

\begin{itemize}
\tightlist
\item
  \texttt{unknown} \textbf{\texttt{is}} \texttt{any\ expression}
  ---\textgreater{} \texttt{bool}
\item
  \textbf{\texttt{is}} (\texttt{unknown} , \texttt{any\ expression})
  ---\textgreater{} \texttt{bool}
\end{itemize}

\subsubsection{Result:}\label{result-274}

returns true if the left operand is of the right operand type, false
otherwise

\subsubsection{Examples:}\label{examples-214}

\begin{verbatim}
 
bool var0 <- 0 is int; // var0 equals true 
bool var1 <- an_agent is node; // var1 equals true 
bool var2 <- 1 is float; // var2 equals false
\end{verbatim}

\begin{center}\rule{0.5\linewidth}{\linethickness}\end{center}

\subsection{\texorpdfstring{\texttt{is\_csv}}{is\_csv}}\label{is_csv}

\subsubsection{Possible use:}\label{possible-use-285}

\begin{itemize}
\tightlist
\item
  \textbf{\texttt{is\_csv}} (\texttt{any}) ---\textgreater{}
  \texttt{bool}
\end{itemize}

\subsubsection{Result:}\label{result-275}

Tests whether the operand is a csv file.

\begin{center}\rule{0.5\linewidth}{\linethickness}\end{center}

\subsection{\texorpdfstring{\texttt{is\_dxf}}{is\_dxf}}\label{is_dxf}

\subsubsection{Possible use:}\label{possible-use-286}

\begin{itemize}
\tightlist
\item
  \textbf{\texttt{is\_dxf}} (\texttt{any}) ---\textgreater{}
  \texttt{bool}
\end{itemize}

\subsubsection{Result:}\label{result-276}

Tests whether the operand is a dxf file.

\begin{center}\rule{0.5\linewidth}{\linethickness}\end{center}

\subsection{\texorpdfstring{\texttt{is\_error}}{is\_error}}\label{is_error}

\subsubsection{Possible use:}\label{possible-use-287}

\begin{itemize}
\tightlist
\item
  \textbf{\texttt{is\_error}} (\texttt{any\ expression})
  ---\textgreater{} \texttt{bool}
\end{itemize}

\subsubsection{Result:}\label{result-277}

Returns whether or not the argument raises an error when evaluated

\begin{center}\rule{0.5\linewidth}{\linethickness}\end{center}

\subsection{\texorpdfstring{\texttt{is\_finite}}{is\_finite}}\label{is_finite}

\subsubsection{Possible use:}\label{possible-use-288}

\begin{itemize}
\tightlist
\item
  \textbf{\texttt{is\_finite}} (\texttt{float}) ---\textgreater{}
  \texttt{bool}
\end{itemize}

\subsubsection{Result:}\label{result-278}

Returns whether the argument is a finite number or not

\subsubsection{Examples:}\label{examples-215}

\begin{verbatim}
 
bool var0 <- is_finite(4.66); // var0 equals true 
bool var1 <- is_finite(#infinity); // var1 equals false
\end{verbatim}

\begin{center}\rule{0.5\linewidth}{\linethickness}\end{center}

\subsection{\texorpdfstring{\texttt{is\_gaml}}{is\_gaml}}\label{is_gaml}

\subsubsection{Possible use:}\label{possible-use-289}

\begin{itemize}
\tightlist
\item
  \textbf{\texttt{is\_gaml}} (\texttt{any}) ---\textgreater{}
  \texttt{bool}
\end{itemize}

\subsubsection{Result:}\label{result-279}

Tests whether the operand is a gaml file.

\begin{center}\rule{0.5\linewidth}{\linethickness}\end{center}

\subsection{\texorpdfstring{\texttt{is\_geojson}}{is\_geojson}}\label{is_geojson}

\subsubsection{Possible use:}\label{possible-use-290}

\begin{itemize}
\tightlist
\item
  \textbf{\texttt{is\_geojson}} (\texttt{any}) ---\textgreater{}
  \texttt{bool}
\end{itemize}

\subsubsection{Result:}\label{result-280}

Tests whether the operand is a geojson file.

\begin{center}\rule{0.5\linewidth}{\linethickness}\end{center}

\subsection{\texorpdfstring{\texttt{is\_gif}}{is\_gif}}\label{is_gif}

\subsubsection{Possible use:}\label{possible-use-291}

\begin{itemize}
\tightlist
\item
  \textbf{\texttt{is\_gif}} (\texttt{any}) ---\textgreater{}
  \texttt{bool}
\end{itemize}

\subsubsection{Result:}\label{result-281}

Tests whether the operand is a gif file.

\begin{center}\rule{0.5\linewidth}{\linethickness}\end{center}

\subsection{\texorpdfstring{\texttt{is\_gml}}{is\_gml}}\label{is_gml}

\subsubsection{Possible use:}\label{possible-use-292}

\begin{itemize}
\tightlist
\item
  \textbf{\texttt{is\_gml}} (\texttt{any}) ---\textgreater{}
  \texttt{bool}
\end{itemize}

\subsubsection{Result:}\label{result-282}

Tests whether the operand is a gml file.

\begin{center}\rule{0.5\linewidth}{\linethickness}\end{center}

\subsection{\texorpdfstring{\texttt{is\_grid}}{is\_grid}}\label{is_grid}

\subsubsection{Possible use:}\label{possible-use-293}

\begin{itemize}
\tightlist
\item
  \textbf{\texttt{is\_grid}} (\texttt{any}) ---\textgreater{}
  \texttt{bool}
\end{itemize}

\subsubsection{Result:}\label{result-283}

Tests whether the operand is a grid file.

\begin{center}\rule{0.5\linewidth}{\linethickness}\end{center}

\subsection{\texorpdfstring{\texttt{is\_image}}{is\_image}}\label{is_image}

\subsubsection{Possible use:}\label{possible-use-294}

\begin{itemize}
\tightlist
\item
  \textbf{\texttt{is\_image}} (\texttt{any}) ---\textgreater{}
  \texttt{bool}
\end{itemize}

\subsubsection{Result:}\label{result-284}

Tests whether the operand is a image file.

\begin{center}\rule{0.5\linewidth}{\linethickness}\end{center}

\subsection{\texorpdfstring{\texttt{is\_json}}{is\_json}}\label{is_json}

\subsubsection{Possible use:}\label{possible-use-295}

\begin{itemize}
\tightlist
\item
  \textbf{\texttt{is\_json}} (\texttt{any}) ---\textgreater{}
  \texttt{bool}
\end{itemize}

\subsubsection{Result:}\label{result-285}

Tests whether the operand is a json file.

\begin{center}\rule{0.5\linewidth}{\linethickness}\end{center}

\subsection{\texorpdfstring{\texttt{is\_number}}{is\_number}}\label{is_number}

\subsubsection{Possible use:}\label{possible-use-296}

\begin{itemize}
\tightlist
\item
  \textbf{\texttt{is\_number}} (\texttt{string}) ---\textgreater{}
  \texttt{bool}
\item
  \textbf{\texttt{is\_number}} (\texttt{float}) ---\textgreater{}
  \texttt{bool}
\end{itemize}

\subsubsection{Result:}\label{result-286}

tests whether the operand represents a numerical value Returns whether
the argument is a real number or not

\subsubsection{Comment:}\label{comment-55}

Note that the symbol . should be used for a float value (a string with ,
will not be considered as a numeric value). Symbols e and E are also
accepted. A hexadecimal value should begin with \#.

\subsubsection{Examples:}\label{examples-216}

\begin{verbatim}
 
bool var0 <- is_number("test"); // var0 equals false 
bool var1 <- is_number("123.56"); // var1 equals true 
bool var2 <- is_number("-1.2e5"); // var2 equals true 
bool var3 <- is_number("1,2"); // var3 equals false 
bool var4 <- is_number("#12FA"); // var4 equals true 
bool var5 <- is_number(4.66); // var5 equals true 
bool var6 <- is_number(#infinity); // var6 equals true 
bool var7 <- is_number(#nan); // var7 equals false
\end{verbatim}

\begin{center}\rule{0.5\linewidth}{\linethickness}\end{center}

\subsection{\texorpdfstring{\texttt{is\_obj}}{is\_obj}}\label{is_obj}

\subsubsection{Possible use:}\label{possible-use-297}

\begin{itemize}
\tightlist
\item
  \textbf{\texttt{is\_obj}} (\texttt{any}) ---\textgreater{}
  \texttt{bool}
\end{itemize}

\subsubsection{Result:}\label{result-287}

Tests whether the operand is a obj file.

\begin{center}\rule{0.5\linewidth}{\linethickness}\end{center}

\subsection{\texorpdfstring{\texttt{is\_osm}}{is\_osm}}\label{is_osm}

\subsubsection{Possible use:}\label{possible-use-298}

\begin{itemize}
\tightlist
\item
  \textbf{\texttt{is\_osm}} (\texttt{any}) ---\textgreater{}
  \texttt{bool}
\end{itemize}

\subsubsection{Result:}\label{result-288}

Tests whether the operand is a osm file.

\begin{center}\rule{0.5\linewidth}{\linethickness}\end{center}

\subsection{\texorpdfstring{\texttt{is\_pgm}}{is\_pgm}}\label{is_pgm}

\subsubsection{Possible use:}\label{possible-use-299}

\begin{itemize}
\tightlist
\item
  \textbf{\texttt{is\_pgm}} (\texttt{any}) ---\textgreater{}
  \texttt{bool}
\end{itemize}

\subsubsection{Result:}\label{result-289}

Tests whether the operand is a pgm file.

\begin{center}\rule{0.5\linewidth}{\linethickness}\end{center}

\subsection{\texorpdfstring{\texttt{is\_property}}{is\_property}}\label{is_property}

\subsubsection{Possible use:}\label{possible-use-300}

\begin{itemize}
\tightlist
\item
  \textbf{\texttt{is\_property}} (\texttt{any}) ---\textgreater{}
  \texttt{bool}
\end{itemize}

\subsubsection{Result:}\label{result-290}

Tests whether the operand is a property file.

\begin{center}\rule{0.5\linewidth}{\linethickness}\end{center}

\subsection{\texorpdfstring{\texttt{is\_R}}{is\_R}}\label{is_r}

\subsubsection{Possible use:}\label{possible-use-301}

\begin{itemize}
\tightlist
\item
  \textbf{\texttt{is\_R}} (\texttt{any}) ---\textgreater{} \texttt{bool}
\end{itemize}

\subsubsection{Result:}\label{result-291}

Tests whether the operand is a R file.

\begin{center}\rule{0.5\linewidth}{\linethickness}\end{center}

\subsection{\texorpdfstring{\texttt{is\_shape}}{is\_shape}}\label{is_shape}

\subsubsection{Possible use:}\label{possible-use-302}

\begin{itemize}
\tightlist
\item
  \textbf{\texttt{is\_shape}} (\texttt{any}) ---\textgreater{}
  \texttt{bool}
\end{itemize}

\subsubsection{Result:}\label{result-292}

Tests whether the operand is a shape file.

\begin{center}\rule{0.5\linewidth}{\linethickness}\end{center}

\subsection{\texorpdfstring{\texttt{is\_skill}}{is\_skill}}\label{is_skill}

\subsubsection{Possible use:}\label{possible-use-303}

\begin{itemize}
\tightlist
\item
  \texttt{unknown} \textbf{\texttt{is\_skill}} \texttt{string}
  ---\textgreater{} \texttt{bool}
\item
  \textbf{\texttt{is\_skill}} (\texttt{unknown} , \texttt{string})
  ---\textgreater{} \texttt{bool}
\end{itemize}

\subsubsection{Result:}\label{result-293}

returns true if the left operand is an agent whose species implements
the right-hand skill name

\subsubsection{Examples:}\label{examples-217}

\begin{verbatim}
 
bool var0 <- agentA is_skill 'moving'; // var0 equals true
\end{verbatim}

\begin{center}\rule{0.5\linewidth}{\linethickness}\end{center}

\subsection{\texorpdfstring{\texttt{is\_svg}}{is\_svg}}\label{is_svg}

\subsubsection{Possible use:}\label{possible-use-304}

\begin{itemize}
\tightlist
\item
  \textbf{\texttt{is\_svg}} (\texttt{any}) ---\textgreater{}
  \texttt{bool}
\end{itemize}

\subsubsection{Result:}\label{result-294}

Tests whether the operand is a svg file.

\begin{center}\rule{0.5\linewidth}{\linethickness}\end{center}

\subsection{\texorpdfstring{\texttt{is\_text}}{is\_text}}\label{is_text}

\subsubsection{Possible use:}\label{possible-use-305}

\begin{itemize}
\tightlist
\item
  \textbf{\texttt{is\_text}} (\texttt{any}) ---\textgreater{}
  \texttt{bool}
\end{itemize}

\subsubsection{Result:}\label{result-295}

Tests whether the operand is a text file.

\begin{center}\rule{0.5\linewidth}{\linethickness}\end{center}

\subsection{\texorpdfstring{\texttt{is\_threeds}}{is\_threeds}}\label{is_threeds}

\subsubsection{Possible use:}\label{possible-use-306}

\begin{itemize}
\tightlist
\item
  \textbf{\texttt{is\_threeds}} (\texttt{any}) ---\textgreater{}
  \texttt{bool}
\end{itemize}

\subsubsection{Result:}\label{result-296}

Tests whether the operand is a threeds file.

\begin{center}\rule{0.5\linewidth}{\linethickness}\end{center}

\subsection{\texorpdfstring{\texttt{is\_URL}}{is\_URL}}\label{is_url}

\subsubsection{Possible use:}\label{possible-use-307}

\begin{itemize}
\tightlist
\item
  \textbf{\texttt{is\_URL}} (\texttt{any}) ---\textgreater{}
  \texttt{bool}
\end{itemize}

\subsubsection{Result:}\label{result-297}

Tests whether the operand is a URL file.

\begin{center}\rule{0.5\linewidth}{\linethickness}\end{center}

\subsection{\texorpdfstring{\texttt{is\_warning}}{is\_warning}}\label{is_warning}

\subsubsection{Possible use:}\label{possible-use-308}

\begin{itemize}
\tightlist
\item
  \textbf{\texttt{is\_warning}} (\texttt{any\ expression})
  ---\textgreater{} \texttt{bool}
\end{itemize}

\subsubsection{Result:}\label{result-298}

Returns whether or not the argument raises a warning when evaluated

\begin{center}\rule{0.5\linewidth}{\linethickness}\end{center}

\subsection{\texorpdfstring{\texttt{is\_xml}}{is\_xml}}\label{is_xml}

\subsubsection{Possible use:}\label{possible-use-309}

\begin{itemize}
\tightlist
\item
  \textbf{\texttt{is\_xml}} (\texttt{any}) ---\textgreater{}
  \texttt{bool}
\end{itemize}

\subsubsection{Result:}\label{result-299}

Tests whether the operand is a xml file.

\begin{center}\rule{0.5\linewidth}{\linethickness}\end{center}

\subsection{\texorpdfstring{\texttt{json\_file}}{json\_file}}\label{json_file}

\subsubsection{Possible use:}\label{possible-use-310}

\begin{itemize}
\tightlist
\item
  \textbf{\texttt{json\_file}} (\texttt{string}) ---\textgreater{}
  \texttt{file}
\end{itemize}

\subsubsection{Result:}\label{result-300}

Constructs a file of type json. Allowed extensions are limited to json

\begin{center}\rule{0.5\linewidth}{\linethickness}\end{center}

\subsection{\texorpdfstring{\texttt{kappa}}{kappa}}\label{kappa}

\subsubsection{Possible use:}\label{possible-use-311}

\begin{itemize}
\tightlist
\item
  \textbf{\texttt{kappa}} (\texttt{list}, \texttt{list}, \texttt{list})
  ---\textgreater{} \texttt{float}
\item
  \textbf{\texttt{kappa}} (\texttt{list}, \texttt{list}, \texttt{list},
  \texttt{list}) ---\textgreater{} \texttt{float}
\end{itemize}

\subsubsection{Result:}\label{result-301}

kappa indicator for 2 map comparisons:
kappa(list\_vals1,list\_vals2,categories). Reference: Cohen, J. A
coefficient of agreement for nominal scales. Educ. Psychol. Meas. 1960,
20. kappa indicator for 2 map comparisons:
kappa(list\_vals1,list\_vals2,categories, weights). Reference: Cohen, J.
A coefficient of agreement for nominal scales. Educ. Psychol. Meas.
1960, 20.

\subsubsection{Examples:}\label{examples-218}

\begin{verbatim}
kappa([cat1,cat1,cat2,cat3,cat2],[cat2,cat1,cat2,cat1,cat2],[cat1,cat2,cat3])  
float var1 <- kappa([1,3,5,1,5],[1,1,1,1,5],[1,3,5]); // var1 equals the similarity between 0 and 1 
float var2 <- kappa([1,1,1,1,5],[1,1,1,1,5],[1,3,5]); // var2 equals 1.0kappa([cat1,cat1,cat2,cat3,cat2],[cat2,cat1,cat2,cat1,cat2],[cat1,cat2,cat3], [1.0, 2.0, 3.0, 1.0, 5.0]) 
\end{verbatim}

\begin{center}\rule{0.5\linewidth}{\linethickness}\end{center}

\subsection{\texorpdfstring{\texttt{kappa\_sim}}{kappa\_sim}}\label{kappa_sim}

\subsubsection{Possible use:}\label{possible-use-312}

\begin{itemize}
\tightlist
\item
  \textbf{\texttt{kappa\_sim}} (\texttt{list}, \texttt{list},
  \texttt{list}, \texttt{list}) ---\textgreater{} \texttt{float}
\item
  \textbf{\texttt{kappa\_sim}} (\texttt{list}, \texttt{list},
  \texttt{list}, \texttt{list}, \texttt{list}) ---\textgreater{}
  \texttt{float}
\end{itemize}

\subsubsection{Result:}\label{result-302}

kappa simulation indicator for 2 map comparisons:
kappa(list\_valsInits,list\_valsObs,list\_valsSim, categories, weights).
Reference: van Vliet, J., Bregt, A.K. \& Hagen-Zanker, A. (2011).
Revisiting Kappa to account for change in the accuracy assessment of
land-use change models, Ecological Modelling 222(8) kappa simulation
indicator for 2 map comparisons:
kappa(list\_valsInits,list\_valsObs,list\_valsSim, categories).
Reference: van Vliet, J., Bregt, A.K. \& Hagen-Zanker, A. (2011).
Revisiting Kappa to account for change in the accuracy assessment of
land-use change models, Ecological Modelling 222(8).

\subsubsection{Examples:}\label{examples-219}

\begin{verbatim}
kappa([cat1,cat1,cat2,cat2,cat2],[cat2,cat1,cat2,cat1,cat3],[cat2,cat1,cat2,cat3,cat3], [cat1,cat2,cat3],[1.0, 2.0, 3.0, 1.0, 5.0]) kappa([cat1,cat1,cat2,cat2,cat2],[cat2,cat1,cat2,cat1,cat3],[cat2,cat1,cat2,cat3,cat3], [cat1,cat2,cat3]) 
\end{verbatim}

\begin{center}\rule{0.5\linewidth}{\linethickness}\end{center}

\subsection{\texorpdfstring{\texttt{kmeans}}{kmeans}}\label{kmeans}

\subsubsection{Possible use:}\label{possible-use-313}

\begin{itemize}
\tightlist
\item
  \texttt{list} \textbf{\texttt{kmeans}} \texttt{int} ---\textgreater{}
  \texttt{list\textless{}list\textgreater{}}
\item
  \textbf{\texttt{kmeans}} (\texttt{list} , \texttt{int})
  ---\textgreater{} \texttt{list\textless{}list\textgreater{}}
\item
  \textbf{\texttt{kmeans}} (\texttt{list}, \texttt{int}, \texttt{int})
  ---\textgreater{} \texttt{list\textless{}list\textgreater{}}
\end{itemize}

\subsubsection{Result:}\label{result-303}

returns the list of clusters (list of instance indices) computed with
the kmeans++ algorithm from the first operand data according to the
number of clusters to split the data into (k) and the maximum number of
iterations to run the algorithm for (If negative, no maximum will be
used) (maxIt). Usage: kmeans(data,k,maxit) returns the list of clusters
(list of instance indices) computed with the kmeans++ algorithm from the
first operand data according to the number of clusters to split the data
into (k). Usage: kmeans(data,k)

\subsubsection{Special cases:}\label{special-cases-82}

\begin{itemize}
\tightlist
\item
  if the lengths of two vectors in the right-hand aren't equal, returns
  0\\
\item
  if the lengths of two vectors in the right-hand aren't equal, returns
  0
\end{itemize}

\subsubsection{Examples:}\label{examples-220}

\begin{verbatim}
kmeans ([[2,4,5], [3,8,2], [1,1,3], [4,3,4]],2,10)  
list<list> var1 <- kmeans ([[2,4,5], [3,8,2], [1,1,3], [4,3,4]],2); // var1 equals []
\end{verbatim}

\begin{center}\rule{0.5\linewidth}{\linethickness}\end{center}

\subsection{\texorpdfstring{\texttt{kurtosis}}{kurtosis}}\label{kurtosis}

\subsubsection{Possible use:}\label{possible-use-314}

\begin{itemize}
\tightlist
\item
  \textbf{\texttt{kurtosis}} (\texttt{list}) ---\textgreater{}
  \texttt{float}
\end{itemize}

\subsubsection{Result:}\label{result-304}

returns kurtosis value computed from the operand list of values

\subsubsection{Special cases:}\label{special-cases-83}

\begin{itemize}
\tightlist
\item
  if the length of the list is lower than 3, returns NaN
\end{itemize}

\subsubsection{Examples:}\label{examples-221}

\begin{verbatim}
 
float var0 <- kurtosis ([1,2,3,4,5]); // var0 equals 1.0
\end{verbatim}

\begin{center}\rule{0.5\linewidth}{\linethickness}\end{center}

\subsection{\texorpdfstring{\texttt{kurtosis}}{kurtosis}}\label{kurtosis-1}

\subsubsection{Possible use:}\label{possible-use-315}

\begin{itemize}
\tightlist
\item
  \textbf{\texttt{kurtosis}} (\texttt{container}) ---\textgreater{}
  \texttt{float}
\item
  \texttt{float} \textbf{\texttt{kurtosis}} \texttt{float}
  ---\textgreater{} \texttt{float}
\item
  \textbf{\texttt{kurtosis}} (\texttt{float} , \texttt{float})
  ---\textgreater{} \texttt{float}
\end{itemize}

\subsubsection{Result:}\label{result-305}

Returns the kurtosis (aka excess) of a data sequence Returns the
kurtosis (aka excess) of a data sequence

\begin{center}\rule{0.5\linewidth}{\linethickness}\end{center}

\subsection{\texorpdfstring{\texttt{last}}{last}}\label{last}

\subsubsection{Possible use:}\label{possible-use-316}

\begin{itemize}
\tightlist
\item
  \textbf{\texttt{last}} (\texttt{string}) ---\textgreater{}
  \texttt{string}
\item
  \textbf{\texttt{last}}
  (\texttt{container\textless{}KeyType,ValueType\textgreater{}})
  ---\textgreater{} \texttt{ValueType}
\item
  \texttt{int} \textbf{\texttt{last}} \texttt{container}
  ---\textgreater{} \texttt{list}
\item
  \textbf{\texttt{last}} (\texttt{int} , \texttt{container})
  ---\textgreater{} \texttt{list}
\end{itemize}

\subsubsection{Result:}\label{result-306}

the last element of the operand

\subsubsection{Comment:}\label{comment-56}

the last operator behavior depends on the nature of the operand

\subsubsection{Special cases:}\label{special-cases-84}

\begin{itemize}
\tightlist
\item
  if it is a map, last returns the value of the last pair (in insertion
  order)\\
\item
  if it is a file, last returns the last element of the content of the
  file (that is also a container)\\
\item
  if it is a population, last returns the last agent of the population\\
\item
  if it is a graph, last returns a list containing the last edge
  created\\
\item
  if it is a matrix, last returns the element at \{length-1,length-1\}
  in the matrix\\
\item
  for a matrix of int or float, it will return 0 if the matrix is
  empty\\
\item
  for a matrix of object or geometry, it will return nil if the matrix
  is empty\\
\item
  if it is a string, last returns a string composed of its last
  character, or an empty string if the operand is empty
\end{itemize}

\begin{verbatim}
 
string var0 <- last ('abce'); // var0 equals 'e'
\end{verbatim}

\begin{itemize}
\tightlist
\item
  if it is a list, last returns the last element of the list, or nil if
  the list is empty
\end{itemize}

\begin{verbatim}
 
int var1 <- last ([1, 2, 3]); // var1 equals 3
\end{verbatim}

\subsubsection{See also:}\label{see-also-122}

\href{OperatorsDH\#first}{first},

\begin{center}\rule{0.5\linewidth}{\linethickness}\end{center}

\subsection{\texorpdfstring{\texttt{last\_index\_of}}{last\_index\_of}}\label{last_index_of}

\subsubsection{Possible use:}\label{possible-use-317}

\begin{itemize}
\tightlist
\item
  \texttt{species} \textbf{\texttt{last\_index\_of}} \texttt{unknown}
  ---\textgreater{} \texttt{int}
\item
  \textbf{\texttt{last\_index\_of}} (\texttt{species} ,
  \texttt{unknown}) ---\textgreater{} \texttt{int}
\item
  \texttt{map} \textbf{\texttt{last\_index\_of}} \texttt{unknown}
  ---\textgreater{} \texttt{unknown}
\item
  \textbf{\texttt{last\_index\_of}} (\texttt{map} , \texttt{unknown})
  ---\textgreater{} \texttt{unknown}
\item
  \texttt{list} \textbf{\texttt{last\_index\_of}} \texttt{unknown}
  ---\textgreater{} \texttt{int}
\item
  \textbf{\texttt{last\_index\_of}} (\texttt{list} , \texttt{unknown})
  ---\textgreater{} \texttt{int}
\item
  \texttt{matrix} \textbf{\texttt{last\_index\_of}} \texttt{unknown}
  ---\textgreater{} \texttt{point}
\item
  \textbf{\texttt{last\_index\_of}} (\texttt{matrix} , \texttt{unknown})
  ---\textgreater{} \texttt{point}
\item
  \texttt{string} \textbf{\texttt{last\_index\_of}} \texttt{string}
  ---\textgreater{} \texttt{int}
\item
  \textbf{\texttt{last\_index\_of}} (\texttt{string} , \texttt{string})
  ---\textgreater{} \texttt{int}
\end{itemize}

\subsubsection{Result:}\label{result-307}

the index of the last occurence of the right operand in the left operand
container

\subsubsection{Comment:}\label{comment-57}

The definition of last\_index\_of and the type of the index depend on
the container

\subsubsection{Special cases:}\label{special-cases-85}

\begin{itemize}
\tightlist
\item
  if the left operand is a species, the last index of an agent is the
  same as its index\\
\item
  if the left operand is a map, last\_index\_of returns the index as an
  int (the key of the pair)
\end{itemize}

\begin{verbatim}
 
unknown var0 <- [1::2, 3::4, 5::4] last_index_of 4; // var0 equals 5
\end{verbatim}

\begin{itemize}
\tightlist
\item
  if the left operand is a list, last\_index\_of returns the index as an
  integer
\end{itemize}

\begin{verbatim}
 
int var1 <- [1,2,3,4,5,6] last_index_of 4; // var1 equals 3 
int var2 <- [4,2,3,4,5,4] last_index_of 4; // var2 equals 5
\end{verbatim}

\begin{itemize}
\tightlist
\item
  if the left operand is a matrix, last\_index\_of returns the index as
  a point
\end{itemize}

\begin{verbatim}
 
point var3 <- matrix([[1,2,3],[4,5,4]]) last_index_of 4; // var3 equals {1.0,2.0}
\end{verbatim}

\begin{itemize}
\tightlist
\item
  if both operands are strings, returns the index within the left-hand
  string of the rightmost occurrence of the given right-hand string
\end{itemize}

\begin{verbatim}
 
int var4 <- "abcabcabc" last_index_of "ca"; // var4 equals 5
\end{verbatim}

\subsubsection{See also:}\label{see-also-123}

\href{OperatorsAA\#at}{at}, \href{OperatorsIM\#index_of}{index\_of},
\href{OperatorsIM\#last_index_of}{last\_index\_of},

\begin{center}\rule{0.5\linewidth}{\linethickness}\end{center}

\subsection{\texorpdfstring{\texttt{last\_of}}{last\_of}}\label{last_of}

Same signification as \href{OperatorsIM\#last}{last}

\begin{center}\rule{0.5\linewidth}{\linethickness}\end{center}

\subsection{\texorpdfstring{\texttt{last\_with}}{last\_with}}\label{last_with}

\subsubsection{Possible use:}\label{possible-use-318}

\begin{itemize}
\tightlist
\item
  \texttt{container} \textbf{\texttt{last\_with}}
  \texttt{any\ expression} ---\textgreater{} \texttt{unknown}
\item
  \textbf{\texttt{last\_with}} (\texttt{container} ,
  \texttt{any\ expression}) ---\textgreater{} \texttt{unknown}
\end{itemize}

\subsubsection{Result:}\label{result-308}

the last element of the left-hand operand that makes the right-hand
operand evaluate to true.

\subsubsection{Comment:}\label{comment-58}

in the right-hand operand, the keyword each can be used to represent, in
turn, each of the right-hand operand elements.

\subsubsection{Special cases:}\label{special-cases-86}

\begin{itemize}
\tightlist
\item
  if the left-hand operand is nil, last\_with throws an error.\\
\item
  If there is no element that satisfies the condition, it returns nil\\
\item
  if the left-operand is a map, the keyword each will contain each value
\end{itemize}

\begin{verbatim}
 
unknown var4 <- [1::2, 3::4, 5::6] last_with (each >= 4); // var4 equals 6 
unknown var5 <- [1::2, 3::4, 5::6].pairs last_with (each.value >= 4); // var5 equals (5::6)
\end{verbatim}

\subsubsection{Examples:}\label{examples-222}

\begin{verbatim}
 
unknown var0 <- [1,2,3,4,5,6,7,8] last_with (each > 3); // var0 equals 8 
unknown var2 <- g2 last_with (length(g2 out_edges_of each) = 0 ); // var2 equals node11 
unknown var3 <- (list(node) last_with (round(node(each).location.x) > 32); // var3 equals node3
\end{verbatim}

\subsubsection{See also:}\label{see-also-124}

\href{OperatorsDH\#group_by}{group\_by},
\href{OperatorsDH\#first_with}{first\_with},
\href{OperatorsSZ\#where}{where},

\begin{center}\rule{0.5\linewidth}{\linethickness}\end{center}

\subsection{\texorpdfstring{\texttt{layout}}{layout}}\label{layout}

\subsubsection{Possible use:}\label{possible-use-319}

\begin{itemize}
\tightlist
\item
  \texttt{graph} \textbf{\texttt{layout}} \texttt{string}
  ---\textgreater{} \texttt{graph}
\item
  \textbf{\texttt{layout}} (\texttt{graph} , \texttt{string})
  ---\textgreater{} \texttt{graph}
\item
  \textbf{\texttt{layout}} (\texttt{graph}, \texttt{string},
  \texttt{int}) ---\textgreater{} \texttt{graph}
\item
  \textbf{\texttt{layout}} (\texttt{graph}, \texttt{string},
  \texttt{int}, \texttt{map\textless{}string,unknown\textgreater{}})
  ---\textgreater{} \texttt{graph}
\end{itemize}

\subsubsection{Result:}\label{result-309}

layouts a GAMA graph.

\begin{center}\rule{0.5\linewidth}{\linethickness}\end{center}

\subsection{\texorpdfstring{\texttt{length}}{length}}\label{length}

\subsubsection{Possible use:}\label{possible-use-320}

\begin{itemize}
\tightlist
\item
  \textbf{\texttt{length}}
  (\texttt{container\textless{}KeyType,ValueType\textgreater{}})
  ---\textgreater{} \texttt{int}
\item
  \textbf{\texttt{length}} (\texttt{string}) ---\textgreater{}
  \texttt{int}
\end{itemize}

\subsubsection{Result:}\label{result-310}

the number of elements contained in the operand

\subsubsection{Comment:}\label{comment-59}

the length operator behavior depends on the nature of the operand

\subsubsection{Special cases:}\label{special-cases-87}

\begin{itemize}
\tightlist
\item
  if it is a population, length returns number of agents of the
  population\\
\item
  if it is a graph, length returns the number of vertexes or of edges
  (depending on the way it was created)\\
\item
  if it is a list or a map, length returns the number of elements in the
  list or map
\end{itemize}

\begin{verbatim}
 
int var0 <- length([12,13]); // var0 equals 2 
int var1 <- length([]); // var1 equals 0
\end{verbatim}

\begin{itemize}
\tightlist
\item
  if it is a matrix, length returns the number of cells
\end{itemize}

\begin{verbatim}
 
int var2 <- length(matrix([["c11","c12","c13"],["c21","c22","c23"]])); // var2 equals 6
\end{verbatim}

\begin{itemize}
\tightlist
\item
  if it is a string, length returns the number of characters
\end{itemize}

\begin{verbatim}
 
int var3 <- length ('I am an agent'); // var3 equals 13
\end{verbatim}

\begin{center}\rule{0.5\linewidth}{\linethickness}\end{center}

\subsection{\texorpdfstring{\texttt{lgamma}}{lgamma}}\label{lgamma}

Same signification as \href{OperatorsIM\#log_gamma}{log\_gamma}

\begin{center}\rule{0.5\linewidth}{\linethickness}\end{center}

\subsection{\texorpdfstring{\texttt{line}}{line}}\label{line}

\subsubsection{Possible use:}\label{possible-use-321}

\begin{itemize}
\tightlist
\item
  \textbf{\texttt{line}}
  (\texttt{container\textless{}geometry\textgreater{}})
  ---\textgreater{} \texttt{geometry}
\item
  \texttt{container\textless{}geometry\textgreater{}}
  \textbf{\texttt{line}} \texttt{float} ---\textgreater{}
  \texttt{geometry}
\item
  \textbf{\texttt{line}}
  (\texttt{container\textless{}geometry\textgreater{}} , \texttt{float})
  ---\textgreater{} \texttt{geometry}
\end{itemize}

\subsubsection{Result:}\label{result-311}

A polyline geometry from the given list of points represented as a
cylinder of radius r. A polyline geometry from the given list of points.

\subsubsection{Special cases:}\label{special-cases-88}

\begin{itemize}
\tightlist
\item
  if the operand is nil, returns the point geometry \{0,0\}\\
\item
  if the operand is composed of a single point, returns a point
  geometry.\\
\item
  if the operand is nil, returns the point geometry \{0,0\}\\
\item
  if the operand is composed of a single point, returns a point
  geometry.\\
\item
  if a radius is added, the given list of points represented as a
  cylinder of radius r
\end{itemize}

\begin{verbatim}
 
geometry var0 <- polyline([{0,0}, {0,10}, {10,10}, {10,0}],0.2); // var0 equals a polyline geometry composed of the 4 points.
\end{verbatim}

\subsubsection{Examples:}\label{examples-223}

\begin{verbatim}
 
geometry var1 <- polyline([{0,0}, {0,10}, {10,10}, {10,0}]); // var1 equals a polyline geometry composed of the 4 points.
\end{verbatim}

\subsubsection{See also:}\label{see-also-125}

\href{OperatorsAA\#around}{around}, \href{OperatorsBC\#circle}{circle},
\href{OperatorsBC\#cone}{cone}, \href{OperatorsIM\#link}{link},
\href{OperatorsNR\#norm}{norm}, \href{OperatorsNR\#point}{point},
\href{OperatorsSZ\#polygone}{polygone},
\href{OperatorsNR\#rectangle}{rectangle},
\href{OperatorsSZ\#square}{square},
\href{OperatorsSZ\#triangle}{triangle},

\begin{center}\rule{0.5\linewidth}{\linethickness}\end{center}

\subsection{\texorpdfstring{\texttt{link}}{link}}\label{link}

\subsubsection{Possible use:}\label{possible-use-322}

\begin{itemize}
\tightlist
\item
  \texttt{geometry} \textbf{\texttt{link}} \texttt{geometry}
  ---\textgreater{} \texttt{geometry}
\item
  \textbf{\texttt{link}} (\texttt{geometry} , \texttt{geometry})
  ---\textgreater{} \texttt{geometry}
\end{itemize}

\subsubsection{Result:}\label{result-312}

A dynamic line geometry between the location of the two operands

\subsubsection{Comment:}\label{comment-60}

The geometry of the link is a line between the locations of the two
operands, which is built and maintained dynamically

\subsubsection{Special cases:}\label{special-cases-89}

\begin{itemize}
\tightlist
\item
  if one of the operands is nil, link returns a point geometry at the
  location of the other. If both are null, it returns a point geometry
  at \{0,0\}
\end{itemize}

\subsubsection{Examples:}\label{examples-224}

\begin{verbatim}
 
geometry var0 <- link (geom1,geom2); // var0 equals a link geometry between geom1 and geom2.
\end{verbatim}

\subsubsection{See also:}\label{see-also-126}

\href{OperatorsAA\#around}{around}, \href{OperatorsBC\#circle}{circle},
\href{OperatorsBC\#cone}{cone}, \href{OperatorsIM\#line}{line},
\href{OperatorsNR\#norm}{norm}, \href{OperatorsNR\#point}{point},
\href{OperatorsNR\#polygon}{polygon},
\href{OperatorsNR\#polyline}{polyline},
\href{OperatorsNR\#rectangle}{rectangle},
\href{OperatorsSZ\#square}{square},
\href{OperatorsSZ\#triangle}{triangle},

\begin{center}\rule{0.5\linewidth}{\linethickness}\end{center}

\subsection{\texorpdfstring{\texttt{list}}{list}}\label{list}

\subsubsection{Possible use:}\label{possible-use-323}

\begin{itemize}
\tightlist
\item
  \textbf{\texttt{list}} (\texttt{any}) ---\textgreater{} \texttt{list}
\end{itemize}

\subsubsection{Result:}\label{result-313}

Casts the operand into the type list

\begin{center}\rule{0.5\linewidth}{\linethickness}\end{center}

\subsection{\texorpdfstring{\texttt{list\_with}}{list\_with}}\label{list_with}

\subsubsection{Possible use:}\label{possible-use-324}

\begin{itemize}
\tightlist
\item
  \texttt{int} \textbf{\texttt{list\_with}} \texttt{any\ expression}
  ---\textgreater{} \texttt{list}
\item
  \textbf{\texttt{list\_with}} (\texttt{int} , \texttt{any\ expression})
  ---\textgreater{} \texttt{list}
\end{itemize}

\subsubsection{Result:}\label{result-314}

creates a list with a size provided by the first operand, and filled
with the second operand

\subsubsection{Comment:}\label{comment-61}

Note that the right operand should be positive, and that the second one
is evaluated for each position in the list.

\subsubsection{See also:}\label{see-also-127}

\href{OperatorsIM\#list}{list},

\begin{center}\rule{0.5\linewidth}{\linethickness}\end{center}

\subsection{\texorpdfstring{\texttt{ln}}{ln}}\label{ln}

\subsubsection{Possible use:}\label{possible-use-325}

\begin{itemize}
\tightlist
\item
  \textbf{\texttt{ln}} (\texttt{float}) ---\textgreater{} \texttt{float}
\item
  \textbf{\texttt{ln}} (\texttt{int}) ---\textgreater{} \texttt{float}
\end{itemize}

\subsubsection{Result:}\label{result-315}

Returns the natural logarithm (base e) of the operand.

\subsubsection{Special cases:}\label{special-cases-90}

\begin{itemize}
\tightlist
\item
  an exception is raised if the operand is less than zero.
\end{itemize}

\subsubsection{Examples:}\label{examples-225}

\begin{verbatim}
 
float var0 <- ln(exp(1)); // var0 equals 1.0 
float var1 <- ln(1); // var1 equals 0.0
\end{verbatim}

\subsubsection{See also:}\label{see-also-128}

\href{OperatorsDH\#exp}{exp},

\begin{center}\rule{0.5\linewidth}{\linethickness}\end{center}

\subsection{\texorpdfstring{\texttt{load\_graph\_from\_file}}{load\_graph\_from\_file}}\label{load_graph_from_file}

\subsubsection{Possible use:}\label{possible-use-326}

\begin{itemize}
\tightlist
\item
  \textbf{\texttt{load\_graph\_from\_file}} (\texttt{string})
  ---\textgreater{} \texttt{graph}
\item
  \texttt{string} \textbf{\texttt{load\_graph\_from\_file}}
  \texttt{file} ---\textgreater{} \texttt{graph}
\item
  \textbf{\texttt{load\_graph\_from\_file}} (\texttt{string} ,
  \texttt{file}) ---\textgreater{} \texttt{graph}
\item
  \texttt{string} \textbf{\texttt{load\_graph\_from\_file}}
  \texttt{string} ---\textgreater{} \texttt{graph}
\item
  \textbf{\texttt{load\_graph\_from\_file}} (\texttt{string} ,
  \texttt{string}) ---\textgreater{} \texttt{graph}
\item
  \textbf{\texttt{load\_graph\_from\_file}} (\texttt{string},
  \texttt{species}, \texttt{species}) ---\textgreater{} \texttt{graph}
\item
  \textbf{\texttt{load\_graph\_from\_file}} (\texttt{string},
  \texttt{file}, \texttt{species}, \texttt{species}) ---\textgreater{}
  \texttt{graph}
\item
  \textbf{\texttt{load\_graph\_from\_file}} (\texttt{string},
  \texttt{string}, \texttt{species}, \texttt{species}) ---\textgreater{}
  \texttt{graph}
\item
  \textbf{\texttt{load\_graph\_from\_file}} (\texttt{string},
  \texttt{string}, \texttt{species}, \texttt{species}, \texttt{bool})
  ---\textgreater{} \texttt{graph}
\end{itemize}

\subsubsection{Result:}\label{result-316}

returns a graph loaded from a given file encoded into a given format.
The last boolean parameter indicates whether the resulting graph will be
considered as spatial or not by GAMA loads a graph from a file

\subsubsection{Comment:}\label{comment-62}

Available formats: ``pajek'': Pajek (Slovene word for Spider) is a
program, for Windows, for analysis and visualization of large networks.
See: \url{http://pajek.imfm.si/doku.php?id=pajek} for more
details.``lgl'': LGL is a compendium of applications for making the
visualization of large networks and trees tractable. See:
\url{http://lgl.sourceforge.net/} for more details.``dot'': DOT is a
plain text graph description language. It is a simple way of describing
graphs that both humans and computer programs can use. See:
\url{http://en.wikipedia.org/wiki/DOT_language} for more
details.``edge'': This format is a simple text file with numeric vertex
ids defining the edges.``gexf'': GEXF (Graph Exchange XML Format) is a
language for describing complex networks structures, their associated
data and dynamics. Started in 2007 at Gephi project by different actors,
deeply involved in graph exchange issues, the gexf specifications are
mature enough to claim being both extensible and open, and suitable for
real specific applications. See: \url{http://gexf.net/format/} for more
details.``graphml'': GraphML is a comprehensive and easy-to-use file
format for graphs based on XML. See:
\url{http://graphml.graphdrawing.org/} for more details.``tlp'' or
``tulip'': TLP is the Tulip software graph format. See:
\url{http://tulip.labri.fr/TulipDrupal/?q=tlp-file-format} for more
details. ``ncol'': This format is used by the Large Graph Layout progra.
It is simply a symbolic weighted edge list. It is a simple text file
with one edge per line. An edge is defined by two symbolic vertex names
separated by whitespace. (The symbolic vertex names themselves cannot
contain whitespace.) They might followed by an optional number, this
will be the weight of the edge. See:
\url{http://bioinformatics.icmb.utexas.edu/lgl} for more details.The map
operand should includes following elements:Available formats: ``pajek'':
Pajek (Slovene word for Spider) is a program, for Windows, for analysis
and visualization of large networks. See:
\url{http://pajek.imfm.si/doku.php?id=pajek} for more details.``lgl'':
LGL is a compendium of applications for making the visualization of
large networks and trees tractable. See:
\url{http://lgl.sourceforge.net/} for more details.``dot'': DOT is a
plain text graph description language. It is a simple way of describing
graphs that both humans and computer programs can use. See:
\url{http://en.wikipedia.org/wiki/DOT_language} for more
details.``edge'': This format is a simple text file with numeric vertex
ids defining the edges.``gexf'': GEXF (Graph Exchange XML Format) is a
language for describing complex networks structures, their associated
data and dynamics. Started in 2007 at Gephi project by different actors,
deeply involved in graph exchange issues, the gexf specifications are
mature enough to claim being both extensible and open, and suitable for
real specific applications. See: \url{http://gexf.net/format/} for more
details.``graphml'': GraphML is a comprehensive and easy-to-use file
format for graphs based on XML. See:
\url{http://graphml.graphdrawing.org/} for more details.``tlp'' or
``tulip'': TLP is the Tulip software graph format. See:
\url{http://tulip.labri.fr/TulipDrupal/?q=tlp-file-format} for more
details. ``ncol'': This format is used by the Large Graph Layout progra.
It is simply a symbolic weighted edge list. It is a simple text file
with one edge per line. An edge is defined by two symbolic vertex names
separated by whitespace. (The symbolic vertex names themselves cannot
contain whitespace.) They might followed by an optional number, this
will be the weight of the edge. See:
\url{http://bioinformatics.icmb.utexas.edu/lgl} for more details.The map
operand should includes following elements:

\subsubsection{Special cases:}\label{special-cases-91}

\begin{itemize}
\tightlist
\item
  ``format'': the format of the file\\
\item
  ``filename'': the filename of the file containing the network\\
\item
  ``edges\_species'': the species of edges\\
\item
  ``vertices\_specy'': the species of vertices\\
\item
  ``format'': the format of the file\\
\item
  ``filename'': the filename of the file containing the network\\
\item
  ``edges\_species'': the species of edges\\
\item
  ``vertices\_specy'': the species of vertices\\
\item
  ``format'': the format of the file, ``file'': the file containing the
  network
\end{itemize}

\begin{verbatim}
graph<myVertexSpecy,myEdgeSpecy> myGraph <- load_graph_from_file(           "pajek",            "example_of_Pajek_file"); 
\end{verbatim}

\begin{itemize}
\tightlist
\item
  ``format'': the format of the file, ``filename'': the filename of the
  file containing the network
\end{itemize}

\begin{verbatim}
graph<myVertexSpecy,myEdgeSpecy> myGraph <- load_graph_from_file(           "pajek",            "example_of_Pajek_file"); 
\end{verbatim}

\begin{itemize}
\tightlist
\item
  ``filename'': the filename of the file containing the network,
  ``edges\_species'': the species of edges, ``vertices\_specy'': the
  species of vertices
\end{itemize}

\begin{verbatim}
graph<myVertexSpecy,myEdgeSpecy> myGraph <- load_graph_from_file(           "pajek",            "./example_of_Pajek_file",          myVertexSpecy,          myEdgeSpecy ); 
\end{verbatim}

\begin{itemize}
\tightlist
\item
  ``format'': the format of the file, ``file'': the file containing the
  network, ``edges\_species'': the species of edges,
  ``vertices\_specy'': the species of vertices
\end{itemize}

\begin{verbatim}
graph<myVertexSpecy,myEdgeSpecy> myGraph <- load_graph_from_file(           "pajek",            "example_of_Pajek_file",            myVertexSpecy,          myEdgeSpecy ); 
\end{verbatim}

\begin{itemize}
\tightlist
\item
  ``file'': the file containing the network
\end{itemize}

\begin{verbatim}
graph<myVertexSpecy,myEdgeSpecy> myGraph <- load_graph_from_file(           "pajek",            "example_of_Pajek_file"); 
\end{verbatim}

\subsubsection{Examples:}\label{examples-226}

\begin{verbatim}
graph<myVertexSpecy,myEdgeSpecy> myGraph <- load_graph_from_file(           "pajek",            "./example_of_Pajek_file",          myVertexSpecy,          myEdgeSpecy , true); graph<myVertexSpecy,myEdgeSpecy> myGraph <- load_graph_from_file(          "pajek",            "./example_of_Pajek_file",          myVertexSpecy,          myEdgeSpecy); 
\end{verbatim}

\begin{center}\rule{0.5\linewidth}{\linethickness}\end{center}

\subsection{\texorpdfstring{\texttt{load\_shortest\_paths}}{load\_shortest\_paths}}\label{load_shortest_paths}

\subsubsection{Possible use:}\label{possible-use-327}

\begin{itemize}
\tightlist
\item
  \texttt{graph} \textbf{\texttt{load\_shortest\_paths}} \texttt{matrix}
  ---\textgreater{} \texttt{graph}
\item
  \textbf{\texttt{load\_shortest\_paths}} (\texttt{graph} ,
  \texttt{matrix}) ---\textgreater{} \texttt{graph}
\end{itemize}

\subsubsection{Result:}\label{result-317}

put in the graph cache the computed shortest paths contained in the
matrix (rows: source, columns: target)

\subsubsection{Examples:}\label{examples-227}

\begin{verbatim}
 
graph var0 <- load_shortest_paths(shortest_paths_matrix); // var0 equals return my_graph with all the shortest paths computed
\end{verbatim}

\begin{center}\rule{0.5\linewidth}{\linethickness}\end{center}

\subsection{\texorpdfstring{\texttt{load\_sub\_model}}{load\_sub\_model}}\label{load_sub_model}

\subsubsection{Possible use:}\label{possible-use-328}

\begin{itemize}
\tightlist
\item
  \texttt{string} \textbf{\texttt{load\_sub\_model}} \texttt{string}
  ---\textgreater{} \texttt{msi.gama.kernel.experiment.IExperimentAgent}
\item
  \textbf{\texttt{load\_sub\_model}} (\texttt{string} , \texttt{string})
  ---\textgreater{} \texttt{msi.gama.kernel.experiment.IExperimentAgent}
\end{itemize}

\subsubsection{Result:}\label{result-318}

Load a submodel

\subsubsection{Comment:}\label{comment-63}

loaded submodel

\begin{center}\rule{0.5\linewidth}{\linethickness}\end{center}

\subsection{\texorpdfstring{\texttt{log}}{log}}\label{log}

\subsubsection{Possible use:}\label{possible-use-329}

\begin{itemize}
\tightlist
\item
  \textbf{\texttt{log}} (\texttt{float}) ---\textgreater{}
  \texttt{float}
\item
  \textbf{\texttt{log}} (\texttt{int}) ---\textgreater{} \texttt{float}
\end{itemize}

\subsubsection{Result:}\label{result-319}

Returns the logarithm (base 10) of the operand.

\subsubsection{Special cases:}\label{special-cases-92}

\begin{itemize}
\tightlist
\item
  an exception is raised if the operand is equals or less than zero.
\end{itemize}

\subsubsection{Examples:}\label{examples-228}

\begin{verbatim}
 
float var0 <- log(10); // var0 equals 1.0 
float var1 <- log(1); // var1 equals 0.0
\end{verbatim}

\subsubsection{See also:}\label{see-also-129}

\href{OperatorsIM\#ln}{ln},

\begin{center}\rule{0.5\linewidth}{\linethickness}\end{center}

\subsection{\texorpdfstring{\texttt{log\_gamma}}{log\_gamma}}\label{log_gamma}

\subsubsection{Possible use:}\label{possible-use-330}

\begin{itemize}
\tightlist
\item
  \textbf{\texttt{log\_gamma}} (\texttt{float}) ---\textgreater{}
  \texttt{float}
\end{itemize}

\subsubsection{Result:}\label{result-320}

Returns the log of the value of the Gamma function at x.

\begin{center}\rule{0.5\linewidth}{\linethickness}\end{center}

\subsection{\texorpdfstring{\texttt{lower\_case}}{lower\_case}}\label{lower_case}

\subsubsection{Possible use:}\label{possible-use-331}

\begin{itemize}
\tightlist
\item
  \textbf{\texttt{lower\_case}} (\texttt{string}) ---\textgreater{}
  \texttt{string}
\end{itemize}

\subsubsection{Result:}\label{result-321}

Converts all of the characters in the string operand to lower case

\subsubsection{Examples:}\label{examples-229}

\begin{verbatim}
 
string var0 <- lower_case("Abc"); // var0 equals 'abc'
\end{verbatim}

\subsubsection{See also:}\label{see-also-130}

\href{OperatorsSZ\#upper_case}{upper\_case},

\begin{center}\rule{0.5\linewidth}{\linethickness}\end{center}

\subsection{\texorpdfstring{\texttt{main\_connected\_component}}{main\_connected\_component}}\label{main_connected_component}

\subsubsection{Possible use:}\label{possible-use-332}

\begin{itemize}
\tightlist
\item
  \textbf{\texttt{main\_connected\_component}} (\texttt{graph})
  ---\textgreater{} \texttt{graph}
\end{itemize}

\subsubsection{Result:}\label{result-322}

returns the sub-graph corresponding to the main connected components of
the graph

\subsubsection{Examples:}\label{examples-230}

\begin{verbatim}
 
graph var0 <- main_connected_component(my_graph); // var0 equals the sub-graph corresponding to the main connected components of the graph
\end{verbatim}

\subsubsection{See also:}\label{see-also-131}

\href{OperatorsBC\#connected_components_of}{connected\_components\_of},

\begin{center}\rule{0.5\linewidth}{\linethickness}\end{center}

\subsection{\texorpdfstring{\texttt{map}}{map}}\label{map}

\subsubsection{Possible use:}\label{possible-use-333}

\begin{itemize}
\tightlist
\item
  \textbf{\texttt{map}} (\texttt{any}) ---\textgreater{} \texttt{map}
\end{itemize}

\subsubsection{Result:}\label{result-323}

Casts the operand into the type map

\begin{center}\rule{0.5\linewidth}{\linethickness}\end{center}

\subsection{\texorpdfstring{\texttt{masked\_by}}{masked\_by}}\label{masked_by}

\subsubsection{Possible use:}\label{possible-use-334}

\begin{itemize}
\tightlist
\item
  \texttt{geometry} \textbf{\texttt{masked\_by}}
  \texttt{container\textless{}geometry\textgreater{}} ---\textgreater{}
  \texttt{geometry}
\item
  \textbf{\texttt{masked\_by}} (\texttt{geometry} ,
  \texttt{container\textless{}geometry\textgreater{}}) ---\textgreater{}
  \texttt{geometry}
\item
  \textbf{\texttt{masked\_by}} (\texttt{geometry},
  \texttt{container\textless{}geometry\textgreater{}}, \texttt{int})
  ---\textgreater{} \texttt{geometry}
\end{itemize}

\subsubsection{Examples:}\label{examples-231}

\begin{verbatim}
 
geometry var0 <- perception_geom masked_by obstacle_list; // var0 equals the geometry representing the part of perception_geom visible from the agent position considering the list of obstacles obstacle_list. 
geometry var1 <- perception_geom masked_by obstacle_list; // var1 equals the geometry representing the part of perception_geom visible from the agent position considering the list of obstacles obstacle_list.
\end{verbatim}

\begin{center}\rule{0.5\linewidth}{\linethickness}\end{center}

\subsection{\texorpdfstring{\texttt{material}}{material}}\label{material-4}

\subsubsection{Possible use:}\label{possible-use-335}

\begin{itemize}
\tightlist
\item
  \texttt{float} \textbf{\texttt{material}} \texttt{float}
  ---\textgreater{} \texttt{msi.gama.util.GamaMaterial}
\item
  \textbf{\texttt{material}} (\texttt{float} , \texttt{float})
  ---\textgreater{} \texttt{msi.gama.util.GamaMaterial}
\end{itemize}

\subsubsection{Result:}\label{result-324}

Returns

\subsubsection{Examples:}\label{examples-232}

\begin{verbatim}
 
\end{verbatim}

\subsubsection{See also:}\label{see-also-132}

\href{OperatorsSZ\#}{},

\begin{center}\rule{0.5\linewidth}{\linethickness}\end{center}

\subsection{\texorpdfstring{\texttt{material}}{material}}\label{material-5}

\subsubsection{Possible use:}\label{possible-use-336}

\begin{itemize}
\tightlist
\item
  \textbf{\texttt{material}} (\texttt{any}) ---\textgreater{}
  \texttt{material}
\end{itemize}

\subsubsection{Result:}\label{result-325}

Casts the operand into the type material

\begin{center}\rule{0.5\linewidth}{\linethickness}\end{center}

\subsection{\texorpdfstring{\texttt{matrix}}{matrix}}\label{matrix}

\subsubsection{Possible use:}\label{possible-use-337}

\begin{itemize}
\tightlist
\item
  \textbf{\texttt{matrix}} (\texttt{any}) ---\textgreater{}
  \texttt{matrix}
\end{itemize}

\subsubsection{Result:}\label{result-326}

Casts the operand into the type matrix

\begin{center}\rule{0.5\linewidth}{\linethickness}\end{center}

\subsection{\texorpdfstring{\texttt{matrix\_with}}{matrix\_with}}\label{matrix_with}

\subsubsection{Possible use:}\label{possible-use-338}

\begin{itemize}
\tightlist
\item
  \texttt{point} \textbf{\texttt{matrix\_with}} \texttt{any\ expression}
  ---\textgreater{} \texttt{matrix}
\item
  \textbf{\texttt{matrix\_with}} (\texttt{point} ,
  \texttt{any\ expression}) ---\textgreater{} \texttt{matrix}
\end{itemize}

\subsubsection{Result:}\label{result-327}

creates a matrix with a size provided by the first operand, and filled
with the second operand

\subsubsection{Comment:}\label{comment-64}

Note that both components of the right operand point should be positive,
otherwise an exception is raised.

\subsubsection{See also:}\label{see-also-133}

\href{OperatorsIM\#matrix}{matrix},
\href{OperatorsAA\#as_matrix}{as\_matrix},

\begin{center}\rule{0.5\linewidth}{\linethickness}\end{center}

\subsection{\texorpdfstring{\texttt{max}}{max}}\label{max}

\subsubsection{Possible use:}\label{possible-use-339}

\begin{itemize}
\tightlist
\item
  \textbf{\texttt{max}} (\texttt{container}) ---\textgreater{}
  \texttt{unknown}
\end{itemize}

\subsubsection{Result:}\label{result-328}

the maximum element found in the operand

\subsubsection{Comment:}\label{comment-65}

the max operator behavior depends on the nature of the operand

\subsubsection{Special cases:}\label{special-cases-93}

\begin{itemize}
\tightlist
\item
  if it is a population of a list of other type: max transforms all
  elements into integer and returns the maximum of them\\
\item
  if it is a map, max returns the maximum among the list of all elements
  value\\
\item
  if it is a file, max returns the maximum of the content of the file
  (that is also a container)\\
\item
  if it is a graph, max returns the maximum of the list of the elements
  of the graph (that can be the list of edges or vertexes depending on
  the graph)\\
\item
  if it is a matrix of int, float or object, max returns the maximum of
  all the numerical elements (thus all elements for integer and float
  matrices)\\
\item
  if it is a matrix of geometry, max returns the maximum of the list of
  the geometries\\
\item
  if it is a matrix of another type, max returns the maximum of the
  elements transformed into float\\
\item
  if it is a list of int of float, max returns the maximum of all the
  elements
\end{itemize}

\begin{verbatim}
 
unknown var0 <- max ([100, 23.2, 34.5]); // var0 equals 100.0
\end{verbatim}

\begin{itemize}
\tightlist
\item
  if it is a list of points: max returns the maximum of all points as a
  point (i.e.~the point with the greatest coordinate on the x-axis, in
  case of equality the point with the greatest coordinate on the y-axis
  is chosen. If all the points are equal, the first one is returned. )
\end{itemize}

\begin{verbatim}
 
unknown var1 <- max([{1.0,3.0},{3.0,5.0},{9.0,1.0},{7.0,8.0}]); // var1 equals {9.0,1.0}
\end{verbatim}

\subsubsection{See also:}\label{see-also-134}

\href{OperatorsIM\#min}{min},

\begin{center}\rule{0.5\linewidth}{\linethickness}\end{center}

\subsection{\texorpdfstring{\texttt{max\_flow\_between}}{max\_flow\_between}}\label{max_flow_between}

\subsubsection{Possible use:}\label{possible-use-340}

\begin{itemize}
\tightlist
\item
  \textbf{\texttt{max\_flow\_between}} (\texttt{graph},
  \texttt{unknown}, \texttt{unknown}) ---\textgreater{}
  \texttt{msi.gama.util.GamaMap\textless{}java.lang.Object,java.lang.Double\textgreater{}}
\end{itemize}

\subsubsection{Result:}\label{result-329}

The max flow (map in a graph between the source and the sink using
Edmonds-Karp algorithm

\subsubsection{Examples:}\label{examples-233}

\begin{verbatim}
max_flow_between(my_graph, vertice1, vertice2) 
\end{verbatim}

\begin{center}\rule{0.5\linewidth}{\linethickness}\end{center}

\subsection{\texorpdfstring{\texttt{max\_of}}{max\_of}}\label{max_of}

\subsubsection{Possible use:}\label{possible-use-341}

\begin{itemize}
\tightlist
\item
  \texttt{container} \textbf{\texttt{max\_of}} \texttt{any\ expression}
  ---\textgreater{} \texttt{unknown}
\item
  \textbf{\texttt{max\_of}} (\texttt{container} ,
  \texttt{any\ expression}) ---\textgreater{} \texttt{unknown}
\end{itemize}

\subsubsection{Result:}\label{result-330}

the maximum value of the right-hand expression evaluated on each of the
elements of the left-hand operand

\subsubsection{Comment:}\label{comment-66}

in the right-hand operand, the keyword each can be used to represent, in
turn, each of the right-hand operand elements.

\subsubsection{Special cases:}\label{special-cases-94}

\begin{itemize}
\tightlist
\item
  As of GAMA 1.6, if the left-hand operand is nil or empty, max\_of
  throws an error\\
\item
  if the left-operand is a map, the keyword each will contain each value
\end{itemize}

\begin{verbatim}
 
unknown var4 <- [1::2, 3::4, 5::6] max_of (each + 3); // var4 equals 9
\end{verbatim}

\subsubsection{Examples:}\label{examples-234}

\begin{verbatim}
 
unknown var0 <- [1,2,4,3,5,7,6,8] max_of (each * 100 ); // var0 equals 800graph g2 <- as_edge_graph([{1,5}::{12,45},{12,45}::{34,56}]);  
unknown var2 <- g2.vertices max_of (g2 degree_of( each )); // var2 equals 2 
unknown var3 <- (list(node) max_of (round(node(each).location.x)); // var3 equals 96
\end{verbatim}

\subsubsection{See also:}\label{see-also-135}

\href{OperatorsIM\#min_of}{min\_of},

\begin{center}\rule{0.5\linewidth}{\linethickness}\end{center}

\subsection{\texorpdfstring{\texttt{maximal\_cliques\_of}}{maximal\_cliques\_of}}\label{maximal_cliques_of}

\subsubsection{Possible use:}\label{possible-use-342}

\begin{itemize}
\tightlist
\item
  \textbf{\texttt{maximal\_cliques\_of}} (\texttt{graph})
  ---\textgreater{} \texttt{list\textless{}list\textgreater{}}
\end{itemize}

\subsubsection{Result:}\label{result-331}

returns the maximal cliques of a graph using the Bron-Kerbosch clique
detection algorithm: A clique is maximal if it is impossible to enlarge
it by adding another vertex from the graph. Note that a maximal clique
is not necessarily the biggest clique in the graph.

\subsubsection{Examples:}\label{examples-235}

\begin{verbatim}
graph my_graph <- graph([]);  
list<list> var1 <- maximal_cliques_of (my_graph); // var1 equals the list of all the maximal cliques as list
\end{verbatim}

\subsubsection{See also:}\label{see-also-136}

\href{OperatorsBC\#biggest_cliques_of}{biggest\_cliques\_of},

\begin{center}\rule{0.5\linewidth}{\linethickness}\end{center}

\subsection{\texorpdfstring{\texttt{mean}}{mean}}\label{mean}

\subsubsection{Possible use:}\label{possible-use-343}

\begin{itemize}
\tightlist
\item
  \textbf{\texttt{mean}} (\texttt{container}) ---\textgreater{}
  \texttt{unknown}
\end{itemize}

\subsubsection{Result:}\label{result-332}

the mean of all the elements of the operand

\subsubsection{Comment:}\label{comment-67}

the elements of the operand are summed (see sum for more details about
the sum of container elements ) and then the sum value is divided by the
number of elements.

\subsubsection{Special cases:}\label{special-cases-95}

\begin{itemize}
\tightlist
\item
  if the container contains points, the result will be a point. If the
  container contains rgb values, the result will be a rgb color
\end{itemize}

\subsubsection{Examples:}\label{examples-236}

\begin{verbatim}
 
unknown var0 <- mean ([4.5, 3.5, 5.5, 7.0]); // var0 equals 5.125 
\end{verbatim}

\subsubsection{See also:}\label{see-also-137}

\href{OperatorsSZ\#sum}{sum},

\begin{center}\rule{0.5\linewidth}{\linethickness}\end{center}

\subsection{\texorpdfstring{\texttt{mean\_deviation}}{mean\_deviation}}\label{mean_deviation}

\subsubsection{Possible use:}\label{possible-use-344}

\begin{itemize}
\tightlist
\item
  \textbf{\texttt{mean\_deviation}} (\texttt{container})
  ---\textgreater{} \texttt{float}
\end{itemize}

\subsubsection{Result:}\label{result-333}

the deviation from the mean of all the elements of the operand. See
Mean\_deviation for more details.

\subsubsection{Comment:}\label{comment-68}

The operator casts all the numerical element of the list into float. The
elements that are not numerical are discarded.

\subsubsection{Examples:}\label{examples-237}

\begin{verbatim}
 
float var0 <- mean_deviation ([4.5, 3.5, 5.5, 7.0]); // var0 equals 1.125
\end{verbatim}

\subsubsection{See also:}\label{see-also-138}

\href{OperatorsIM\#mean}{mean},
\href{OperatorsSZ\#standard_deviation}{standard\_deviation},

\begin{center}\rule{0.5\linewidth}{\linethickness}\end{center}

\subsection{\texorpdfstring{\texttt{mean\_of}}{mean\_of}}\label{mean_of}

\subsubsection{Possible use:}\label{possible-use-345}

\begin{itemize}
\tightlist
\item
  \texttt{container} \textbf{\texttt{mean\_of}} \texttt{any\ expression}
  ---\textgreater{} \texttt{unknown}
\item
  \textbf{\texttt{mean\_of}} (\texttt{container} ,
  \texttt{any\ expression}) ---\textgreater{} \texttt{unknown}
\end{itemize}

\subsubsection{Result:}\label{result-334}

the mean of the right-hand expression evaluated on each of the elements
of the left-hand operand

\subsubsection{Comment:}\label{comment-69}

in the right-hand operand, the keyword each can be used to represent, in
turn, each of the right-hand operand elements.

\subsubsection{Special cases:}\label{special-cases-96}

\begin{itemize}
\tightlist
\item
  if the left-operand is a map, the keyword each will contain each value
\end{itemize}

\begin{verbatim}
 
unknown var1 <- [1::2, 3::4, 5::6] mean_of (each); // var1 equals 4
\end{verbatim}

\subsubsection{Examples:}\label{examples-238}

\begin{verbatim}
 
unknown var0 <- [1,2] mean_of (each * 10 ); // var0 equals 15
\end{verbatim}

\subsubsection{See also:}\label{see-also-139}

\href{OperatorsIM\#min_of}{min\_of},
\href{OperatorsIM\#max_of}{max\_of},
\href{OperatorsSZ\#sum_of}{sum\_of},
\href{OperatorsNR\#product_of}{product\_of},

\begin{center}\rule{0.5\linewidth}{\linethickness}\end{center}

\subsection{\texorpdfstring{\texttt{meanR}}{meanR}}\label{meanr}

\subsubsection{Possible use:}\label{possible-use-346}

\begin{itemize}
\tightlist
\item
  \textbf{\texttt{meanR}} (\texttt{container}) ---\textgreater{}
  \texttt{unknown}
\end{itemize}

\subsubsection{Result:}\label{result-335}

returns the mean value of given vector (right-hand operand) in given
variable (left-hand operand).

\subsubsection{Examples:}\label{examples-239}

\begin{verbatim}
list<int> X <- [2, 3, 1];  
int var1 <- meanR(X); // var1 equals 2
\end{verbatim}

\begin{center}\rule{0.5\linewidth}{\linethickness}\end{center}

\subsection{\texorpdfstring{\texttt{median}}{median}}\label{median}

\subsubsection{Possible use:}\label{possible-use-347}

\begin{itemize}
\tightlist
\item
  \textbf{\texttt{median}} (\texttt{container}) ---\textgreater{}
  \texttt{unknown}
\end{itemize}

\subsubsection{Result:}\label{result-336}

the median of all the elements of the operand.

\subsubsection{Special cases:}\label{special-cases-97}

\begin{itemize}
\tightlist
\item
  if the container contains points, the result will be a point. If the
  container contains rgb values, the result will be a rgb color
\end{itemize}

\subsubsection{Examples:}\label{examples-240}

\begin{verbatim}
 
unknown var0 <- median ([4.5, 3.5, 5.5, 3.4, 7.0]); // var0 equals 4.5
\end{verbatim}

\subsubsection{See also:}\label{see-also-140}

\href{OperatorsIM\#mean}{mean},

\begin{center}\rule{0.5\linewidth}{\linethickness}\end{center}

\subsection{\texorpdfstring{\texttt{mental\_state}}{mental\_state}}\label{mental_state}

\subsubsection{Possible use:}\label{possible-use-348}

\begin{itemize}
\tightlist
\item
  \textbf{\texttt{mental\_state}} (\texttt{any}) ---\textgreater{}
  \texttt{mental\_state}
\end{itemize}

\subsubsection{Result:}\label{result-337}

Casts the operand into the type mental\_state

\begin{center}\rule{0.5\linewidth}{\linethickness}\end{center}

\subsection{\texorpdfstring{\texttt{message}}{message}}\label{message}

\subsubsection{Possible use:}\label{possible-use-349}

\begin{itemize}
\tightlist
\item
  \textbf{\texttt{message}} (\texttt{unknown}) ---\textgreater{}
  \texttt{msi.gama.extensions.messaging.GamaMessage}
\end{itemize}

\subsubsection{Result:}\label{result-338}

to be added

\begin{center}\rule{0.5\linewidth}{\linethickness}\end{center}

\subsection{\texorpdfstring{\texttt{milliseconds\_between}}{milliseconds\_between}}\label{milliseconds_between}

\subsubsection{Possible use:}\label{possible-use-350}

\begin{itemize}
\tightlist
\item
  \texttt{date} \textbf{\texttt{milliseconds\_between}} \texttt{date}
  ---\textgreater{} \texttt{float}
\item
  \textbf{\texttt{milliseconds\_between}} (\texttt{date} ,
  \texttt{date}) ---\textgreater{} \texttt{float}
\end{itemize}

\subsubsection{Result:}\label{result-339}

Provide the exact number of milliseconds between two dates. This number
can be positive or negative (if the second operand is smaller than the
first one)

\subsubsection{Examples:}\label{examples-241}

\begin{verbatim}
 
float var0 <- milliseconds_between(date('2000-01-01'), date('2000-02-01')); // var0 equals 2.6784E9
\end{verbatim}

\begin{center}\rule{0.5\linewidth}{\linethickness}\end{center}

\subsection{\texorpdfstring{\texttt{min}}{min}}\label{min}

\subsubsection{Possible use:}\label{possible-use-351}

\begin{itemize}
\tightlist
\item
  \textbf{\texttt{min}} (\texttt{container}) ---\textgreater{}
  \texttt{unknown}
\end{itemize}

\subsubsection{Result:}\label{result-340}

the minimum element found in the operand.

\subsubsection{Comment:}\label{comment-70}

the min operator behavior depends on the nature of the operand

\subsubsection{Special cases:}\label{special-cases-98}

\begin{itemize}
\tightlist
\item
  if it is a list of points: min returns the minimum of all points as a
  point (i.e.~the point with the smallest coordinate on the x-axis, in
  case of equality the point with the smallest coordinate on the y-axis
  is chosen. If all the points are equal, the first one is returned. )\\
\item
  if it is a population of a list of other types: min transforms all
  elements into integer and returns the minimum of them\\
\item
  if it is a map, min returns the minimum among the list of all elements
  value\\
\item
  if it is a file, min returns the minimum of the content of the file
  (that is also a container)\\
\item
  if it is a graph, min returns the minimum of the list of the elements
  of the graph (that can be the list of edges or vertexes depending on
  the graph)\\
\item
  if it is a matrix of int, float or object, min returns the minimum of
  all the numerical elements (thus all elements for integer and float
  matrices)\\
\item
  if it is a matrix of geometry, min returns the minimum of the list of
  the geometries\\
\item
  if it is a matrix of another type, min returns the minimum of the
  elements transformed into float\\
\item
  if it is a list of int or float: min returns the minimum of all the
  elements
\end{itemize}

\begin{verbatim}
 
unknown var0 <- min ([100, 23.2, 34.5]); // var0 equals 23.2
\end{verbatim}

\subsubsection{See also:}\label{see-also-141}

\href{OperatorsIM\#max}{max},

\begin{center}\rule{0.5\linewidth}{\linethickness}\end{center}

\subsection{\texorpdfstring{\texttt{min\_of}}{min\_of}}\label{min_of}

\subsubsection{Possible use:}\label{possible-use-352}

\begin{itemize}
\tightlist
\item
  \texttt{container} \textbf{\texttt{min\_of}} \texttt{any\ expression}
  ---\textgreater{} \texttt{unknown}
\item
  \textbf{\texttt{min\_of}} (\texttt{container} ,
  \texttt{any\ expression}) ---\textgreater{} \texttt{unknown}
\end{itemize}

\subsubsection{Result:}\label{result-341}

the minimum value of the right-hand expression evaluated on each of the
elements of the left-hand operand

\subsubsection{Comment:}\label{comment-71}

in the right-hand operand, the keyword each can be used to represent, in
turn, each of the right-hand operand elements.

\subsubsection{Special cases:}\label{special-cases-99}

\begin{itemize}
\tightlist
\item
  if the left-hand operand is nil or empty, min\_of throws an error\\
\item
  if the left-operand is a map, the keyword each will contain each value
\end{itemize}

\begin{verbatim}
 
unknown var4 <- [1::2, 3::4, 5::6] min_of (each + 3); // var4 equals 5
\end{verbatim}

\subsubsection{Examples:}\label{examples-242}

\begin{verbatim}
 
unknown var0 <- [1,2,4,3,5,7,6,8] min_of (each * 100 ); // var0 equals 100graph g2 <- as_edge_graph([{1,5}::{12,45},{12,45}::{34,56}]);  
unknown var2 <- g2 min_of (length(g2 out_edges_of each) ); // var2 equals 0 
unknown var3 <- (list(node) min_of (round(node(each).location.x)); // var3 equals 4
\end{verbatim}

\subsubsection{See also:}\label{see-also-142}

\href{OperatorsIM\#max_of}{max\_of},

\begin{center}\rule{0.5\linewidth}{\linethickness}\end{center}

\subsection{\texorpdfstring{\texttt{minus\_days}}{minus\_days}}\label{minus_days}

\subsubsection{Possible use:}\label{possible-use-353}

\begin{itemize}
\tightlist
\item
  \texttt{date} \textbf{\texttt{minus\_days}} \texttt{int}
  ---\textgreater{} \texttt{date}
\item
  \textbf{\texttt{minus\_days}} (\texttt{date} , \texttt{int})
  ---\textgreater{} \texttt{date}
\end{itemize}

\subsubsection{Result:}\label{result-342}

Subtract a given number of days from a date

\subsubsection{Examples:}\label{examples-243}

\begin{verbatim}
 
date var0 <- date('2000-01-01') minus_days 20; // var0 equals date('1999-12-12')
\end{verbatim}

\begin{center}\rule{0.5\linewidth}{\linethickness}\end{center}

\subsection{\texorpdfstring{\texttt{minus\_hours}}{minus\_hours}}\label{minus_hours}

\subsubsection{Possible use:}\label{possible-use-354}

\begin{itemize}
\tightlist
\item
  \texttt{date} \textbf{\texttt{minus\_hours}} \texttt{int}
  ---\textgreater{} \texttt{date}
\item
  \textbf{\texttt{minus\_hours}} (\texttt{date} , \texttt{int})
  ---\textgreater{} \texttt{date}
\end{itemize}

\subsubsection{Result:}\label{result-343}

Remove a given number of hours from a date

\subsubsection{Examples:}\label{examples-244}

\begin{verbatim}
// equivalent to date1 - 15 #h  
date var1 <- date('2000-01-01') minus_hours 15 ; // var1 equals date('1999-12-31 09:00:00')
\end{verbatim}

\begin{center}\rule{0.5\linewidth}{\linethickness}\end{center}

\subsection{\texorpdfstring{\texttt{minus\_minutes}}{minus\_minutes}}\label{minus_minutes}

\subsubsection{Possible use:}\label{possible-use-355}

\begin{itemize}
\tightlist
\item
  \texttt{date} \textbf{\texttt{minus\_minutes}} \texttt{int}
  ---\textgreater{} \texttt{date}
\item
  \textbf{\texttt{minus\_minutes}} (\texttt{date} , \texttt{int})
  ---\textgreater{} \texttt{date}
\end{itemize}

\subsubsection{Result:}\label{result-344}

Subtract a given number of minutes from a date

\subsubsection{Examples:}\label{examples-245}

\begin{verbatim}
// date('2000-01-01') to date1 - 5#mn  
date var1 <- date('2000-01-01') minus_minutes 5 ; // var1 equals date('1999-12-31 23:55:00')
\end{verbatim}

\begin{center}\rule{0.5\linewidth}{\linethickness}\end{center}

\subsection{\texorpdfstring{\texttt{minus\_months}}{minus\_months}}\label{minus_months}

\subsubsection{Possible use:}\label{possible-use-356}

\begin{itemize}
\tightlist
\item
  \texttt{date} \textbf{\texttt{minus\_months}} \texttt{int}
  ---\textgreater{} \texttt{date}
\item
  \textbf{\texttt{minus\_months}} (\texttt{date} , \texttt{int})
  ---\textgreater{} \texttt{date}
\end{itemize}

\subsubsection{Result:}\label{result-345}

Subtract a given number of months from a date

\subsubsection{Examples:}\label{examples-246}

\begin{verbatim}
 
date var0 <- date('2000-01-01') minus_months 5; // var0 equals date('1999-08-01')
\end{verbatim}

\begin{center}\rule{0.5\linewidth}{\linethickness}\end{center}

\subsection{\texorpdfstring{\texttt{minus\_ms}}{minus\_ms}}\label{minus_ms}

\subsubsection{Possible use:}\label{possible-use-357}

\begin{itemize}
\tightlist
\item
  \texttt{date} \textbf{\texttt{minus\_ms}} \texttt{int}
  ---\textgreater{} \texttt{date}
\item
  \textbf{\texttt{minus\_ms}} (\texttt{date} , \texttt{int})
  ---\textgreater{} \texttt{date}
\end{itemize}

\subsubsection{Result:}\label{result-346}

Remove a given number of milliseconds from a date

\subsubsection{Examples:}\label{examples-247}

\begin{verbatim}
// equivalent to date1 - 15 #ms  
date var1 <- date('2000-01-01') minus_ms 1000 ; // var1 equals date('1999-12-31 23:59:59')
\end{verbatim}

\begin{center}\rule{0.5\linewidth}{\linethickness}\end{center}

\subsection{\texorpdfstring{\texttt{minus\_seconds}}{minus\_seconds}}\label{minus_seconds}

Same signification as \href{OperatorsAA\#-}{-}

\begin{center}\rule{0.5\linewidth}{\linethickness}\end{center}

\subsection{\texorpdfstring{\texttt{minus\_weeks}}{minus\_weeks}}\label{minus_weeks}

\subsubsection{Possible use:}\label{possible-use-358}

\begin{itemize}
\tightlist
\item
  \texttt{date} \textbf{\texttt{minus\_weeks}} \texttt{int}
  ---\textgreater{} \texttt{date}
\item
  \textbf{\texttt{minus\_weeks}} (\texttt{date} , \texttt{int})
  ---\textgreater{} \texttt{date}
\end{itemize}

\subsubsection{Result:}\label{result-347}

Subtract a given number of weeks from a date

\subsubsection{Examples:}\label{examples-248}

\begin{verbatim}
 
date var0 <- date('2000-01-01') minus_weeks 15; // var0 equals date('1999-09-18')
\end{verbatim}

\begin{center}\rule{0.5\linewidth}{\linethickness}\end{center}

\subsection{\texorpdfstring{\texttt{minus\_years}}{minus\_years}}\label{minus_years}

\subsubsection{Possible use:}\label{possible-use-359}

\begin{itemize}
\tightlist
\item
  \texttt{date} \textbf{\texttt{minus\_years}} \texttt{int}
  ---\textgreater{} \texttt{date}
\item
  \textbf{\texttt{minus\_years}} (\texttt{date} , \texttt{int})
  ---\textgreater{} \texttt{date}
\end{itemize}

\subsubsection{Result:}\label{result-348}

Subtract a given number of year from a date

\subsubsection{Examples:}\label{examples-249}

\begin{verbatim}
 
date var0 <- date('2000-01-01') minus_years 3; // var0 equals date('1997-01-01')
\end{verbatim}

\begin{center}\rule{0.5\linewidth}{\linethickness}\end{center}

\subsection{\texorpdfstring{\texttt{mod}}{mod}}\label{mod}

\subsubsection{Possible use:}\label{possible-use-360}

\begin{itemize}
\tightlist
\item
  \texttt{int} \textbf{\texttt{mod}} \texttt{int} ---\textgreater{}
  \texttt{int}
\item
  \textbf{\texttt{mod}} (\texttt{int} , \texttt{int}) ---\textgreater{}
  \texttt{int}
\end{itemize}

\subsubsection{Result:}\label{result-349}

Returns the remainder of the integer division of the left-hand operand
by the right-hand operand.

\subsubsection{Special cases:}\label{special-cases-100}

\begin{itemize}
\tightlist
\item
  if operands are float, they are truncated\\
\item
  if the right-hand operand is equal to zero, raises an exception.
\end{itemize}

\subsubsection{Examples:}\label{examples-250}

\begin{verbatim}
 
int var0 <- 40 mod 3; // var0 equals 1
\end{verbatim}

\subsubsection{See also:}\label{see-also-143}

\href{OperatorsDH\#div}{div},

\begin{center}\rule{0.5\linewidth}{\linethickness}\end{center}

\subsection{\texorpdfstring{\texttt{moment}}{moment}}\label{moment}

\subsubsection{Possible use:}\label{possible-use-361}

\begin{itemize}
\tightlist
\item
  \textbf{\texttt{moment}} (\texttt{container}, \texttt{int},
  \texttt{float}) ---\textgreater{} \texttt{float}
\end{itemize}

\subsubsection{Result:}\label{result-350}

Returns the moment of k-th order with constant c of a data sequence

\begin{center}\rule{0.5\linewidth}{\linethickness}\end{center}

\subsection{\texorpdfstring{\texttt{months\_between}}{months\_between}}\label{months_between}

\subsubsection{Possible use:}\label{possible-use-362}

\begin{itemize}
\tightlist
\item
  \texttt{date} \textbf{\texttt{months\_between}} \texttt{date}
  ---\textgreater{} \texttt{int}
\item
  \textbf{\texttt{months\_between}} (\texttt{date} , \texttt{date})
  ---\textgreater{} \texttt{int}
\end{itemize}

\subsubsection{Result:}\label{result-351}

Provide the exact number of months between two dates. This number can be
positive or negative (if the second operand is smaller than the first
one)

\subsubsection{Examples:}\label{examples-251}

\begin{verbatim}
 
int var0 <- months_between(date('2000-01-01'), date('2000-02-01')); // var0 equals 1
\end{verbatim}

\begin{center}\rule{0.5\linewidth}{\linethickness}\end{center}

\subsection{\texorpdfstring{\texttt{moran}}{moran}}\label{moran}

\subsubsection{Possible use:}\label{possible-use-363}

\begin{itemize}
\tightlist
\item
  \texttt{list\textless{}float\textgreater{}} \textbf{\texttt{moran}}
  \texttt{matrix\textless{}float\textgreater{}} ---\textgreater{}
  \texttt{float}
\item
  \textbf{\texttt{moran}} (\texttt{list\textless{}float\textgreater{}} ,
  \texttt{matrix\textless{}float\textgreater{}}) ---\textgreater{}
  \texttt{float}
\end{itemize}

\subsubsection{Special cases:}\label{special-cases-101}

\begin{itemize}
\tightlist
\item
  return the Moran Index of the given list of interest points (list of
  floats) and the weight matrix (matrix of float)
\end{itemize}

\begin{verbatim}
 
float var0 <- moran([1.0, 0.5, 2.0], weight_matrix); // var0 equals the Moran index computed
\end{verbatim}

\begin{center}\rule{0.5\linewidth}{\linethickness}\end{center}

\subsection{\texorpdfstring{\texttt{mul}}{mul}}\label{mul}

\subsubsection{Possible use:}\label{possible-use-364}

\begin{itemize}
\tightlist
\item
  \textbf{\texttt{mul}} (\texttt{container}) ---\textgreater{}
  \texttt{unknown}
\end{itemize}

\subsubsection{Result:}\label{result-352}

the product of all the elements of the operand

\subsubsection{Comment:}\label{comment-72}

the mul operator behavior depends on the nature of the operand

\subsubsection{Special cases:}\label{special-cases-102}

\begin{itemize}
\tightlist
\item
  if it is a list of points: mul returns the product of all points as a
  point (each coordinate is the product of the corresponding coordinate
  of each element)\\
\item
  if it is a list of other types: mul transforms all elements into
  integer and multiplies them\\
\item
  if it is a map, mul returns the product of the value of all elements\\
\item
  if it is a file, mul returns the product of the content of the file
  (that is also a container)\\
\item
  if it is a graph, mul returns the product of the list of the elements
  of the graph (that can be the list of edges or vertexes depending on
  the graph)\\
\item
  if it is a matrix of int, float or object, mul returns the product of
  all the numerical elements (thus all elements for integer and float
  matrices)\\
\item
  if it is a matrix of geometry, mul returns the product of the list of
  the geometries\\
\item
  if it is a matrix of other types: mul transforms all elements into
  float and multiplies them\\
\item
  if it is a list of int or float: mul returns the product of all the
  elements
\end{itemize}

\begin{verbatim}
 
unknown var0 <- mul ([100, 23.2, 34.5]); // var0 equals 80040.0
\end{verbatim}

\subsubsection{See also:}\label{see-also-144}

\href{OperatorsSZ\#sum}{sum},

\chapter{Operators (N to R)}\label{operators-n-to-r}

\begin{center}\rule{0.5\linewidth}{\linethickness}\end{center}

\textbf{This file is automatically generated from java files. Do Not
Edit It.}

\begin{center}\rule{0.5\linewidth}{\linethickness}\end{center}

\section{Definition}\label{definition-4}

Operators in the GAML language are used to compose complex expressions.
An operator performs a function on one, two, or n operands (which are
other expressions and thus may be themselves composed of operators) and
returns the result of this function.

Most of them use a classical prefixed functional syntax (i.e.
\texttt{operator\_name(operand1,\ operand2,\ operand3)}, see below),
with the exception of arithmetic (e.g. \texttt{+}, \texttt{/}), logical
(\texttt{and}, \texttt{or}), comparison (e.g. \texttt{\textgreater{}},
\texttt{\textless{}}), access (\texttt{.}, \texttt{{[}..{]}}) and pair
(\texttt{::}) operators, which require an infixed notation (i.e.
\texttt{operand1\ operator\_symbol\ operand1}).

The ternary functional if-else operator, \texttt{?\ :}, uses a special
infixed syntax composed with two symbols (e.g.
\texttt{operand1\ ?\ operand2\ :\ operand3}). Two unary operators
(\texttt{-} and \texttt{!}) use a traditional prefixed syntax that does
not require parentheses unless the operand is itself a complex
expression (e.g. \texttt{-\ 10}, \texttt{!\ (operand1\ or\ operand2)}).

Finally, special constructor operators (\texttt{\{...\}} for
constructing points, \texttt{{[}...{]}} for constructing lists and maps)
will require their operands to be placed between their two symbols (e.g.
\texttt{\{1,2,3\}}, \texttt{{[}operand1,\ operand2,\ ...,\ operandn{]}}
or \texttt{{[}key1::value1,\ key2::value2...\ keyn::valuen{]}}).

With the exception of these special cases above, the following rules
apply to the syntax of operators: * if they only have one operand, the
functional prefixed syntax is mandatory (e.g.
\texttt{operator\_name(operand1)}) * if they have two arguments, either
the functional prefixed syntax (e.g.
\texttt{operator\_name(operand1,\ operand2)}) or the infixed syntax
(e.g. \texttt{operand1\ operator\_name\ operand2}) can be used. * if
they have more than two arguments, either the functional prefixed syntax
(e.g. \texttt{operator\_name(operand1,\ operand2,\ ...,\ operand)}) or a
special infixed syntax with the first operand on the left-hand side of
the operator name (e.g.
\texttt{operand1\ operator\_name(operand2,\ ...,\ operand)}) can be
used.

All of these alternative syntaxes are completely equivalent.

Operators in GAML are purely functional, i.e.~they are guaranteed to not
have any side effects on their operands. For instance, the
\texttt{shuffle} operator, which randomizes the positions of elements in
a list, does not modify its list operand but returns a new shuffled
list.

\section{\texorpdfstring{}{ }}\label{section-22}

\section{Priority between operators}\label{priority-between-operators-4}

The priority of operators determines, in the case of complex expressions
composed of several operators, which one(s) will be evaluated first.

GAML follows in general the traditional priorities attributed to
arithmetic, boolean, comparison operators, with some twists. Namely: *
the constructor operators, like \texttt{::}, used to compose pairs of
operands, have the lowest priority of all operators (e.g.
\texttt{a\ \textgreater{}\ b\ ::\ b\ \textgreater{}\ c} will return a
pair of boolean values, which means that the two comparisons are
evaluated before the operator applies. Similarly,
\texttt{{[}a\ \textgreater{}\ 10,\ b\ \textgreater{}\ 5{]}} will return
a list of boolean values. * it is followed by the \texttt{?:} operator,
the functional if-else (e.g.
\texttt{a\ \textgreater{}\ b\ ?\ a\ +\ 10\ :\ a\ -\ 10} will return the
result of the if-else). * next are the logical operators, \texttt{and}
and \texttt{or} (e.g.
\texttt{a\ \textgreater{}\ b\ or\ b\ \textgreater{}\ c} will return the
value of the test) * next are the comparison operators (i.e.
\texttt{\textgreater{}}, \texttt{\textless{}}, \texttt{\textless{}=},
\texttt{\textgreater{}=}, \texttt{=}, \texttt{!=}) * next the arithmetic
operators in their logical order (multiplicative operators have a higher
priority than additive operators) * next the unary operators \texttt{-}
and \texttt{!} * next the access operators \texttt{.} and
\texttt{{[}{]}} (e.g.
\texttt{\{1,2,3\}.x\ \textgreater{}\ 20\ +\ \{4,5,6\}.y} will return the
result of the comparison between the x and y ordinates of the two
points) * and finally the functional operators, which have the highest
priority of all.

\begin{center}\rule{0.5\linewidth}{\linethickness}\end{center}

\section{Using actions as operators}\label{using-actions-as-operators-4}

Actions defined in species can be used as operators, provided they are
called on the correct agent. The syntax is that of normal functional
operators, but the agent that will perform the action must be added as
the first operand.

For instance, if the following species is defined:

\begin{verbatim}
species spec1 {
        int min(int x, int y) {
                return x > y ? x : y;
        }
}
\end{verbatim}

Any agent instance of spec1 can use \texttt{min} as an operator (if the
action conflicts with an existing operator, a warning will be emitted).
For instance, in the same model, the following line is perfectly
acceptable:

\begin{verbatim}
global {
        init {
                create spec1;
                spec1 my_agent <- spec1[0];
                int the_min <- my_agent min(10,20); // or min(my_agent, 10, 20);
        }
}
\end{verbatim}

If the action doesn't have any operands, the syntax to use is
\texttt{my\_agent\ the\_action()}. Finally, if it does not return a
value, it might still be used but is considering as returning a value of
type \texttt{unknown} (e.g.
\texttt{unknown\ result\ \textless{}-\ my\_agent\ the\_action(op1,\ op2);}).

Note that due to the fact that actions are written by modelers, the
general functional contract is not respected in that case: actions might
perfectly have side effects on their operands (including the agent).

\begin{center}\rule{0.5\linewidth}{\linethickness}\end{center}

\section{Table of Contents}\label{table-of-contents-4}

\begin{center}\rule{0.5\linewidth}{\linethickness}\end{center}

\section{Operators by categories}\label{operators-by-categories-4}

\begin{center}\rule{0.5\linewidth}{\linethickness}\end{center}

\subsection{3D}\label{d-4}

\href{OperatorsBC\#box}{box}, \href{OperatorsBC\#cone3d}{cone3D},
\href{OperatorsBC\#cube}{cube}, \href{OperatorsBC\#cylinder}{cylinder},
\href{OperatorsDH\#dem}{dem}, \href{OperatorsDH\#hexagon}{hexagon},
\href{OperatorsNR\#pyramid}{pyramid},
\href{OperatorsNR\#rgb_to_xyz}{rgb\_to\_xyz},
\href{OperatorsSZ\#set_z}{set\_z}, \href{OperatorsSZ\#sphere}{sphere},
\href{OperatorsSZ\#teapot}{teapot},

\begin{center}\rule{0.5\linewidth}{\linethickness}\end{center}

\subsection{Arithmetic operators}\label{arithmetic-operators-4}

\href{OperatorsAA\#-}{-}, \href{OperatorsAA\#/}{/},
{[}\textsuperscript{{]}(OperatorsAA\#}), \href{OperatorsAA\#*}{*},
\href{OperatorsAA\#+}{+}, \href{OperatorsAA\#abs}{abs},
\href{OperatorsAA\#acos}{acos}, \href{OperatorsAA\#asin}{asin},
\href{OperatorsAA\#atan}{atan}, \href{OperatorsAA\#atan2}{atan2},
\href{OperatorsBC\#ceil}{ceil}, \href{OperatorsBC\#cos}{cos},
\href{OperatorsBC\#cos_rad}{cos\_rad}, \href{OperatorsDH\#div}{div},
\href{OperatorsDH\#even}{even}, \href{OperatorsDH\#exp}{exp},
\href{OperatorsDH\#fact}{fact}, \href{OperatorsDH\#floor}{floor},
\href{OperatorsDH\#hypot}{hypot},
\href{OperatorsIM\#is_finite}{is\_finite},
\href{OperatorsIM\#is_number}{is\_number}, \href{OperatorsIM\#ln}{ln},
\href{OperatorsIM\#log}{log}, \href{OperatorsIM\#mod}{mod},
\href{OperatorsNR\#round}{round}, \href{OperatorsSZ\#signum}{signum},
\href{OperatorsSZ\#sin}{sin}, \href{OperatorsSZ\#sin_rad}{sin\_rad},
\href{OperatorsSZ\#sqrt}{sqrt}, \href{OperatorsSZ\#tan}{tan},
\href{OperatorsSZ\#tan_rad}{tan\_rad}, \href{OperatorsSZ\#tanh}{tanh},
\href{OperatorsSZ\#with_precision}{with\_precision},

\begin{center}\rule{0.5\linewidth}{\linethickness}\end{center}

\subsection{BDI}\label{bdi-4}

\href{OperatorsAA\#and}{and}, \href{OperatorsDH\#eval_when}{eval\_when},
\href{OperatorsDH\#get_about}{get\_about},
\href{OperatorsDH\#get_agent}{get\_agent},
\href{OperatorsDH\#get_agent_cause}{get\_agent\_cause},
\href{OperatorsDH\#get_belief_op}{get\_belief\_op},
\href{OperatorsDH\#get_belief_with_name_op}{get\_belief\_with\_name\_op},
\href{OperatorsDH\#get_beliefs_op}{get\_beliefs\_op},
\href{OperatorsDH\#get_beliefs_with_name_op}{get\_beliefs\_with\_name\_op},
\href{OperatorsDH\#get_current_intention_op}{get\_current\_intention\_op},
\href{OperatorsDH\#get_decay}{get\_decay},
\href{OperatorsDH\#get_desire_op}{get\_desire\_op},
\href{OperatorsDH\#get_desire_with_name_op}{get\_desire\_with\_name\_op},
\href{OperatorsDH\#get_desires_op}{get\_desires\_op},
\href{OperatorsDH\#get_desires_with_name_op}{get\_desires\_with\_name\_op},
\href{OperatorsDH\#get_dominance}{get\_dominance},
\href{OperatorsDH\#get_familiarity}{get\_familiarity},
\href{OperatorsDH\#get_ideal_op}{get\_ideal\_op},
\href{OperatorsDH\#get_ideal_with_name_op}{get\_ideal\_with\_name\_op},
\href{OperatorsDH\#get_ideals_op}{get\_ideals\_op},
\href{OperatorsDH\#get_ideals_with_name_op}{get\_ideals\_with\_name\_op},
\href{OperatorsDH\#get_intensity}{get\_intensity},
\href{OperatorsDH\#get_intention_op}{get\_intention\_op},
\href{OperatorsDH\#get_intention_with_name_op}{get\_intention\_with\_name\_op},
\href{OperatorsDH\#get_intentions_op}{get\_intentions\_op},
\href{OperatorsDH\#get_intentions_with_name_op}{get\_intentions\_with\_name\_op},
\href{OperatorsDH\#get_lifetime}{get\_lifetime},
\href{OperatorsDH\#get_liking}{get\_liking},
\href{OperatorsDH\#get_modality}{get\_modality},
\href{OperatorsDH\#get_obligation_op}{get\_obligation\_op},
\href{OperatorsDH\#get_obligation_with_name_op}{get\_obligation\_with\_name\_op},
\href{OperatorsDH\#get_obligations_op}{get\_obligations\_op},
\href{OperatorsDH\#get_obligations_with_name_op}{get\_obligations\_with\_name\_op},
\href{OperatorsDH\#get_plan_name}{get\_plan\_name},
\href{OperatorsDH\#get_predicate}{get\_predicate},
\href{OperatorsDH\#get_solidarity}{get\_solidarity},
\href{OperatorsDH\#get_strength}{get\_strength},
\href{OperatorsDH\#get_super_intention}{get\_super\_intention},
\href{OperatorsDH\#get_trust}{get\_trust},
\href{OperatorsDH\#get_truth}{get\_truth},
\href{OperatorsDH\#get_uncertainties_op}{get\_uncertainties\_op},
\href{OperatorsDH\#get_uncertainties_with_name_op}{get\_uncertainties\_with\_name\_op},
\href{OperatorsDH\#get_uncertainty_op}{get\_uncertainty\_op},
\href{OperatorsDH\#get_uncertainty_with_name_op}{get\_uncertainty\_with\_name\_op},
\href{OperatorsDH\#has_belief_op}{has\_belief\_op},
\href{OperatorsDH\#has_belief_with_name_op}{has\_belief\_with\_name\_op},
\href{OperatorsDH\#has_desire_op}{has\_desire\_op},
\href{OperatorsDH\#has_desire_with_name_op}{has\_desire\_with\_name\_op},
\href{OperatorsDH\#has_ideal_op}{has\_ideal\_op},
\href{OperatorsDH\#has_ideal_with_name_op}{has\_ideal\_with\_name\_op},
\href{OperatorsDH\#has_intention_op}{has\_intention\_op},
\href{OperatorsDH\#has_intention_with_name_op}{has\_intention\_with\_name\_op},
\href{OperatorsDH\#has_obligation_op}{has\_obligation\_op},
\href{OperatorsDH\#has_obligation_with_name_op}{has\_obligation\_with\_name\_op},
\href{OperatorsDH\#has_uncertainty_op}{has\_uncertainty\_op},
\href{OperatorsDH\#has_uncertainty_with_name_op}{has\_uncertainty\_with\_name\_op},
\href{OperatorsNR\#new_emotion}{new\_emotion},
\href{OperatorsNR\#new_mental_state}{new\_mental\_state},
\href{OperatorsNR\#new_predicate}{new\_predicate},
\href{OperatorsNR\#new_social_link}{new\_social\_link},
\href{OperatorsNR\#or}{or}, \href{OperatorsSZ\#set_about}{set\_about},
\href{OperatorsSZ\#set_agent}{set\_agent},
\href{OperatorsSZ\#set_agent_cause}{set\_agent\_cause},
\href{OperatorsSZ\#set_decay}{set\_decay},
\href{OperatorsSZ\#set_dominance}{set\_dominance},
\href{OperatorsSZ\#set_familiarity}{set\_familiarity},
\href{OperatorsSZ\#set_intensity}{set\_intensity},
\href{OperatorsSZ\#set_lifetime}{set\_lifetime},
\href{OperatorsSZ\#set_liking}{set\_liking},
\href{OperatorsSZ\#set_modality}{set\_modality},
\href{OperatorsSZ\#set_predicate}{set\_predicate},
\href{OperatorsSZ\#set_solidarity}{set\_solidarity},
\href{OperatorsSZ\#set_strength}{set\_strength},
\href{OperatorsSZ\#set_trust}{set\_trust},
\href{OperatorsSZ\#set_truth}{set\_truth},
\href{OperatorsSZ\#with_lifetime}{with\_lifetime},
\href{OperatorsSZ\#with_values}{with\_values},

\begin{center}\rule{0.5\linewidth}{\linethickness}\end{center}

\subsection{Casting operators}\label{casting-operators-4}

\href{OperatorsAA\#as}{as}, \href{OperatorsAA\#as_int}{as\_int},
\href{OperatorsAA\#as_matrix}{as\_matrix},
\href{OperatorsDH\#font}{font}, \href{OperatorsIM\#is}{is},
\href{OperatorsIM\#is_skill}{is\_skill},
\href{OperatorsIM\#list_with}{list\_with},
\href{OperatorsIM\#matrix_with}{matrix\_with},
\href{OperatorsSZ\#species}{species},
\href{OperatorsSZ\#to_gaml}{to\_gaml},
\href{OperatorsSZ\#topology}{topology},

\begin{center}\rule{0.5\linewidth}{\linethickness}\end{center}

\subsection{Color-related operators}\label{color-related-operators-4}

\href{OperatorsAA\#-}{-}, \href{OperatorsAA\#/}{/},
\href{OperatorsAA\#*}{*}, \href{OperatorsAA\#+}{+},
\href{OperatorsBC\#blend}{blend},
\href{OperatorsBC\#brewer_colors}{brewer\_colors},
\href{OperatorsBC\#brewer_palettes}{brewer\_palettes},
\href{OperatorsDH\#grayscale}{grayscale}, \href{OperatorsDH\#hsb}{hsb},
\href{OperatorsIM\#mean}{mean}, \href{OperatorsIM\#median}{median},
\href{OperatorsNR\#rgb}{rgb}, \href{OperatorsNR\#rnd_color}{rnd\_color},
\href{OperatorsSZ\#sum}{sum},

\begin{center}\rule{0.5\linewidth}{\linethickness}\end{center}

\subsection{Comparison operators}\label{comparison-operators-4}

\href{OperatorsAA\#!=}{!=}, \href{OperatorsAA\#\%3C}{\textless{}},
\href{OperatorsAA\#\%3C=}{\textless{}=}, \href{OperatorsAA\#=}{=},
\href{OperatorsAA\#\%3E}{\textgreater{}},
\href{OperatorsAA\#\%3E=}{\textgreater{}=},
\href{OperatorsBC\#between}{between},

\begin{center}\rule{0.5\linewidth}{\linethickness}\end{center}

\subsection{Containers-related
operators}\label{containers-related-operators-4}

\href{OperatorsAA\#-}{-}, \href{OperatorsAA\#::}{::},
\href{OperatorsAA\#+}{+}, \href{OperatorsAA\#accumulate}{accumulate},
\href{OperatorsAA\#among}{among}, \href{OperatorsAA\#at}{at},
\href{OperatorsBC\#collect}{collect},
\href{OperatorsBC\#contains}{contains},
\href{OperatorsBC\#contains_all}{contains\_all},
\href{OperatorsBC\#contains_any}{contains\_any},
\href{OperatorsBC\#count}{count},
\href{OperatorsDH\#distinct}{distinct},
\href{OperatorsDH\#empty}{empty}, \href{OperatorsDH\#every}{every},
\href{OperatorsDH\#first}{first},
\href{OperatorsDH\#first_with}{first\_with},
\href{OperatorsDH\#get}{get}, \href{OperatorsDH\#group_by}{group\_by},
\href{OperatorsIM\#in}{in}, \href{OperatorsIM\#index_by}{index\_by},
\href{OperatorsIM\#inter}{inter},
\href{OperatorsIM\#interleave}{interleave},
\href{OperatorsIM\#internal_at}{internal\_at},
\href{OperatorsIM\#internal_integrated_value}{internal\_integrated\_value},
\href{OperatorsIM\#last}{last},
\href{OperatorsIM\#last_with}{last\_with},
\href{OperatorsIM\#length}{length}, \href{OperatorsIM\#max}{max},
\href{OperatorsIM\#max_of}{max\_of}, \href{OperatorsIM\#mean}{mean},
\href{OperatorsIM\#mean_of}{mean\_of},
\href{OperatorsIM\#median}{median}, \href{OperatorsIM\#min}{min},
\href{OperatorsIM\#min_of}{min\_of}, \href{OperatorsIM\#mul}{mul},
\href{OperatorsNR\#one_of}{one\_of},
\href{OperatorsNR\#product_of}{product\_of},
\href{OperatorsNR\#range}{range}, \href{OperatorsNR\#reverse}{reverse},
\href{OperatorsSZ\#shuffle}{shuffle},
\href{OperatorsSZ\#sort_by}{sort\_by}, \href{OperatorsSZ\#split}{split},
\href{OperatorsSZ\#split_in}{split\_in},
\href{OperatorsSZ\#split_using}{split\_using},
\href{OperatorsSZ\#sum}{sum}, \href{OperatorsSZ\#sum_of}{sum\_of},
\href{OperatorsSZ\#union}{union},
\href{OperatorsSZ\#variance_of}{variance\_of},
\href{OperatorsSZ\#where}{where},
\href{OperatorsSZ\#with_max_of}{with\_max\_of},
\href{OperatorsSZ\#with_min_of}{with\_min\_of},

\begin{center}\rule{0.5\linewidth}{\linethickness}\end{center}

\subsection{Date-related operators}\label{date-related-operators-4}

\href{OperatorsAA\#-}{-}, \href{OperatorsAA\#!=}{!=},
\href{OperatorsAA\#+}{+}, \href{OperatorsAA\#\%3C}{\textless{}},
\href{OperatorsAA\#\%3C=}{\textless{}=}, \href{OperatorsAA\#=}{=},
\href{OperatorsAA\#\%3E}{\textgreater{}},
\href{OperatorsAA\#\%3E=}{\textgreater{}=},
\href{OperatorsAA\#after}{after}, \href{OperatorsBC\#before}{before},
\href{OperatorsBC\#between}{between}, \href{OperatorsDH\#every}{every},
\href{OperatorsIM\#milliseconds_between}{milliseconds\_between},
\href{OperatorsIM\#minus_days}{minus\_days},
\href{OperatorsIM\#minus_hours}{minus\_hours},
\href{OperatorsIM\#minus_minutes}{minus\_minutes},
\href{OperatorsIM\#minus_months}{minus\_months},
\href{OperatorsIM\#minus_ms}{minus\_ms},
\href{OperatorsIM\#minus_weeks}{minus\_weeks},
\href{OperatorsIM\#minus_years}{minus\_years},
\href{OperatorsIM\#months_between}{months\_between},
\href{OperatorsNR\#plus_days}{plus\_days},
\href{OperatorsNR\#plus_hours}{plus\_hours},
\href{OperatorsNR\#plus_minutes}{plus\_minutes},
\href{OperatorsNR\#plus_months}{plus\_months},
\href{OperatorsNR\#plus_ms}{plus\_ms},
\href{OperatorsNR\#plus_weeks}{plus\_weeks},
\href{OperatorsNR\#plus_years}{plus\_years},
\href{OperatorsSZ\#since}{since}, \href{OperatorsSZ\#to}{to},
\href{OperatorsSZ\#until}{until},
\href{OperatorsSZ\#years_between}{years\_between},

\begin{center}\rule{0.5\linewidth}{\linethickness}\end{center}

\subsection{Dates}\label{dates-4}

\begin{center}\rule{0.5\linewidth}{\linethickness}\end{center}

\subsection{DescriptiveStatistics}\label{descriptivestatistics-4}

\href{OperatorsAA\#auto_correlation}{auto\_correlation},
\href{OperatorsBC\#correlation}{correlation},
\href{OperatorsBC\#covariance}{covariance},
\href{OperatorsDH\#durbin_watson}{durbin\_watson},
\href{OperatorsIM\#kurtosis}{kurtosis},
\href{OperatorsIM\#moment}{moment},
\href{OperatorsNR\#quantile}{quantile},
\href{OperatorsNR\#quantile_inverse}{quantile\_inverse},
\href{OperatorsNR\#rank_interpolated}{rank\_interpolated},
\href{OperatorsNR\#rms}{rms}, \href{OperatorsSZ\#skew}{skew},
\href{OperatorsSZ\#variance}{variance},

\begin{center}\rule{0.5\linewidth}{\linethickness}\end{center}

\subsection{Displays}\label{displays-4}

\href{OperatorsDH\#horizontal}{horizontal},
\href{OperatorsSZ\#stack}{stack},
\href{OperatorsSZ\#vertical}{vertical},

\begin{center}\rule{0.5\linewidth}{\linethickness}\end{center}

\subsection{Distributions}\label{distributions-4}

\href{OperatorsBC\#binomial_coeff}{binomial\_coeff},
\href{OperatorsBC\#binomial_complemented}{binomial\_complemented},
\href{OperatorsBC\#binomial_sum}{binomial\_sum},
\href{OperatorsBC\#chi_square}{chi\_square},
\href{OperatorsBC\#chi_square_complemented}{chi\_square\_complemented},
\href{OperatorsDH\#gamma_distribution}{gamma\_distribution},
\href{OperatorsDH\#gamma_distribution_complemented}{gamma\_distribution\_complemented},
\href{OperatorsNR\#normal_area}{normal\_area},
\href{OperatorsNR\#normal_density}{normal\_density},
\href{OperatorsNR\#normal_inverse}{normal\_inverse},
\href{OperatorsNR\#pvalue_for_fstat}{pValue\_for\_fStat},
\href{OperatorsNR\#pvalue_for_tstat}{pValue\_for\_tStat},
\href{OperatorsSZ\#student_area}{student\_area},
\href{OperatorsSZ\#student_t_inverse}{student\_t\_inverse},

\begin{center}\rule{0.5\linewidth}{\linethickness}\end{center}

\subsection{Driving operators}\label{driving-operators-4}

\href{OperatorsAA\#as_driving_graph}{as\_driving\_graph},

\begin{center}\rule{0.5\linewidth}{\linethickness}\end{center}

\subsection{edge}\label{edge-5}

\href{OperatorsDH\#edge_between}{edge\_between},
\href{OperatorsSZ\#strahler}{strahler},

\begin{center}\rule{0.5\linewidth}{\linethickness}\end{center}

\subsection{EDP-related operators}\label{edp-related-operators-4}

\href{OperatorsDH\#diff}{diff}, \href{OperatorsDH\#diff2}{diff2},
\href{OperatorsIM\#internal_zero_order_equation}{internal\_zero\_order\_equation},

\begin{center}\rule{0.5\linewidth}{\linethickness}\end{center}

\subsection{Files-related operators}\label{files-related-operators-4}

\href{OperatorsBC\#crs}{crs},
\href{OperatorsDH\#evaluate_sub_model}{evaluate\_sub\_model},
\href{OperatorsDH\#file}{file},
\href{OperatorsDH\#file_exists}{file\_exists},
\href{OperatorsDH\#folder}{folder}, \href{OperatorsDH\#get}{get},
\href{OperatorsIM\#load_sub_model}{load\_sub\_model},
\href{OperatorsNR\#new_folder}{new\_folder},
\href{OperatorsNR\#osm_file}{osm\_file}, \href{OperatorsNR\#read}{read},
\href{OperatorsSZ\#step_sub_model}{step\_sub\_model},
\href{OperatorsSZ\#writable}{writable},

\begin{center}\rule{0.5\linewidth}{\linethickness}\end{center}

\subsection{FIPA-related operators}\label{fipa-related-operators-4}

\href{OperatorsBC\#conversation}{conversation},
\href{OperatorsIM\#message}{message},

\begin{center}\rule{0.5\linewidth}{\linethickness}\end{center}

\subsection{GamaMetaType}\label{gamametatype-4}

\href{OperatorsSZ\#type_of}{type\_of},

\begin{center}\rule{0.5\linewidth}{\linethickness}\end{center}

\subsection{GammaFunction}\label{gammafunction-4}

\href{OperatorsBC\#beta}{beta}, \href{OperatorsDH\#gamma}{gamma},
\href{OperatorsIM\#incomplete_beta}{incomplete\_beta},
\href{OperatorsIM\#incomplete_gamma}{incomplete\_gamma},
\href{OperatorsIM\#incomplete_gamma_complement}{incomplete\_gamma\_complement},
\href{OperatorsIM\#log_gamma}{log\_gamma},

\begin{center}\rule{0.5\linewidth}{\linethickness}\end{center}

\subsection{Graphs-related operators}\label{graphs-related-operators-4}

\href{OperatorsAA\#add_edge}{add\_edge},
\href{OperatorsAA\#add_node}{add\_node},
\href{OperatorsAA\#adjacency}{adjacency},
\href{OperatorsAA\#agent_from_geometry}{agent\_from\_geometry},
\href{OperatorsAA\#all_pairs_shortest_path}{all\_pairs\_shortest\_path},
\href{OperatorsAA\#alpha_index}{alpha\_index},
\href{OperatorsAA\#as_distance_graph}{as\_distance\_graph},
\href{OperatorsAA\#as_edge_graph}{as\_edge\_graph},
\href{OperatorsAA\#as_intersection_graph}{as\_intersection\_graph},
\href{OperatorsAA\#as_path}{as\_path},
\href{OperatorsBC\#beta_index}{beta\_index},
\href{OperatorsBC\#betweenness_centrality}{betweenness\_centrality},
\href{OperatorsBC\#biggest_cliques_of}{biggest\_cliques\_of},
\href{OperatorsBC\#connected_components_of}{connected\_components\_of},
\href{OperatorsBC\#connectivity_index}{connectivity\_index},
\href{OperatorsBC\#contains_edge}{contains\_edge},
\href{OperatorsBC\#contains_vertex}{contains\_vertex},
\href{OperatorsDH\#degree_of}{degree\_of},
\href{OperatorsDH\#directed}{directed}, \href{OperatorsDH\#edge}{edge},
\href{OperatorsDH\#edge_between}{edge\_between},
\href{OperatorsDH\#edge_betweenness}{edge\_betweenness},
\href{OperatorsDH\#edges}{edges},
\href{OperatorsDH\#gamma_index}{gamma\_index},
\href{OperatorsDH\#generate_barabasi_albert}{generate\_barabasi\_albert},
\href{OperatorsDH\#generate_complete_graph}{generate\_complete\_graph},
\href{OperatorsDH\#generate_watts_strogatz}{generate\_watts\_strogatz},
\href{OperatorsDH\#grid_cells_to_graph}{grid\_cells\_to\_graph},
\href{OperatorsIM\#in_degree_of}{in\_degree\_of},
\href{OperatorsIM\#in_edges_of}{in\_edges\_of},
\href{OperatorsIM\#layout}{layout},
\href{OperatorsIM\#load_graph_from_file}{load\_graph\_from\_file},
\href{OperatorsIM\#load_shortest_paths}{load\_shortest\_paths},
\href{OperatorsIM\#main_connected_component}{main\_connected\_component},
\href{OperatorsIM\#max_flow_between}{max\_flow\_between},
\href{OperatorsIM\#maximal_cliques_of}{maximal\_cliques\_of},
\href{OperatorsNR\#nb_cycles}{nb\_cycles},
\href{OperatorsNR\#neighbors_of}{neighbors\_of},
\href{OperatorsNR\#node}{node}, \href{OperatorsNR\#nodes}{nodes},
\href{OperatorsNR\#out_degree_of}{out\_degree\_of},
\href{OperatorsNR\#out_edges_of}{out\_edges\_of},
\href{OperatorsNR\#path_between}{path\_between},
\href{OperatorsNR\#paths_between}{paths\_between},
\href{OperatorsNR\#predecessors_of}{predecessors\_of},
\href{OperatorsNR\#remove_node_from}{remove\_node\_from},
\href{OperatorsNR\#rewire_n}{rewire\_n},
\href{OperatorsSZ\#source_of}{source\_of},
\href{OperatorsSZ\#spatial_graph}{spatial\_graph},
\href{OperatorsSZ\#strahler}{strahler},
\href{OperatorsSZ\#successors_of}{successors\_of},
\href{OperatorsSZ\#sum}{sum}, \href{OperatorsSZ\#target_of}{target\_of},
\href{OperatorsSZ\#undirected}{undirected},
\href{OperatorsSZ\#use_cache}{use\_cache},
\href{OperatorsSZ\#weight_of}{weight\_of},
\href{OperatorsSZ\#with_optimizer_type}{with\_optimizer\_type},
\href{OperatorsSZ\#with_weights}{with\_weights},

\begin{center}\rule{0.5\linewidth}{\linethickness}\end{center}

\subsection{Grid-related operators}\label{grid-related-operators-4}

\href{OperatorsAA\#as_4_grid}{as\_4\_grid},
\href{OperatorsAA\#as_grid}{as\_grid},
\href{OperatorsAA\#as_hexagonal_grid}{as\_hexagonal\_grid},
\href{OperatorsDH\#grid_at}{grid\_at},
\href{OperatorsNR\#path_between}{path\_between},

\begin{center}\rule{0.5\linewidth}{\linethickness}\end{center}

\subsection{Iterator operators}\label{iterator-operators-4}

\href{OperatorsAA\#accumulate}{accumulate},
\href{OperatorsAA\#as_map}{as\_map},
\href{OperatorsBC\#collect}{collect}, \href{OperatorsBC\#count}{count},
\href{OperatorsBC\#create_map}{create\_map},
\href{OperatorsDH\#distribution_of}{distribution\_of},
\href{OperatorsDH\#distribution_of}{distribution\_of},
\href{OperatorsDH\#distribution_of}{distribution\_of},
\href{OperatorsDH\#distribution2d_of}{distribution2d\_of},
\href{OperatorsDH\#distribution2d_of}{distribution2d\_of},
\href{OperatorsDH\#distribution2d_of}{distribution2d\_of},
\href{OperatorsDH\#first_with}{first\_with},
\href{OperatorsDH\#frequency_of}{frequency\_of},
\href{OperatorsDH\#group_by}{group\_by},
\href{OperatorsIM\#index_by}{index\_by},
\href{OperatorsIM\#last_with}{last\_with},
\href{OperatorsIM\#max_of}{max\_of},
\href{OperatorsIM\#mean_of}{mean\_of},
\href{OperatorsIM\#min_of}{min\_of},
\href{OperatorsNR\#product_of}{product\_of},
\href{OperatorsSZ\#sort_by}{sort\_by},
\href{OperatorsSZ\#sum_of}{sum\_of},
\href{OperatorsSZ\#variance_of}{variance\_of},
\href{OperatorsSZ\#where}{where},
\href{OperatorsSZ\#with_max_of}{with\_max\_of},
\href{OperatorsSZ\#with_min_of}{with\_min\_of},

\begin{center}\rule{0.5\linewidth}{\linethickness}\end{center}

\subsection{List-related operators}\label{list-related-operators-4}

\href{OperatorsBC\#copy_between}{copy\_between},
\href{OperatorsIM\#index_of}{index\_of},
\href{OperatorsIM\#last_index_of}{last\_index\_of},

\begin{center}\rule{0.5\linewidth}{\linethickness}\end{center}

\subsection{Logical operators}\label{logical-operators-4}

\href{OperatorsAA\#:}{:}, \href{OperatorsAA\#!}{!},
\href{OperatorsAA\#?}{?}, \href{OperatorsAA\#and}{and},
\href{OperatorsNR\#or}{or}, \href{OperatorsSZ\#xor}{xor},

\begin{center}\rule{0.5\linewidth}{\linethickness}\end{center}

\subsection{Map comparaison
operators}\label{map-comparaison-operators-4}

\href{OperatorsDH\#fuzzy_kappa}{fuzzy\_kappa},
\href{OperatorsDH\#fuzzy_kappa_sim}{fuzzy\_kappa\_sim},
\href{OperatorsIM\#kappa}{kappa},
\href{OperatorsIM\#kappa_sim}{kappa\_sim},
\href{OperatorsNR\#percent_absolute_deviation}{percent\_absolute\_deviation},

\begin{center}\rule{0.5\linewidth}{\linethickness}\end{center}

\subsection{Map-related operators}\label{map-related-operators-4}

\href{OperatorsAA\#as_map}{as\_map},
\href{OperatorsBC\#create_map}{create\_map},
\href{OperatorsIM\#index_of}{index\_of},
\href{OperatorsIM\#last_index_of}{last\_index\_of},

\begin{center}\rule{0.5\linewidth}{\linethickness}\end{center}

\subsection{Material}\label{material-6}

\href{OperatorsIM\#material}{material},

\begin{center}\rule{0.5\linewidth}{\linethickness}\end{center}

\subsection{Matrix-related operators}\label{matrix-related-operators-4}

\href{OperatorsAA\#-}{-}, \href{OperatorsAA\#/}{/},
\href{OperatorsAA\#.}{.}, \href{OperatorsAA\#*}{*},
\href{OperatorsAA\#+}{+},
\href{OperatorsAA\#append_horizontally}{append\_horizontally},
\href{OperatorsAA\#append_vertically}{append\_vertically},
\href{OperatorsBC\#column_at}{column\_at},
\href{OperatorsBC\#columns_list}{columns\_list},
\href{OperatorsDH\#determinant}{determinant},
\href{OperatorsDH\#eigenvalues}{eigenvalues},
\href{OperatorsIM\#index_of}{index\_of},
\href{OperatorsIM\#inverse}{inverse},
\href{OperatorsIM\#last_index_of}{last\_index\_of},
\href{OperatorsNR\#row_at}{row\_at},
\href{OperatorsNR\#rows_list}{rows\_list},
\href{OperatorsSZ\#shuffle}{shuffle}, \href{OperatorsSZ\#trace}{trace},
\href{OperatorsSZ\#transpose}{transpose},

\begin{center}\rule{0.5\linewidth}{\linethickness}\end{center}

\subsection{multicriteria operators}\label{multicriteria-operators-4}

\href{OperatorsDH\#electre_dm}{electre\_DM},
\href{OperatorsDH\#evidence_theory_dm}{evidence\_theory\_DM},
\href{OperatorsDH\#fuzzy_choquet_dm}{fuzzy\_choquet\_DM},
\href{OperatorsNR\#promethee_dm}{promethee\_DM},
\href{OperatorsSZ\#weighted_means_dm}{weighted\_means\_DM},

\begin{center}\rule{0.5\linewidth}{\linethickness}\end{center}

\subsection{Path-related operators}\label{path-related-operators-4}

\href{OperatorsAA\#agent_from_geometry}{agent\_from\_geometry},
\href{OperatorsAA\#all_pairs_shortest_path}{all\_pairs\_shortest\_path},
\href{OperatorsAA\#as_path}{as\_path},
\href{OperatorsIM\#load_shortest_paths}{load\_shortest\_paths},
\href{OperatorsIM\#max_flow_between}{max\_flow\_between},
\href{OperatorsNR\#path_between}{path\_between},
\href{OperatorsNR\#path_to}{path\_to},
\href{OperatorsNR\#paths_between}{paths\_between},
\href{OperatorsSZ\#use_cache}{use\_cache},

\begin{center}\rule{0.5\linewidth}{\linethickness}\end{center}

\subsection{Points-related operators}\label{points-related-operators-4}

\href{OperatorsAA\#-}{-}, \href{OperatorsAA\#/}{/},
\href{OperatorsAA\#*}{*}, \href{OperatorsAA\#+}{+},
\href{OperatorsAA\#\%3C}{\textless{}},
\href{OperatorsAA\#\%3C=}{\textless{}=},
\href{OperatorsAA\#\%3E}{\textgreater{}},
\href{OperatorsAA\#\%3E=}{\textgreater{}=},
\href{OperatorsAA\#add_point}{add\_point},
\href{OperatorsAA\#angle_between}{angle\_between},
\href{OperatorsAA\#any_location_in}{any\_location\_in},
\href{OperatorsBC\#centroid}{centroid},
\href{OperatorsBC\#closest_points_with}{closest\_points\_with},
\href{OperatorsDH\#farthest_point_to}{farthest\_point\_to},
\href{OperatorsDH\#grid_at}{grid\_at}, \href{OperatorsNR\#norm}{norm},
\href{OperatorsNR\#point}{point},
\href{OperatorsNR\#points_along}{points\_along},
\href{OperatorsNR\#points_at}{points\_at},
\href{OperatorsNR\#points_on}{points\_on},

\begin{center}\rule{0.5\linewidth}{\linethickness}\end{center}

\subsection{Random operators}\label{random-operators-4}

\href{OperatorsBC\#binomial}{binomial}, \href{OperatorsDH\#flip}{flip},
\href{OperatorsDH\#gauss}{gauss},
\href{OperatorsIM\#improved_generator}{improved\_generator},
\href{OperatorsNR\#open_simplex_generator}{open\_simplex\_generator},
\href{OperatorsNR\#poisson}{poisson}, \href{OperatorsNR\#rnd}{rnd},
\href{OperatorsNR\#rnd_choice}{rnd\_choice},
\href{OperatorsSZ\#sample}{sample},
\href{OperatorsSZ\#shuffle}{shuffle},
\href{OperatorsSZ\#simplex_generator}{simplex\_generator},
\href{OperatorsSZ\#skew_gauss}{skew\_gauss},
\href{OperatorsSZ\#truncated_gauss}{truncated\_gauss},

\begin{center}\rule{0.5\linewidth}{\linethickness}\end{center}

\subsection{ReverseOperators}\label{reverseoperators-4}

\href{OperatorsSZ\#savesimulation}{saveSimulation},
\href{OperatorsSZ\#serialize}{serialize},
\href{OperatorsSZ\#serializeagent}{serializeAgent},
\href{OperatorsSZ\#unserializesimulation}{unSerializeSimulation},
\href{OperatorsSZ\#unserializesimulationfromfile}{unSerializeSimulationFromFile},

\begin{center}\rule{0.5\linewidth}{\linethickness}\end{center}

\subsection{Shape}\label{shape-4}

\href{OperatorsAA\#arc}{arc}, \href{OperatorsBC\#box}{box},
\href{OperatorsBC\#circle}{circle}, \href{OperatorsBC\#cone}{cone},
\href{OperatorsBC\#cone3d}{cone3D}, \href{OperatorsBC\#cross}{cross},
\href{OperatorsBC\#cube}{cube}, \href{OperatorsBC\#curve}{curve},
\href{OperatorsBC\#cylinder}{cylinder},
\href{OperatorsDH\#ellipse}{ellipse},
\href{OperatorsDH\#envelope}{envelope},
\href{OperatorsDH\#geometry_collection}{geometry\_collection},
\href{OperatorsDH\#hexagon}{hexagon}, \href{OperatorsIM\#line}{line},
\href{OperatorsIM\#link}{link}, \href{OperatorsNR\#plan}{plan},
\href{OperatorsNR\#polygon}{polygon},
\href{OperatorsNR\#polyhedron}{polyhedron},
\href{OperatorsNR\#pyramid}{pyramid},
\href{OperatorsNR\#rectangle}{rectangle},
\href{OperatorsSZ\#sphere}{sphere}, \href{OperatorsSZ\#square}{square},
\href{OperatorsSZ\#squircle}{squircle},
\href{OperatorsSZ\#teapot}{teapot},
\href{OperatorsSZ\#triangle}{triangle},

\begin{center}\rule{0.5\linewidth}{\linethickness}\end{center}

\subsection{Spatial operators}\label{spatial-operators-4}

\href{OperatorsAA\#-}{-}, \href{OperatorsAA\#*}{*},
\href{OperatorsAA\#+}{+}, \href{OperatorsAA\#add_point}{add\_point},
\href{OperatorsAA\#agent_closest_to}{agent\_closest\_to},
\href{OperatorsAA\#agent_farthest_to}{agent\_farthest\_to},
\href{OperatorsAA\#agents_at_distance}{agents\_at\_distance},
\href{OperatorsAA\#agents_inside}{agents\_inside},
\href{OperatorsAA\#agents_overlapping}{agents\_overlapping},
\href{OperatorsAA\#angle_between}{angle\_between},
\href{OperatorsAA\#any_location_in}{any\_location\_in},
\href{OperatorsAA\#arc}{arc}, \href{OperatorsAA\#around}{around},
\href{OperatorsAA\#as_4_grid}{as\_4\_grid},
\href{OperatorsAA\#as_grid}{as\_grid},
\href{OperatorsAA\#as_hexagonal_grid}{as\_hexagonal\_grid},
\href{OperatorsAA\#at_distance}{at\_distance},
\href{OperatorsAA\#at_location}{at\_location},
\href{OperatorsBC\#box}{box}, \href{OperatorsBC\#centroid}{centroid},
\href{OperatorsBC\#circle}{circle}, \href{OperatorsBC\#clean}{clean},
\href{OperatorsBC\#clean_network}{clean\_network},
\href{OperatorsBC\#closest_points_with}{closest\_points\_with},
\href{OperatorsBC\#closest_to}{closest\_to},
\href{OperatorsBC\#cone}{cone}, \href{OperatorsBC\#cone3d}{cone3D},
\href{OperatorsBC\#convex_hull}{convex\_hull},
\href{OperatorsBC\#covers}{covers}, \href{OperatorsBC\#cross}{cross},
\href{OperatorsBC\#crosses}{crosses}, \href{OperatorsBC\#crs}{crs},
\href{OperatorsBC\#crs_transform}{CRS\_transform},
\href{OperatorsBC\#cube}{cube}, \href{OperatorsBC\#curve}{curve},
\href{OperatorsBC\#cylinder}{cylinder}, \href{OperatorsDH\#dem}{dem},
\href{OperatorsDH\#direction_between}{direction\_between},
\href{OperatorsDH\#disjoint_from}{disjoint\_from},
\href{OperatorsDH\#distance_between}{distance\_between},
\href{OperatorsDH\#distance_to}{distance\_to},
\href{OperatorsDH\#ellipse}{ellipse},
\href{OperatorsDH\#envelope}{envelope},
\href{OperatorsDH\#farthest_point_to}{farthest\_point\_to},
\href{OperatorsDH\#farthest_to}{farthest\_to},
\href{OperatorsDH\#geometry_collection}{geometry\_collection},
\href{OperatorsDH\#gini}{gini}, \href{OperatorsDH\#hexagon}{hexagon},
\href{OperatorsDH\#hierarchical_clustering}{hierarchical\_clustering},
\href{OperatorsIM\#idw}{IDW}, \href{OperatorsIM\#inside}{inside},
\href{OperatorsIM\#inter}{inter},
\href{OperatorsIM\#intersects}{intersects},
\href{OperatorsIM\#line}{line}, \href{OperatorsIM\#link}{link},
\href{OperatorsIM\#masked_by}{masked\_by},
\href{OperatorsIM\#moran}{moran},
\href{OperatorsNR\#neighbors_at}{neighbors\_at},
\href{OperatorsNR\#neighbors_of}{neighbors\_of},
\href{OperatorsNR\#overlapping}{overlapping},
\href{OperatorsNR\#overlaps}{overlaps},
\href{OperatorsNR\#partially_overlaps}{partially\_overlaps},
\href{OperatorsNR\#path_between}{path\_between},
\href{OperatorsNR\#path_to}{path\_to}, \href{OperatorsNR\#plan}{plan},
\href{OperatorsNR\#points_along}{points\_along},
\href{OperatorsNR\#points_at}{points\_at},
\href{OperatorsNR\#points_on}{points\_on},
\href{OperatorsNR\#polygon}{polygon},
\href{OperatorsNR\#polyhedron}{polyhedron},
\href{OperatorsNR\#pyramid}{pyramid},
\href{OperatorsNR\#rectangle}{rectangle},
\href{OperatorsNR\#rgb_to_xyz}{rgb\_to\_xyz},
\href{OperatorsNR\#rotated_by}{rotated\_by},
\href{OperatorsNR\#round}{round},
\href{OperatorsSZ\#scaled_to}{scaled\_to},
\href{OperatorsSZ\#set_z}{set\_z},
\href{OperatorsSZ\#simple_clustering_by_distance}{simple\_clustering\_by\_distance},
\href{OperatorsSZ\#simplification}{simplification},
\href{OperatorsSZ\#skeletonize}{skeletonize},
\href{OperatorsSZ\#smooth}{smooth}, \href{OperatorsSZ\#sphere}{sphere},
\href{OperatorsSZ\#split_at}{split\_at},
\href{OperatorsSZ\#split_geometry}{split\_geometry},
\href{OperatorsSZ\#split_lines}{split\_lines},
\href{OperatorsSZ\#square}{square},
\href{OperatorsSZ\#squircle}{squircle},
\href{OperatorsSZ\#teapot}{teapot},
\href{OperatorsSZ\#to_gama_crs}{to\_GAMA\_CRS},
\href{OperatorsSZ\#to_rectangles}{to\_rectangles},
\href{OperatorsSZ\#to_squares}{to\_squares},
\href{OperatorsSZ\#to_sub_geometries}{to\_sub\_geometries},
\href{OperatorsSZ\#touches}{touches},
\href{OperatorsSZ\#towards}{towards},
\href{OperatorsSZ\#transformed_by}{transformed\_by},
\href{OperatorsSZ\#translated_by}{translated\_by},
\href{OperatorsSZ\#triangle}{triangle},
\href{OperatorsSZ\#triangulate}{triangulate},
\href{OperatorsSZ\#union}{union}, \href{OperatorsSZ\#using}{using},
\href{OperatorsSZ\#voronoi}{voronoi},
\href{OperatorsSZ\#with_precision}{with\_precision},
\href{OperatorsSZ\#without_holes}{without\_holes},

\begin{center}\rule{0.5\linewidth}{\linethickness}\end{center}

\subsection{Spatial properties
operators}\label{spatial-properties-operators-4}

\href{OperatorsBC\#covers}{covers},
\href{OperatorsBC\#crosses}{crosses},
\href{OperatorsIM\#intersects}{intersects},
\href{OperatorsNR\#partially_overlaps}{partially\_overlaps},
\href{OperatorsSZ\#touches}{touches},

\begin{center}\rule{0.5\linewidth}{\linethickness}\end{center}

\subsection{Spatial queries
operators}\label{spatial-queries-operators-4}

\href{OperatorsAA\#agent_closest_to}{agent\_closest\_to},
\href{OperatorsAA\#agent_farthest_to}{agent\_farthest\_to},
\href{OperatorsAA\#agents_at_distance}{agents\_at\_distance},
\href{OperatorsAA\#agents_inside}{agents\_inside},
\href{OperatorsAA\#agents_overlapping}{agents\_overlapping},
\href{OperatorsAA\#at_distance}{at\_distance},
\href{OperatorsBC\#closest_to}{closest\_to},
\href{OperatorsDH\#farthest_to}{farthest\_to},
\href{OperatorsIM\#inside}{inside},
\href{OperatorsNR\#neighbors_at}{neighbors\_at},
\href{OperatorsNR\#neighbors_of}{neighbors\_of},
\href{OperatorsNR\#overlapping}{overlapping},

\begin{center}\rule{0.5\linewidth}{\linethickness}\end{center}

\subsection{Spatial relations
operators}\label{spatial-relations-operators-4}

\href{OperatorsDH\#direction_between}{direction\_between},
\href{OperatorsDH\#distance_between}{distance\_between},
\href{OperatorsDH\#distance_to}{distance\_to},
\href{OperatorsNR\#path_between}{path\_between},
\href{OperatorsNR\#path_to}{path\_to},
\href{OperatorsSZ\#towards}{towards},

\begin{center}\rule{0.5\linewidth}{\linethickness}\end{center}

\subsection{Spatial statistical
operators}\label{spatial-statistical-operators-4}

\href{OperatorsDH\#hierarchical_clustering}{hierarchical\_clustering},
\href{OperatorsSZ\#simple_clustering_by_distance}{simple\_clustering\_by\_distance},

\begin{center}\rule{0.5\linewidth}{\linethickness}\end{center}

\subsection{Spatial transformations
operators}\label{spatial-transformations-operators-4}

\href{OperatorsAA\#-}{-}, \href{OperatorsAA\#*}{*},
\href{OperatorsAA\#+}{+}, \href{OperatorsAA\#as_4_grid}{as\_4\_grid},
\href{OperatorsAA\#as_grid}{as\_grid},
\href{OperatorsAA\#as_hexagonal_grid}{as\_hexagonal\_grid},
\href{OperatorsAA\#at_location}{at\_location},
\href{OperatorsBC\#clean}{clean},
\href{OperatorsBC\#clean_network}{clean\_network},
\href{OperatorsBC\#convex_hull}{convex\_hull},
\href{OperatorsBC\#crs_transform}{CRS\_transform},
\href{OperatorsNR\#rotated_by}{rotated\_by},
\href{OperatorsSZ\#scaled_to}{scaled\_to},
\href{OperatorsSZ\#simplification}{simplification},
\href{OperatorsSZ\#skeletonize}{skeletonize},
\href{OperatorsSZ\#smooth}{smooth},
\href{OperatorsSZ\#split_geometry}{split\_geometry},
\href{OperatorsSZ\#split_lines}{split\_lines},
\href{OperatorsSZ\#to_gama_crs}{to\_GAMA\_CRS},
\href{OperatorsSZ\#to_rectangles}{to\_rectangles},
\href{OperatorsSZ\#to_squares}{to\_squares},
\href{OperatorsSZ\#to_sub_geometries}{to\_sub\_geometries},
\href{OperatorsSZ\#transformed_by}{transformed\_by},
\href{OperatorsSZ\#translated_by}{translated\_by},
\href{OperatorsSZ\#triangulate}{triangulate},
\href{OperatorsSZ\#voronoi}{voronoi},
\href{OperatorsSZ\#with_precision}{with\_precision},
\href{OperatorsSZ\#without_holes}{without\_holes},

\begin{center}\rule{0.5\linewidth}{\linethickness}\end{center}

\subsection{Species-related
operators}\label{species-related-operators-4}

\href{OperatorsIM\#index_of}{index\_of},
\href{OperatorsIM\#last_index_of}{last\_index\_of},
\href{OperatorsNR\#of_generic_species}{of\_generic\_species},
\href{OperatorsNR\#of_species}{of\_species},

\begin{center}\rule{0.5\linewidth}{\linethickness}\end{center}

\subsection{Statistical operators}\label{statistical-operators-4}

\href{OperatorsBC\#build}{build}, \href{OperatorsBC\#corr}{corR},
\href{OperatorsDH\#dbscan}{dbscan},
\href{OperatorsDH\#distribution_of}{distribution\_of},
\href{OperatorsDH\#distribution2d_of}{distribution2d\_of},
\href{OperatorsDH\#dtw}{dtw},
\href{OperatorsDH\#frequency_of}{frequency\_of},
\href{OperatorsDH\#gamma_rnd}{gamma\_rnd},
\href{OperatorsDH\#geometric_mean}{geometric\_mean},
\href{OperatorsDH\#gini}{gini},
\href{OperatorsDH\#harmonic_mean}{harmonic\_mean},
\href{OperatorsDH\#hierarchical_clustering}{hierarchical\_clustering},
\href{OperatorsIM\#kmeans}{kmeans},
\href{OperatorsIM\#kurtosis}{kurtosis}, \href{OperatorsIM\#max}{max},
\href{OperatorsIM\#mean}{mean},
\href{OperatorsIM\#mean_deviation}{mean\_deviation},
\href{OperatorsIM\#meanr}{meanR}, \href{OperatorsIM\#median}{median},
\href{OperatorsIM\#min}{min}, \href{OperatorsIM\#moran}{moran},
\href{OperatorsIM\#mul}{mul}, \href{OperatorsNR\#predict}{predict},
\href{OperatorsSZ\#simple_clustering_by_distance}{simple\_clustering\_by\_distance},
\href{OperatorsSZ\#skewness}{skewness},
\href{OperatorsSZ\#split}{split},
\href{OperatorsSZ\#split_in}{split\_in},
\href{OperatorsSZ\#split_using}{split\_using},
\href{OperatorsSZ\#standard_deviation}{standard\_deviation},
\href{OperatorsSZ\#sum}{sum}, \href{OperatorsSZ\#variance}{variance},

\begin{center}\rule{0.5\linewidth}{\linethickness}\end{center}

\subsection{Strings-related
operators}\label{strings-related-operators-4}

\href{OperatorsAA\#+}{+}, \href{OperatorsAA\#\%3C}{\textless{}},
\href{OperatorsAA\#\%3C=}{\textless{}=},
\href{OperatorsAA\#\%3E}{\textgreater{}},
\href{OperatorsAA\#\%3E=}{\textgreater{}=}, \href{OperatorsAA\#at}{at},
\href{OperatorsBC\#char}{char}, \href{OperatorsBC\#contains}{contains},
\href{OperatorsBC\#contains_all}{contains\_all},
\href{OperatorsBC\#contains_any}{contains\_any},
\href{OperatorsBC\#copy_between}{copy\_between},
\href{OperatorsDH\#date}{date}, \href{OperatorsDH\#empty}{empty},
\href{OperatorsDH\#first}{first}, \href{OperatorsIM\#in}{in},
\href{OperatorsIM\#indented_by}{indented\_by},
\href{OperatorsIM\#index_of}{index\_of},
\href{OperatorsIM\#is_number}{is\_number},
\href{OperatorsIM\#last}{last},
\href{OperatorsIM\#last_index_of}{last\_index\_of},
\href{OperatorsIM\#length}{length},
\href{OperatorsIM\#lower_case}{lower\_case},
\href{OperatorsNR\#replace}{replace},
\href{OperatorsNR\#replace_regex}{replace\_regex},
\href{OperatorsNR\#reverse}{reverse},
\href{OperatorsSZ\#sample}{sample},
\href{OperatorsSZ\#shuffle}{shuffle},
\href{OperatorsSZ\#split_with}{split\_with},
\href{OperatorsSZ\#string}{string},
\href{OperatorsSZ\#upper_case}{upper\_case},

\begin{center}\rule{0.5\linewidth}{\linethickness}\end{center}

\subsection{System}\label{system-4}

\href{OperatorsAA\#.}{.}, \href{OperatorsBC\#command}{command},
\href{OperatorsBC\#copy}{copy}, \href{OperatorsDH\#dead}{dead},
\href{OperatorsDH\#eval_gaml}{eval\_gaml},
\href{OperatorsDH\#every}{every},
\href{OperatorsIM\#is_error}{is\_error},
\href{OperatorsIM\#is_warning}{is\_warning},
\href{OperatorsSZ\#user_input}{user\_input},

\begin{center}\rule{0.5\linewidth}{\linethickness}\end{center}

\subsection{Time-related operators}\label{time-related-operators-4}

\href{OperatorsDH\#date}{date}, \href{OperatorsSZ\#string}{string},

\begin{center}\rule{0.5\linewidth}{\linethickness}\end{center}

\subsection{Types-related operators}\label{types-related-operators-4}

\begin{center}\rule{0.5\linewidth}{\linethickness}\end{center}

\subsection{User control operators}\label{user-control-operators-4}

\href{OperatorsSZ\#user_input}{user\_input},

\begin{center}\rule{0.5\linewidth}{\linethickness}\end{center}

\section{Operators}\label{operators-4}

\begin{center}\rule{0.5\linewidth}{\linethickness}\end{center}

\subsection{\texorpdfstring{\texttt{nb\_cycles}}{nb\_cycles}}\label{nb_cycles}

\subsubsection{Possible use:}\label{possible-use-365}

\begin{itemize}
\tightlist
\item
  \textbf{\texttt{nb\_cycles}} (\texttt{graph}) ---\textgreater{}
  \texttt{int}
\end{itemize}

\subsubsection{Result:}\label{result-353}

returns the maximum number of independent cycles in a graph. This number
(u) is estimated through the number of nodes (v), links (e) and of
sub-graphs (p): u = e - v + p.

\subsubsection{Examples:}\label{examples-252}

\begin{verbatim}
graph graphEpidemio <- graph([]);  
int var1 <- nb_cycles(graphEpidemio); // var1 equals the number of cycles in the graph
\end{verbatim}

\subsubsection{See also:}\label{see-also-145}

\href{OperatorsAA\#alpha_index}{alpha\_index},
\href{OperatorsBC\#beta_index}{beta\_index},
\href{OperatorsDH\#gamma_index}{gamma\_index},
\href{OperatorsBC\#connectivity_index}{connectivity\_index},

\begin{center}\rule{0.5\linewidth}{\linethickness}\end{center}

\subsection{\texorpdfstring{\texttt{neighbors\_at}}{neighbors\_at}}\label{neighbors_at}

\subsubsection{Possible use:}\label{possible-use-366}

\begin{itemize}
\tightlist
\item
  \texttt{geometry} \textbf{\texttt{neighbors\_at}} \texttt{float}
  ---\textgreater{} \texttt{list}
\item
  \textbf{\texttt{neighbors\_at}} (\texttt{geometry} , \texttt{float})
  ---\textgreater{} \texttt{list}
\end{itemize}

\subsubsection{Result:}\label{result-354}

a list, containing all the agents of the same species than the left
argument (if it is an agent) located at a distance inferior or equal to
the right-hand operand to the left-hand operand (geometry, agent,
point).

\subsubsection{Comment:}\label{comment-73}

The topology used to compute the neighborhood is the one of the
left-operand if this one is an agent; otherwise the one of the agent
applying the operator.

\subsubsection{Examples:}\label{examples-253}

\begin{verbatim}
 
list var0 <- (self neighbors_at (10)); // var0 equals all the agents located at a distance lower or equal to 10 to the agent applying the operator.
\end{verbatim}

\subsubsection{See also:}\label{see-also-146}

\href{OperatorsNR\#neighbors_of}{neighbors\_of},
\href{OperatorsBC\#closest_to}{closest\_to},
\href{OperatorsNR\#overlapping}{overlapping},
\href{OperatorsAA\#agents_overlapping}{agents\_overlapping},
\href{OperatorsAA\#agents_inside}{agents\_inside},
\href{OperatorsAA\#agent_closest_to}{agent\_closest\_to},
\href{OperatorsAA\#at_distance}{at\_distance},

\begin{center}\rule{0.5\linewidth}{\linethickness}\end{center}

\subsection{\texorpdfstring{\texttt{neighbors\_of}}{neighbors\_of}}\label{neighbors_of}

\subsubsection{Possible use:}\label{possible-use-367}

\begin{itemize}
\tightlist
\item
  \texttt{graph} \textbf{\texttt{neighbors\_of}} \texttt{unknown}
  ---\textgreater{} \texttt{list}
\item
  \textbf{\texttt{neighbors\_of}} (\texttt{graph} , \texttt{unknown})
  ---\textgreater{} \texttt{list}
\item
  \texttt{topology} \textbf{\texttt{neighbors\_of}} \texttt{agent}
  ---\textgreater{} \texttt{list}
\item
  \textbf{\texttt{neighbors\_of}} (\texttt{topology} , \texttt{agent})
  ---\textgreater{} \texttt{list}
\item
  \textbf{\texttt{neighbors\_of}} (\texttt{topology}, \texttt{geometry},
  \texttt{float}) ---\textgreater{} \texttt{list}
\end{itemize}

\subsubsection{Result:}\label{result-355}

a list, containing all the agents of the same species than the argument
(if it is an agent) located at a distance inferior or equal to 1 to the
right-hand operand agent considering the left-hand operand topology.

\subsubsection{Special cases:}\label{special-cases-103}

\begin{itemize}
\tightlist
\item
  a list, containing all the agents of the same species than the left
  argument (if it is an agent) located at a distance inferior or equal
  to the third argument to the second argument (agent, geometry or
  point) considering the first operand topology.
\end{itemize}

\begin{verbatim}
 
list var3 <- neighbors_of (topology(self), self,10); // var3 equals all the agents located at a distance lower or equal to 10 to the agent applying the operator considering its topology.
\end{verbatim}

\subsubsection{Examples:}\label{examples-254}

\begin{verbatim}
 
list var0 <- graphEpidemio neighbors_of (node(3)); // var0 equals [node0,node2] 
list var1 <- graphFromMap neighbors_of node({12,45}); // var1 equals [{1.0,5.0},{34.0,56.0}] 
list var2 <- topology(self) neighbors_of self; // var2 equals returns all the agents located at a distance lower or equal to 1 to the agent applying the operator considering its topology.
\end{verbatim}

\subsubsection{See also:}\label{see-also-147}

\href{OperatorsNR\#predecessors_of}{predecessors\_of},
\href{OperatorsSZ\#successors_of}{successors\_of},
\href{OperatorsNR\#neighbors_at}{neighbors\_at},
\href{OperatorsBC\#closest_to}{closest\_to},
\href{OperatorsNR\#overlapping}{overlapping},
\href{OperatorsAA\#agents_overlapping}{agents\_overlapping},
\href{OperatorsAA\#agents_inside}{agents\_inside},
\href{OperatorsAA\#agent_closest_to}{agent\_closest\_to},

\begin{center}\rule{0.5\linewidth}{\linethickness}\end{center}

\subsection{\texorpdfstring{\texttt{new\_emotion}}{new\_emotion}}\label{new_emotion}

\subsubsection{Possible use:}\label{possible-use-368}

\begin{itemize}
\tightlist
\item
  \textbf{\texttt{new\_emotion}} (\texttt{string}) ---\textgreater{}
  \texttt{emotion}
\item
  \texttt{string} \textbf{\texttt{new\_emotion}} \texttt{predicate}
  ---\textgreater{} \texttt{emotion}
\item
  \textbf{\texttt{new\_emotion}} (\texttt{string} , \texttt{predicate})
  ---\textgreater{} \texttt{emotion}
\item
  \texttt{string} \textbf{\texttt{new\_emotion}} \texttt{float}
  ---\textgreater{} \texttt{emotion}
\item
  \textbf{\texttt{new\_emotion}} (\texttt{string} , \texttt{float})
  ---\textgreater{} \texttt{emotion}
\item
  \texttt{string} \textbf{\texttt{new\_emotion}} \texttt{agent}
  ---\textgreater{} \texttt{emotion}
\item
  \textbf{\texttt{new\_emotion}} (\texttt{string} , \texttt{agent})
  ---\textgreater{} \texttt{emotion}
\item
  \textbf{\texttt{new\_emotion}} (\texttt{string}, \texttt{float},
  \texttt{float}) ---\textgreater{} \texttt{emotion}
\item
  \textbf{\texttt{new\_emotion}} (\texttt{string}, \texttt{float},
  \texttt{agent}) ---\textgreater{} \texttt{emotion}
\item
  \textbf{\texttt{new\_emotion}} (\texttt{string}, \texttt{float},
  \texttt{predicate}) ---\textgreater{} \texttt{emotion}
\item
  \textbf{\texttt{new\_emotion}} (\texttt{string}, \texttt{predicate},
  \texttt{agent}) ---\textgreater{} \texttt{emotion}
\item
  \textbf{\texttt{new\_emotion}} (\texttt{string}, \texttt{float},
  \texttt{float}, \texttt{agent}) ---\textgreater{} \texttt{emotion}
\item
  \textbf{\texttt{new\_emotion}} (\texttt{string}, \texttt{float},
  \texttt{predicate}, \texttt{agent}) ---\textgreater{} \texttt{emotion}
\item
  \textbf{\texttt{new\_emotion}} (\texttt{string}, \texttt{float},
  \texttt{predicate}, \texttt{float}) ---\textgreater{} \texttt{emotion}
\item
  \textbf{\texttt{new\_emotion}} (\texttt{string}, \texttt{float},
  \texttt{predicate}, \texttt{float}, \texttt{agent}) ---\textgreater{}
  \texttt{emotion}
\end{itemize}

\subsubsection{Result:}\label{result-356}

a new emotion with the given properties (name) a new emotion with the
given properties (name,intensity,decay) a new emotion with the given
properties (name,about) a new emotion with the given properties (name) a
new emotion with the given properties (name) a new emotion with the
given properties (name,intensity,about) a new emotion with the given
properties (name) a new emotion with the given properties (name) a new
emotion with the given properties (name) a new emotion with the given
properties (name, intensity) a new emotion with the given properties
(name) a new emotion with the given properties (name)

\subsubsection{Examples:}\label{examples-255}

\begin{verbatim}
emotion("joy",12.3,eatFood,4) emotion("joy",12.3,4) emotion("joy",eatFood) emotion("joy",12.3,eatFood,4) emotion("joy") emotion("joy",12.3,eatFood) emotion("joy",12.3,eatFood,4) emotion("joy",12.3,eatFood,4) emotion("joy",12.3,eatFood,4) emotion("joy",12.3) emotion("joy",12.3,eatFood,4) emotion("joy",12.3,eatFood,4) 
\end{verbatim}

\begin{center}\rule{0.5\linewidth}{\linethickness}\end{center}

\subsection{\texorpdfstring{\texttt{new\_folder}}{new\_folder}}\label{new_folder}

\subsubsection{Possible use:}\label{possible-use-369}

\begin{itemize}
\tightlist
\item
  \textbf{\texttt{new\_folder}} (\texttt{string}) ---\textgreater{}
  \texttt{file}
\end{itemize}

\subsubsection{Result:}\label{result-357}

opens an existing repository or create a new folder if it does not
exist.

\subsubsection{Special cases:}\label{special-cases-104}

\begin{itemize}
\tightlist
\item
  If the specified string does not refer to an existing repository, the
  repository is created.\\
\item
  If the string refers to an existing file, an exception is risen.
\end{itemize}

\subsubsection{Examples:}\label{examples-256}

\begin{verbatim}
file dirNewT <- new_folder("incl/");    // dirNewT represents the repository "../incl/"                                                             // eventually creates the directory ../incl 
\end{verbatim}

\subsubsection{See also:}\label{see-also-148}

\href{OperatorsDH\#folder}{folder}, \href{OperatorsDH\#file}{file},

\begin{center}\rule{0.5\linewidth}{\linethickness}\end{center}

\subsection{\texorpdfstring{\texttt{new\_mental\_state}}{new\_mental\_state}}\label{new_mental_state}

\subsubsection{Possible use:}\label{possible-use-370}

\begin{itemize}
\tightlist
\item
  \textbf{\texttt{new\_mental\_state}} (\texttt{string})
  ---\textgreater{} \texttt{mental\_state}
\item
  \texttt{string} \textbf{\texttt{new\_mental\_state}}
  \texttt{predicate} ---\textgreater{} \texttt{mental\_state}
\item
  \textbf{\texttt{new\_mental\_state}} (\texttt{string} ,
  \texttt{predicate}) ---\textgreater{} \texttt{mental\_state}
\item
  \texttt{string} \textbf{\texttt{new\_mental\_state}} \texttt{emotion}
  ---\textgreater{} \texttt{mental\_state}
\item
  \textbf{\texttt{new\_mental\_state}} (\texttt{string} ,
  \texttt{emotion}) ---\textgreater{} \texttt{mental\_state}
\item
  \texttt{string} \textbf{\texttt{new\_mental\_state}}
  \texttt{mental\_state} ---\textgreater{} \texttt{mental\_state}
\item
  \textbf{\texttt{new\_mental\_state}} (\texttt{string} ,
  \texttt{mental\_state}) ---\textgreater{} \texttt{mental\_state}
\item
  \textbf{\texttt{new\_mental\_state}} (\texttt{string},
  \texttt{predicate}, \texttt{int}) ---\textgreater{}
  \texttt{mental\_state}
\item
  \textbf{\texttt{new\_mental\_state}} (\texttt{string},
  \texttt{mental\_state}, \texttt{agent}) ---\textgreater{}
  \texttt{mental\_state}
\item
  \textbf{\texttt{new\_mental\_state}} (\texttt{string},
  \texttt{emotion}, \texttt{agent}) ---\textgreater{}
  \texttt{mental\_state}
\item
  \textbf{\texttt{new\_mental\_state}} (\texttt{string},
  \texttt{emotion}, \texttt{float}) ---\textgreater{}
  \texttt{mental\_state}
\item
  \textbf{\texttt{new\_mental\_state}} (\texttt{string},
  \texttt{mental\_state}, \texttt{float}) ---\textgreater{}
  \texttt{mental\_state}
\item
  \textbf{\texttt{new\_mental\_state}} (\texttt{string},
  \texttt{predicate}, \texttt{agent}) ---\textgreater{}
  \texttt{mental\_state}
\item
  \textbf{\texttt{new\_mental\_state}} (\texttt{string},
  \texttt{mental\_state}, \texttt{int}) ---\textgreater{}
  \texttt{mental\_state}
\item
  \textbf{\texttt{new\_mental\_state}} (\texttt{string},
  \texttt{emotion}, \texttt{int}) ---\textgreater{}
  \texttt{mental\_state}
\item
  \textbf{\texttt{new\_mental\_state}} (\texttt{string},
  \texttt{predicate}, \texttt{float}) ---\textgreater{}
  \texttt{mental\_state}
\item
  \textbf{\texttt{new\_mental\_state}} (\texttt{string},
  \texttt{predicate}, \texttt{int}, \texttt{agent}) ---\textgreater{}
  \texttt{mental\_state}
\item
  \textbf{\texttt{new\_mental\_state}} (\texttt{string},
  \texttt{emotion}, \texttt{float}, \texttt{int}) ---\textgreater{}
  \texttt{mental\_state}
\item
  \textbf{\texttt{new\_mental\_state}} (\texttt{string},
  \texttt{predicate}, \texttt{float}, \texttt{int}) ---\textgreater{}
  \texttt{mental\_state}
\item
  \textbf{\texttt{new\_mental\_state}} (\texttt{string},
  \texttt{mental\_state}, \texttt{float}, \texttt{int})
  ---\textgreater{} \texttt{mental\_state}
\item
  \textbf{\texttt{new\_mental\_state}} (\texttt{string},
  \texttt{mental\_state}, \texttt{float}, \texttt{agent})
  ---\textgreater{} \texttt{mental\_state}
\item
  \textbf{\texttt{new\_mental\_state}} (\texttt{string},
  \texttt{emotion}, \texttt{int}, \texttt{agent}) ---\textgreater{}
  \texttt{mental\_state}
\item
  \textbf{\texttt{new\_mental\_state}} (\texttt{string},
  \texttt{mental\_state}, \texttt{int}, \texttt{agent})
  ---\textgreater{} \texttt{mental\_state}
\item
  \textbf{\texttt{new\_mental\_state}} (\texttt{string},
  \texttt{emotion}, \texttt{float}, \texttt{agent}) ---\textgreater{}
  \texttt{mental\_state}
\item
  \textbf{\texttt{new\_mental\_state}} (\texttt{string},
  \texttt{predicate}, \texttt{float}, \texttt{agent}) ---\textgreater{}
  \texttt{mental\_state}
\item
  \textbf{\texttt{new\_mental\_state}} (\texttt{string},
  \texttt{emotion}, \texttt{float}, \texttt{int}, \texttt{agent})
  ---\textgreater{} \texttt{mental\_state}
\item
  \textbf{\texttt{new\_mental\_state}} (\texttt{string},
  \texttt{mental\_state}, \texttt{float}, \texttt{int}, \texttt{agent})
  ---\textgreater{} \texttt{mental\_state}
\item
  \textbf{\texttt{new\_mental\_state}} (\texttt{string},
  \texttt{predicate}, \texttt{float}, \texttt{int}, \texttt{agent})
  ---\textgreater{} \texttt{mental\_state}
\end{itemize}

\subsubsection{Result:}\label{result-358}

a new mental state a new mental state a new mental state a new mental
state a new mental state a new mental state a new mental state a new
mental state a new mental state a new mental state a new mental state a
new mental state a new mental state a new mental state a new mental
state a new mental state a new mental state a new mental state a new
mental state a new mental state a new mental state a new mental state a
new mental state a new mental state a new mental state

\subsubsection{Examples:}\label{examples-257}

\begin{verbatim}
new_mental-state(belief) new_mental-state(belief) new_mental-state(belief) new_mental-state(belief) new_mental-state(belief) new_mental-state(belief) new_mental-state(belief) new_mental-state(belief) new_mental-state(belief) new_mental-state(belief) new_mental-state(belief) new_mental-state(belief) new_mental-state(belief) new_mental-state(belief) new_mental-state(belief) new_mental-state(belief) new_mental-state(belief) new_mental-state(belief) new_mental-state(belief) new_mental-state(belief) new_mental-state(belief) new_mental-state(belief) new_mental-state(belief) new_mental-state(belief) new_mental-state(belief) 
\end{verbatim}

\begin{center}\rule{0.5\linewidth}{\linethickness}\end{center}

\subsection{\texorpdfstring{\texttt{new\_predicate}}{new\_predicate}}\label{new_predicate}

\subsubsection{Possible use:}\label{possible-use-371}

\begin{itemize}
\tightlist
\item
  \textbf{\texttt{new\_predicate}} (\texttt{string}) ---\textgreater{}
  \texttt{predicate}
\item
  \texttt{string} \textbf{\texttt{new\_predicate}} \texttt{int}
  ---\textgreater{} \texttt{predicate}
\item
  \textbf{\texttt{new\_predicate}} (\texttt{string} , \texttt{int})
  ---\textgreater{} \texttt{predicate}
\item
  \texttt{string} \textbf{\texttt{new\_predicate}} \texttt{map}
  ---\textgreater{} \texttt{predicate}
\item
  \textbf{\texttt{new\_predicate}} (\texttt{string} , \texttt{map})
  ---\textgreater{} \texttt{predicate}
\item
  \texttt{string} \textbf{\texttt{new\_predicate}} \texttt{bool}
  ---\textgreater{} \texttt{predicate}
\item
  \textbf{\texttt{new\_predicate}} (\texttt{string} , \texttt{bool})
  ---\textgreater{} \texttt{predicate}
\item
  \texttt{string} \textbf{\texttt{new\_predicate}} \texttt{agent}
  ---\textgreater{} \texttt{predicate}
\item
  \textbf{\texttt{new\_predicate}} (\texttt{string} , \texttt{agent})
  ---\textgreater{} \texttt{predicate}
\item
  \textbf{\texttt{new\_predicate}} (\texttt{string}, \texttt{map},
  \texttt{int}) ---\textgreater{} \texttt{predicate}
\item
  \textbf{\texttt{new\_predicate}} (\texttt{string}, \texttt{map},
  \texttt{agent}) ---\textgreater{} \texttt{predicate}
\item
  \textbf{\texttt{new\_predicate}} (\texttt{string}, \texttt{map},
  \texttt{bool}) ---\textgreater{} \texttt{predicate}
\item
  \textbf{\texttt{new\_predicate}} (\texttt{string}, \texttt{map},
  \texttt{int}, \texttt{bool}) ---\textgreater{} \texttt{predicate}
\item
  \textbf{\texttt{new\_predicate}} (\texttt{string}, \texttt{map},
  \texttt{int}, \texttt{agent}) ---\textgreater{} \texttt{predicate}
\item
  \textbf{\texttt{new\_predicate}} (\texttt{string}, \texttt{map},
  \texttt{bool}, \texttt{agent}) ---\textgreater{} \texttt{predicate}
\item
  \textbf{\texttt{new\_predicate}} (\texttt{string}, \texttt{map},
  \texttt{int}, \texttt{bool}, \texttt{agent}) ---\textgreater{}
  \texttt{predicate}
\end{itemize}

\subsubsection{Result:}\label{result-359}

a new predicate with the given properties (name) a new predicate with
the given is\_true (name, lifetime) a new predicate with the given
properties (name, values, lifetime) a new predicate with the given
properties (name, values, agentCause) a new predicate with the given
properties (name, values, lifetime, is\_true) a new predicate with the
given properties (name, values, is\_true) a new predicate with the given
properties (name, values) a new predicate with the given properties
(name, values, lifetime, agentCause) a new predicate with the given
properties (name, values, lifetime, is\_true, agentCause) a new
predicate with the given properties (name, values, is\_true, agentCause)
a new predicate with the given is\_true (name, is\_true) a new predicate
with the given properties (name, values, lifetime)

\subsubsection{Examples:}\label{examples-258}

\begin{verbatim}
predicate("people to meet") predicate("hasWater", 10  predicate("people to meet", ["time"::10], true) predicate("people to meet", ["time"::10], agentA) predicate("people to meet", ["time"::10], 10,true) predicate("people to meet", ["time"::10], true) predicate("people to meet", people1 ) predicate("people to meet", ["time"::10], 10, agentA) predicate("people to meet", ["time"::10], 10, true, agentA) predicate("people to meet", ["time"::10], true, agentA) predicate("hasWater", true) predicate("people to meet", ["time"::10], true) 
\end{verbatim}

\begin{center}\rule{0.5\linewidth}{\linethickness}\end{center}

\subsection{\texorpdfstring{\texttt{new\_social\_link}}{new\_social\_link}}\label{new_social_link}

\subsubsection{Possible use:}\label{possible-use-372}

\begin{itemize}
\tightlist
\item
  \textbf{\texttt{new\_social\_link}} (\texttt{agent}) ---\textgreater{}
  \texttt{msi.gaml.architecture.simplebdi.SocialLink}
\item
  \textbf{\texttt{new\_social\_link}} (\texttt{agent}, \texttt{float},
  \texttt{float}, \texttt{float}, \texttt{float}) ---\textgreater{}
  \texttt{msi.gaml.architecture.simplebdi.SocialLink}
\end{itemize}

\subsubsection{Result:}\label{result-360}

a new social link a new social link

\subsubsection{Examples:}\label{examples-259}

\begin{verbatim}
new_social_link(agentA) new_social_link(agentA,0.0,-0.1,0.2,0.1) 
\end{verbatim}

\begin{center}\rule{0.5\linewidth}{\linethickness}\end{center}

\subsection{\texorpdfstring{\texttt{node}}{node}}\label{node}

\subsubsection{Possible use:}\label{possible-use-373}

\begin{itemize}
\tightlist
\item
  \textbf{\texttt{node}} (\texttt{unknown}) ---\textgreater{}
  \texttt{unknown}
\item
  \texttt{unknown} \textbf{\texttt{node}} \texttt{float}
  ---\textgreater{} \texttt{unknown}
\item
  \textbf{\texttt{node}} (\texttt{unknown} , \texttt{float})
  ---\textgreater{} \texttt{unknown}
\end{itemize}

\begin{center}\rule{0.5\linewidth}{\linethickness}\end{center}

\subsection{\texorpdfstring{\texttt{nodes}}{nodes}}\label{nodes}

\subsubsection{Possible use:}\label{possible-use-374}

\begin{itemize}
\tightlist
\item
  \textbf{\texttt{nodes}} (\texttt{container}) ---\textgreater{}
  \texttt{container}
\end{itemize}

\begin{center}\rule{0.5\linewidth}{\linethickness}\end{center}

\subsection{\texorpdfstring{\texttt{norm}}{norm}}\label{norm}

\subsubsection{Possible use:}\label{possible-use-375}

\begin{itemize}
\tightlist
\item
  \textbf{\texttt{norm}} (\texttt{point}) ---\textgreater{}
  \texttt{float}
\end{itemize}

\subsubsection{Result:}\label{result-361}

the norm of the vector with the coordinates of the point operand.

\subsubsection{Examples:}\label{examples-260}

\begin{verbatim}
 
float var0 <- norm({3,4}); // var0 equals 5.0
\end{verbatim}

\begin{center}\rule{0.5\linewidth}{\linethickness}\end{center}

\subsection{\texorpdfstring{\texttt{Norm}}{Norm}}\label{norm-1}

\subsubsection{Possible use:}\label{possible-use-376}

\begin{itemize}
\tightlist
\item
  \textbf{\texttt{Norm}} (\texttt{any}) ---\textgreater{} \texttt{Norm}
\end{itemize}

\subsubsection{Result:}\label{result-362}

Casts the operand into the type Norm

\begin{center}\rule{0.5\linewidth}{\linethickness}\end{center}

\subsection{\texorpdfstring{\texttt{normal\_area}}{normal\_area}}\label{normal_area}

\subsubsection{Possible use:}\label{possible-use-377}

\begin{itemize}
\tightlist
\item
  \textbf{\texttt{normal\_area}} (\texttt{float}, \texttt{float},
  \texttt{float}) ---\textgreater{} \texttt{float}
\end{itemize}

\subsubsection{Result:}\label{result-363}

Returns the area to the left of x in the normal distribution with the
given mean and standard deviation.

\begin{center}\rule{0.5\linewidth}{\linethickness}\end{center}

\subsection{\texorpdfstring{\texttt{normal\_density}}{normal\_density}}\label{normal_density}

\subsubsection{Possible use:}\label{possible-use-378}

\begin{itemize}
\tightlist
\item
  \textbf{\texttt{normal\_density}} (\texttt{float}, \texttt{float},
  \texttt{float}) ---\textgreater{} \texttt{float}
\end{itemize}

\subsubsection{Result:}\label{result-364}

Returns the probability of x in the normal distribution with the given
mean and standard deviation.

\begin{center}\rule{0.5\linewidth}{\linethickness}\end{center}

\subsection{\texorpdfstring{\texttt{normal\_inverse}}{normal\_inverse}}\label{normal_inverse}

\subsubsection{Possible use:}\label{possible-use-379}

\begin{itemize}
\tightlist
\item
  \textbf{\texttt{normal\_inverse}} (\texttt{float}, \texttt{float},
  \texttt{float}) ---\textgreater{} \texttt{float}
\end{itemize}

\subsubsection{Result:}\label{result-365}

Returns the x in the normal distribution with the given mean and
standard deviation, to the left of which lies the given area.
normal.Inverse returns the value in terms of standard deviations from
the mean, so we need to adjust it for the given mean and standard
deviation.

\begin{center}\rule{0.5\linewidth}{\linethickness}\end{center}

\subsection{\texorpdfstring{\texttt{not}}{not}}\label{not}

Same signification as \href{OperatorsAA\#!}{!}

\begin{center}\rule{0.5\linewidth}{\linethickness}\end{center}

\subsection{\texorpdfstring{\texttt{obj\_file}}{obj\_file}}\label{obj_file}

\subsubsection{Possible use:}\label{possible-use-380}

\begin{itemize}
\tightlist
\item
  \textbf{\texttt{obj\_file}} (\texttt{string}) ---\textgreater{}
  \texttt{file}
\end{itemize}

\subsubsection{Result:}\label{result-366}

Constructs a file of type obj. Allowed extensions are limited to obj,
OBJ

\begin{center}\rule{0.5\linewidth}{\linethickness}\end{center}

\subsection{\texorpdfstring{\texttt{of}}{of}}\label{of}

Same signification as \href{OperatorsAA\#.}{.}

\begin{center}\rule{0.5\linewidth}{\linethickness}\end{center}

\subsection{\texorpdfstring{\texttt{of\_generic\_species}}{of\_generic\_species}}\label{of_generic_species}

\subsubsection{Possible use:}\label{possible-use-381}

\begin{itemize}
\tightlist
\item
  \texttt{container} \textbf{\texttt{of\_generic\_species}}
  \texttt{species} ---\textgreater{} \texttt{list}
\item
  \textbf{\texttt{of\_generic\_species}} (\texttt{container} ,
  \texttt{species}) ---\textgreater{} \texttt{list}
\end{itemize}

\subsubsection{Result:}\label{result-367}

a list, containing the agents of the left-hand operand whose species is
that denoted by the right-hand operand and whose species extends the
right-hand operand species

\subsubsection{Examples:}\label{examples-261}

\begin{verbatim}
// species test {} // species sous_test parent: test {}  
list var2 <- [sous_test(0),sous_test(1),test(2),test(3)] of_generic_species test; // var2 equals [sous_test0,sous_test1,test2,test3] 
list var3 <- [sous_test(0),sous_test(1),test(2),test(3)] of_generic_species sous_test; // var3 equals [sous_test0,sous_test1] 
list var4 <- [sous_test(0),sous_test(1),test(2),test(3)] of_species test; // var4 equals [test2,test3] 
list var5 <- [sous_test(0),sous_test(1),test(2),test(3)] of_species sous_test; // var5 equals [sous_test0,sous_test1]
\end{verbatim}

\subsubsection{See also:}\label{see-also-149}

\href{OperatorsNR\#of_species}{of\_species},

\begin{center}\rule{0.5\linewidth}{\linethickness}\end{center}

\subsection{\texorpdfstring{\texttt{of\_species}}{of\_species}}\label{of_species}

\subsubsection{Possible use:}\label{possible-use-382}

\begin{itemize}
\tightlist
\item
  \texttt{container} \textbf{\texttt{of\_species}} \texttt{species}
  ---\textgreater{} \texttt{list}
\item
  \textbf{\texttt{of\_species}} (\texttt{container} , \texttt{species})
  ---\textgreater{} \texttt{list}
\end{itemize}

\subsubsection{Result:}\label{result-368}

a list, containing the agents of the left-hand operand whose species is
the one denoted by the right-hand operand.The expression agents
of\_species (species self) is equivalent to agents where (species each =
species self); however, the advantage of using the first syntax is that
the resulting list is correctly typed with the right species, whereas,
in the second syntax, the parser cannot determine the species of the
agents within the list (resulting in the need to cast it explicitly if
it is to be used in an ask statement, for instance).

\subsubsection{Special cases:}\label{special-cases-105}

\begin{itemize}
\tightlist
\item
  if the right operand is nil, of\_species returns the right operand
\end{itemize}

\subsubsection{Examples:}\label{examples-262}

\begin{verbatim}
 
list var0 <- (self neighbors_at 10) of_species (species (self)); // var0 equals all the neighboring agents of the same species. 
list var1 <- [test(0),test(1),node(1),node(2)] of_species test; // var1 equals [test0,test1]
\end{verbatim}

\subsubsection{See also:}\label{see-also-150}

\href{OperatorsNR\#of_generic_species}{of\_generic\_species},

\begin{center}\rule{0.5\linewidth}{\linethickness}\end{center}

\subsection{\texorpdfstring{\texttt{one\_of}}{one\_of}}\label{one_of}

\subsubsection{Possible use:}\label{possible-use-383}

\begin{itemize}
\tightlist
\item
  \textbf{\texttt{one\_of}}
  (\texttt{container\textless{}KeyType,ValueType\textgreater{}})
  ---\textgreater{} \texttt{ValueType}
\end{itemize}

\subsubsection{Result:}\label{result-369}

one of the values stored in this container at a random key

\subsubsection{Comment:}\label{comment-74}

the one\_of operator behavior depends on the nature of the operand

\subsubsection{Special cases:}\label{special-cases-106}

\begin{itemize}
\tightlist
\item
  if it is a graph, one\_of returns one of the lists of edges\\
\item
  if it is a file, one\_of returns one of the elements of the content of
  the file (that is also a container)\\
\item
  if the operand is empty, one\_of returns nil
\end{itemize}

\begin{verbatim}
\end{verbatim}

\begin{itemize}
\tightlist
\item
  if it is a list or a matrix, one\_of returns one of the values of the
  list or of the matrix
\end{itemize}

\begin{verbatim}
int 
i <- any ([1,2,3]); //i equals 1, 2 or 3string sMat <- one_of(matrix([["c11","c12","c13"],["c21","c22","c23"]]));   // sMat equals "c11","c12","c13", "c21","c22" or "c23" 
\end{verbatim}

\begin{itemize}
\tightlist
\item
  if it is a map, one\_of returns one the value of a random pair of the
  map
\end{itemize}

\begin{verbatim}
int im <- one_of ([2::3, 4::5, 6::7]);  // im equals 3, 5 or 7  
bool var6 <- [2::3, 4::5, 6::7].values contains im; // var6 equals true
\end{verbatim}

\begin{itemize}
\tightlist
\item
  if it is a population, one\_of returns one of the agents of the
  population
\end{itemize}

\begin{verbatim}
bug b <- one_of(bug);   // Given a previously defined species bug, b is one of the created bugs, e.g. bug3 
\end{verbatim}

\subsubsection{See also:}\label{see-also-151}

\href{OperatorsBC\#contains}{contains},

\begin{center}\rule{0.5\linewidth}{\linethickness}\end{center}

\subsection{\texorpdfstring{\texttt{open\_simplex\_generator}}{open\_simplex\_generator}}\label{open_simplex_generator}

\subsubsection{Possible use:}\label{possible-use-384}

\begin{itemize}
\tightlist
\item
  \textbf{\texttt{open\_simplex\_generator}} (\texttt{float},
  \texttt{float}, \texttt{float}) ---\textgreater{} \texttt{float}
\end{itemize}

\subsubsection{Result:}\label{result-370}

take a x, y and a bias parameters and gives a value

\subsubsection{Examples:}\label{examples-263}

\begin{verbatim}
 
float var0 <- open_simplex_generator(2,3,253); // var0 equals 10.2
\end{verbatim}

\begin{center}\rule{0.5\linewidth}{\linethickness}\end{center}

\subsection{\texorpdfstring{\texttt{or}}{or}}\label{or}

\subsubsection{Possible use:}\label{possible-use-385}

\begin{itemize}
\tightlist
\item
  \texttt{bool} \textbf{\texttt{or}} \texttt{any\ expression}
  ---\textgreater{} \texttt{bool}
\item
  \textbf{\texttt{or}} (\texttt{bool} , \texttt{any\ expression})
  ---\textgreater{} \texttt{bool}
\end{itemize}

\subsubsection{Result:}\label{result-371}

a bool value, equal to the logical or between the left-hand operand and
the right-hand operand.

\subsubsection{Comment:}\label{comment-75}

both operands are always casted to bool before applying the operator.
Thus, an expression like 1 or 0 is accepted and returns true.

\subsubsection{See also:}\label{see-also-152}

\href{OperatorsBC\#bool}{bool}, \href{OperatorsAA\#and}{and},
\href{OperatorsAA\#!}{!},

\begin{center}\rule{0.5\linewidth}{\linethickness}\end{center}

\subsection{\texorpdfstring{\texttt{or}}{or}}\label{or-1}

\subsubsection{Possible use:}\label{possible-use-386}

\begin{itemize}
\tightlist
\item
  \texttt{predicate} \textbf{\texttt{or}} \texttt{predicate}
  ---\textgreater{} \texttt{predicate}
\item
  \textbf{\texttt{or}} (\texttt{predicate} , \texttt{predicate})
  ---\textgreater{} \texttt{predicate}
\end{itemize}

\subsubsection{Result:}\label{result-372}

create a new predicate from two others by including them as
subintentions. It's an exclusive ``or''

\subsubsection{Examples:}\label{examples-264}

\begin{verbatim}
predicate1 or predicate2 
\end{verbatim}

\begin{center}\rule{0.5\linewidth}{\linethickness}\end{center}

\subsection{\texorpdfstring{\texttt{osm\_file}}{osm\_file}}\label{osm_file}

\subsubsection{Possible use:}\label{possible-use-387}

\begin{itemize}
\tightlist
\item
  \texttt{string} \textbf{\texttt{osm\_file}}
  \texttt{map\textless{}string,list\textgreater{}} ---\textgreater{}
  \texttt{file}
\item
  \textbf{\texttt{osm\_file}} (\texttt{string} ,
  \texttt{map\textless{}string,list\textgreater{}}) ---\textgreater{}
  \texttt{file}
\item
  \textbf{\texttt{osm\_file}} (\texttt{string},
  \texttt{map\textless{}string,list\textgreater{}}, \texttt{int})
  ---\textgreater{} \texttt{file}
\end{itemize}

\subsubsection{Result:}\label{result-373}

opens a file that a is a kind of OSM file with some filtering, forcing
the initial CRS to be the one indicated by the second int parameter (see
\url{http://spatialreference.org/ref/epsg/}). If this int parameter is
equal to 0, the data is considered as already projected. opens a file
that a is a kind of OSM file with some filtering.

\subsubsection{Comment:}\label{comment-76}

The file should have a OSM file extension, cf.~file type definition for
supported file extensions.The file should have a OSM file extension,
cf.~file type definition for supported file extensions.

\subsubsection{Special cases:}\label{special-cases-107}

\begin{itemize}
\tightlist
\item
  If the specified string does not refer to an existing OSM file, an
  exception is risen.\\
\item
  If the specified string does not refer to an existing OSM file, an
  exception is risen.
\end{itemize}

\subsubsection{Examples:}\label{examples-265}

\begin{verbatim}
file myOSMfile2 <- osm_file("../includes/rouen.osm",["highway"::["primary","motorway"]], 0); file myOSMfile <- osm_file("../includes/rouen.osm", ["highway"::["primary","motorway"]]); 
\end{verbatim}

\subsubsection{See also:}\label{see-also-153}

\href{OperatorsDH\#file}{file},

\begin{center}\rule{0.5\linewidth}{\linethickness}\end{center}

\subsection{\texorpdfstring{\texttt{out\_degree\_of}}{out\_degree\_of}}\label{out_degree_of}

\subsubsection{Possible use:}\label{possible-use-388}

\begin{itemize}
\tightlist
\item
  \texttt{graph} \textbf{\texttt{out\_degree\_of}} \texttt{unknown}
  ---\textgreater{} \texttt{int}
\item
  \textbf{\texttt{out\_degree\_of}} (\texttt{graph} , \texttt{unknown})
  ---\textgreater{} \texttt{int}
\end{itemize}

\subsubsection{Result:}\label{result-374}

returns the out degree of a vertex (right-hand operand) in the graph
given as left-hand operand.

\subsubsection{Examples:}\label{examples-266}

\begin{verbatim}
 
int var1 <- graphFromMap out_degree_of (node(3)); // var1 equals 4
\end{verbatim}

\subsubsection{See also:}\label{see-also-154}

\href{OperatorsIM\#in_degree_of}{in\_degree\_of},
\href{OperatorsDH\#degree_of}{degree\_of},

\begin{center}\rule{0.5\linewidth}{\linethickness}\end{center}

\subsection{\texorpdfstring{\texttt{out\_edges\_of}}{out\_edges\_of}}\label{out_edges_of}

\subsubsection{Possible use:}\label{possible-use-389}

\begin{itemize}
\tightlist
\item
  \texttt{graph} \textbf{\texttt{out\_edges\_of}} \texttt{unknown}
  ---\textgreater{} \texttt{list}
\item
  \textbf{\texttt{out\_edges\_of}} (\texttt{graph} , \texttt{unknown})
  ---\textgreater{} \texttt{list}
\end{itemize}

\subsubsection{Result:}\label{result-375}

returns the list of the out-edges of a vertex (right-hand operand) in
the graph given as left-hand operand.

\subsubsection{Examples:}\label{examples-267}

\begin{verbatim}
 
list var1 <- graphFromMap out_edges_of (node(3)); // var1 equals 3
\end{verbatim}

\subsubsection{See also:}\label{see-also-155}

\href{OperatorsIM\#in_edges_of}{in\_edges\_of},

\begin{center}\rule{0.5\linewidth}{\linethickness}\end{center}

\subsection{\texorpdfstring{\texttt{overlapping}}{overlapping}}\label{overlapping}

\subsubsection{Possible use:}\label{possible-use-390}

\begin{itemize}
\tightlist
\item
  \texttt{container\textless{}agent\textgreater{}}
  \textbf{\texttt{overlapping}} \texttt{geometry} ---\textgreater{}
  \texttt{list\textless{}geometry\textgreater{}}
\item
  \textbf{\texttt{overlapping}}
  (\texttt{container\textless{}agent\textgreater{}} , \texttt{geometry})
  ---\textgreater{} \texttt{list\textless{}geometry\textgreater{}}
\end{itemize}

\subsubsection{Result:}\label{result-376}

A list of agents or geometries among the left-operand list, species or
meta-population (addition of species), overlapping the operand (casted
as a geometry).

\subsubsection{Examples:}\label{examples-268}

\begin{verbatim}
 
list<geometry> var0 <- [ag1, ag2, ag3] overlapping(self); // var0 equals return the agents among ag1, ag2 and ag3 that overlap the shape of the agent applying the operator.(species1 + species2) overlapping self 
\end{verbatim}

\subsubsection{See also:}\label{see-also-156}

\href{OperatorsNR\#neighbors_at}{neighbors\_at},
\href{OperatorsNR\#neighbors_of}{neighbors\_of},
\href{OperatorsAA\#agent_closest_to}{agent\_closest\_to},
\href{OperatorsAA\#agents_inside}{agents\_inside},
\href{OperatorsBC\#closest_to}{closest\_to},
\href{OperatorsIM\#inside}{inside},
\href{OperatorsAA\#agents_overlapping}{agents\_overlapping},

\begin{center}\rule{0.5\linewidth}{\linethickness}\end{center}

\subsection{\texorpdfstring{\texttt{overlaps}}{overlaps}}\label{overlaps}

\subsubsection{Possible use:}\label{possible-use-391}

\begin{itemize}
\tightlist
\item
  \texttt{geometry} \textbf{\texttt{overlaps}} \texttt{geometry}
  ---\textgreater{} \texttt{bool}
\item
  \textbf{\texttt{overlaps}} (\texttt{geometry} , \texttt{geometry})
  ---\textgreater{} \texttt{bool}
\end{itemize}

\subsubsection{Result:}\label{result-377}

A boolean, equal to true if the left-geometry (or agent/point) overlaps
the right-geometry (or agent/point).

\subsubsection{Special cases:}\label{special-cases-108}

\begin{itemize}
\tightlist
\item
  if one of the operand is null, returns false.\\
\item
  if one operand is a point, returns true if the point is included in
  the geometry
\end{itemize}

\subsubsection{Examples:}\label{examples-269}

\begin{verbatim}
 
bool var0 <- polyline([{10,10},{20,20}]) overlaps polyline([{15,15},{25,25}]); // var0 equals true 
bool var1 <- polygon([{10,10},{10,20},{20,20},{20,10}]) overlaps polygon([{15,15},{15,25},{25,25},{25,15}]); // var1 equals true 
bool var2 <- polygon([{10,10},{10,20},{20,20},{20,10}]) overlaps {25,25}; // var2 equals false 
bool var3 <- polygon([{10,10},{10,20},{20,20},{20,10}]) overlaps polygon([{35,35},{35,45},{45,45},{45,35}]); // var3 equals false 
bool var4 <- polygon([{10,10},{10,20},{20,20},{20,10}]) overlaps polyline([{10,10},{20,20}]); // var4 equals true 
bool var5 <- polygon([{10,10},{10,20},{20,20},{20,10}]) overlaps {15,15}; // var5 equals true 
bool var6 <- polygon([{10,10},{10,20},{20,20},{20,10}]) overlaps polygon([{0,0},{0,30},{30,30}, {30,0}]); // var6 equals true 
bool var7 <- polygon([{10,10},{10,20},{20,20},{20,10}]) overlaps polygon([{15,15},{15,25},{25,25},{25,15}]); // var7 equals true 
bool var8 <- polygon([{10,10},{10,20},{20,20},{20,10}]) overlaps polygon([{10,20},{20,20},{20,30},{10,30}]); // var8 equals true
\end{verbatim}

\subsubsection{See also:}\label{see-also-157}

\href{OperatorsDH\#disjoint_from}{disjoint\_from},
\href{OperatorsBC\#crosses}{crosses},
\href{OperatorsIM\#intersects}{intersects},
\href{OperatorsNR\#partially_overlaps}{partially\_overlaps},
\href{OperatorsSZ\#touches}{touches},

\begin{center}\rule{0.5\linewidth}{\linethickness}\end{center}

\subsection{\texorpdfstring{\texttt{pair}}{pair}}\label{pair}

\subsubsection{Possible use:}\label{possible-use-392}

\begin{itemize}
\tightlist
\item
  \textbf{\texttt{pair}} (\texttt{any}) ---\textgreater{} \texttt{pair}
\end{itemize}

\subsubsection{Result:}\label{result-378}

Casts the operand into the type pair

\begin{center}\rule{0.5\linewidth}{\linethickness}\end{center}

\subsection{\texorpdfstring{\texttt{partially\_overlaps}}{partially\_overlaps}}\label{partially_overlaps}

\subsubsection{Possible use:}\label{possible-use-393}

\begin{itemize}
\tightlist
\item
  \texttt{geometry} \textbf{\texttt{partially\_overlaps}}
  \texttt{geometry} ---\textgreater{} \texttt{bool}
\item
  \textbf{\texttt{partially\_overlaps}} (\texttt{geometry} ,
  \texttt{geometry}) ---\textgreater{} \texttt{bool}
\end{itemize}

\subsubsection{Result:}\label{result-379}

A boolean, equal to true if the left-geometry (or agent/point) partially
overlaps the right-geometry (or agent/point).

\subsubsection{Comment:}\label{comment-77}

if one geometry operand fully covers the other geometry operand, returns
false (contrarily to the overlaps operator).

\subsubsection{Special cases:}\label{special-cases-109}

\begin{itemize}
\tightlist
\item
  if one of the operand is null, returns false.
\end{itemize}

\subsubsection{Examples:}\label{examples-270}

\begin{verbatim}
 
bool var0 <- polyline([{10,10},{20,20}]) partially_overlaps polyline([{15,15},{25,25}]); // var0 equals true 
bool var1 <- polygon([{10,10},{10,20},{20,20},{20,10}]) partially_overlaps polygon([{15,15},{15,25},{25,25},{25,15}]); // var1 equals true 
bool var2 <- polygon([{10,10},{10,20},{20,20},{20,10}]) partially_overlaps {25,25}; // var2 equals false 
bool var3 <- polygon([{10,10},{10,20},{20,20},{20,10}]) partially_overlaps polygon([{35,35},{35,45},{45,45},{45,35}]); // var3 equals false 
bool var4 <- polygon([{10,10},{10,20},{20,20},{20,10}]) partially_overlaps polyline([{10,10},{20,20}]); // var4 equals false 
bool var5 <- polygon([{10,10},{10,20},{20,20},{20,10}]) partially_overlaps {15,15}; // var5 equals false 
bool var6 <- polygon([{10,10},{10,20},{20,20},{20,10}]) partially_overlaps polygon([{0,0},{0,30},{30,30}, {30,0}]); // var6 equals false 
bool var7 <- polygon([{10,10},{10,20},{20,20},{20,10}]) partially_overlaps polygon([{15,15},{15,25},{25,25},{25,15}]); // var7 equals true 
bool var8 <- polygon([{10,10},{10,20},{20,20},{20,10}]) partially_overlaps polygon([{10,20},{20,20},{20,30},{10,30}]); // var8 equals false
\end{verbatim}

\subsubsection{See also:}\label{see-also-158}

\href{OperatorsDH\#disjoint_from}{disjoint\_from},
\href{OperatorsBC\#crosses}{crosses},
\href{OperatorsNR\#overlaps}{overlaps},
\href{OperatorsIM\#intersects}{intersects},
\href{OperatorsSZ\#touches}{touches},

\begin{center}\rule{0.5\linewidth}{\linethickness}\end{center}

\subsection{\texorpdfstring{\texttt{path}}{path}}\label{path}

\subsubsection{Possible use:}\label{possible-use-394}

\begin{itemize}
\tightlist
\item
  \textbf{\texttt{path}} (\texttt{any}) ---\textgreater{} \texttt{path}
\end{itemize}

\subsubsection{Result:}\label{result-380}

Casts the operand into the type path

\begin{center}\rule{0.5\linewidth}{\linethickness}\end{center}

\subsection{\texorpdfstring{\texttt{path\_between}}{path\_between}}\label{path_between}

\subsubsection{Possible use:}\label{possible-use-395}

\begin{itemize}
\tightlist
\item
  \texttt{list\textless{}agent\textgreater{}}
  \textbf{\texttt{path\_between}}
  \texttt{container\textless{}geometry\textgreater{}} ---\textgreater{}
  \texttt{path}
\item
  \textbf{\texttt{path\_between}}
  (\texttt{list\textless{}agent\textgreater{}} ,
  \texttt{container\textless{}geometry\textgreater{}}) ---\textgreater{}
  \texttt{path}
\item
  \texttt{topology} \textbf{\texttt{path\_between}}
  \texttt{container\textless{}geometry\textgreater{}} ---\textgreater{}
  \texttt{path}
\item
  \textbf{\texttt{path\_between}} (\texttt{topology} ,
  \texttt{container\textless{}geometry\textgreater{}}) ---\textgreater{}
  \texttt{path}
\item
  \texttt{msi.gama.util.GamaMap\textless{}msi.gama.metamodel.agent.IAgent,java.lang.Object\textgreater{}}
  \textbf{\texttt{path\_between}}
  \texttt{container\textless{}geometry\textgreater{}} ---\textgreater{}
  \texttt{path}
\item
  \textbf{\texttt{path\_between}}
  (\texttt{msi.gama.util.GamaMap\textless{}msi.gama.metamodel.agent.IAgent,java.lang.Object\textgreater{}}
  , \texttt{container\textless{}geometry\textgreater{}})
  ---\textgreater{} \texttt{path}
\item
  \textbf{\texttt{path\_between}} (\texttt{topology}, \texttt{geometry},
  \texttt{geometry}) ---\textgreater{} \texttt{path}
\item
  \textbf{\texttt{path\_between}}
  (\texttt{msi.gama.util.GamaMap\textless{}msi.gama.metamodel.agent.IAgent,java.lang.Object\textgreater{}},
  \texttt{geometry}, \texttt{geometry}) ---\textgreater{} \texttt{path}
\item
  \textbf{\texttt{path\_between}}
  (\texttt{list\textless{}agent\textgreater{}}, \texttt{geometry},
  \texttt{geometry}) ---\textgreater{} \texttt{path}
\item
  \textbf{\texttt{path\_between}} (\texttt{graph}, \texttt{geometry},
  \texttt{geometry}) ---\textgreater{} \texttt{path}
\end{itemize}

\subsubsection{Result:}\label{result-381}

The shortest path between several objects according to set of cells The
shortest path between two objects according to set of cells with
corresponding weights The shortest path between two objects according to
set of cells The shortest path between several objects according to set
of cells with corresponding weights The shortest path between a list of
two objects in a graph

\subsubsection{Examples:}\label{examples-271}

\begin{verbatim}
 
path var0 <- path_between (cell_grid where each.is_free, [ag1, ag2, ag3]); // var0 equals A path between ag1 and ag2 and ag3 passing through the given cell_grid agents 
path var1 <- my_topology path_between (ag1, ag2); // var1 equals A path between ag1 and ag2 
path var2 <- path_between (cell_grid as_map (each::each.is_obstacle ? 9999.0 : 1.0), ag1, ag2); // var2 equals A path between ag1 and ag2 passing through the given cell_grid agents with a minimal cost 
path var3 <- path_between (cell_grid where each.is_free, ag1, ag2); // var3 equals A path between ag1 and ag2 passing through the given cell_grid agents 
path var4 <- my_topology path_between [ag1, ag2]; // var4 equals A path between ag1 and ag2 
path var5 <- path_between (cell_grid as_map (each::each.is_obstacle ? 9999.0 : 1.0), [ag1, ag2, ag3]); // var5 equals A path between ag1 and ag2 and ag3 passing through the given cell_grid agents with minimal cost 
path var6 <- path_between (my_graph, ag1, ag2); // var6 equals A path between ag1 and ag2
\end{verbatim}

\subsubsection{See also:}\label{see-also-159}

\href{OperatorsSZ\#towards}{towards},
\href{OperatorsDH\#direction_to}{direction\_to},
\href{OperatorsDH\#distance_between}{distance\_between},
\href{OperatorsDH\#direction_between}{direction\_between},
\href{OperatorsNR\#path_to}{path\_to},
\href{OperatorsDH\#distance_to}{distance\_to},

\begin{center}\rule{0.5\linewidth}{\linethickness}\end{center}

\subsection{\texorpdfstring{\texttt{path\_to}}{path\_to}}\label{path_to}

\subsubsection{Possible use:}\label{possible-use-396}

\begin{itemize}
\tightlist
\item
  \texttt{geometry} \textbf{\texttt{path\_to}} \texttt{geometry}
  ---\textgreater{} \texttt{path}
\item
  \textbf{\texttt{path\_to}} (\texttt{geometry} , \texttt{geometry})
  ---\textgreater{} \texttt{path}
\item
  \texttt{point} \textbf{\texttt{path\_to}} \texttt{point}
  ---\textgreater{} \texttt{path}
\item
  \textbf{\texttt{path\_to}} (\texttt{point} , \texttt{point})
  ---\textgreater{} \texttt{path}
\end{itemize}

\subsubsection{Result:}\label{result-382}

A path between two geometries (geometries, agents or points) considering
the topology of the agent applying the operator.

\subsubsection{Examples:}\label{examples-272}

\begin{verbatim}
 
path var0 <- ag1 path_to ag2; // var0 equals the path between ag1 and ag2 considering the topology of the agent applying the operator
\end{verbatim}

\subsubsection{See also:}\label{see-also-160}

\href{OperatorsSZ\#towards}{towards},
\href{OperatorsDH\#direction_to}{direction\_to},
\href{OperatorsDH\#distance_between}{distance\_between},
\href{OperatorsDH\#direction_between}{direction\_between},
\href{OperatorsNR\#path_between}{path\_between},
\href{OperatorsDH\#distance_to}{distance\_to},

\begin{center}\rule{0.5\linewidth}{\linethickness}\end{center}

\subsection{\texorpdfstring{\texttt{paths\_between}}{paths\_between}}\label{paths_between}

\subsubsection{Possible use:}\label{possible-use-397}

\begin{itemize}
\tightlist
\item
  \textbf{\texttt{paths\_between}} (\texttt{graph}, \texttt{pair},
  \texttt{int}) ---\textgreater{}
  \texttt{msi.gama.util.IList\textless{}msi.gama.util.path.GamaSpatialPath\textgreater{}}
\end{itemize}

\subsubsection{Result:}\label{result-383}

The K shortest paths between a list of two objects in a graph

\subsubsection{Examples:}\label{examples-273}

\begin{verbatim}
 
msi.gama.util.IList<msi.gama.util.path.GamaSpatialPath> var0 <- paths_between(my_graph, ag1:: ag2, 2); // var0 equals the 2 shortest paths (ordered by length) between ag1 and ag2
\end{verbatim}

\begin{center}\rule{0.5\linewidth}{\linethickness}\end{center}

\subsection{\texorpdfstring{\texttt{pbinom}}{pbinom}}\label{pbinom}

Same signification as \href{OperatorsBC\#binomial_sum}{binomial\_sum}

\begin{center}\rule{0.5\linewidth}{\linethickness}\end{center}

\subsection{\texorpdfstring{\texttt{pchisq}}{pchisq}}\label{pchisq}

Same signification as \href{OperatorsBC\#chi_square}{chi\_square}

\begin{center}\rule{0.5\linewidth}{\linethickness}\end{center}

\subsection{\texorpdfstring{\texttt{percent\_absolute\_deviation}}{percent\_absolute\_deviation}}\label{percent_absolute_deviation}

\subsubsection{Possible use:}\label{possible-use-398}

\begin{itemize}
\tightlist
\item
  \texttt{list\textless{}float\textgreater{}}
  \textbf{\texttt{percent\_absolute\_deviation}}
  \texttt{list\textless{}float\textgreater{}} ---\textgreater{}
  \texttt{float}
\item
  \textbf{\texttt{percent\_absolute\_deviation}}
  (\texttt{list\textless{}float\textgreater{}} ,
  \texttt{list\textless{}float\textgreater{}}) ---\textgreater{}
  \texttt{float}
\end{itemize}

\subsubsection{Result:}\label{result-384}

percent absolute deviation indicator for 2 series of values:
percent\_absolute\_deviation(list\_vals\_observe,list\_vals\_sim)

\subsubsection{Examples:}\label{examples-274}

\begin{verbatim}
percent_absolute_deviation([200,300,150,150,200],[250,250,100,200,200]) 
\end{verbatim}

\begin{center}\rule{0.5\linewidth}{\linethickness}\end{center}

\subsection{\texorpdfstring{\texttt{percentile}}{percentile}}\label{percentile}

Same signification as
\href{OperatorsNR\#quantile_inverse}{quantile\_inverse}

\begin{center}\rule{0.5\linewidth}{\linethickness}\end{center}

\subsection{\texorpdfstring{\texttt{pgamma}}{pgamma}}\label{pgamma}

Same signification as
\href{OperatorsDH\#gamma_distribution}{gamma\_distribution}

\begin{center}\rule{0.5\linewidth}{\linethickness}\end{center}

\subsection{\texorpdfstring{\texttt{pgm\_file}}{pgm\_file}}\label{pgm_file}

\subsubsection{Possible use:}\label{possible-use-399}

\begin{itemize}
\tightlist
\item
  \textbf{\texttt{pgm\_file}} (\texttt{string}) ---\textgreater{}
  \texttt{file}
\end{itemize}

\subsubsection{Result:}\label{result-385}

Constructs a file of type pgm. Allowed extensions are limited to pgm

\begin{center}\rule{0.5\linewidth}{\linethickness}\end{center}

\subsection{\texorpdfstring{\texttt{plan}}{plan}}\label{plan}

\subsubsection{Possible use:}\label{possible-use-400}

\begin{itemize}
\tightlist
\item
  \texttt{container\textless{}geometry\textgreater{}}
  \textbf{\texttt{plan}} \texttt{float} ---\textgreater{}
  \texttt{geometry}
\item
  \textbf{\texttt{plan}}
  (\texttt{container\textless{}geometry\textgreater{}} , \texttt{float})
  ---\textgreater{} \texttt{geometry}
\end{itemize}

\subsubsection{Result:}\label{result-386}

A polyline geometry from the given list of points.

\subsubsection{Special cases:}\label{special-cases-110}

\begin{itemize}
\tightlist
\item
  if the operand is nil, returns the point geometry \{0,0\}\\
\item
  if the operand is composed of a single point, returns a point
  geometry.
\end{itemize}

\subsubsection{Examples:}\label{examples-275}

\begin{verbatim}
 
geometry var0 <- polyplan([{0,0}, {0,10}, {10,10}, {10,0}],10); // var0 equals a polyline geometry composed of the 4 points with a depth of 10.
\end{verbatim}

\subsubsection{See also:}\label{see-also-161}

\href{OperatorsAA\#around}{around}, \href{OperatorsBC\#circle}{circle},
\href{OperatorsBC\#cone}{cone}, \href{OperatorsIM\#link}{link},
\href{OperatorsNR\#norm}{norm}, \href{OperatorsNR\#point}{point},
\href{OperatorsSZ\#polygone}{polygone},
\href{OperatorsNR\#rectangle}{rectangle},
\href{OperatorsSZ\#square}{square},
\href{OperatorsSZ\#triangle}{triangle},

\begin{center}\rule{0.5\linewidth}{\linethickness}\end{center}

\subsection{\texorpdfstring{\texttt{plus\_days}}{plus\_days}}\label{plus_days}

\subsubsection{Possible use:}\label{possible-use-401}

\begin{itemize}
\tightlist
\item
  \texttt{date} \textbf{\texttt{plus\_days}} \texttt{int}
  ---\textgreater{} \texttt{date}
\item
  \textbf{\texttt{plus\_days}} (\texttt{date} , \texttt{int})
  ---\textgreater{} \texttt{date}
\end{itemize}

\subsubsection{Result:}\label{result-387}

Add a given number of days to a date

\subsubsection{Examples:}\label{examples-276}

\begin{verbatim}
 
date var0 <- date('2000-01-01') plus_days 12; // var0 equals date('2000-01-13')
\end{verbatim}

\begin{center}\rule{0.5\linewidth}{\linethickness}\end{center}

\subsection{\texorpdfstring{\texttt{plus\_hours}}{plus\_hours}}\label{plus_hours}

\subsubsection{Possible use:}\label{possible-use-402}

\begin{itemize}
\tightlist
\item
  \texttt{date} \textbf{\texttt{plus\_hours}} \texttt{int}
  ---\textgreater{} \texttt{date}
\item
  \textbf{\texttt{plus\_hours}} (\texttt{date} , \texttt{int})
  ---\textgreater{} \texttt{date}
\end{itemize}

\subsubsection{Result:}\label{result-388}

Add a given number of hours to a date

\subsubsection{Examples:}\label{examples-277}

\begin{verbatim}
// equivalent to date1 + 15 #h  
date var1 <- date('2000-01-01') plus_hours 24; // var1 equals date('2000-01-02')
\end{verbatim}

\begin{center}\rule{0.5\linewidth}{\linethickness}\end{center}

\subsection{\texorpdfstring{\texttt{plus\_minutes}}{plus\_minutes}}\label{plus_minutes}

\subsubsection{Possible use:}\label{possible-use-403}

\begin{itemize}
\tightlist
\item
  \texttt{date} \textbf{\texttt{plus\_minutes}} \texttt{int}
  ---\textgreater{} \texttt{date}
\item
  \textbf{\texttt{plus\_minutes}} (\texttt{date} , \texttt{int})
  ---\textgreater{} \texttt{date}
\end{itemize}

\subsubsection{Result:}\label{result-389}

Add a given number of minutes to a date

\subsubsection{Examples:}\label{examples-278}

\begin{verbatim}
// equivalent to date1 + 5 #mn  
date var1 <- date('2000-01-01') plus_minutes 5 ; // var1 equals date('2000-01-01 00:05:00')
\end{verbatim}

\begin{center}\rule{0.5\linewidth}{\linethickness}\end{center}

\subsection{\texorpdfstring{\texttt{plus\_months}}{plus\_months}}\label{plus_months}

\subsubsection{Possible use:}\label{possible-use-404}

\begin{itemize}
\tightlist
\item
  \texttt{date} \textbf{\texttt{plus\_months}} \texttt{int}
  ---\textgreater{} \texttt{date}
\item
  \textbf{\texttt{plus\_months}} (\texttt{date} , \texttt{int})
  ---\textgreater{} \texttt{date}
\end{itemize}

\subsubsection{Result:}\label{result-390}

Add a given number of months to a date

\subsubsection{Examples:}\label{examples-279}

\begin{verbatim}
 
date var0 <- date('2000-01-01') plus_months 5; // var0 equals date('2000-06-01')
\end{verbatim}

\begin{center}\rule{0.5\linewidth}{\linethickness}\end{center}

\subsection{\texorpdfstring{\texttt{plus\_ms}}{plus\_ms}}\label{plus_ms}

\subsubsection{Possible use:}\label{possible-use-405}

\begin{itemize}
\tightlist
\item
  \texttt{date} \textbf{\texttt{plus\_ms}} \texttt{int}
  ---\textgreater{} \texttt{date}
\item
  \textbf{\texttt{plus\_ms}} (\texttt{date} , \texttt{int})
  ---\textgreater{} \texttt{date}
\end{itemize}

\subsubsection{Result:}\label{result-391}

Add a given number of milliseconds to a date

\subsubsection{Examples:}\label{examples-280}

\begin{verbatim}
// equivalent to date('2000-01-01') + 15 #ms  
date var1 <- date('2000-01-01') plus_ms 1000 ; // var1 equals date('2000-01-01 00:00:01')
\end{verbatim}

\begin{center}\rule{0.5\linewidth}{\linethickness}\end{center}

\subsection{\texorpdfstring{\texttt{plus\_seconds}}{plus\_seconds}}\label{plus_seconds}

Same signification as \href{OperatorsAA\#+}{+}

\begin{center}\rule{0.5\linewidth}{\linethickness}\end{center}

\subsection{\texorpdfstring{\texttt{plus\_weeks}}{plus\_weeks}}\label{plus_weeks}

\subsubsection{Possible use:}\label{possible-use-406}

\begin{itemize}
\tightlist
\item
  \texttt{date} \textbf{\texttt{plus\_weeks}} \texttt{int}
  ---\textgreater{} \texttt{date}
\item
  \textbf{\texttt{plus\_weeks}} (\texttt{date} , \texttt{int})
  ---\textgreater{} \texttt{date}
\end{itemize}

\subsubsection{Result:}\label{result-392}

Add a given number of weeks to a date

\subsubsection{Examples:}\label{examples-281}

\begin{verbatim}
 
date var0 <- date('2000-01-01') plus_weeks 15; // var0 equals date('2000-04-15')
\end{verbatim}

\begin{center}\rule{0.5\linewidth}{\linethickness}\end{center}

\subsection{\texorpdfstring{\texttt{plus\_years}}{plus\_years}}\label{plus_years}

\subsubsection{Possible use:}\label{possible-use-407}

\begin{itemize}
\tightlist
\item
  \texttt{date} \textbf{\texttt{plus\_years}} \texttt{int}
  ---\textgreater{} \texttt{date}
\item
  \textbf{\texttt{plus\_years}} (\texttt{date} , \texttt{int})
  ---\textgreater{} \texttt{date}
\end{itemize}

\subsubsection{Result:}\label{result-393}

Add a given number of years to a date

\subsubsection{Examples:}\label{examples-282}

\begin{verbatim}
 
date var0 <- date('2000-01-01') plus_years 15; // var0 equals date('2015-01-01')
\end{verbatim}

\begin{center}\rule{0.5\linewidth}{\linethickness}\end{center}

\subsection{\texorpdfstring{\texttt{pnorm}}{pnorm}}\label{pnorm}

Same signification as \href{OperatorsNR\#normal_area}{normal\_area}

\begin{center}\rule{0.5\linewidth}{\linethickness}\end{center}

\subsection{\texorpdfstring{\texttt{point}}{point}}\label{point}

\subsubsection{Possible use:}\label{possible-use-408}

\begin{itemize}
\tightlist
\item
  \texttt{float} \textbf{\texttt{point}} \texttt{float}
  ---\textgreater{} \texttt{point}
\item
  \textbf{\texttt{point}} (\texttt{float} , \texttt{float})
  ---\textgreater{} \texttt{point}
\item
  \texttt{int} \textbf{\texttt{point}} \texttt{int} ---\textgreater{}
  \texttt{point}
\item
  \textbf{\texttt{point}} (\texttt{int} , \texttt{int})
  ---\textgreater{} \texttt{point}
\item
  \texttt{float} \textbf{\texttt{point}} \texttt{int} ---\textgreater{}
  \texttt{point}
\item
  \textbf{\texttt{point}} (\texttt{float} , \texttt{int})
  ---\textgreater{} \texttt{point}
\item
  \texttt{int} \textbf{\texttt{point}} \texttt{float} ---\textgreater{}
  \texttt{point}
\item
  \textbf{\texttt{point}} (\texttt{int} , \texttt{float})
  ---\textgreater{} \texttt{point}
\item
  \textbf{\texttt{point}} (\texttt{float}, \texttt{int}, \texttt{float})
  ---\textgreater{} \texttt{point}
\item
  \textbf{\texttt{point}} (\texttt{float}, \texttt{float},
  \texttt{float}) ---\textgreater{} \texttt{point}
\item
  \textbf{\texttt{point}} (\texttt{int}, \texttt{int}, \texttt{int})
  ---\textgreater{} \texttt{point}
\item
  \textbf{\texttt{point}} (\texttt{int}, \texttt{float}, \texttt{float})
  ---\textgreater{} \texttt{point}
\item
  \textbf{\texttt{point}} (\texttt{int}, \texttt{int}, \texttt{float})
  ---\textgreater{} \texttt{point}
\item
  \textbf{\texttt{point}} (\texttt{float}, \texttt{float}, \texttt{int})
  ---\textgreater{} \texttt{point}
\item
  \textbf{\texttt{point}} (\texttt{float}, \texttt{int}, \texttt{int})
  ---\textgreater{} \texttt{point}
\end{itemize}

\subsubsection{Result:}\label{result-394}

internal use only. Use the standard construction \{x,y, z\} instead.
internal use only. Use the standard construction \{x,y\} instead.
internal use only. Use the standard construction \{x,y, z\} instead.
internal use only. Use the standard construction \{x,y, z\} instead.
internal use only. Use the standard construction \{x,y\} instead.
internal use only. Use the standard construction \{x,y\} instead.
internal use only. Use the standard construction \{x,y\} instead.
internal use only. Use the standard construction \{x,y, z\} instead.
internal use only. Use the standard construction \{x,y, z\} instead.
internal use only. Use the standard construction \{x,y, z\} instead.
internal use only. Use the standard construction \{x,y, z\} instead.

\begin{center}\rule{0.5\linewidth}{\linethickness}\end{center}

\subsection{\texorpdfstring{\texttt{points\_along}}{points\_along}}\label{points_along}

\subsubsection{Possible use:}\label{possible-use-409}

\begin{itemize}
\tightlist
\item
  \texttt{geometry} \textbf{\texttt{points\_along}}
  \texttt{list\textless{}float\textgreater{}} ---\textgreater{}
  \texttt{list}
\item
  \textbf{\texttt{points\_along}} (\texttt{geometry} ,
  \texttt{list\textless{}float\textgreater{}}) ---\textgreater{}
  \texttt{list}
\end{itemize}

\subsubsection{Result:}\label{result-395}

A list of points along the operand-geometry given its location in terms
of rate of distance from the starting points of the geometry.

\subsubsection{Examples:}\label{examples-283}

\begin{verbatim}
 
list var0 <-  line([{10,10},{80,80}]) points_along ([0.3, 0.5, 0.9]); // var0 equals the list of following points: [{31.0,31.0,0.0},{45.0,45.0,0.0},{73.0,73.0,0.0}]
\end{verbatim}

\subsubsection{See also:}\label{see-also-162}

\href{OperatorsBC\#closest_points_with}{closest\_points\_with},
\href{OperatorsDH\#farthest_point_to}{farthest\_point\_to},
\href{OperatorsNR\#points_at}{points\_at},
\href{OperatorsNR\#points_on}{points\_on},

\begin{center}\rule{0.5\linewidth}{\linethickness}\end{center}

\subsection{\texorpdfstring{\texttt{points\_at}}{points\_at}}\label{points_at}

\subsubsection{Possible use:}\label{possible-use-410}

\begin{itemize}
\tightlist
\item
  \texttt{int} \textbf{\texttt{points\_at}} \texttt{float}
  ---\textgreater{} \texttt{list\textless{}point\textgreater{}}
\item
  \textbf{\texttt{points\_at}} (\texttt{int} , \texttt{float})
  ---\textgreater{} \texttt{list\textless{}point\textgreater{}}
\end{itemize}

\subsubsection{Result:}\label{result-396}

A list of left-operand number of points located at a the right-operand
distance to the agent location.

\subsubsection{Examples:}\label{examples-284}

\begin{verbatim}
 
list<point> var0 <- 3 points_at(20.0); // var0 equals returns [pt1, pt2, pt3] with pt1, pt2 and pt3 located at a distance of 20.0 to the agent location
\end{verbatim}

\subsubsection{See also:}\label{see-also-163}

\href{OperatorsAA\#any_location_in}{any\_location\_in},
\href{OperatorsAA\#any_point_in}{any\_point\_in},
\href{OperatorsBC\#closest_points_with}{closest\_points\_with},
\href{OperatorsDH\#farthest_point_to}{farthest\_point\_to},

\begin{center}\rule{0.5\linewidth}{\linethickness}\end{center}

\subsection{\texorpdfstring{\texttt{points\_on}}{points\_on}}\label{points_on}

\subsubsection{Possible use:}\label{possible-use-411}

\begin{itemize}
\tightlist
\item
  \texttt{geometry} \textbf{\texttt{points\_on}} \texttt{float}
  ---\textgreater{} \texttt{list}
\item
  \textbf{\texttt{points\_on}} (\texttt{geometry} , \texttt{float})
  ---\textgreater{} \texttt{list}
\end{itemize}

\subsubsection{Result:}\label{result-397}

A list of points of the operand-geometry distant from each other to the
float right-operand .

\subsubsection{Examples:}\label{examples-285}

\begin{verbatim}
 
list var0 <-  square(5) points_on(2); // var0 equals a list of points belonging to the exterior ring of the square distant from each other of 2.
\end{verbatim}

\subsubsection{See also:}\label{see-also-164}

\href{OperatorsBC\#closest_points_with}{closest\_points\_with},
\href{OperatorsDH\#farthest_point_to}{farthest\_point\_to},
\href{OperatorsNR\#points_at}{points\_at},

\begin{center}\rule{0.5\linewidth}{\linethickness}\end{center}

\subsection{\texorpdfstring{\texttt{poisson}}{poisson}}\label{poisson}

\subsubsection{Possible use:}\label{possible-use-412}

\begin{itemize}
\tightlist
\item
  \textbf{\texttt{poisson}} (\texttt{float}) ---\textgreater{}
  \texttt{int}
\end{itemize}

\subsubsection{Result:}\label{result-398}

A value from a random variable following a Poisson distribution (with
the positive expected number of occurence lambda as operand).

\subsubsection{Comment:}\label{comment-78}

The Poisson distribution is a discrete probability distribution that
expresses the probability of a given number of events occurring in a
fixed interval of time and/or space if these events occur with a known
average rate and independently of the time since the last event,
cf.~Poisson distribution on Wikipedia.

\subsubsection{Examples:}\label{examples-286}

\begin{verbatim}
 
int var0 <- poisson(3.5); // var0 equals a random positive integer
\end{verbatim}

\subsubsection{See also:}\label{see-also-165}

\href{OperatorsBC\#binomial}{binomial},
\href{OperatorsDH\#gauss}{gauss},

\begin{center}\rule{0.5\linewidth}{\linethickness}\end{center}

\subsection{\texorpdfstring{\texttt{polygon}}{polygon}}\label{polygon}

\subsubsection{Possible use:}\label{possible-use-413}

\begin{itemize}
\tightlist
\item
  \textbf{\texttt{polygon}}
  (\texttt{container\textless{}agent\textgreater{}}) ---\textgreater{}
  \texttt{geometry}
\end{itemize}

\subsubsection{Result:}\label{result-399}

A polygon geometry from the given list of points.

\subsubsection{Special cases:}\label{special-cases-111}

\begin{itemize}
\tightlist
\item
  if the operand is nil, returns the point geometry \{0,0\}\\
\item
  if the operand is composed of a single point, returns a point
  geometry\\
\item
  if the operand is composed of 2 points, returns a polyline geometry.
\end{itemize}

\subsubsection{Examples:}\label{examples-287}

\begin{verbatim}
 
geometry var0 <- polygon([{0,0}, {0,10}, {10,10}, {10,0}]); // var0 equals a polygon geometry composed of the 4 points.
\end{verbatim}

\subsubsection{See also:}\label{see-also-166}

\href{OperatorsAA\#around}{around}, \href{OperatorsBC\#circle}{circle},
\href{OperatorsBC\#cone}{cone}, \href{OperatorsIM\#line}{line},
\href{OperatorsIM\#link}{link}, \href{OperatorsNR\#norm}{norm},
\href{OperatorsNR\#point}{point},
\href{OperatorsNR\#polyline}{polyline},
\href{OperatorsNR\#rectangle}{rectangle},
\href{OperatorsSZ\#square}{square},
\href{OperatorsSZ\#triangle}{triangle},

\begin{center}\rule{0.5\linewidth}{\linethickness}\end{center}

\subsection{\texorpdfstring{\texttt{polyhedron}}{polyhedron}}\label{polyhedron}

\subsubsection{Possible use:}\label{possible-use-414}

\begin{itemize}
\tightlist
\item
  \texttt{container\textless{}geometry\textgreater{}}
  \textbf{\texttt{polyhedron}} \texttt{float} ---\textgreater{}
  \texttt{geometry}
\item
  \textbf{\texttt{polyhedron}}
  (\texttt{container\textless{}geometry\textgreater{}} , \texttt{float})
  ---\textgreater{} \texttt{geometry}
\end{itemize}

\subsubsection{Result:}\label{result-400}

A polyhedron geometry from the given list of points.

\subsubsection{Special cases:}\label{special-cases-112}

\begin{itemize}
\tightlist
\item
  if the operand is nil, returns the point geometry \{0,0\}\\
\item
  if the operand is composed of a single point, returns a point
  geometry\\
\item
  if the operand is composed of 2 points, returns a polyline geometry.
\end{itemize}

\subsubsection{Examples:}\label{examples-288}

\begin{verbatim}
 
geometry var0 <- polyhedron([{0,0}, {0,10}, {10,10}, {10,0}],10); // var0 equals a polygon geometry composed of the 4 points and of depth 10.
\end{verbatim}

\subsubsection{See also:}\label{see-also-167}

\href{OperatorsAA\#around}{around}, \href{OperatorsBC\#circle}{circle},
\href{OperatorsBC\#cone}{cone}, \href{OperatorsIM\#line}{line},
\href{OperatorsIM\#link}{link}, \href{OperatorsNR\#norm}{norm},
\href{OperatorsNR\#point}{point},
\href{OperatorsNR\#polyline}{polyline},
\href{OperatorsNR\#rectangle}{rectangle},
\href{OperatorsSZ\#square}{square},
\href{OperatorsSZ\#triangle}{triangle},

\begin{center}\rule{0.5\linewidth}{\linethickness}\end{center}

\subsection{\texorpdfstring{\texttt{polyline}}{polyline}}\label{polyline}

Same signification as \href{OperatorsIM\#line}{line}

\begin{center}\rule{0.5\linewidth}{\linethickness}\end{center}

\subsection{\texorpdfstring{\texttt{polyplan}}{polyplan}}\label{polyplan}

Same signification as \href{OperatorsNR\#plan}{plan}

\begin{center}\rule{0.5\linewidth}{\linethickness}\end{center}

\subsection{\texorpdfstring{\texttt{predecessors\_of}}{predecessors\_of}}\label{predecessors_of}

\subsubsection{Possible use:}\label{possible-use-415}

\begin{itemize}
\tightlist
\item
  \texttt{graph} \textbf{\texttt{predecessors\_of}} \texttt{unknown}
  ---\textgreater{} \texttt{list}
\item
  \textbf{\texttt{predecessors\_of}} (\texttt{graph} , \texttt{unknown})
  ---\textgreater{} \texttt{list}
\end{itemize}

\subsubsection{Result:}\label{result-401}

returns the list of predecessors (i.e.~sources of in edges) of the given
vertex (right-hand operand) in the given graph (left-hand operand)

\subsubsection{Examples:}\label{examples-289}

\begin{verbatim}
 
list var1 <- graphEpidemio predecessors_of ({1,5}); // var1 equals [] 
list var2 <- graphEpidemio predecessors_of node({34,56}); // var2 equals [{12;45}]
\end{verbatim}

\subsubsection{See also:}\label{see-also-168}

\href{OperatorsNR\#neighbors_of}{neighbors\_of},
\href{OperatorsSZ\#successors_of}{successors\_of},

\begin{center}\rule{0.5\linewidth}{\linethickness}\end{center}

\subsection{\texorpdfstring{\texttt{predicate}}{predicate}}\label{predicate}

\subsubsection{Possible use:}\label{possible-use-416}

\begin{itemize}
\tightlist
\item
  \textbf{\texttt{predicate}} (\texttt{any}) ---\textgreater{}
  \texttt{predicate}
\end{itemize}

\subsubsection{Result:}\label{result-402}

Casts the operand into the type predicate

\begin{center}\rule{0.5\linewidth}{\linethickness}\end{center}

\subsection{\texorpdfstring{\texttt{predict}}{predict}}\label{predict}

\subsubsection{Possible use:}\label{possible-use-417}

\begin{itemize}
\tightlist
\item
  \texttt{regression} \textbf{\texttt{predict}}
  \texttt{list\textless{}float\textgreater{}} ---\textgreater{}
  \texttt{float}
\item
  \textbf{\texttt{predict}} (\texttt{regression} ,
  \texttt{list\textless{}float\textgreater{}}) ---\textgreater{}
  \texttt{float}
\end{itemize}

\subsubsection{Result:}\label{result-403}

returns the value predict by the regression parameters for a given
instance. Usage: predict(regression, instance)

\subsubsection{Examples:}\label{examples-290}

\begin{verbatim}
predict(my_regression, [1,2,3]) 
\end{verbatim}

\begin{center}\rule{0.5\linewidth}{\linethickness}\end{center}

\subsection{\texorpdfstring{\texttt{product}}{product}}\label{product}

Same signification as \href{OperatorsIM\#mul}{mul}

\begin{center}\rule{0.5\linewidth}{\linethickness}\end{center}

\subsection{\texorpdfstring{\texttt{product\_of}}{product\_of}}\label{product_of}

\subsubsection{Possible use:}\label{possible-use-418}

\begin{itemize}
\tightlist
\item
  \texttt{container} \textbf{\texttt{product\_of}}
  \texttt{any\ expression} ---\textgreater{} \texttt{unknown}
\item
  \textbf{\texttt{product\_of}} (\texttt{container} ,
  \texttt{any\ expression}) ---\textgreater{} \texttt{unknown}
\end{itemize}

\subsubsection{Result:}\label{result-404}

the product of the right-hand expression evaluated on each of the
elements of the left-hand operand

\subsubsection{Comment:}\label{comment-79}

in the right-hand operand, the keyword each can be used to represent, in
turn, each of the right-hand operand elements.

\subsubsection{Special cases:}\label{special-cases-113}

\begin{itemize}
\tightlist
\item
  if the left-operand is a map, the keyword each will contain each value
\end{itemize}

\begin{verbatim}
 
unknown var1 <- [1::2, 3::4, 5::6] product_of (each); // var1 equals 48
\end{verbatim}

\subsubsection{Examples:}\label{examples-291}

\begin{verbatim}
 
unknown var0 <- [1,2] product_of (each * 10 ); // var0 equals 200
\end{verbatim}

\subsubsection{See also:}\label{see-also-169}

\href{OperatorsIM\#min_of}{min\_of},
\href{OperatorsIM\#max_of}{max\_of},
\href{OperatorsSZ\#sum_of}{sum\_of},
\href{OperatorsIM\#mean_of}{mean\_of},

\begin{center}\rule{0.5\linewidth}{\linethickness}\end{center}

\subsection{\texorpdfstring{\texttt{promethee\_DM}}{promethee\_DM}}\label{promethee_dm}

\subsubsection{Possible use:}\label{possible-use-419}

\begin{itemize}
\tightlist
\item
  \texttt{msi.gama.util.IList\textless{}java.util.List\textgreater{}}
  \textbf{\texttt{promethee\_DM}}
  \texttt{msi.gama.util.IList\textless{}java.util.Map\textless{}java.lang.String,java.lang.Object\textgreater{}\textgreater{}}
  ---\textgreater{} \texttt{int}
\item
  \textbf{\texttt{promethee\_DM}}
  (\texttt{msi.gama.util.IList\textless{}java.util.List\textgreater{}} ,
  \texttt{msi.gama.util.IList\textless{}java.util.Map\textless{}java.lang.String,java.lang.Object\textgreater{}\textgreater{}})
  ---\textgreater{} \texttt{int}
\end{itemize}

\subsubsection{Result:}\label{result-405}

The index of the best candidate according to the Promethee II method.
This method is based on a comparison per pair of possible candidates
along each criterion: all candidates are compared to each other by pair
and ranked. More information about this method can be found in
{[}\url{http://www.sciencedirect.com/science?_ob=ArticleURL\&_udi=B6VCT-4VF56TV-1\&_user=10\&_coverDate=01\%2F01\%2F2010\&_rdoc=1\&_fmt=high\&_orig=search\&_sort=d\&_docanchor}=\&view=c\&\_searchStrId=1389284642\&\_rerunOrigin=google\&\_acct=C000050221\&\_version=1\&\_urlVersion=0\&\_userid=10\&md5=d334de2a4e0d6281199a39857648cd36
Behzadian, M., Kazemzadeh, R., Albadvi, A., M., A.: PROMETHEE: A
comprehensive literature review on methodologies and applications.
European Journal of Operational Research(2009){]}. The first operand is
the list of candidates (a candidate is a list of criterion values); the
second operand the list of criterion: A criterion is a map that contains
fours elements: a name, a weight, a preference value (p) and an
indifference value (q). The preference value represents the threshold
from which the difference between two criterion values allows to prefer
one vector of values over another. The indifference value represents the
threshold from which the difference between two criterion values is
considered significant.

\subsubsection{Special cases:}\label{special-cases-114}

\begin{itemize}
\tightlist
\item
  returns -1 is the list of candidates is nil or empty
\end{itemize}

\subsubsection{Examples:}\label{examples-292}

\begin{verbatim}
 
int var0 <- promethee_DM([[1.0, 7.0],[4.0,2.0],[3.0, 3.0]], [["name"::"utility", "weight" :: 2.0,"p"::0.5, "q"::0.0, "s"::1.0, "maximize" :: true],["name"::"price", "weight" :: 1.0,"p"::0.5, "q"::0.0, "s"::1.0, "maximize" :: false]]); // var0 equals 1
\end{verbatim}

\subsubsection{See also:}\label{see-also-170}

\href{OperatorsSZ\#weighted_means_dm}{weighted\_means\_DM},
\href{OperatorsDH\#electre_dm}{electre\_DM},
\href{OperatorsDH\#evidence_theory_dm}{evidence\_theory\_DM},

\begin{center}\rule{0.5\linewidth}{\linethickness}\end{center}

\subsection{\texorpdfstring{\texttt{property\_file}}{property\_file}}\label{property_file}

\subsubsection{Possible use:}\label{possible-use-420}

\begin{itemize}
\tightlist
\item
  \textbf{\texttt{property\_file}} (\texttt{string}) ---\textgreater{}
  \texttt{file}
\end{itemize}

\subsubsection{Result:}\label{result-406}

Constructs a file of type property. Allowed extensions are limited to
properties

\begin{center}\rule{0.5\linewidth}{\linethickness}\end{center}

\subsection{\texorpdfstring{\texttt{pValue\_for\_fStat}}{pValue\_for\_fStat}}\label{pvalue_for_fstat}

\subsubsection{Possible use:}\label{possible-use-421}

\begin{itemize}
\tightlist
\item
  \textbf{\texttt{pValue\_for\_fStat}} (\texttt{float}, \texttt{int},
  \texttt{int}) ---\textgreater{} \texttt{float}
\end{itemize}

\subsubsection{Result:}\label{result-407}

Returns the P value of F statistic fstat with numerator degrees of
freedom dfn and denominator degress of freedom dfd. Uses the incomplete
Beta function.

\begin{center}\rule{0.5\linewidth}{\linethickness}\end{center}

\subsection{\texorpdfstring{\texttt{pValue\_for\_tStat}}{pValue\_for\_tStat}}\label{pvalue_for_tstat}

\subsubsection{Possible use:}\label{possible-use-422}

\begin{itemize}
\tightlist
\item
  \texttt{float} \textbf{\texttt{pValue\_for\_tStat}} \texttt{int}
  ---\textgreater{} \texttt{float}
\item
  \textbf{\texttt{pValue\_for\_tStat}} (\texttt{float} , \texttt{int})
  ---\textgreater{} \texttt{float}
\end{itemize}

\subsubsection{Result:}\label{result-408}

Returns the P value of the T statistic tstat with df degrees of freedom.
This is a two-tailed test so we just double the right tail which is
given by studentT of -\textbar{}tstat\textbar{}.

\begin{center}\rule{0.5\linewidth}{\linethickness}\end{center}

\subsection{\texorpdfstring{\texttt{pyramid}}{pyramid}}\label{pyramid}

\subsubsection{Possible use:}\label{possible-use-423}

\begin{itemize}
\tightlist
\item
  \textbf{\texttt{pyramid}} (\texttt{float}) ---\textgreater{}
  \texttt{geometry}
\end{itemize}

\subsubsection{Result:}\label{result-409}

A square geometry which side size is given by the operand.

\subsubsection{Comment:}\label{comment-80}

the center of the pyramid is by default the location of the current
agent in which has been called this operator.

\subsubsection{Special cases:}\label{special-cases-115}

\begin{itemize}
\tightlist
\item
  returns nil if the operand is nil.
\end{itemize}

\subsubsection{Examples:}\label{examples-293}

\begin{verbatim}
 
geometry var0 <- pyramid(5); // var0 equals a geometry as a square with side_size = 5.
\end{verbatim}

\subsubsection{See also:}\label{see-also-171}

\href{OperatorsAA\#around}{around}, \href{OperatorsBC\#circle}{circle},
\href{OperatorsBC\#cone}{cone}, \href{OperatorsIM\#line}{line},
\href{OperatorsIM\#link}{link}, \href{OperatorsNR\#norm}{norm},
\href{OperatorsNR\#point}{point}, \href{OperatorsNR\#polygon}{polygon},
\href{OperatorsNR\#polyline}{polyline},
\href{OperatorsNR\#rectangle}{rectangle},
\href{OperatorsSZ\#square}{square},

\begin{center}\rule{0.5\linewidth}{\linethickness}\end{center}

\subsection{\texorpdfstring{\texttt{quantile}}{quantile}}\label{quantile}

\subsubsection{Possible use:}\label{possible-use-424}

\begin{itemize}
\tightlist
\item
  \texttt{container} \textbf{\texttt{quantile}} \texttt{float}
  ---\textgreater{} \texttt{float}
\item
  \textbf{\texttt{quantile}} (\texttt{container} , \texttt{float})
  ---\textgreater{} \texttt{float}
\end{itemize}

\subsubsection{Result:}\label{result-410}

Returns the phi-quantile; that is, an element elem for which holds that
phi percent of data elements are less than elem. The quantile need not
necessarily be contained in the data sequence, it can be a linear
interpolation.

\begin{center}\rule{0.5\linewidth}{\linethickness}\end{center}

\subsection{\texorpdfstring{\texttt{quantile\_inverse}}{quantile\_inverse}}\label{quantile_inverse}

\subsubsection{Possible use:}\label{possible-use-425}

\begin{itemize}
\tightlist
\item
  \texttt{container} \textbf{\texttt{quantile\_inverse}} \texttt{float}
  ---\textgreater{} \texttt{float}
\item
  \textbf{\texttt{quantile\_inverse}} (\texttt{container} ,
  \texttt{float}) ---\textgreater{} \texttt{float}
\end{itemize}

\subsubsection{Result:}\label{result-411}

Returns how many percent of the elements contained in the receiver are
\textless{}= element. Does linear interpolation if the element is not
contained but lies in between two contained elements.

\begin{center}\rule{0.5\linewidth}{\linethickness}\end{center}

\subsection{\texorpdfstring{\texttt{R\_correlation}}{R\_correlation}}\label{r_correlation}

Same signification as \href{OperatorsBC\#corR}{corR}

\begin{center}\rule{0.5\linewidth}{\linethickness}\end{center}

\subsection{\texorpdfstring{\texttt{R\_file}}{R\_file}}\label{r_file}

\subsubsection{Possible use:}\label{possible-use-426}

\begin{itemize}
\tightlist
\item
  \textbf{\texttt{R\_file}} (\texttt{string}) ---\textgreater{}
  \texttt{file}
\end{itemize}

\subsubsection{Result:}\label{result-412}

Constructs a file of type R. Allowed extensions are limited to r

\begin{center}\rule{0.5\linewidth}{\linethickness}\end{center}

\subsection{\texorpdfstring{\texttt{R\_mean}}{R\_mean}}\label{r_mean}

Same signification as \href{OperatorsIM\#meanR}{meanR}

\begin{center}\rule{0.5\linewidth}{\linethickness}\end{center}

\subsection{\texorpdfstring{\texttt{range}}{range}}\label{range}

\subsubsection{Possible use:}\label{possible-use-427}

\begin{itemize}
\tightlist
\item
  \textbf{\texttt{range}} (\texttt{int}) ---\textgreater{} \texttt{list}
\item
  \texttt{int} \textbf{\texttt{range}} \texttt{int} ---\textgreater{}
  \texttt{list}
\item
  \textbf{\texttt{range}} (\texttt{int} , \texttt{int})
  ---\textgreater{} \texttt{list}
\item
  \textbf{\texttt{range}} (\texttt{int}, \texttt{int}, \texttt{int})
  ---\textgreater{} \texttt{list}
\end{itemize}

\subsubsection{Result:}\label{result-413}

Allows to build a list of int representing all contiguous values from
the first to the second argument. The range can be increasing or
decreasing. Passing the same value for both will return a singleton list
with this value Allows to build a list of int representing all
contiguous values from zero to the argument. The range can be increasing
or decreasing. Passing 0 will return a singleton list with 0 Allows to
build a list of int representing all contiguous values from the first to
the second argument, using the step represented by the third argument.
The range can be increasing or decreasing. Passing the same value for
both will return a singleton list with this value. Passing a step of 0
will result in an exception. Attempting to build infinite ranges
(e.g.~end \textgreater{} start with a negative step) will similarly not
be accepted and yield an exception

\begin{center}\rule{0.5\linewidth}{\linethickness}\end{center}

\subsection{\texorpdfstring{\texttt{rank\_interpolated}}{rank\_interpolated}}\label{rank_interpolated}

\subsubsection{Possible use:}\label{possible-use-428}

\begin{itemize}
\tightlist
\item
  \texttt{container} \textbf{\texttt{rank\_interpolated}} \texttt{float}
  ---\textgreater{} \texttt{float}
\item
  \textbf{\texttt{rank\_interpolated}} (\texttt{container} ,
  \texttt{float}) ---\textgreater{} \texttt{float}
\end{itemize}

\subsubsection{Result:}\label{result-414}

Returns the linearly interpolated number of elements in a list less or
equal to a given element. The rank is the number of elements
\textless{}= element. Ranks are of the form \{0, 1, 2,\ldots{},
sortedList.size()\}. If no element is \textless{}= element, then the
rank is zero. If the element lies in between two contained elements,
then linear interpolation is used and a non integer value is returned.

\begin{center}\rule{0.5\linewidth}{\linethickness}\end{center}

\subsection{\texorpdfstring{\texttt{read}}{read}}\label{read}

\subsubsection{Possible use:}\label{possible-use-429}

\begin{itemize}
\tightlist
\item
  \textbf{\texttt{read}} (\texttt{string}) ---\textgreater{}
  \texttt{unknown}
\end{itemize}

\subsubsection{Result:}\label{result-415}

Reads an attribute of the agent. The attribute's name is specified by
the operand.

\subsubsection{Examples:}\label{examples-294}

\begin{verbatim}
unknown 
agent_name <- read ('name'); //agent_name equals reads the 'name' variable of agent then assigns the returned value to the 'agent_name' variable. 
\end{verbatim}

\begin{center}\rule{0.5\linewidth}{\linethickness}\end{center}

\subsection{\texorpdfstring{\texttt{rectangle}}{rectangle}}\label{rectangle}

\subsubsection{Possible use:}\label{possible-use-430}

\begin{itemize}
\tightlist
\item
  \textbf{\texttt{rectangle}} (\texttt{point}) ---\textgreater{}
  \texttt{geometry}
\item
  \texttt{float} \textbf{\texttt{rectangle}} \texttt{float}
  ---\textgreater{} \texttt{geometry}
\item
  \textbf{\texttt{rectangle}} (\texttt{float} , \texttt{float})
  ---\textgreater{} \texttt{geometry}
\item
  \texttt{point} \textbf{\texttt{rectangle}} \texttt{point}
  ---\textgreater{} \texttt{geometry}
\item
  \textbf{\texttt{rectangle}} (\texttt{point} , \texttt{point})
  ---\textgreater{} \texttt{geometry}
\end{itemize}

\subsubsection{Result:}\label{result-416}

A rectangle geometry which side sizes are given by the operands.

\subsubsection{Comment:}\label{comment-81}

the center of the rectangle is by default the location of the current
agent in which has been called this operator.the center of the rectangle
is by default the location of the current agent in which has been called
this operator.

\subsubsection{Special cases:}\label{special-cases-116}

\begin{itemize}
\tightlist
\item
  returns nil if the operand is nil.\\
\item
  returns nil if the operand is nil.\\
\item
  returns nil if the operand is nil.
\end{itemize}

\subsubsection{Examples:}\label{examples-295}

\begin{verbatim}
 
geometry var0 <- rectangle(10, 5); // var0 equals a geometry as a rectangle with width = 10 and height = 5. 
geometry var1 <- rectangle({10, 5}); // var1 equals a geometry as a rectangle with width = 10 and height = 5. 
geometry var2 <- rectangle({2.0,6.0}, {6.0,20.0}); // var2 equals a geometry as a rectangle with {2.0,6.0} as the upper-left corner, {6.0,20.0} as the lower-right corner.
\end{verbatim}

\subsubsection{See also:}\label{see-also-172}

\href{OperatorsAA\#around}{around}, \href{OperatorsBC\#circle}{circle},
\href{OperatorsBC\#cone}{cone}, \href{OperatorsIM\#line}{line},
\href{OperatorsIM\#link}{link}, \href{OperatorsNR\#norm}{norm},
\href{OperatorsNR\#point}{point}, \href{OperatorsNR\#polygon}{polygon},
\href{OperatorsNR\#polyline}{polyline},
\href{OperatorsSZ\#square}{square},
\href{OperatorsSZ\#triangle}{triangle},

\begin{center}\rule{0.5\linewidth}{\linethickness}\end{center}

\subsection{\texorpdfstring{\texttt{reduced\_by}}{reduced\_by}}\label{reduced_by}

Same signification as \href{OperatorsAA\#-}{-}

\begin{center}\rule{0.5\linewidth}{\linethickness}\end{center}

\subsection{\texorpdfstring{\texttt{regression}}{regression}}\label{regression}

\subsubsection{Possible use:}\label{possible-use-431}

\begin{itemize}
\tightlist
\item
  \textbf{\texttt{regression}} (\texttt{any}) ---\textgreater{}
  \texttt{regression}
\end{itemize}

\subsubsection{Result:}\label{result-417}

Casts the operand into the type regression

\begin{center}\rule{0.5\linewidth}{\linethickness}\end{center}

\subsection{\texorpdfstring{\texttt{remove\_duplicates}}{remove\_duplicates}}\label{remove_duplicates}

Same signification as \href{OperatorsDH\#distinct}{distinct}

\begin{center}\rule{0.5\linewidth}{\linethickness}\end{center}

\subsection{\texorpdfstring{\texttt{remove\_node\_from}}{remove\_node\_from}}\label{remove_node_from}

\subsubsection{Possible use:}\label{possible-use-432}

\begin{itemize}
\tightlist
\item
  \texttt{geometry} \textbf{\texttt{remove\_node\_from}} \texttt{graph}
  ---\textgreater{} \texttt{graph}
\item
  \textbf{\texttt{remove\_node\_from}} (\texttt{geometry} ,
  \texttt{graph}) ---\textgreater{} \texttt{graph}
\end{itemize}

\subsubsection{Result:}\label{result-418}

removes a node from a graph.

\subsubsection{Comment:}\label{comment-82}

all the edges containing this node are also removed.

\subsubsection{Examples:}\label{examples-296}

\begin{verbatim}
 
graph var0 <- node(0) remove_node_from graphEpidemio; // var0 equals the graph without node(0)
\end{verbatim}

\begin{center}\rule{0.5\linewidth}{\linethickness}\end{center}

\subsection{\texorpdfstring{\texttt{replace}}{replace}}\label{replace}

\subsubsection{Possible use:}\label{possible-use-433}

\begin{itemize}
\tightlist
\item
  \textbf{\texttt{replace}} (\texttt{string}, \texttt{string},
  \texttt{string}) ---\textgreater{} \texttt{string}
\end{itemize}

\subsubsection{Result:}\label{result-419}

Returns the String resulting by replacing for the first operand all the
sub-strings corresponding the second operand by the third operand

\subsubsection{Examples:}\label{examples-297}

\begin{verbatim}
 
string var0 <- replace('to be or not to be,that is the question','to', 'do'); // var0 equals 'do be or not do be,that is the question'
\end{verbatim}

\subsubsection{See also:}\label{see-also-173}

\href{OperatorsNR\#replace_regex}{replace\_regex},

\begin{center}\rule{0.5\linewidth}{\linethickness}\end{center}

\subsection{\texorpdfstring{\texttt{replace\_regex}}{replace\_regex}}\label{replace_regex}

\subsubsection{Possible use:}\label{possible-use-434}

\begin{itemize}
\tightlist
\item
  \textbf{\texttt{replace\_regex}} (\texttt{string}, \texttt{string},
  \texttt{string}) ---\textgreater{} \texttt{string}
\end{itemize}

\subsubsection{Result:}\label{result-420}

Returns the String resulting by replacing for the first operand all the
sub-strings corresponding to the regular expression given in the second
operand by the third operand

\subsubsection{Examples:}\label{examples-298}

\begin{verbatim}
 
string var0 <- replace_regex("colour, color", "colou?r", "col"); // var0 equals 'col, col'
\end{verbatim}

\subsubsection{See also:}\label{see-also-174}

\href{OperatorsNR\#replace}{replace},

\begin{center}\rule{0.5\linewidth}{\linethickness}\end{center}

\subsection{\texorpdfstring{\texttt{reverse}}{reverse}}\label{reverse}

\subsubsection{Possible use:}\label{possible-use-435}

\begin{itemize}
\tightlist
\item
  \textbf{\texttt{reverse}}
  (\texttt{msi.gama.util.GamaMap\textless{}K,V\textgreater{}})
  ---\textgreater{} \texttt{container}
\item
  \textbf{\texttt{reverse}}
  (\texttt{container\textless{}KeyType,ValueType\textgreater{}})
  ---\textgreater{} \texttt{container}
\item
  \textbf{\texttt{reverse}} (\texttt{string}) ---\textgreater{}
  \texttt{string}
\end{itemize}

\subsubsection{Result:}\label{result-421}

the operand elements in the reversed order in a copy of the operand.

\subsubsection{Comment:}\label{comment-83}

the reverse operator behavior depends on the nature of the operand

\subsubsection{Special cases:}\label{special-cases-117}

\begin{itemize}
\tightlist
\item
  if it is a file, reverse returns a copy of the file with a reversed
  content\\
\item
  if it is a population, reverse returns a copy of the population with
  elements in the reversed order\\
\item
  if it is a graph, reverse returns a copy of the graph (with all edges
  and vertexes), with all of the edges reversed\\
\item
  if it is a list, reverse returns a copy of the operand list with
  elements in the reversed order
\end{itemize}

\begin{verbatim}
 
container var0 <- reverse ([10,12,14]); // var0 equals [14, 12, 10]
\end{verbatim}

\begin{itemize}
\tightlist
\item
  if it is a map, reverse returns a copy of the operand map with each
  pair in the reversed order (i.e.~all keys become values and values
  become keys)
\end{itemize}

\begin{verbatim}
 
map<int,string> var1 <- reverse (['k1'::44, 'k2'::32, 'k3'::12]); // var1 equals [44::'k1', 32::'k2', 12::'k3']
\end{verbatim}

\begin{itemize}
\tightlist
\item
  if it is a matrix, reverse returns a new matrix containing the
  transpose of the operand.
\end{itemize}

\begin{verbatim}
 
container var2 <- reverse(matrix([["c11","c12","c13"],["c21","c22","c23"]])); // var2 equals matrix([["c11","c21"],["c12","c22"],["c13","c23"]])
\end{verbatim}

\begin{itemize}
\tightlist
\item
  if it is a string, reverse returns a new string with characters in the
  reversed order
\end{itemize}

\begin{verbatim}
 
string var3 <- reverse ('abcd'); // var3 equals 'dcba'
\end{verbatim}

\begin{center}\rule{0.5\linewidth}{\linethickness}\end{center}

\subsection{\texorpdfstring{\texttt{rewire\_n}}{rewire\_n}}\label{rewire_n}

\subsubsection{Possible use:}\label{possible-use-436}

\begin{itemize}
\tightlist
\item
  \texttt{graph} \textbf{\texttt{rewire\_n}} \texttt{int}
  ---\textgreater{} \texttt{graph}
\item
  \textbf{\texttt{rewire\_n}} (\texttt{graph} , \texttt{int})
  ---\textgreater{} \texttt{graph}
\end{itemize}

\subsubsection{Result:}\label{result-422}

rewires the given count of edges.

\subsubsection{Comment:}\label{comment-84}

If there are too many edges, all the edges will be rewired.

\subsubsection{Examples:}\label{examples-299}

\begin{verbatim}
 
graph var1 <- graphEpidemio rewire_n 10; // var1 equals the graph with 3 edges rewired
\end{verbatim}

\begin{center}\rule{0.5\linewidth}{\linethickness}\end{center}

\subsection{\texorpdfstring{\texttt{rgb}}{rgb}}\label{rgb}

\subsubsection{Possible use:}\label{possible-use-437}

\begin{itemize}
\tightlist
\item
  \texttt{rgb} \textbf{\texttt{rgb}} \texttt{int} ---\textgreater{}
  \texttt{rgb}
\item
  \textbf{\texttt{rgb}} (\texttt{rgb} , \texttt{int}) ---\textgreater{}
  \texttt{rgb}
\item
  \texttt{rgb} \textbf{\texttt{rgb}} \texttt{float} ---\textgreater{}
  \texttt{rgb}
\item
  \textbf{\texttt{rgb}} (\texttt{rgb} , \texttt{float})
  ---\textgreater{} \texttt{rgb}
\item
  \texttt{string} \textbf{\texttt{rgb}} \texttt{int} ---\textgreater{}
  \texttt{rgb}
\item
  \textbf{\texttt{rgb}} (\texttt{string} , \texttt{int})
  ---\textgreater{} \texttt{rgb}
\item
  \textbf{\texttt{rgb}} (\texttt{int}, \texttt{int}, \texttt{int})
  ---\textgreater{} \texttt{rgb}
\item
  \textbf{\texttt{rgb}} (\texttt{int}, \texttt{int}, \texttt{int},
  \texttt{float}) ---\textgreater{} \texttt{rgb}
\item
  \textbf{\texttt{rgb}} (\texttt{int}, \texttt{int}, \texttt{int},
  \texttt{int}) ---\textgreater{} \texttt{rgb}
\end{itemize}

\subsubsection{Result:}\label{result-423}

Returns a color defined by red, green, blue components and an alpha
blending value.

\subsubsection{Special cases:}\label{special-cases-118}

\begin{itemize}
\tightlist
\item
  It can be used with r=red, g=green, b=blue (each between 0 and 255),
  a=alpha (between 0.0 and 1.0)\\
\item
  It can be used with r=red, g=green, b=blue (each between 0 and 255),
  a=alpha (between 0 and 255)\\
\item
  It can be used with r=red, g=green, b=blue, each between 0 and 255\\
\item
  It can be used with a color and an alpha between 0 and 255\\
\item
  It can be used with a color and an alpha between 0 and 1\\
\item
  It can be used with a name of color and alpha (between 0 and 255)
\end{itemize}

\subsubsection{Examples:}\label{examples-300}

\begin{verbatim}
 
rgb var0 <- rgb (255,0,0,0.5); // var0 equals a light red color 
rgb var1 <- rgb (255,0,0,125); // var1 equals a light red color 
rgb var3 <- rgb (255,0,0); // var3 equals #red 
rgb var4 <- rgb(rgb(255,0,0),125); // var4 equals a light red color 
rgb var5 <- rgb(rgb(255,0,0),0.5); // var5 equals a light red color 
rgb var6 <- rgb ("red"); // var6 equals rgb(255,0,0)
\end{verbatim}

\subsubsection{See also:}\label{see-also-175}

\href{OperatorsDH\#hsb}{hsb},

\begin{center}\rule{0.5\linewidth}{\linethickness}\end{center}

\subsection{\texorpdfstring{\texttt{rgb}}{rgb}}\label{rgb-1}

\subsubsection{Possible use:}\label{possible-use-438}

\begin{itemize}
\tightlist
\item
  \textbf{\texttt{rgb}} (\texttt{any}) ---\textgreater{} \texttt{rgb}
\end{itemize}

\subsubsection{Result:}\label{result-424}

Casts the operand into the type rgb

\begin{center}\rule{0.5\linewidth}{\linethickness}\end{center}

\subsection{\texorpdfstring{\texttt{rgb\_to\_xyz}}{rgb\_to\_xyz}}\label{rgb_to_xyz}

\subsubsection{Possible use:}\label{possible-use-439}

\begin{itemize}
\tightlist
\item
  \textbf{\texttt{rgb\_to\_xyz}} (\texttt{file}) ---\textgreater{}
  \texttt{list\textless{}point\textgreater{}}
\end{itemize}

\subsubsection{Result:}\label{result-425}

A list of point corresponding to RGB value of an image (r:x , g:y, b:z)

\subsubsection{Examples:}\label{examples-301}

\begin{verbatim}
 
list<point> var0 <- rgb_to_xyz(texture); // var0 equals a list of points
\end{verbatim}

\begin{center}\rule{0.5\linewidth}{\linethickness}\end{center}

\subsection{\texorpdfstring{\texttt{rms}}{rms}}\label{rms}

\subsubsection{Possible use:}\label{possible-use-440}

\begin{itemize}
\tightlist
\item
  \texttt{int} \textbf{\texttt{rms}} \texttt{float} ---\textgreater{}
  \texttt{float}
\item
  \textbf{\texttt{rms}} (\texttt{int} , \texttt{float})
  ---\textgreater{} \texttt{float}
\end{itemize}

\subsubsection{Result:}\label{result-426}

Returns the RMS (Root-Mean-Square) of a data sequence. The RMS of data
sequence is the square-root of the mean of the squares of the elements
in the data sequence. It is a measure of the average size of the
elements of a data sequence.

\begin{center}\rule{0.5\linewidth}{\linethickness}\end{center}

\subsection{\texorpdfstring{\texttt{rnd}}{rnd}}\label{rnd}

\subsubsection{Possible use:}\label{possible-use-441}

\begin{itemize}
\tightlist
\item
  \textbf{\texttt{rnd}} (\texttt{float}) ---\textgreater{}
  \texttt{float}
\item
  \textbf{\texttt{rnd}} (\texttt{int}) ---\textgreater{} \texttt{int}
\item
  \textbf{\texttt{rnd}} (\texttt{point}) ---\textgreater{}
  \texttt{point}
\item
  \texttt{float} \textbf{\texttt{rnd}} \texttt{float} ---\textgreater{}
  \texttt{float}
\item
  \textbf{\texttt{rnd}} (\texttt{float} , \texttt{float})
  ---\textgreater{} \texttt{float}
\item
  \texttt{point} \textbf{\texttt{rnd}} \texttt{point} ---\textgreater{}
  \texttt{point}
\item
  \textbf{\texttt{rnd}} (\texttt{point} , \texttt{point})
  ---\textgreater{} \texttt{point}
\item
  \texttt{int} \textbf{\texttt{rnd}} \texttt{int} ---\textgreater{}
  \texttt{int}
\item
  \textbf{\texttt{rnd}} (\texttt{int} , \texttt{int}) ---\textgreater{}
  \texttt{int}
\item
  \textbf{\texttt{rnd}} (\texttt{float}, \texttt{float}, \texttt{float})
  ---\textgreater{} \texttt{float}
\item
  \textbf{\texttt{rnd}} (\texttt{point}, \texttt{point}, \texttt{float})
  ---\textgreater{} \texttt{point}
\item
  \textbf{\texttt{rnd}} (\texttt{int}, \texttt{int}, \texttt{int})
  ---\textgreater{} \texttt{int}
\end{itemize}

\subsubsection{Result:}\label{result-427}

a random integer in the interval {[}0, operand{]}

\subsubsection{Comment:}\label{comment-85}

to obtain a probability between 0 and 1, use the expression (rnd n) / n,
where n is used to indicate the precision

\subsubsection{Special cases:}\label{special-cases-119}

\begin{itemize}
\tightlist
\item
  if the operand is a float, returns an uniformly distributed float
  random number in {[}0.0, to{]}\\
\item
  if the operand is a point, returns a point with three random float
  ordinates, each in the interval {[}0, ordinate of argument{]}
\end{itemize}

\subsubsection{Examples:}\label{examples-302}

\begin{verbatim}
 
float var0 <- rnd(3.4); // var0 equals a random float between 0.0 and 3.4 
float var1 <- rnd (2.0, 4.0); // var1 equals a float number between 2.0 and 4.0 
float var2 <- rnd (2.0, 4.0, 0.5); // var2 equals a float number between 2.0 and 4.0 every 0.5 
int var3 <- rnd (2); // var3 equals 0, 1 or 2 
float var4 <- rnd (1000) / 1000; // var4 equals a float between 0 and 1 with a precision of 0.001 
point var5 <- rnd ({2.5,3, 0.0}); // var5 equals {x,y} with x in [0.0,2.0], y in [0.0,3.0], z = 0.0 
point var6 <- rnd ({2.0, 4.0}, {2.0, 5.0, 10.0}); // var6 equals a point with x = 2.0, y between 2.0 and 4.0 and z between 0.0 and 10.0 
int var7 <- rnd (2, 4); // var7 equals 2, 3 or 4 
point var8 <- rnd ({2.0, 4.0}, {2.0, 5.0, 10.0}, 1); // var8 equals a point with x = 2.0, y equal to 2.0, 3.0 or 4.0 and z between 0.0 and 10.0 every 1.0 
int var9 <- rnd (2, 12, 4); // var9 equals 2, 6 or 10
\end{verbatim}

\subsubsection{See also:}\label{see-also-176}

\href{OperatorsDH\#flip}{flip},

\begin{center}\rule{0.5\linewidth}{\linethickness}\end{center}

\subsection{\texorpdfstring{\texttt{rnd\_choice}}{rnd\_choice}}\label{rnd_choice}

\subsubsection{Possible use:}\label{possible-use-442}

\begin{itemize}
\tightlist
\item
  \textbf{\texttt{rnd\_choice}} (\texttt{list}) ---\textgreater{}
  \texttt{int}
\end{itemize}

\subsubsection{Result:}\label{result-428}

returns an index of the given list with a probability following the
(normalized) distribution described in the list (a form of lottery)

\subsubsection{Examples:}\label{examples-303}

\begin{verbatim}
 
int var0 <- rnd_choice([0.2,0.5,0.3]); // var0 equals 2/10 chances to return 0, 5/10 chances to return 1, 3/10 chances to return 2
\end{verbatim}

\subsubsection{See also:}\label{see-also-177}

\href{OperatorsNR\#rnd}{rnd},

\begin{center}\rule{0.5\linewidth}{\linethickness}\end{center}

\subsection{\texorpdfstring{\texttt{rnd\_color}}{rnd\_color}}\label{rnd_color}

\subsubsection{Possible use:}\label{possible-use-443}

\begin{itemize}
\tightlist
\item
  \textbf{\texttt{rnd\_color}} (\texttt{int}) ---\textgreater{}
  \texttt{rgb}
\item
  \texttt{int} \textbf{\texttt{rnd\_color}} \texttt{int}
  ---\textgreater{} \texttt{rgb}
\item
  \textbf{\texttt{rnd\_color}} (\texttt{int} , \texttt{int})
  ---\textgreater{} \texttt{rgb}
\end{itemize}

\subsubsection{Result:}\label{result-429}

rgb color rgb color

\subsubsection{Comment:}\label{comment-86}

Return a random color equivalent to rgb(rnd(first\_op,
last\_op),rnd(first\_op, last\_op),rnd(first\_op, last\_op))Return a
random color equivalent to rgb(rnd(operand),rnd(operand),rnd(operand))

\subsubsection{Examples:}\label{examples-304}

\begin{verbatim}
 
rgb var0 <- rnd_color(100, 200); // var0 equals a random color, equivalent to rgb(rnd(100, 200),rnd(100, 200),rnd(100, 200)) 
rgb var1 <- rnd_color(255); // var1 equals a random color, equivalent to rgb(rnd(255),rnd(255),rnd(255))
\end{verbatim}

\subsubsection{See also:}\label{see-also-178}

\href{OperatorsNR\#rgb}{rgb}, \href{OperatorsDH\#hsb}{hsb},

\begin{center}\rule{0.5\linewidth}{\linethickness}\end{center}

\subsection{\texorpdfstring{\texttt{rotated\_by}}{rotated\_by}}\label{rotated_by}

\subsubsection{Possible use:}\label{possible-use-444}

\begin{itemize}
\tightlist
\item
  \texttt{geometry} \textbf{\texttt{rotated\_by}} \texttt{float}
  ---\textgreater{} \texttt{geometry}
\item
  \textbf{\texttt{rotated\_by}} (\texttt{geometry} , \texttt{float})
  ---\textgreater{} \texttt{geometry}
\item
  \texttt{geometry} \textbf{\texttt{rotated\_by}} \texttt{int}
  ---\textgreater{} \texttt{geometry}
\item
  \textbf{\texttt{rotated\_by}} (\texttt{geometry} , \texttt{int})
  ---\textgreater{} \texttt{geometry}
\item
  \textbf{\texttt{rotated\_by}} (\texttt{geometry}, \texttt{float},
  \texttt{point}) ---\textgreater{} \texttt{geometry}
\end{itemize}

\subsubsection{Result:}\label{result-430}

A geometry resulting from the application of a rotation by the
right-hand operand angle (degree) to the left-hand operand (geometry,
agent, point) A geometry resulting from the application of a rotation by
the right-hand operand angles (degree) along the three axis (x,y,z) to
the left-hand operand (geometry, agent, point)

\subsubsection{Comment:}\label{comment-87}

the right-hand operand can be a float or a int

\subsubsection{Examples:}\label{examples-305}

\begin{verbatim}
 
geometry var0 <- self rotated_by 45; // var0 equals the geometry resulting from a 45 degrees rotation to the geometry of the agent applying the operator. 
geometry var1 <- rotated_by(pyramid(10),45, {1,0,0}); // var1 equals the geometry resulting from a 45 degrees rotation along the {1,0,0} vector to the geometry of the agent applying the operator.
\end{verbatim}

\subsubsection{See also:}\label{see-also-179}

\href{OperatorsSZ\#transformed_by}{transformed\_by},
\href{OperatorsSZ\#translated_by}{translated\_by},

\begin{center}\rule{0.5\linewidth}{\linethickness}\end{center}

\subsection{\texorpdfstring{\texttt{round}}{round}}\label{round}

\subsubsection{Possible use:}\label{possible-use-445}

\begin{itemize}
\tightlist
\item
  \textbf{\texttt{round}} (\texttt{int}) ---\textgreater{} \texttt{int}
\item
  \textbf{\texttt{round}} (\texttt{point}) ---\textgreater{}
  \texttt{point}
\item
  \textbf{\texttt{round}} (\texttt{float}) ---\textgreater{}
  \texttt{int}
\end{itemize}

\subsubsection{Result:}\label{result-431}

Returns the rounded value of the operand.

\subsubsection{Special cases:}\label{special-cases-120}

\begin{itemize}
\tightlist
\item
  if the operand is an int, round returns it
\end{itemize}

\subsubsection{Examples:}\label{examples-306}

\begin{verbatim}
 
point var0 <- {12345.78943,  12345.78943, 12345.78943} with_precision 2; // var0 equals {12345.79,12345.79,12345.79} 
int var1 <- round (0.51); // var1 equals 1 
int var2 <- round (100.2); // var2 equals 100 
int var3 <- round(-0.51); // var3 equals -1
\end{verbatim}

\subsubsection{See also:}\label{see-also-180}

\href{OperatorsNR\#round}{round}, \href{OperatorsIM\#int}{int},
\href{OperatorsSZ\#with_precision}{with\_precision},

\begin{center}\rule{0.5\linewidth}{\linethickness}\end{center}

\subsection{\texorpdfstring{\texttt{row\_at}}{row\_at}}\label{row_at}

\subsubsection{Possible use:}\label{possible-use-446}

\begin{itemize}
\tightlist
\item
  \texttt{matrix} \textbf{\texttt{row\_at}} \texttt{int}
  ---\textgreater{} \texttt{list}
\item
  \textbf{\texttt{row\_at}} (\texttt{matrix} , \texttt{int})
  ---\textgreater{} \texttt{list}
\end{itemize}

\subsubsection{Result:}\label{result-432}

returns the row at a num\_line (right-hand operand)

\subsubsection{Examples:}\label{examples-307}

\begin{verbatim}
 
list var0 <- matrix([["el11","el12","el13"],["el21","el22","el23"],["el31","el32","el33"]]) row_at 2; // var0 equals ["el13","el23","el33"]
\end{verbatim}

\subsubsection{See also:}\label{see-also-181}

\href{OperatorsBC\#column_at}{column\_at},
\href{OperatorsBC\#columns_list}{columns\_list},

\begin{center}\rule{0.5\linewidth}{\linethickness}\end{center}

\subsection{\texorpdfstring{\texttt{rows\_list}}{rows\_list}}\label{rows_list}

\subsubsection{Possible use:}\label{possible-use-447}

\begin{itemize}
\tightlist
\item
  \textbf{\texttt{rows\_list}} (\texttt{matrix}) ---\textgreater{}
  \texttt{list\textless{}list\textgreater{}}
\end{itemize}

\subsubsection{Result:}\label{result-433}

returns a list of the rows of the matrix, with each row as a list of
elements

\subsubsection{Examples:}\label{examples-308}

\begin{verbatim}
 
list<list> var0 <- rows_list(matrix([["el11","el12","el13"],["el21","el22","el23"],["el31","el32","el33"]])); // var0 equals [["el11","el21","el31"],["el12","el22","el32"],["el13","el23","el33"]]
\end{verbatim}

\subsubsection{See also:}\label{see-also-182}

\href{OperatorsBC\#columns_list}{columns\_list},

\chapter{Operators (S to Z)}\label{operators-s-to-z}

\begin{center}\rule{0.5\linewidth}{\linethickness}\end{center}

\textbf{This file is automatically generated from java files. Do Not
Edit It.}

\begin{center}\rule{0.5\linewidth}{\linethickness}\end{center}

\section{Definition}\label{definition-5}

Operators in the GAML language are used to compose complex expressions.
An operator performs a function on one, two, or n operands (which are
other expressions and thus may be themselves composed of operators) and
returns the result of this function.

Most of them use a classical prefixed functional syntax (i.e.
\texttt{operator\_name(operand1,\ operand2,\ operand3)}, see below),
with the exception of arithmetic (e.g. \texttt{+}, \texttt{/}), logical
(\texttt{and}, \texttt{or}), comparison (e.g. \texttt{\textgreater{}},
\texttt{\textless{}}), access (\texttt{.}, \texttt{{[}..{]}}) and pair
(\texttt{::}) operators, which require an infixed notation (i.e.
\texttt{operand1\ operator\_symbol\ operand1}).

The ternary functional if-else operator, \texttt{?\ :}, uses a special
infixed syntax composed with two symbols (e.g.
\texttt{operand1\ ?\ operand2\ :\ operand3}). Two unary operators
(\texttt{-} and \texttt{!}) use a traditional prefixed syntax that does
not require parentheses unless the operand is itself a complex
expression (e.g. \texttt{-\ 10}, \texttt{!\ (operand1\ or\ operand2)}).

Finally, special constructor operators (\texttt{\{...\}} for
constructing points, \texttt{{[}...{]}} for constructing lists and maps)
will require their operands to be placed between their two symbols (e.g.
\texttt{\{1,2,3\}}, \texttt{{[}operand1,\ operand2,\ ...,\ operandn{]}}
or \texttt{{[}key1::value1,\ key2::value2...\ keyn::valuen{]}}).

With the exception of these special cases above, the following rules
apply to the syntax of operators: * if they only have one operand, the
functional prefixed syntax is mandatory (e.g.
\texttt{operator\_name(operand1)}) * if they have two arguments, either
the functional prefixed syntax (e.g.
\texttt{operator\_name(operand1,\ operand2)}) or the infixed syntax
(e.g. \texttt{operand1\ operator\_name\ operand2}) can be used. * if
they have more than two arguments, either the functional prefixed syntax
(e.g. \texttt{operator\_name(operand1,\ operand2,\ ...,\ operand)}) or a
special infixed syntax with the first operand on the left-hand side of
the operator name (e.g.
\texttt{operand1\ operator\_name(operand2,\ ...,\ operand)}) can be
used.

All of these alternative syntaxes are completely equivalent.

Operators in GAML are purely functional, i.e.~they are guaranteed to not
have any side effects on their operands. For instance, the
\texttt{shuffle} operator, which randomizes the positions of elements in
a list, does not modify its list operand but returns a new shuffled
list.

\section{\texorpdfstring{}{ }}\label{section-23}

\section{Priority between operators}\label{priority-between-operators-5}

The priority of operators determines, in the case of complex expressions
composed of several operators, which one(s) will be evaluated first.

GAML follows in general the traditional priorities attributed to
arithmetic, boolean, comparison operators, with some twists. Namely: *
the constructor operators, like \texttt{::}, used to compose pairs of
operands, have the lowest priority of all operators (e.g.
\texttt{a\ \textgreater{}\ b\ ::\ b\ \textgreater{}\ c} will return a
pair of boolean values, which means that the two comparisons are
evaluated before the operator applies. Similarly,
\texttt{{[}a\ \textgreater{}\ 10,\ b\ \textgreater{}\ 5{]}} will return
a list of boolean values. * it is followed by the \texttt{?:} operator,
the functional if-else (e.g.
\texttt{a\ \textgreater{}\ b\ ?\ a\ +\ 10\ :\ a\ -\ 10} will return the
result of the if-else). * next are the logical operators, \texttt{and}
and \texttt{or} (e.g.
\texttt{a\ \textgreater{}\ b\ or\ b\ \textgreater{}\ c} will return the
value of the test) * next are the comparison operators (i.e.
\texttt{\textgreater{}}, \texttt{\textless{}}, \texttt{\textless{}=},
\texttt{\textgreater{}=}, \texttt{=}, \texttt{!=}) * next the arithmetic
operators in their logical order (multiplicative operators have a higher
priority than additive operators) * next the unary operators \texttt{-}
and \texttt{!} * next the access operators \texttt{.} and
\texttt{{[}{]}} (e.g.
\texttt{\{1,2,3\}.x\ \textgreater{}\ 20\ +\ \{4,5,6\}.y} will return the
result of the comparison between the x and y ordinates of the two
points) * and finally the functional operators, which have the highest
priority of all.

\begin{center}\rule{0.5\linewidth}{\linethickness}\end{center}

\section{Using actions as operators}\label{using-actions-as-operators-5}

Actions defined in species can be used as operators, provided they are
called on the correct agent. The syntax is that of normal functional
operators, but the agent that will perform the action must be added as
the first operand.

For instance, if the following species is defined:

\begin{verbatim}
species spec1 {
        int min(int x, int y) {
                return x > y ? x : y;
        }
}
\end{verbatim}

Any agent instance of spec1 can use \texttt{min} as an operator (if the
action conflicts with an existing operator, a warning will be emitted).
For instance, in the same model, the following line is perfectly
acceptable:

\begin{verbatim}
global {
        init {
                create spec1;
                spec1 my_agent <- spec1[0];
                int the_min <- my_agent min(10,20); // or min(my_agent, 10, 20);
        }
}
\end{verbatim}

If the action doesn't have any operands, the syntax to use is
\texttt{my\_agent\ the\_action()}. Finally, if it does not return a
value, it might still be used but is considering as returning a value of
type \texttt{unknown} (e.g.
\texttt{unknown\ result\ \textless{}-\ my\_agent\ the\_action(op1,\ op2);}).

Note that due to the fact that actions are written by modelers, the
general functional contract is not respected in that case: actions might
perfectly have side effects on their operands (including the agent).

\begin{center}\rule{0.5\linewidth}{\linethickness}\end{center}

\section{Table of Contents}\label{table-of-contents-5}

\begin{center}\rule{0.5\linewidth}{\linethickness}\end{center}

\section{Operators by categories}\label{operators-by-categories-5}

\begin{center}\rule{0.5\linewidth}{\linethickness}\end{center}

\subsection{3D}\label{d-5}

\href{OperatorsBC\#box}{box}, \href{OperatorsBC\#cone3d}{cone3D},
\href{OperatorsBC\#cube}{cube}, \href{OperatorsBC\#cylinder}{cylinder},
\href{OperatorsDH\#dem}{dem}, \href{OperatorsDH\#hexagon}{hexagon},
\href{OperatorsNR\#pyramid}{pyramid},
\href{OperatorsNR\#rgb_to_xyz}{rgb\_to\_xyz},
\href{OperatorsSZ\#set_z}{set\_z}, \href{OperatorsSZ\#sphere}{sphere},
\href{OperatorsSZ\#teapot}{teapot},

\begin{center}\rule{0.5\linewidth}{\linethickness}\end{center}

\subsection{Arithmetic operators}\label{arithmetic-operators-5}

\href{OperatorsAA\#-}{-}, \href{OperatorsAA\#/}{/},
{[}\textsuperscript{{]}(OperatorsAA\#}), \href{OperatorsAA\#*}{*},
\href{OperatorsAA\#+}{+}, \href{OperatorsAA\#abs}{abs},
\href{OperatorsAA\#acos}{acos}, \href{OperatorsAA\#asin}{asin},
\href{OperatorsAA\#atan}{atan}, \href{OperatorsAA\#atan2}{atan2},
\href{OperatorsBC\#ceil}{ceil}, \href{OperatorsBC\#cos}{cos},
\href{OperatorsBC\#cos_rad}{cos\_rad}, \href{OperatorsDH\#div}{div},
\href{OperatorsDH\#even}{even}, \href{OperatorsDH\#exp}{exp},
\href{OperatorsDH\#fact}{fact}, \href{OperatorsDH\#floor}{floor},
\href{OperatorsDH\#hypot}{hypot},
\href{OperatorsIM\#is_finite}{is\_finite},
\href{OperatorsIM\#is_number}{is\_number}, \href{OperatorsIM\#ln}{ln},
\href{OperatorsIM\#log}{log}, \href{OperatorsIM\#mod}{mod},
\href{OperatorsNR\#round}{round}, \href{OperatorsSZ\#signum}{signum},
\href{OperatorsSZ\#sin}{sin}, \href{OperatorsSZ\#sin_rad}{sin\_rad},
\href{OperatorsSZ\#sqrt}{sqrt}, \href{OperatorsSZ\#tan}{tan},
\href{OperatorsSZ\#tan_rad}{tan\_rad}, \href{OperatorsSZ\#tanh}{tanh},
\href{OperatorsSZ\#with_precision}{with\_precision},

\begin{center}\rule{0.5\linewidth}{\linethickness}\end{center}

\subsection{BDI}\label{bdi-5}

\href{OperatorsAA\#and}{and}, \href{OperatorsDH\#eval_when}{eval\_when},
\href{OperatorsDH\#get_about}{get\_about},
\href{OperatorsDH\#get_agent}{get\_agent},
\href{OperatorsDH\#get_agent_cause}{get\_agent\_cause},
\href{OperatorsDH\#get_belief_op}{get\_belief\_op},
\href{OperatorsDH\#get_belief_with_name_op}{get\_belief\_with\_name\_op},
\href{OperatorsDH\#get_beliefs_op}{get\_beliefs\_op},
\href{OperatorsDH\#get_beliefs_with_name_op}{get\_beliefs\_with\_name\_op},
\href{OperatorsDH\#get_current_intention_op}{get\_current\_intention\_op},
\href{OperatorsDH\#get_decay}{get\_decay},
\href{OperatorsDH\#get_desire_op}{get\_desire\_op},
\href{OperatorsDH\#get_desire_with_name_op}{get\_desire\_with\_name\_op},
\href{OperatorsDH\#get_desires_op}{get\_desires\_op},
\href{OperatorsDH\#get_desires_with_name_op}{get\_desires\_with\_name\_op},
\href{OperatorsDH\#get_dominance}{get\_dominance},
\href{OperatorsDH\#get_familiarity}{get\_familiarity},
\href{OperatorsDH\#get_ideal_op}{get\_ideal\_op},
\href{OperatorsDH\#get_ideal_with_name_op}{get\_ideal\_with\_name\_op},
\href{OperatorsDH\#get_ideals_op}{get\_ideals\_op},
\href{OperatorsDH\#get_ideals_with_name_op}{get\_ideals\_with\_name\_op},
\href{OperatorsDH\#get_intensity}{get\_intensity},
\href{OperatorsDH\#get_intention_op}{get\_intention\_op},
\href{OperatorsDH\#get_intention_with_name_op}{get\_intention\_with\_name\_op},
\href{OperatorsDH\#get_intentions_op}{get\_intentions\_op},
\href{OperatorsDH\#get_intentions_with_name_op}{get\_intentions\_with\_name\_op},
\href{OperatorsDH\#get_lifetime}{get\_lifetime},
\href{OperatorsDH\#get_liking}{get\_liking},
\href{OperatorsDH\#get_modality}{get\_modality},
\href{OperatorsDH\#get_obligation_op}{get\_obligation\_op},
\href{OperatorsDH\#get_obligation_with_name_op}{get\_obligation\_with\_name\_op},
\href{OperatorsDH\#get_obligations_op}{get\_obligations\_op},
\href{OperatorsDH\#get_obligations_with_name_op}{get\_obligations\_with\_name\_op},
\href{OperatorsDH\#get_plan_name}{get\_plan\_name},
\href{OperatorsDH\#get_predicate}{get\_predicate},
\href{OperatorsDH\#get_solidarity}{get\_solidarity},
\href{OperatorsDH\#get_strength}{get\_strength},
\href{OperatorsDH\#get_super_intention}{get\_super\_intention},
\href{OperatorsDH\#get_trust}{get\_trust},
\href{OperatorsDH\#get_truth}{get\_truth},
\href{OperatorsDH\#get_uncertainties_op}{get\_uncertainties\_op},
\href{OperatorsDH\#get_uncertainties_with_name_op}{get\_uncertainties\_with\_name\_op},
\href{OperatorsDH\#get_uncertainty_op}{get\_uncertainty\_op},
\href{OperatorsDH\#get_uncertainty_with_name_op}{get\_uncertainty\_with\_name\_op},
\href{OperatorsDH\#has_belief_op}{has\_belief\_op},
\href{OperatorsDH\#has_belief_with_name_op}{has\_belief\_with\_name\_op},
\href{OperatorsDH\#has_desire_op}{has\_desire\_op},
\href{OperatorsDH\#has_desire_with_name_op}{has\_desire\_with\_name\_op},
\href{OperatorsDH\#has_ideal_op}{has\_ideal\_op},
\href{OperatorsDH\#has_ideal_with_name_op}{has\_ideal\_with\_name\_op},
\href{OperatorsDH\#has_intention_op}{has\_intention\_op},
\href{OperatorsDH\#has_intention_with_name_op}{has\_intention\_with\_name\_op},
\href{OperatorsDH\#has_obligation_op}{has\_obligation\_op},
\href{OperatorsDH\#has_obligation_with_name_op}{has\_obligation\_with\_name\_op},
\href{OperatorsDH\#has_uncertainty_op}{has\_uncertainty\_op},
\href{OperatorsDH\#has_uncertainty_with_name_op}{has\_uncertainty\_with\_name\_op},
\href{OperatorsNR\#new_emotion}{new\_emotion},
\href{OperatorsNR\#new_mental_state}{new\_mental\_state},
\href{OperatorsNR\#new_predicate}{new\_predicate},
\href{OperatorsNR\#new_social_link}{new\_social\_link},
\href{OperatorsNR\#or}{or}, \href{OperatorsSZ\#set_about}{set\_about},
\href{OperatorsSZ\#set_agent}{set\_agent},
\href{OperatorsSZ\#set_agent_cause}{set\_agent\_cause},
\href{OperatorsSZ\#set_decay}{set\_decay},
\href{OperatorsSZ\#set_dominance}{set\_dominance},
\href{OperatorsSZ\#set_familiarity}{set\_familiarity},
\href{OperatorsSZ\#set_intensity}{set\_intensity},
\href{OperatorsSZ\#set_lifetime}{set\_lifetime},
\href{OperatorsSZ\#set_liking}{set\_liking},
\href{OperatorsSZ\#set_modality}{set\_modality},
\href{OperatorsSZ\#set_predicate}{set\_predicate},
\href{OperatorsSZ\#set_solidarity}{set\_solidarity},
\href{OperatorsSZ\#set_strength}{set\_strength},
\href{OperatorsSZ\#set_trust}{set\_trust},
\href{OperatorsSZ\#set_truth}{set\_truth},
\href{OperatorsSZ\#with_lifetime}{with\_lifetime},
\href{OperatorsSZ\#with_values}{with\_values},

\begin{center}\rule{0.5\linewidth}{\linethickness}\end{center}

\subsection{Casting operators}\label{casting-operators-5}

\href{OperatorsAA\#as}{as}, \href{OperatorsAA\#as_int}{as\_int},
\href{OperatorsAA\#as_matrix}{as\_matrix},
\href{OperatorsDH\#font}{font}, \href{OperatorsIM\#is}{is},
\href{OperatorsIM\#is_skill}{is\_skill},
\href{OperatorsIM\#list_with}{list\_with},
\href{OperatorsIM\#matrix_with}{matrix\_with},
\href{OperatorsSZ\#species}{species},
\href{OperatorsSZ\#to_gaml}{to\_gaml},
\href{OperatorsSZ\#topology}{topology},

\begin{center}\rule{0.5\linewidth}{\linethickness}\end{center}

\subsection{Color-related operators}\label{color-related-operators-5}

\href{OperatorsAA\#-}{-}, \href{OperatorsAA\#/}{/},
\href{OperatorsAA\#*}{*}, \href{OperatorsAA\#+}{+},
\href{OperatorsBC\#blend}{blend},
\href{OperatorsBC\#brewer_colors}{brewer\_colors},
\href{OperatorsBC\#brewer_palettes}{brewer\_palettes},
\href{OperatorsDH\#grayscale}{grayscale}, \href{OperatorsDH\#hsb}{hsb},
\href{OperatorsIM\#mean}{mean}, \href{OperatorsIM\#median}{median},
\href{OperatorsNR\#rgb}{rgb}, \href{OperatorsNR\#rnd_color}{rnd\_color},
\href{OperatorsSZ\#sum}{sum},

\begin{center}\rule{0.5\linewidth}{\linethickness}\end{center}

\subsection{Comparison operators}\label{comparison-operators-5}

\href{OperatorsAA\#!=}{!=}, \href{OperatorsAA\#\%3C}{\textless{}},
\href{OperatorsAA\#\%3C=}{\textless{}=}, \href{OperatorsAA\#=}{=},
\href{OperatorsAA\#\%3E}{\textgreater{}},
\href{OperatorsAA\#\%3E=}{\textgreater{}=},
\href{OperatorsBC\#between}{between},

\begin{center}\rule{0.5\linewidth}{\linethickness}\end{center}

\subsection{Containers-related
operators}\label{containers-related-operators-5}

\href{OperatorsAA\#-}{-}, \href{OperatorsAA\#::}{::},
\href{OperatorsAA\#+}{+}, \href{OperatorsAA\#accumulate}{accumulate},
\href{OperatorsAA\#among}{among}, \href{OperatorsAA\#at}{at},
\href{OperatorsBC\#collect}{collect},
\href{OperatorsBC\#contains}{contains},
\href{OperatorsBC\#contains_all}{contains\_all},
\href{OperatorsBC\#contains_any}{contains\_any},
\href{OperatorsBC\#count}{count},
\href{OperatorsDH\#distinct}{distinct},
\href{OperatorsDH\#empty}{empty}, \href{OperatorsDH\#every}{every},
\href{OperatorsDH\#first}{first},
\href{OperatorsDH\#first_with}{first\_with},
\href{OperatorsDH\#get}{get}, \href{OperatorsDH\#group_by}{group\_by},
\href{OperatorsIM\#in}{in}, \href{OperatorsIM\#index_by}{index\_by},
\href{OperatorsIM\#inter}{inter},
\href{OperatorsIM\#interleave}{interleave},
\href{OperatorsIM\#internal_at}{internal\_at},
\href{OperatorsIM\#internal_integrated_value}{internal\_integrated\_value},
\href{OperatorsIM\#last}{last},
\href{OperatorsIM\#last_with}{last\_with},
\href{OperatorsIM\#length}{length}, \href{OperatorsIM\#max}{max},
\href{OperatorsIM\#max_of}{max\_of}, \href{OperatorsIM\#mean}{mean},
\href{OperatorsIM\#mean_of}{mean\_of},
\href{OperatorsIM\#median}{median}, \href{OperatorsIM\#min}{min},
\href{OperatorsIM\#min_of}{min\_of}, \href{OperatorsIM\#mul}{mul},
\href{OperatorsNR\#one_of}{one\_of},
\href{OperatorsNR\#product_of}{product\_of},
\href{OperatorsNR\#range}{range}, \href{OperatorsNR\#reverse}{reverse},
\href{OperatorsSZ\#shuffle}{shuffle},
\href{OperatorsSZ\#sort_by}{sort\_by}, \href{OperatorsSZ\#split}{split},
\href{OperatorsSZ\#split_in}{split\_in},
\href{OperatorsSZ\#split_using}{split\_using},
\href{OperatorsSZ\#sum}{sum}, \href{OperatorsSZ\#sum_of}{sum\_of},
\href{OperatorsSZ\#union}{union},
\href{OperatorsSZ\#variance_of}{variance\_of},
\href{OperatorsSZ\#where}{where},
\href{OperatorsSZ\#with_max_of}{with\_max\_of},
\href{OperatorsSZ\#with_min_of}{with\_min\_of},

\begin{center}\rule{0.5\linewidth}{\linethickness}\end{center}

\subsection{Date-related operators}\label{date-related-operators-5}

\href{OperatorsAA\#-}{-}, \href{OperatorsAA\#!=}{!=},
\href{OperatorsAA\#+}{+}, \href{OperatorsAA\#\%3C}{\textless{}},
\href{OperatorsAA\#\%3C=}{\textless{}=}, \href{OperatorsAA\#=}{=},
\href{OperatorsAA\#\%3E}{\textgreater{}},
\href{OperatorsAA\#\%3E=}{\textgreater{}=},
\href{OperatorsAA\#after}{after}, \href{OperatorsBC\#before}{before},
\href{OperatorsBC\#between}{between}, \href{OperatorsDH\#every}{every},
\href{OperatorsIM\#milliseconds_between}{milliseconds\_between},
\href{OperatorsIM\#minus_days}{minus\_days},
\href{OperatorsIM\#minus_hours}{minus\_hours},
\href{OperatorsIM\#minus_minutes}{minus\_minutes},
\href{OperatorsIM\#minus_months}{minus\_months},
\href{OperatorsIM\#minus_ms}{minus\_ms},
\href{OperatorsIM\#minus_weeks}{minus\_weeks},
\href{OperatorsIM\#minus_years}{minus\_years},
\href{OperatorsIM\#months_between}{months\_between},
\href{OperatorsNR\#plus_days}{plus\_days},
\href{OperatorsNR\#plus_hours}{plus\_hours},
\href{OperatorsNR\#plus_minutes}{plus\_minutes},
\href{OperatorsNR\#plus_months}{plus\_months},
\href{OperatorsNR\#plus_ms}{plus\_ms},
\href{OperatorsNR\#plus_weeks}{plus\_weeks},
\href{OperatorsNR\#plus_years}{plus\_years},
\href{OperatorsSZ\#since}{since}, \href{OperatorsSZ\#to}{to},
\href{OperatorsSZ\#until}{until},
\href{OperatorsSZ\#years_between}{years\_between},

\begin{center}\rule{0.5\linewidth}{\linethickness}\end{center}

\subsection{Dates}\label{dates-5}

\begin{center}\rule{0.5\linewidth}{\linethickness}\end{center}

\subsection{DescriptiveStatistics}\label{descriptivestatistics-5}

\href{OperatorsAA\#auto_correlation}{auto\_correlation},
\href{OperatorsBC\#correlation}{correlation},
\href{OperatorsBC\#covariance}{covariance},
\href{OperatorsDH\#durbin_watson}{durbin\_watson},
\href{OperatorsIM\#kurtosis}{kurtosis},
\href{OperatorsIM\#moment}{moment},
\href{OperatorsNR\#quantile}{quantile},
\href{OperatorsNR\#quantile_inverse}{quantile\_inverse},
\href{OperatorsNR\#rank_interpolated}{rank\_interpolated},
\href{OperatorsNR\#rms}{rms}, \href{OperatorsSZ\#skew}{skew},
\href{OperatorsSZ\#variance}{variance},

\begin{center}\rule{0.5\linewidth}{\linethickness}\end{center}

\subsection{Displays}\label{displays-5}

\href{OperatorsDH\#horizontal}{horizontal},
\href{OperatorsSZ\#stack}{stack},
\href{OperatorsSZ\#vertical}{vertical},

\begin{center}\rule{0.5\linewidth}{\linethickness}\end{center}

\subsection{Distributions}\label{distributions-5}

\href{OperatorsBC\#binomial_coeff}{binomial\_coeff},
\href{OperatorsBC\#binomial_complemented}{binomial\_complemented},
\href{OperatorsBC\#binomial_sum}{binomial\_sum},
\href{OperatorsBC\#chi_square}{chi\_square},
\href{OperatorsBC\#chi_square_complemented}{chi\_square\_complemented},
\href{OperatorsDH\#gamma_distribution}{gamma\_distribution},
\href{OperatorsDH\#gamma_distribution_complemented}{gamma\_distribution\_complemented},
\href{OperatorsNR\#normal_area}{normal\_area},
\href{OperatorsNR\#normal_density}{normal\_density},
\href{OperatorsNR\#normal_inverse}{normal\_inverse},
\href{OperatorsNR\#pvalue_for_fstat}{pValue\_for\_fStat},
\href{OperatorsNR\#pvalue_for_tstat}{pValue\_for\_tStat},
\href{OperatorsSZ\#student_area}{student\_area},
\href{OperatorsSZ\#student_t_inverse}{student\_t\_inverse},

\begin{center}\rule{0.5\linewidth}{\linethickness}\end{center}

\subsection{Driving operators}\label{driving-operators-5}

\href{OperatorsAA\#as_driving_graph}{as\_driving\_graph},

\begin{center}\rule{0.5\linewidth}{\linethickness}\end{center}

\subsection{edge}\label{edge-6}

\href{OperatorsDH\#edge_between}{edge\_between},
\href{OperatorsSZ\#strahler}{strahler},

\begin{center}\rule{0.5\linewidth}{\linethickness}\end{center}

\subsection{EDP-related operators}\label{edp-related-operators-5}

\href{OperatorsDH\#diff}{diff}, \href{OperatorsDH\#diff2}{diff2},
\href{OperatorsIM\#internal_zero_order_equation}{internal\_zero\_order\_equation},

\begin{center}\rule{0.5\linewidth}{\linethickness}\end{center}

\subsection{Files-related operators}\label{files-related-operators-5}

\href{OperatorsBC\#crs}{crs},
\href{OperatorsDH\#evaluate_sub_model}{evaluate\_sub\_model},
\href{OperatorsDH\#file}{file},
\href{OperatorsDH\#file_exists}{file\_exists},
\href{OperatorsDH\#folder}{folder}, \href{OperatorsDH\#get}{get},
\href{OperatorsIM\#load_sub_model}{load\_sub\_model},
\href{OperatorsNR\#new_folder}{new\_folder},
\href{OperatorsNR\#osm_file}{osm\_file}, \href{OperatorsNR\#read}{read},
\href{OperatorsSZ\#step_sub_model}{step\_sub\_model},
\href{OperatorsSZ\#writable}{writable},

\begin{center}\rule{0.5\linewidth}{\linethickness}\end{center}

\subsection{FIPA-related operators}\label{fipa-related-operators-5}

\href{OperatorsBC\#conversation}{conversation},
\href{OperatorsIM\#message}{message},

\begin{center}\rule{0.5\linewidth}{\linethickness}\end{center}

\subsection{GamaMetaType}\label{gamametatype-5}

\href{OperatorsSZ\#type_of}{type\_of},

\begin{center}\rule{0.5\linewidth}{\linethickness}\end{center}

\subsection{GammaFunction}\label{gammafunction-5}

\href{OperatorsBC\#beta}{beta}, \href{OperatorsDH\#gamma}{gamma},
\href{OperatorsIM\#incomplete_beta}{incomplete\_beta},
\href{OperatorsIM\#incomplete_gamma}{incomplete\_gamma},
\href{OperatorsIM\#incomplete_gamma_complement}{incomplete\_gamma\_complement},
\href{OperatorsIM\#log_gamma}{log\_gamma},

\begin{center}\rule{0.5\linewidth}{\linethickness}\end{center}

\subsection{Graphs-related operators}\label{graphs-related-operators-5}

\href{OperatorsAA\#add_edge}{add\_edge},
\href{OperatorsAA\#add_node}{add\_node},
\href{OperatorsAA\#adjacency}{adjacency},
\href{OperatorsAA\#agent_from_geometry}{agent\_from\_geometry},
\href{OperatorsAA\#all_pairs_shortest_path}{all\_pairs\_shortest\_path},
\href{OperatorsAA\#alpha_index}{alpha\_index},
\href{OperatorsAA\#as_distance_graph}{as\_distance\_graph},
\href{OperatorsAA\#as_edge_graph}{as\_edge\_graph},
\href{OperatorsAA\#as_intersection_graph}{as\_intersection\_graph},
\href{OperatorsAA\#as_path}{as\_path},
\href{OperatorsBC\#beta_index}{beta\_index},
\href{OperatorsBC\#betweenness_centrality}{betweenness\_centrality},
\href{OperatorsBC\#biggest_cliques_of}{biggest\_cliques\_of},
\href{OperatorsBC\#connected_components_of}{connected\_components\_of},
\href{OperatorsBC\#connectivity_index}{connectivity\_index},
\href{OperatorsBC\#contains_edge}{contains\_edge},
\href{OperatorsBC\#contains_vertex}{contains\_vertex},
\href{OperatorsDH\#degree_of}{degree\_of},
\href{OperatorsDH\#directed}{directed}, \href{OperatorsDH\#edge}{edge},
\href{OperatorsDH\#edge_between}{edge\_between},
\href{OperatorsDH\#edge_betweenness}{edge\_betweenness},
\href{OperatorsDH\#edges}{edges},
\href{OperatorsDH\#gamma_index}{gamma\_index},
\href{OperatorsDH\#generate_barabasi_albert}{generate\_barabasi\_albert},
\href{OperatorsDH\#generate_complete_graph}{generate\_complete\_graph},
\href{OperatorsDH\#generate_watts_strogatz}{generate\_watts\_strogatz},
\href{OperatorsDH\#grid_cells_to_graph}{grid\_cells\_to\_graph},
\href{OperatorsIM\#in_degree_of}{in\_degree\_of},
\href{OperatorsIM\#in_edges_of}{in\_edges\_of},
\href{OperatorsIM\#layout}{layout},
\href{OperatorsIM\#load_graph_from_file}{load\_graph\_from\_file},
\href{OperatorsIM\#load_shortest_paths}{load\_shortest\_paths},
\href{OperatorsIM\#main_connected_component}{main\_connected\_component},
\href{OperatorsIM\#max_flow_between}{max\_flow\_between},
\href{OperatorsIM\#maximal_cliques_of}{maximal\_cliques\_of},
\href{OperatorsNR\#nb_cycles}{nb\_cycles},
\href{OperatorsNR\#neighbors_of}{neighbors\_of},
\href{OperatorsNR\#node}{node}, \href{OperatorsNR\#nodes}{nodes},
\href{OperatorsNR\#out_degree_of}{out\_degree\_of},
\href{OperatorsNR\#out_edges_of}{out\_edges\_of},
\href{OperatorsNR\#path_between}{path\_between},
\href{OperatorsNR\#paths_between}{paths\_between},
\href{OperatorsNR\#predecessors_of}{predecessors\_of},
\href{OperatorsNR\#remove_node_from}{remove\_node\_from},
\href{OperatorsNR\#rewire_n}{rewire\_n},
\href{OperatorsSZ\#source_of}{source\_of},
\href{OperatorsSZ\#spatial_graph}{spatial\_graph},
\href{OperatorsSZ\#strahler}{strahler},
\href{OperatorsSZ\#successors_of}{successors\_of},
\href{OperatorsSZ\#sum}{sum}, \href{OperatorsSZ\#target_of}{target\_of},
\href{OperatorsSZ\#undirected}{undirected},
\href{OperatorsSZ\#use_cache}{use\_cache},
\href{OperatorsSZ\#weight_of}{weight\_of},
\href{OperatorsSZ\#with_optimizer_type}{with\_optimizer\_type},
\href{OperatorsSZ\#with_weights}{with\_weights},

\begin{center}\rule{0.5\linewidth}{\linethickness}\end{center}

\subsection{Grid-related operators}\label{grid-related-operators-5}

\href{OperatorsAA\#as_4_grid}{as\_4\_grid},
\href{OperatorsAA\#as_grid}{as\_grid},
\href{OperatorsAA\#as_hexagonal_grid}{as\_hexagonal\_grid},
\href{OperatorsDH\#grid_at}{grid\_at},
\href{OperatorsNR\#path_between}{path\_between},

\begin{center}\rule{0.5\linewidth}{\linethickness}\end{center}

\subsection{Iterator operators}\label{iterator-operators-5}

\href{OperatorsAA\#accumulate}{accumulate},
\href{OperatorsAA\#as_map}{as\_map},
\href{OperatorsBC\#collect}{collect}, \href{OperatorsBC\#count}{count},
\href{OperatorsBC\#create_map}{create\_map},
\href{OperatorsDH\#distribution_of}{distribution\_of},
\href{OperatorsDH\#distribution_of}{distribution\_of},
\href{OperatorsDH\#distribution_of}{distribution\_of},
\href{OperatorsDH\#distribution2d_of}{distribution2d\_of},
\href{OperatorsDH\#distribution2d_of}{distribution2d\_of},
\href{OperatorsDH\#distribution2d_of}{distribution2d\_of},
\href{OperatorsDH\#first_with}{first\_with},
\href{OperatorsDH\#frequency_of}{frequency\_of},
\href{OperatorsDH\#group_by}{group\_by},
\href{OperatorsIM\#index_by}{index\_by},
\href{OperatorsIM\#last_with}{last\_with},
\href{OperatorsIM\#max_of}{max\_of},
\href{OperatorsIM\#mean_of}{mean\_of},
\href{OperatorsIM\#min_of}{min\_of},
\href{OperatorsNR\#product_of}{product\_of},
\href{OperatorsSZ\#sort_by}{sort\_by},
\href{OperatorsSZ\#sum_of}{sum\_of},
\href{OperatorsSZ\#variance_of}{variance\_of},
\href{OperatorsSZ\#where}{where},
\href{OperatorsSZ\#with_max_of}{with\_max\_of},
\href{OperatorsSZ\#with_min_of}{with\_min\_of},

\begin{center}\rule{0.5\linewidth}{\linethickness}\end{center}

\subsection{List-related operators}\label{list-related-operators-5}

\href{OperatorsBC\#copy_between}{copy\_between},
\href{OperatorsIM\#index_of}{index\_of},
\href{OperatorsIM\#last_index_of}{last\_index\_of},

\begin{center}\rule{0.5\linewidth}{\linethickness}\end{center}

\subsection{Logical operators}\label{logical-operators-5}

\href{OperatorsAA\#:}{:}, \href{OperatorsAA\#!}{!},
\href{OperatorsAA\#?}{?}, \href{OperatorsAA\#and}{and},
\href{OperatorsNR\#or}{or}, \href{OperatorsSZ\#xor}{xor},

\begin{center}\rule{0.5\linewidth}{\linethickness}\end{center}

\subsection{Map comparaison
operators}\label{map-comparaison-operators-5}

\href{OperatorsDH\#fuzzy_kappa}{fuzzy\_kappa},
\href{OperatorsDH\#fuzzy_kappa_sim}{fuzzy\_kappa\_sim},
\href{OperatorsIM\#kappa}{kappa},
\href{OperatorsIM\#kappa_sim}{kappa\_sim},
\href{OperatorsNR\#percent_absolute_deviation}{percent\_absolute\_deviation},

\begin{center}\rule{0.5\linewidth}{\linethickness}\end{center}

\subsection{Map-related operators}\label{map-related-operators-5}

\href{OperatorsAA\#as_map}{as\_map},
\href{OperatorsBC\#create_map}{create\_map},
\href{OperatorsIM\#index_of}{index\_of},
\href{OperatorsIM\#last_index_of}{last\_index\_of},

\begin{center}\rule{0.5\linewidth}{\linethickness}\end{center}

\subsection{Material}\label{material-7}

\href{OperatorsIM\#material}{material},

\begin{center}\rule{0.5\linewidth}{\linethickness}\end{center}

\subsection{Matrix-related operators}\label{matrix-related-operators-5}

\href{OperatorsAA\#-}{-}, \href{OperatorsAA\#/}{/},
\href{OperatorsAA\#.}{.}, \href{OperatorsAA\#*}{*},
\href{OperatorsAA\#+}{+},
\href{OperatorsAA\#append_horizontally}{append\_horizontally},
\href{OperatorsAA\#append_vertically}{append\_vertically},
\href{OperatorsBC\#column_at}{column\_at},
\href{OperatorsBC\#columns_list}{columns\_list},
\href{OperatorsDH\#determinant}{determinant},
\href{OperatorsDH\#eigenvalues}{eigenvalues},
\href{OperatorsIM\#index_of}{index\_of},
\href{OperatorsIM\#inverse}{inverse},
\href{OperatorsIM\#last_index_of}{last\_index\_of},
\href{OperatorsNR\#row_at}{row\_at},
\href{OperatorsNR\#rows_list}{rows\_list},
\href{OperatorsSZ\#shuffle}{shuffle}, \href{OperatorsSZ\#trace}{trace},
\href{OperatorsSZ\#transpose}{transpose},

\begin{center}\rule{0.5\linewidth}{\linethickness}\end{center}

\subsection{multicriteria operators}\label{multicriteria-operators-5}

\href{OperatorsDH\#electre_dm}{electre\_DM},
\href{OperatorsDH\#evidence_theory_dm}{evidence\_theory\_DM},
\href{OperatorsDH\#fuzzy_choquet_dm}{fuzzy\_choquet\_DM},
\href{OperatorsNR\#promethee_dm}{promethee\_DM},
\href{OperatorsSZ\#weighted_means_dm}{weighted\_means\_DM},

\begin{center}\rule{0.5\linewidth}{\linethickness}\end{center}

\subsection{Path-related operators}\label{path-related-operators-5}

\href{OperatorsAA\#agent_from_geometry}{agent\_from\_geometry},
\href{OperatorsAA\#all_pairs_shortest_path}{all\_pairs\_shortest\_path},
\href{OperatorsAA\#as_path}{as\_path},
\href{OperatorsIM\#load_shortest_paths}{load\_shortest\_paths},
\href{OperatorsIM\#max_flow_between}{max\_flow\_between},
\href{OperatorsNR\#path_between}{path\_between},
\href{OperatorsNR\#path_to}{path\_to},
\href{OperatorsNR\#paths_between}{paths\_between},
\href{OperatorsSZ\#use_cache}{use\_cache},

\begin{center}\rule{0.5\linewidth}{\linethickness}\end{center}

\subsection{Points-related operators}\label{points-related-operators-5}

\href{OperatorsAA\#-}{-}, \href{OperatorsAA\#/}{/},
\href{OperatorsAA\#*}{*}, \href{OperatorsAA\#+}{+},
\href{OperatorsAA\#\%3C}{\textless{}},
\href{OperatorsAA\#\%3C=}{\textless{}=},
\href{OperatorsAA\#\%3E}{\textgreater{}},
\href{OperatorsAA\#\%3E=}{\textgreater{}=},
\href{OperatorsAA\#add_point}{add\_point},
\href{OperatorsAA\#angle_between}{angle\_between},
\href{OperatorsAA\#any_location_in}{any\_location\_in},
\href{OperatorsBC\#centroid}{centroid},
\href{OperatorsBC\#closest_points_with}{closest\_points\_with},
\href{OperatorsDH\#farthest_point_to}{farthest\_point\_to},
\href{OperatorsDH\#grid_at}{grid\_at}, \href{OperatorsNR\#norm}{norm},
\href{OperatorsNR\#point}{point},
\href{OperatorsNR\#points_along}{points\_along},
\href{OperatorsNR\#points_at}{points\_at},
\href{OperatorsNR\#points_on}{points\_on},

\begin{center}\rule{0.5\linewidth}{\linethickness}\end{center}

\subsection{Random operators}\label{random-operators-5}

\href{OperatorsBC\#binomial}{binomial}, \href{OperatorsDH\#flip}{flip},
\href{OperatorsDH\#gauss}{gauss},
\href{OperatorsIM\#improved_generator}{improved\_generator},
\href{OperatorsNR\#open_simplex_generator}{open\_simplex\_generator},
\href{OperatorsNR\#poisson}{poisson}, \href{OperatorsNR\#rnd}{rnd},
\href{OperatorsNR\#rnd_choice}{rnd\_choice},
\href{OperatorsSZ\#sample}{sample},
\href{OperatorsSZ\#shuffle}{shuffle},
\href{OperatorsSZ\#simplex_generator}{simplex\_generator},
\href{OperatorsSZ\#skew_gauss}{skew\_gauss},
\href{OperatorsSZ\#truncated_gauss}{truncated\_gauss},

\begin{center}\rule{0.5\linewidth}{\linethickness}\end{center}

\subsection{ReverseOperators}\label{reverseoperators-5}

\href{OperatorsSZ\#savesimulation}{saveSimulation},
\href{OperatorsSZ\#serialize}{serialize},
\href{OperatorsSZ\#serializeagent}{serializeAgent},
\href{OperatorsSZ\#unserializesimulation}{unSerializeSimulation},
\href{OperatorsSZ\#unserializesimulationfromfile}{unSerializeSimulationFromFile},

\begin{center}\rule{0.5\linewidth}{\linethickness}\end{center}

\subsection{Shape}\label{shape-5}

\href{OperatorsAA\#arc}{arc}, \href{OperatorsBC\#box}{box},
\href{OperatorsBC\#circle}{circle}, \href{OperatorsBC\#cone}{cone},
\href{OperatorsBC\#cone3d}{cone3D}, \href{OperatorsBC\#cross}{cross},
\href{OperatorsBC\#cube}{cube}, \href{OperatorsBC\#curve}{curve},
\href{OperatorsBC\#cylinder}{cylinder},
\href{OperatorsDH\#ellipse}{ellipse},
\href{OperatorsDH\#envelope}{envelope},
\href{OperatorsDH\#geometry_collection}{geometry\_collection},
\href{OperatorsDH\#hexagon}{hexagon}, \href{OperatorsIM\#line}{line},
\href{OperatorsIM\#link}{link}, \href{OperatorsNR\#plan}{plan},
\href{OperatorsNR\#polygon}{polygon},
\href{OperatorsNR\#polyhedron}{polyhedron},
\href{OperatorsNR\#pyramid}{pyramid},
\href{OperatorsNR\#rectangle}{rectangle},
\href{OperatorsSZ\#sphere}{sphere}, \href{OperatorsSZ\#square}{square},
\href{OperatorsSZ\#squircle}{squircle},
\href{OperatorsSZ\#teapot}{teapot},
\href{OperatorsSZ\#triangle}{triangle},

\begin{center}\rule{0.5\linewidth}{\linethickness}\end{center}

\subsection{Spatial operators}\label{spatial-operators-5}

\href{OperatorsAA\#-}{-}, \href{OperatorsAA\#*}{*},
\href{OperatorsAA\#+}{+}, \href{OperatorsAA\#add_point}{add\_point},
\href{OperatorsAA\#agent_closest_to}{agent\_closest\_to},
\href{OperatorsAA\#agent_farthest_to}{agent\_farthest\_to},
\href{OperatorsAA\#agents_at_distance}{agents\_at\_distance},
\href{OperatorsAA\#agents_inside}{agents\_inside},
\href{OperatorsAA\#agents_overlapping}{agents\_overlapping},
\href{OperatorsAA\#angle_between}{angle\_between},
\href{OperatorsAA\#any_location_in}{any\_location\_in},
\href{OperatorsAA\#arc}{arc}, \href{OperatorsAA\#around}{around},
\href{OperatorsAA\#as_4_grid}{as\_4\_grid},
\href{OperatorsAA\#as_grid}{as\_grid},
\href{OperatorsAA\#as_hexagonal_grid}{as\_hexagonal\_grid},
\href{OperatorsAA\#at_distance}{at\_distance},
\href{OperatorsAA\#at_location}{at\_location},
\href{OperatorsBC\#box}{box}, \href{OperatorsBC\#centroid}{centroid},
\href{OperatorsBC\#circle}{circle}, \href{OperatorsBC\#clean}{clean},
\href{OperatorsBC\#clean_network}{clean\_network},
\href{OperatorsBC\#closest_points_with}{closest\_points\_with},
\href{OperatorsBC\#closest_to}{closest\_to},
\href{OperatorsBC\#cone}{cone}, \href{OperatorsBC\#cone3d}{cone3D},
\href{OperatorsBC\#convex_hull}{convex\_hull},
\href{OperatorsBC\#covers}{covers}, \href{OperatorsBC\#cross}{cross},
\href{OperatorsBC\#crosses}{crosses}, \href{OperatorsBC\#crs}{crs},
\href{OperatorsBC\#crs_transform}{CRS\_transform},
\href{OperatorsBC\#cube}{cube}, \href{OperatorsBC\#curve}{curve},
\href{OperatorsBC\#cylinder}{cylinder}, \href{OperatorsDH\#dem}{dem},
\href{OperatorsDH\#direction_between}{direction\_between},
\href{OperatorsDH\#disjoint_from}{disjoint\_from},
\href{OperatorsDH\#distance_between}{distance\_between},
\href{OperatorsDH\#distance_to}{distance\_to},
\href{OperatorsDH\#ellipse}{ellipse},
\href{OperatorsDH\#envelope}{envelope},
\href{OperatorsDH\#farthest_point_to}{farthest\_point\_to},
\href{OperatorsDH\#farthest_to}{farthest\_to},
\href{OperatorsDH\#geometry_collection}{geometry\_collection},
\href{OperatorsDH\#gini}{gini}, \href{OperatorsDH\#hexagon}{hexagon},
\href{OperatorsDH\#hierarchical_clustering}{hierarchical\_clustering},
\href{OperatorsIM\#idw}{IDW}, \href{OperatorsIM\#inside}{inside},
\href{OperatorsIM\#inter}{inter},
\href{OperatorsIM\#intersects}{intersects},
\href{OperatorsIM\#line}{line}, \href{OperatorsIM\#link}{link},
\href{OperatorsIM\#masked_by}{masked\_by},
\href{OperatorsIM\#moran}{moran},
\href{OperatorsNR\#neighbors_at}{neighbors\_at},
\href{OperatorsNR\#neighbors_of}{neighbors\_of},
\href{OperatorsNR\#overlapping}{overlapping},
\href{OperatorsNR\#overlaps}{overlaps},
\href{OperatorsNR\#partially_overlaps}{partially\_overlaps},
\href{OperatorsNR\#path_between}{path\_between},
\href{OperatorsNR\#path_to}{path\_to}, \href{OperatorsNR\#plan}{plan},
\href{OperatorsNR\#points_along}{points\_along},
\href{OperatorsNR\#points_at}{points\_at},
\href{OperatorsNR\#points_on}{points\_on},
\href{OperatorsNR\#polygon}{polygon},
\href{OperatorsNR\#polyhedron}{polyhedron},
\href{OperatorsNR\#pyramid}{pyramid},
\href{OperatorsNR\#rectangle}{rectangle},
\href{OperatorsNR\#rgb_to_xyz}{rgb\_to\_xyz},
\href{OperatorsNR\#rotated_by}{rotated\_by},
\href{OperatorsNR\#round}{round},
\href{OperatorsSZ\#scaled_to}{scaled\_to},
\href{OperatorsSZ\#set_z}{set\_z},
\href{OperatorsSZ\#simple_clustering_by_distance}{simple\_clustering\_by\_distance},
\href{OperatorsSZ\#simplification}{simplification},
\href{OperatorsSZ\#skeletonize}{skeletonize},
\href{OperatorsSZ\#smooth}{smooth}, \href{OperatorsSZ\#sphere}{sphere},
\href{OperatorsSZ\#split_at}{split\_at},
\href{OperatorsSZ\#split_geometry}{split\_geometry},
\href{OperatorsSZ\#split_lines}{split\_lines},
\href{OperatorsSZ\#square}{square},
\href{OperatorsSZ\#squircle}{squircle},
\href{OperatorsSZ\#teapot}{teapot},
\href{OperatorsSZ\#to_gama_crs}{to\_GAMA\_CRS},
\href{OperatorsSZ\#to_rectangles}{to\_rectangles},
\href{OperatorsSZ\#to_squares}{to\_squares},
\href{OperatorsSZ\#to_sub_geometries}{to\_sub\_geometries},
\href{OperatorsSZ\#touches}{touches},
\href{OperatorsSZ\#towards}{towards},
\href{OperatorsSZ\#transformed_by}{transformed\_by},
\href{OperatorsSZ\#translated_by}{translated\_by},
\href{OperatorsSZ\#triangle}{triangle},
\href{OperatorsSZ\#triangulate}{triangulate},
\href{OperatorsSZ\#union}{union}, \href{OperatorsSZ\#using}{using},
\href{OperatorsSZ\#voronoi}{voronoi},
\href{OperatorsSZ\#with_precision}{with\_precision},
\href{OperatorsSZ\#without_holes}{without\_holes},

\begin{center}\rule{0.5\linewidth}{\linethickness}\end{center}

\subsection{Spatial properties
operators}\label{spatial-properties-operators-5}

\href{OperatorsBC\#covers}{covers},
\href{OperatorsBC\#crosses}{crosses},
\href{OperatorsIM\#intersects}{intersects},
\href{OperatorsNR\#partially_overlaps}{partially\_overlaps},
\href{OperatorsSZ\#touches}{touches},

\begin{center}\rule{0.5\linewidth}{\linethickness}\end{center}

\subsection{Spatial queries
operators}\label{spatial-queries-operators-5}

\href{OperatorsAA\#agent_closest_to}{agent\_closest\_to},
\href{OperatorsAA\#agent_farthest_to}{agent\_farthest\_to},
\href{OperatorsAA\#agents_at_distance}{agents\_at\_distance},
\href{OperatorsAA\#agents_inside}{agents\_inside},
\href{OperatorsAA\#agents_overlapping}{agents\_overlapping},
\href{OperatorsAA\#at_distance}{at\_distance},
\href{OperatorsBC\#closest_to}{closest\_to},
\href{OperatorsDH\#farthest_to}{farthest\_to},
\href{OperatorsIM\#inside}{inside},
\href{OperatorsNR\#neighbors_at}{neighbors\_at},
\href{OperatorsNR\#neighbors_of}{neighbors\_of},
\href{OperatorsNR\#overlapping}{overlapping},

\begin{center}\rule{0.5\linewidth}{\linethickness}\end{center}

\subsection{Spatial relations
operators}\label{spatial-relations-operators-5}

\href{OperatorsDH\#direction_between}{direction\_between},
\href{OperatorsDH\#distance_between}{distance\_between},
\href{OperatorsDH\#distance_to}{distance\_to},
\href{OperatorsNR\#path_between}{path\_between},
\href{OperatorsNR\#path_to}{path\_to},
\href{OperatorsSZ\#towards}{towards},

\begin{center}\rule{0.5\linewidth}{\linethickness}\end{center}

\subsection{Spatial statistical
operators}\label{spatial-statistical-operators-5}

\href{OperatorsDH\#hierarchical_clustering}{hierarchical\_clustering},
\href{OperatorsSZ\#simple_clustering_by_distance}{simple\_clustering\_by\_distance},

\begin{center}\rule{0.5\linewidth}{\linethickness}\end{center}

\subsection{Spatial transformations
operators}\label{spatial-transformations-operators-5}

\href{OperatorsAA\#-}{-}, \href{OperatorsAA\#*}{*},
\href{OperatorsAA\#+}{+}, \href{OperatorsAA\#as_4_grid}{as\_4\_grid},
\href{OperatorsAA\#as_grid}{as\_grid},
\href{OperatorsAA\#as_hexagonal_grid}{as\_hexagonal\_grid},
\href{OperatorsAA\#at_location}{at\_location},
\href{OperatorsBC\#clean}{clean},
\href{OperatorsBC\#clean_network}{clean\_network},
\href{OperatorsBC\#convex_hull}{convex\_hull},
\href{OperatorsBC\#crs_transform}{CRS\_transform},
\href{OperatorsNR\#rotated_by}{rotated\_by},
\href{OperatorsSZ\#scaled_to}{scaled\_to},
\href{OperatorsSZ\#simplification}{simplification},
\href{OperatorsSZ\#skeletonize}{skeletonize},
\href{OperatorsSZ\#smooth}{smooth},
\href{OperatorsSZ\#split_geometry}{split\_geometry},
\href{OperatorsSZ\#split_lines}{split\_lines},
\href{OperatorsSZ\#to_gama_crs}{to\_GAMA\_CRS},
\href{OperatorsSZ\#to_rectangles}{to\_rectangles},
\href{OperatorsSZ\#to_squares}{to\_squares},
\href{OperatorsSZ\#to_sub_geometries}{to\_sub\_geometries},
\href{OperatorsSZ\#transformed_by}{transformed\_by},
\href{OperatorsSZ\#translated_by}{translated\_by},
\href{OperatorsSZ\#triangulate}{triangulate},
\href{OperatorsSZ\#voronoi}{voronoi},
\href{OperatorsSZ\#with_precision}{with\_precision},
\href{OperatorsSZ\#without_holes}{without\_holes},

\begin{center}\rule{0.5\linewidth}{\linethickness}\end{center}

\subsection{Species-related
operators}\label{species-related-operators-5}

\href{OperatorsIM\#index_of}{index\_of},
\href{OperatorsIM\#last_index_of}{last\_index\_of},
\href{OperatorsNR\#of_generic_species}{of\_generic\_species},
\href{OperatorsNR\#of_species}{of\_species},

\begin{center}\rule{0.5\linewidth}{\linethickness}\end{center}

\subsection{Statistical operators}\label{statistical-operators-5}

\href{OperatorsBC\#build}{build}, \href{OperatorsBC\#corr}{corR},
\href{OperatorsDH\#dbscan}{dbscan},
\href{OperatorsDH\#distribution_of}{distribution\_of},
\href{OperatorsDH\#distribution2d_of}{distribution2d\_of},
\href{OperatorsDH\#dtw}{dtw},
\href{OperatorsDH\#frequency_of}{frequency\_of},
\href{OperatorsDH\#gamma_rnd}{gamma\_rnd},
\href{OperatorsDH\#geometric_mean}{geometric\_mean},
\href{OperatorsDH\#gini}{gini},
\href{OperatorsDH\#harmonic_mean}{harmonic\_mean},
\href{OperatorsDH\#hierarchical_clustering}{hierarchical\_clustering},
\href{OperatorsIM\#kmeans}{kmeans},
\href{OperatorsIM\#kurtosis}{kurtosis}, \href{OperatorsIM\#max}{max},
\href{OperatorsIM\#mean}{mean},
\href{OperatorsIM\#mean_deviation}{mean\_deviation},
\href{OperatorsIM\#meanr}{meanR}, \href{OperatorsIM\#median}{median},
\href{OperatorsIM\#min}{min}, \href{OperatorsIM\#moran}{moran},
\href{OperatorsIM\#mul}{mul}, \href{OperatorsNR\#predict}{predict},
\href{OperatorsSZ\#simple_clustering_by_distance}{simple\_clustering\_by\_distance},
\href{OperatorsSZ\#skewness}{skewness},
\href{OperatorsSZ\#split}{split},
\href{OperatorsSZ\#split_in}{split\_in},
\href{OperatorsSZ\#split_using}{split\_using},
\href{OperatorsSZ\#standard_deviation}{standard\_deviation},
\href{OperatorsSZ\#sum}{sum}, \href{OperatorsSZ\#variance}{variance},

\begin{center}\rule{0.5\linewidth}{\linethickness}\end{center}

\subsection{Strings-related
operators}\label{strings-related-operators-5}

\href{OperatorsAA\#+}{+}, \href{OperatorsAA\#\%3C}{\textless{}},
\href{OperatorsAA\#\%3C=}{\textless{}=},
\href{OperatorsAA\#\%3E}{\textgreater{}},
\href{OperatorsAA\#\%3E=}{\textgreater{}=}, \href{OperatorsAA\#at}{at},
\href{OperatorsBC\#char}{char}, \href{OperatorsBC\#contains}{contains},
\href{OperatorsBC\#contains_all}{contains\_all},
\href{OperatorsBC\#contains_any}{contains\_any},
\href{OperatorsBC\#copy_between}{copy\_between},
\href{OperatorsDH\#date}{date}, \href{OperatorsDH\#empty}{empty},
\href{OperatorsDH\#first}{first}, \href{OperatorsIM\#in}{in},
\href{OperatorsIM\#indented_by}{indented\_by},
\href{OperatorsIM\#index_of}{index\_of},
\href{OperatorsIM\#is_number}{is\_number},
\href{OperatorsIM\#last}{last},
\href{OperatorsIM\#last_index_of}{last\_index\_of},
\href{OperatorsIM\#length}{length},
\href{OperatorsIM\#lower_case}{lower\_case},
\href{OperatorsNR\#replace}{replace},
\href{OperatorsNR\#replace_regex}{replace\_regex},
\href{OperatorsNR\#reverse}{reverse},
\href{OperatorsSZ\#sample}{sample},
\href{OperatorsSZ\#shuffle}{shuffle},
\href{OperatorsSZ\#split_with}{split\_with},
\href{OperatorsSZ\#string}{string},
\href{OperatorsSZ\#upper_case}{upper\_case},

\begin{center}\rule{0.5\linewidth}{\linethickness}\end{center}

\subsection{System}\label{system-5}

\href{OperatorsAA\#.}{.}, \href{OperatorsBC\#command}{command},
\href{OperatorsBC\#copy}{copy}, \href{OperatorsDH\#dead}{dead},
\href{OperatorsDH\#eval_gaml}{eval\_gaml},
\href{OperatorsDH\#every}{every},
\href{OperatorsIM\#is_error}{is\_error},
\href{OperatorsIM\#is_warning}{is\_warning},
\href{OperatorsSZ\#user_input}{user\_input},

\begin{center}\rule{0.5\linewidth}{\linethickness}\end{center}

\subsection{Time-related operators}\label{time-related-operators-5}

\href{OperatorsDH\#date}{date}, \href{OperatorsSZ\#string}{string},

\begin{center}\rule{0.5\linewidth}{\linethickness}\end{center}

\subsection{Types-related operators}\label{types-related-operators-5}

\begin{center}\rule{0.5\linewidth}{\linethickness}\end{center}

\subsection{User control operators}\label{user-control-operators-5}

\href{OperatorsSZ\#user_input}{user\_input},

\begin{center}\rule{0.5\linewidth}{\linethickness}\end{center}

\section{Operators}\label{operators-5}

\begin{center}\rule{0.5\linewidth}{\linethickness}\end{center}

\subsection{\texorpdfstring{\texttt{sample}}{sample}}\label{sample}

\subsubsection{Possible use:}\label{possible-use-448}

\begin{itemize}
\tightlist
\item
  \textbf{\texttt{sample}} (\texttt{any\ expression}) ---\textgreater{}
  \texttt{string}
\item
  \texttt{string} \textbf{\texttt{sample}} \texttt{any\ expression}
  ---\textgreater{} \texttt{string}
\item
  \textbf{\texttt{sample}} (\texttt{string} , \texttt{any\ expression})
  ---\textgreater{} \texttt{string}
\item
  \textbf{\texttt{sample}} (\texttt{list}, \texttt{int}, \texttt{bool})
  ---\textgreater{} \texttt{list}
\item
  \textbf{\texttt{sample}} (\texttt{list}, \texttt{int}, \texttt{bool},
  \texttt{list}) ---\textgreater{} \texttt{list}
\end{itemize}

\subsubsection{Result:}\label{result-434}

takes a sample of the specified size from the elements of x using either
with or without replacement with given weights takes a sample of the
specified size from the elements of x using either with or without
replacement

\subsubsection{Examples:}\label{examples-309}

\begin{verbatim}
 
list var0 <- sample([2,10,1],2,false,[0.1,0.7,0.2]); // var0 equals [10,2] 
list var1 <- sample([2,10,1],2,false); // var1 equals [1,2]
\end{verbatim}

\begin{center}\rule{0.5\linewidth}{\linethickness}\end{center}

\subsection{\texorpdfstring{\texttt{Sanction}}{Sanction}}\label{sanction}

\subsubsection{Possible use:}\label{possible-use-449}

\begin{itemize}
\tightlist
\item
  \textbf{\texttt{Sanction}} (\texttt{any}) ---\textgreater{}
  \texttt{Sanction}
\end{itemize}

\subsubsection{Result:}\label{result-435}

Casts the operand into the type Sanction

\begin{center}\rule{0.5\linewidth}{\linethickness}\end{center}

\subsection{\texorpdfstring{\texttt{saveSimulation}}{saveSimulation}}\label{savesimulation}

\subsubsection{Possible use:}\label{possible-use-450}

\begin{itemize}
\tightlist
\item
  \textbf{\texttt{saveSimulation}} (\texttt{string}) ---\textgreater{}
  \texttt{int}
\end{itemize}

\begin{center}\rule{0.5\linewidth}{\linethickness}\end{center}

\subsection{\texorpdfstring{\texttt{scaled\_by}}{scaled\_by}}\label{scaled_by}

Same signification as \href{OperatorsAA\#*}{*}

\begin{center}\rule{0.5\linewidth}{\linethickness}\end{center}

\subsection{\texorpdfstring{\texttt{scaled\_to}}{scaled\_to}}\label{scaled_to}

\subsubsection{Possible use:}\label{possible-use-451}

\begin{itemize}
\tightlist
\item
  \texttt{geometry} \textbf{\texttt{scaled\_to}} \texttt{point}
  ---\textgreater{} \texttt{geometry}
\item
  \textbf{\texttt{scaled\_to}} (\texttt{geometry} , \texttt{point})
  ---\textgreater{} \texttt{geometry}
\end{itemize}

\subsubsection{Result:}\label{result-436}

allows to restrict the size of a geometry so that it fits in the
envelope \{width, height, depth\} defined by the second operand

\subsubsection{Examples:}\label{examples-310}

\begin{verbatim}
 
geometry var0 <- shape scaled_to {10,10}; // var0 equals a geometry corresponding to the geometry of the agent applying the operator scaled so that it fits a square of 10x10
\end{verbatim}

\begin{center}\rule{0.5\linewidth}{\linethickness}\end{center}

\subsection{\texorpdfstring{\texttt{select}}{select}}\label{select}

Same signification as \href{OperatorsSZ\#where}{where}

\begin{center}\rule{0.5\linewidth}{\linethickness}\end{center}

\subsection{\texorpdfstring{\texttt{serialize}}{serialize}}\label{serialize}

\subsubsection{Possible use:}\label{possible-use-452}

\begin{itemize}
\tightlist
\item
  \textbf{\texttt{serialize}} (\texttt{unknown}) ---\textgreater{}
  \texttt{string}
\end{itemize}

\subsubsection{Result:}\label{result-437}

It serializes any object, i.e.~transform it into a string.

\begin{center}\rule{0.5\linewidth}{\linethickness}\end{center}

\subsection{\texorpdfstring{\texttt{serializeAgent}}{serializeAgent}}\label{serializeagent}

\subsubsection{Possible use:}\label{possible-use-453}

\begin{itemize}
\tightlist
\item
  \textbf{\texttt{serializeAgent}} (\texttt{agent}) ---\textgreater{}
  \texttt{string}
\end{itemize}

\begin{center}\rule{0.5\linewidth}{\linethickness}\end{center}

\subsection{\texorpdfstring{\texttt{set\_about}}{set\_about}}\label{set_about}

\subsubsection{Possible use:}\label{possible-use-454}

\begin{itemize}
\tightlist
\item
  \texttt{emotion} \textbf{\texttt{set\_about}} \texttt{predicate}
  ---\textgreater{} \texttt{emotion}
\item
  \textbf{\texttt{set\_about}} (\texttt{emotion} , \texttt{predicate})
  ---\textgreater{} \texttt{emotion}
\end{itemize}

\subsubsection{Result:}\label{result-438}

change the about value of the given emotion

\subsubsection{Examples:}\label{examples-311}

\begin{verbatim}
emotion set_about predicate1 
\end{verbatim}

\begin{center}\rule{0.5\linewidth}{\linethickness}\end{center}

\subsection{\texorpdfstring{\texttt{set\_agent}}{set\_agent}}\label{set_agent}

\subsubsection{Possible use:}\label{possible-use-455}

\begin{itemize}
\tightlist
\item
  \texttt{msi.gaml.architecture.simplebdi.SocialLink}
  \textbf{\texttt{set\_agent}} \texttt{agent} ---\textgreater{}
  \texttt{msi.gaml.architecture.simplebdi.SocialLink}
\item
  \textbf{\texttt{set\_agent}}
  (\texttt{msi.gaml.architecture.simplebdi.SocialLink} , \texttt{agent})
  ---\textgreater{} \texttt{msi.gaml.architecture.simplebdi.SocialLink}
\end{itemize}

\subsubsection{Result:}\label{result-439}

change the agent value of the given social link

\subsubsection{Examples:}\label{examples-312}

\begin{verbatim}
social_link set_agent agentA 
\end{verbatim}

\begin{center}\rule{0.5\linewidth}{\linethickness}\end{center}

\subsection{\texorpdfstring{\texttt{set\_agent\_cause}}{set\_agent\_cause}}\label{set_agent_cause}

\subsubsection{Possible use:}\label{possible-use-456}

\begin{itemize}
\tightlist
\item
  \texttt{predicate} \textbf{\texttt{set\_agent\_cause}} \texttt{agent}
  ---\textgreater{} \texttt{predicate}
\item
  \textbf{\texttt{set\_agent\_cause}} (\texttt{predicate} ,
  \texttt{agent}) ---\textgreater{} \texttt{predicate}
\item
  \texttt{emotion} \textbf{\texttt{set\_agent\_cause}} \texttt{agent}
  ---\textgreater{} \texttt{emotion}
\item
  \textbf{\texttt{set\_agent\_cause}} (\texttt{emotion} ,
  \texttt{agent}) ---\textgreater{} \texttt{emotion}
\end{itemize}

\subsubsection{Result:}\label{result-440}

change the agentCause value of the given predicate change the agentCause
value of the given emotion

\subsubsection{Examples:}\label{examples-313}

\begin{verbatim}
predicate set_agent_cause agentA emotion set_agent_cause agentA 
\end{verbatim}

\begin{center}\rule{0.5\linewidth}{\linethickness}\end{center}

\subsection{\texorpdfstring{\texttt{set\_decay}}{set\_decay}}\label{set_decay}

\subsubsection{Possible use:}\label{possible-use-457}

\begin{itemize}
\tightlist
\item
  \texttt{emotion} \textbf{\texttt{set\_decay}} \texttt{float}
  ---\textgreater{} \texttt{emotion}
\item
  \textbf{\texttt{set\_decay}} (\texttt{emotion} , \texttt{float})
  ---\textgreater{} \texttt{emotion}
\end{itemize}

\subsubsection{Result:}\label{result-441}

change the decay value of the given emotion

\subsubsection{Examples:}\label{examples-314}

\begin{verbatim}
emotion set_decay 12 
\end{verbatim}

\begin{center}\rule{0.5\linewidth}{\linethickness}\end{center}

\subsection{\texorpdfstring{\texttt{set\_dominance}}{set\_dominance}}\label{set_dominance}

\subsubsection{Possible use:}\label{possible-use-458}

\begin{itemize}
\tightlist
\item
  \texttt{msi.gaml.architecture.simplebdi.SocialLink}
  \textbf{\texttt{set\_dominance}} \texttt{float} ---\textgreater{}
  \texttt{msi.gaml.architecture.simplebdi.SocialLink}
\item
  \textbf{\texttt{set\_dominance}}
  (\texttt{msi.gaml.architecture.simplebdi.SocialLink} , \texttt{float})
  ---\textgreater{} \texttt{msi.gaml.architecture.simplebdi.SocialLink}
\end{itemize}

\subsubsection{Result:}\label{result-442}

change the dominance value of the given social link

\subsubsection{Examples:}\label{examples-315}

\begin{verbatim}
social_link set_dominance 0.4 
\end{verbatim}

\begin{center}\rule{0.5\linewidth}{\linethickness}\end{center}

\subsection{\texorpdfstring{\texttt{set\_familiarity}}{set\_familiarity}}\label{set_familiarity}

\subsubsection{Possible use:}\label{possible-use-459}

\begin{itemize}
\tightlist
\item
  \texttt{msi.gaml.architecture.simplebdi.SocialLink}
  \textbf{\texttt{set\_familiarity}} \texttt{float} ---\textgreater{}
  \texttt{msi.gaml.architecture.simplebdi.SocialLink}
\item
  \textbf{\texttt{set\_familiarity}}
  (\texttt{msi.gaml.architecture.simplebdi.SocialLink} , \texttt{float})
  ---\textgreater{} \texttt{msi.gaml.architecture.simplebdi.SocialLink}
\end{itemize}

\subsubsection{Result:}\label{result-443}

change the familiarity value of the given social link

\subsubsection{Examples:}\label{examples-316}

\begin{verbatim}
social_link set_familiarity 0.4 
\end{verbatim}

\begin{center}\rule{0.5\linewidth}{\linethickness}\end{center}

\subsection{\texorpdfstring{\texttt{set\_intensity}}{set\_intensity}}\label{set_intensity}

\subsubsection{Possible use:}\label{possible-use-460}

\begin{itemize}
\tightlist
\item
  \texttt{emotion} \textbf{\texttt{set\_intensity}} \texttt{float}
  ---\textgreater{} \texttt{emotion}
\item
  \textbf{\texttt{set\_intensity}} (\texttt{emotion} , \texttt{float})
  ---\textgreater{} \texttt{emotion}
\end{itemize}

\subsubsection{Result:}\label{result-444}

change the intensity value of the given emotion

\subsubsection{Examples:}\label{examples-317}

\begin{verbatim}
emotion set_intensity 12 
\end{verbatim}

\begin{center}\rule{0.5\linewidth}{\linethickness}\end{center}

\subsection{\texorpdfstring{\texttt{set\_lifetime}}{set\_lifetime}}\label{set_lifetime}

\subsubsection{Possible use:}\label{possible-use-461}

\begin{itemize}
\tightlist
\item
  \texttt{mental\_state} \textbf{\texttt{set\_lifetime}} \texttt{int}
  ---\textgreater{} \texttt{mental\_state}
\item
  \textbf{\texttt{set\_lifetime}} (\texttt{mental\_state} ,
  \texttt{int}) ---\textgreater{} \texttt{mental\_state}
\end{itemize}

\subsubsection{Result:}\label{result-445}

change the lifetime value of the given mental state

\subsubsection{Examples:}\label{examples-318}

\begin{verbatim}
mental state set_lifetime 1 
\end{verbatim}

\begin{center}\rule{0.5\linewidth}{\linethickness}\end{center}

\subsection{\texorpdfstring{\texttt{set\_liking}}{set\_liking}}\label{set_liking}

\subsubsection{Possible use:}\label{possible-use-462}

\begin{itemize}
\tightlist
\item
  \texttt{msi.gaml.architecture.simplebdi.SocialLink}
  \textbf{\texttt{set\_liking}} \texttt{float} ---\textgreater{}
  \texttt{msi.gaml.architecture.simplebdi.SocialLink}
\item
  \textbf{\texttt{set\_liking}}
  (\texttt{msi.gaml.architecture.simplebdi.SocialLink} , \texttt{float})
  ---\textgreater{} \texttt{msi.gaml.architecture.simplebdi.SocialLink}
\end{itemize}

\subsubsection{Result:}\label{result-446}

change the liking value of the given social link

\subsubsection{Examples:}\label{examples-319}

\begin{verbatim}
social_link set_liking 0.4 
\end{verbatim}

\begin{center}\rule{0.5\linewidth}{\linethickness}\end{center}

\subsection{\texorpdfstring{\texttt{set\_modality}}{set\_modality}}\label{set_modality}

\subsubsection{Possible use:}\label{possible-use-463}

\begin{itemize}
\tightlist
\item
  \texttt{mental\_state} \textbf{\texttt{set\_modality}} \texttt{string}
  ---\textgreater{} \texttt{mental\_state}
\item
  \textbf{\texttt{set\_modality}} (\texttt{mental\_state} ,
  \texttt{string}) ---\textgreater{} \texttt{mental\_state}
\end{itemize}

\subsubsection{Result:}\label{result-447}

change the modality value of the given mental state

\subsubsection{Examples:}\label{examples-320}

\begin{verbatim}
mental state set_modality belief 
\end{verbatim}

\begin{center}\rule{0.5\linewidth}{\linethickness}\end{center}

\subsection{\texorpdfstring{\texttt{set\_predicate}}{set\_predicate}}\label{set_predicate}

\subsubsection{Possible use:}\label{possible-use-464}

\begin{itemize}
\tightlist
\item
  \texttt{mental\_state} \textbf{\texttt{set\_predicate}}
  \texttt{predicate} ---\textgreater{} \texttt{mental\_state}
\item
  \textbf{\texttt{set\_predicate}} (\texttt{mental\_state} ,
  \texttt{predicate}) ---\textgreater{} \texttt{mental\_state}
\end{itemize}

\subsubsection{Result:}\label{result-448}

change the predicate value of the given mental state

\subsubsection{Examples:}\label{examples-321}

\begin{verbatim}
mental state set_predicate pred1 
\end{verbatim}

\begin{center}\rule{0.5\linewidth}{\linethickness}\end{center}

\subsection{\texorpdfstring{\texttt{set\_solidarity}}{set\_solidarity}}\label{set_solidarity}

\subsubsection{Possible use:}\label{possible-use-465}

\begin{itemize}
\tightlist
\item
  \texttt{msi.gaml.architecture.simplebdi.SocialLink}
  \textbf{\texttt{set\_solidarity}} \texttt{float} ---\textgreater{}
  \texttt{msi.gaml.architecture.simplebdi.SocialLink}
\item
  \textbf{\texttt{set\_solidarity}}
  (\texttt{msi.gaml.architecture.simplebdi.SocialLink} , \texttt{float})
  ---\textgreater{} \texttt{msi.gaml.architecture.simplebdi.SocialLink}
\end{itemize}

\subsubsection{Result:}\label{result-449}

change the solidarity value of the given social link

\subsubsection{Examples:}\label{examples-322}

\begin{verbatim}
social_link set_solidarity 0.4 
\end{verbatim}

\begin{center}\rule{0.5\linewidth}{\linethickness}\end{center}

\subsection{\texorpdfstring{\texttt{set\_strength}}{set\_strength}}\label{set_strength}

\subsubsection{Possible use:}\label{possible-use-466}

\begin{itemize}
\tightlist
\item
  \texttt{mental\_state} \textbf{\texttt{set\_strength}} \texttt{float}
  ---\textgreater{} \texttt{mental\_state}
\item
  \textbf{\texttt{set\_strength}} (\texttt{mental\_state} ,
  \texttt{float}) ---\textgreater{} \texttt{mental\_state}
\end{itemize}

\subsubsection{Result:}\label{result-450}

change the strength value of the given mental state

\subsubsection{Examples:}\label{examples-323}

\begin{verbatim}
mental state set_strength 1.0 
\end{verbatim}

\begin{center}\rule{0.5\linewidth}{\linethickness}\end{center}

\subsection{\texorpdfstring{\texttt{set\_trust}}{set\_trust}}\label{set_trust}

\subsubsection{Possible use:}\label{possible-use-467}

\begin{itemize}
\tightlist
\item
  \texttt{msi.gaml.architecture.simplebdi.SocialLink}
  \textbf{\texttt{set\_trust}} \texttt{float} ---\textgreater{}
  \texttt{msi.gaml.architecture.simplebdi.SocialLink}
\item
  \textbf{\texttt{set\_trust}}
  (\texttt{msi.gaml.architecture.simplebdi.SocialLink} , \texttt{float})
  ---\textgreater{} \texttt{msi.gaml.architecture.simplebdi.SocialLink}
\end{itemize}

\subsubsection{Result:}\label{result-451}

change the trust value of the given social link

\subsubsection{Examples:}\label{examples-324}

\begin{verbatim}
social_link set_familiarity 0.4 
\end{verbatim}

\begin{center}\rule{0.5\linewidth}{\linethickness}\end{center}

\subsection{\texorpdfstring{\texttt{set\_truth}}{set\_truth}}\label{set_truth}

\subsubsection{Possible use:}\label{possible-use-468}

\begin{itemize}
\tightlist
\item
  \texttt{predicate} \textbf{\texttt{set\_truth}} \texttt{bool}
  ---\textgreater{} \texttt{predicate}
\item
  \textbf{\texttt{set\_truth}} (\texttt{predicate} , \texttt{bool})
  ---\textgreater{} \texttt{predicate}
\end{itemize}

\subsubsection{Result:}\label{result-452}

change the is\_true value of the given predicate

\subsubsection{Examples:}\label{examples-325}

\begin{verbatim}
predicate set_truth false 
\end{verbatim}

\begin{center}\rule{0.5\linewidth}{\linethickness}\end{center}

\subsection{\texorpdfstring{\texttt{set\_z}}{set\_z}}\label{set_z}

\subsubsection{Possible use:}\label{possible-use-469}

\begin{itemize}
\tightlist
\item
  \texttt{geometry} \textbf{\texttt{set\_z}}
  \texttt{container\textless{}float\textgreater{}} ---\textgreater{}
  \texttt{geometry}
\item
  \textbf{\texttt{set\_z}} (\texttt{geometry} ,
  \texttt{container\textless{}float\textgreater{}}) ---\textgreater{}
  \texttt{geometry}
\item
  \textbf{\texttt{set\_z}} (\texttt{geometry}, \texttt{int},
  \texttt{float}) ---\textgreater{} \texttt{geometry}
\end{itemize}

\subsubsection{Result:}\label{result-453}

Sets the z ordinate of the n-th point of a geometry to the value
provided by the third argument

\subsubsection{Examples:}\label{examples-326}

\begin{verbatim}
loop i from: 0 to: length(shape.points) - 1{set shape <-  set_z (shape, i, 3.0);} shape <- triangle(3) set_z [5,10,14]; 
\end{verbatim}

\begin{center}\rule{0.5\linewidth}{\linethickness}\end{center}

\subsection{\texorpdfstring{\texttt{shape\_file}}{shape\_file}}\label{shape_file}

\subsubsection{Possible use:}\label{possible-use-470}

\begin{itemize}
\tightlist
\item
  \textbf{\texttt{shape\_file}} (\texttt{string}) ---\textgreater{}
  \texttt{file}
\end{itemize}

\subsubsection{Result:}\label{result-454}

Constructs a file of type shape. Allowed extensions are limited to shp

\begin{center}\rule{0.5\linewidth}{\linethickness}\end{center}

\subsection{\texorpdfstring{\texttt{shuffle}}{shuffle}}\label{shuffle}

\subsubsection{Possible use:}\label{possible-use-471}

\begin{itemize}
\tightlist
\item
  \textbf{\texttt{shuffle}} (\texttt{container}) ---\textgreater{}
  \texttt{list}
\item
  \textbf{\texttt{shuffle}} (\texttt{matrix}) ---\textgreater{}
  \texttt{matrix}
\item
  \textbf{\texttt{shuffle}} (\texttt{string}) ---\textgreater{}
  \texttt{string}
\end{itemize}

\subsubsection{Result:}\label{result-455}

The elements of the operand in random order.

\subsubsection{Special cases:}\label{special-cases-121}

\begin{itemize}
\tightlist
\item
  if the operand is empty, returns an empty list (or string, matrix)
\end{itemize}

\subsubsection{Examples:}\label{examples-327}

\begin{verbatim}
 
list var0 <- shuffle ([12, 13, 14]); // var0 equals [14,12,13] (for example) 
matrix var1 <- shuffle (matrix([["c11","c12","c13"],["c21","c22","c23"]])); // var1 equals matrix([["c12","c21","c11"],["c13","c22","c23"]]) (for example) 
string var2 <- shuffle ('abc'); // var2 equals 'bac' (for example)
\end{verbatim}

\subsubsection{See also:}\label{see-also-183}

\href{OperatorsNR\#reverse}{reverse},

\begin{center}\rule{0.5\linewidth}{\linethickness}\end{center}

\subsection{\texorpdfstring{\texttt{signum}}{signum}}\label{signum}

\subsubsection{Possible use:}\label{possible-use-472}

\begin{itemize}
\tightlist
\item
  \textbf{\texttt{signum}} (\texttt{float}) ---\textgreater{}
  \texttt{int}
\end{itemize}

\subsubsection{Result:}\label{result-456}

Returns -1 if the argument is negative, +1 if it is positive, 0 if it is
equal to zero or not a number

\subsubsection{Examples:}\label{examples-328}

\begin{verbatim}
 
int var0 <- signum(-12); // var0 equals -1 
int var1 <- signum(14); // var1 equals 1 
int var2 <- signum(0); // var2 equals 0
\end{verbatim}

\begin{center}\rule{0.5\linewidth}{\linethickness}\end{center}

\subsection{\texorpdfstring{\texttt{simple\_clustering\_by\_distance}}{simple\_clustering\_by\_distance}}\label{simple_clustering_by_distance}

\subsubsection{Possible use:}\label{possible-use-473}

\begin{itemize}
\tightlist
\item
  \texttt{container\textless{}agent\textgreater{}}
  \textbf{\texttt{simple\_clustering\_by\_distance}} \texttt{float}
  ---\textgreater{}
  \texttt{list\textless{}list\textless{}agent\textgreater{}\textgreater{}}
\item
  \textbf{\texttt{simple\_clustering\_by\_distance}}
  (\texttt{container\textless{}agent\textgreater{}} , \texttt{float})
  ---\textgreater{}
  \texttt{list\textless{}list\textless{}agent\textgreater{}\textgreater{}}
\end{itemize}

\subsubsection{Result:}\label{result-457}

A list of agent groups clustered by distance considering a distance min
between two groups.

\subsubsection{Examples:}\label{examples-329}

\begin{verbatim}
 
list<list<agent>> var0 <- [ag1, ag2, ag3, ag4, ag5] simpleClusteringByDistance 20.0; // var0 equals for example, can return [[ag1, ag3], [ag2], [ag4, ag5]]
\end{verbatim}

\subsubsection{See also:}\label{see-also-184}

\href{OperatorsDH\#hierarchical_clustering}{hierarchical\_clustering},

\begin{center}\rule{0.5\linewidth}{\linethickness}\end{center}

\subsection{\texorpdfstring{\texttt{simple\_clustering\_by\_envelope\_distance}}{simple\_clustering\_by\_envelope\_distance}}\label{simple_clustering_by_envelope_distance}

Same signification as
\href{OperatorsSZ\#simple_clustering_by_distance}{simple\_clustering\_by\_distance}

\begin{center}\rule{0.5\linewidth}{\linethickness}\end{center}

\subsection{\texorpdfstring{\texttt{simplex\_generator}}{simplex\_generator}}\label{simplex_generator}

\subsubsection{Possible use:}\label{possible-use-474}

\begin{itemize}
\tightlist
\item
  \textbf{\texttt{simplex\_generator}} (\texttt{float}, \texttt{float},
  \texttt{float}) ---\textgreater{} \texttt{float}
\end{itemize}

\subsubsection{Result:}\label{result-458}

take a x, y and a bias parameters and gives a value

\subsubsection{Examples:}\label{examples-330}

\begin{verbatim}
 
float var0 <- simplex_generator(2,3,253); // var0 equals 10.2
\end{verbatim}

\begin{center}\rule{0.5\linewidth}{\linethickness}\end{center}

\subsection{\texorpdfstring{\texttt{simplification}}{simplification}}\label{simplification}

\subsubsection{Possible use:}\label{possible-use-475}

\begin{itemize}
\tightlist
\item
  \texttt{geometry} \textbf{\texttt{simplification}} \texttt{float}
  ---\textgreater{} \texttt{geometry}
\item
  \textbf{\texttt{simplification}} (\texttt{geometry} , \texttt{float})
  ---\textgreater{} \texttt{geometry}
\end{itemize}

\subsubsection{Result:}\label{result-459}

A geometry corresponding to the simplification of the operand (geometry,
agent, point) considering a tolerance distance.

\subsubsection{Comment:}\label{comment-88}

The algorithm used for the simplification is Douglas-Peucker

\subsubsection{Examples:}\label{examples-331}

\begin{verbatim}
 
geometry var0 <- self simplification 0.1; // var0 equals the geometry resulting from the application of the Douglas-Peuker algorithm on the geometry of the agent applying the operator with a tolerance distance of 0.1.
\end{verbatim}

\begin{center}\rule{0.5\linewidth}{\linethickness}\end{center}

\subsection{\texorpdfstring{\texttt{sin}}{sin}}\label{sin}

\subsubsection{Possible use:}\label{possible-use-476}

\begin{itemize}
\tightlist
\item
  \textbf{\texttt{sin}} (\texttt{int}) ---\textgreater{} \texttt{float}
\item
  \textbf{\texttt{sin}} (\texttt{float}) ---\textgreater{}
  \texttt{float}
\end{itemize}

\subsubsection{Result:}\label{result-460}

Returns the value (in {[}-1,1{]}) of the sinus of the operand (in
decimal degrees). The argument is casted to an int before being
evaluated.

\subsubsection{Special cases:}\label{special-cases-122}

\begin{itemize}
\tightlist
\item
  Operand values out of the range {[}0-359{]} are normalized.
\end{itemize}

\subsubsection{Examples:}\label{examples-332}

\begin{verbatim}
 
float var0 <- sin (0); // var0 equals 0.0 
float var1 <- sin(360) with_precision 10 with_precision 10; // var1 equals 0.0
\end{verbatim}

\subsubsection{See also:}\label{see-also-185}

\href{OperatorsBC\#cos}{cos}, \href{OperatorsSZ\#tan}{tan},

\begin{center}\rule{0.5\linewidth}{\linethickness}\end{center}

\subsection{\texorpdfstring{\texttt{sin\_rad}}{sin\_rad}}\label{sin_rad}

\subsubsection{Possible use:}\label{possible-use-477}

\begin{itemize}
\tightlist
\item
  \textbf{\texttt{sin\_rad}} (\texttt{float}) ---\textgreater{}
  \texttt{float}
\end{itemize}

\subsubsection{Result:}\label{result-461}

Returns the value (in {[}-1,1{]}) of the sinus of the operand (in
radians).

\subsubsection{Examples:}\label{examples-333}

\begin{verbatim}
 
float var0 <- sin_rad(#pi); // var0 equals 0.0
\end{verbatim}

\subsubsection{See also:}\label{see-also-186}

\href{OperatorsBC\#cos_rad}{cos\_rad},
\href{OperatorsSZ\#tan_rad}{tan\_rad},

\begin{center}\rule{0.5\linewidth}{\linethickness}\end{center}

\subsection{\texorpdfstring{\texttt{since}}{since}}\label{since}

\subsubsection{Possible use:}\label{possible-use-478}

\begin{itemize}
\tightlist
\item
  \textbf{\texttt{since}} (\texttt{date}) ---\textgreater{}
  \texttt{bool}
\item
  \texttt{any\ expression} \textbf{\texttt{since}} \texttt{date}
  ---\textgreater{} \texttt{bool}
\item
  \textbf{\texttt{since}} (\texttt{any\ expression} , \texttt{date})
  ---\textgreater{} \texttt{bool}
\end{itemize}

\subsubsection{Result:}\label{result-462}

Returns true if the current\_date of the model is after (or equal to)
the date passed in argument. Synonym of `current\_date \textgreater{}=
argument'. Can be used, like `after', in its composed form with 2
arguments to express the lowest boundary of the computation of a
frequency. However, contrary to `after', there is a subtle difference:
the lowest boundary will be tested against the frequency as well

\subsubsection{Examples:}\label{examples-334}

\begin{verbatim}
reflex when: since(starting_date) {}    // this reflex will always be run every(2#days) since (starting_date + 1#day) // the computation will return true 1 day after the starting date and every two days after this reference date 
\end{verbatim}

\begin{center}\rule{0.5\linewidth}{\linethickness}\end{center}

\subsection{\texorpdfstring{\texttt{skeletonize}}{skeletonize}}\label{skeletonize}

\subsubsection{Possible use:}\label{possible-use-479}

\begin{itemize}
\tightlist
\item
  \textbf{\texttt{skeletonize}} (\texttt{geometry}) ---\textgreater{}
  \texttt{list\textless{}geometry\textgreater{}}
\item
  \texttt{geometry} \textbf{\texttt{skeletonize}} \texttt{float}
  ---\textgreater{} \texttt{list\textless{}geometry\textgreater{}}
\item
  \textbf{\texttt{skeletonize}} (\texttt{geometry} , \texttt{float})
  ---\textgreater{} \texttt{list\textless{}geometry\textgreater{}}
\item
  \textbf{\texttt{skeletonize}} (\texttt{geometry}, \texttt{float},
  \texttt{float}) ---\textgreater{}
  \texttt{list\textless{}geometry\textgreater{}}
\end{itemize}

\subsubsection{Result:}\label{result-463}

A list of geometries (polylines) corresponding to the skeleton of the
operand geometry (geometry, agent) with the given tolerance for the
clipping A list of geometries (polylines) corresponding to the skeleton
of the operand geometry (geometry, agent) with the given tolerance for
the clipping and for the triangulation A list of geometries (polylines)
corresponding to the skeleton of the operand geometry (geometry, agent)

\subsubsection{Examples:}\label{examples-335}

\begin{verbatim}
 
list<geometry> var0 <- skeletonize(self); // var0 equals the list of geometries corresponding to the skeleton of the geometry of the agent applying the operator. 
list<geometry> var1 <- skeletonize(self); // var1 equals the list of geometries corresponding to the skeleton of the geometry of the agent applying the operator. 
list<geometry> var2 <- skeletonize(self); // var2 equals the list of geometries corresponding to the skeleton of the geometry of the agent applying the operator.
\end{verbatim}

\begin{center}\rule{0.5\linewidth}{\linethickness}\end{center}

\subsection{\texorpdfstring{\texttt{skew}}{skew}}\label{skew}

\subsubsection{Possible use:}\label{possible-use-480}

\begin{itemize}
\tightlist
\item
  \textbf{\texttt{skew}} (\texttt{container}) ---\textgreater{}
  \texttt{float}
\item
  \texttt{float} \textbf{\texttt{skew}} \texttt{float} ---\textgreater{}
  \texttt{float}
\item
  \textbf{\texttt{skew}} (\texttt{float} , \texttt{float})
  ---\textgreater{} \texttt{float}
\end{itemize}

\subsubsection{Result:}\label{result-464}

Returns the skew of a data sequence. Returns the skew of a data
sequence, which is moment(data,3,mean) / standardDeviation3

\begin{center}\rule{0.5\linewidth}{\linethickness}\end{center}

\subsection{\texorpdfstring{\texttt{skew\_gauss}}{skew\_gauss}}\label{skew_gauss}

\subsubsection{Possible use:}\label{possible-use-481}

\begin{itemize}
\tightlist
\item
  \textbf{\texttt{skew\_gauss}} (\texttt{float}, \texttt{float},
  \texttt{float}, \texttt{float}) ---\textgreater{} \texttt{float}
\end{itemize}

\subsubsection{Result:}\label{result-465}

A value from a skew normally distributed random variable with min value
(the minimum skewed value possible), max value (the maximum skewed value
possible), skew (the degree to which the values cluster around the mode
of the distribution; higher values mean tighter clustering) and bias
(the tendency of the mode to approach the min, max or midpoint value;
positive values bias toward max, negative values toward min).The
algorithm was taken from
\url{http://stackoverflow.com/questions/5853187/skewing-java-random-number-generation-toward-a-certain-number}

\subsubsection{Examples:}\label{examples-336}

\begin{verbatim}
 
float var0 <- skew_gauss(0.0, 1.0, 0.7,0.1); // var0 equals 0.1729218460343077
\end{verbatim}

\subsubsection{See also:}\label{see-also-187}

\href{OperatorsDH\#gauss}{gauss},
\href{OperatorsSZ\#truncated_gauss}{truncated\_gauss},
\href{OperatorsNR\#poisson}{poisson},

\begin{center}\rule{0.5\linewidth}{\linethickness}\end{center}

\subsection{\texorpdfstring{\texttt{skewness}}{skewness}}\label{skewness}

\subsubsection{Possible use:}\label{possible-use-482}

\begin{itemize}
\tightlist
\item
  \textbf{\texttt{skewness}} (\texttt{list}) ---\textgreater{}
  \texttt{float}
\end{itemize}

\subsubsection{Result:}\label{result-466}

returns skewness value computed from the operand list of values

\subsubsection{Special cases:}\label{special-cases-123}

\begin{itemize}
\tightlist
\item
  if the length of the list is lower than 3, returns NaN
\end{itemize}

\subsubsection{Examples:}\label{examples-337}

\begin{verbatim}
 
float var0 <- skewness ([1,2,3,4,5]); // var0 equals 0.0
\end{verbatim}

\begin{center}\rule{0.5\linewidth}{\linethickness}\end{center}

\subsection{\texorpdfstring{\texttt{skill}}{skill}}\label{skill}

\subsubsection{Possible use:}\label{possible-use-483}

\begin{itemize}
\tightlist
\item
  \textbf{\texttt{skill}} (\texttt{any}) ---\textgreater{}
  \texttt{skill}
\end{itemize}

\subsubsection{Result:}\label{result-467}

Casts the operand into the type skill

\begin{center}\rule{0.5\linewidth}{\linethickness}\end{center}

\subsection{\texorpdfstring{\texttt{smooth}}{smooth}}\label{smooth}

\subsubsection{Possible use:}\label{possible-use-484}

\begin{itemize}
\tightlist
\item
  \texttt{geometry} \textbf{\texttt{smooth}} \texttt{float}
  ---\textgreater{} \texttt{geometry}
\item
  \textbf{\texttt{smooth}} (\texttt{geometry} , \texttt{float})
  ---\textgreater{} \texttt{geometry}
\end{itemize}

\subsubsection{Result:}\label{result-468}

Returns a `smoothed' geometry, where straight lines are replaces by
polynomial (bicubic) curves. The first parameter is the original
geometry, the second is the `fit' parameter which can be in the range 0
(loose fit) to 1 (tightest fit).

\subsubsection{Examples:}\label{examples-338}

\begin{verbatim}
 
geometry var0 <- smooth(square(10), 0.0); // var0 equals a 'rounded' square
\end{verbatim}

\begin{center}\rule{0.5\linewidth}{\linethickness}\end{center}

\subsection{\texorpdfstring{\texttt{social\_link}}{social\_link}}\label{social_link}

\subsubsection{Possible use:}\label{possible-use-485}

\begin{itemize}
\tightlist
\item
  \textbf{\texttt{social\_link}} (\texttt{any}) ---\textgreater{}
  \texttt{social\_link}
\end{itemize}

\subsubsection{Result:}\label{result-469}

Casts the operand into the type social\_link

\begin{center}\rule{0.5\linewidth}{\linethickness}\end{center}

\subsection{\texorpdfstring{\texttt{solid}}{solid}}\label{solid}

Same signification as \href{OperatorsSZ\#without_holes}{without\_holes}

\begin{center}\rule{0.5\linewidth}{\linethickness}\end{center}

\subsection{\texorpdfstring{\texttt{sort}}{sort}}\label{sort}

Same signification as \href{OperatorsSZ\#sort_by}{sort\_by}

\begin{center}\rule{0.5\linewidth}{\linethickness}\end{center}

\subsection{\texorpdfstring{\texttt{sort\_by}}{sort\_by}}\label{sort_by}

\subsubsection{Possible use:}\label{possible-use-486}

\begin{itemize}
\tightlist
\item
  \texttt{container} \textbf{\texttt{sort\_by}} \texttt{any\ expression}
  ---\textgreater{} \texttt{list}
\item
  \textbf{\texttt{sort\_by}} (\texttt{container} ,
  \texttt{any\ expression}) ---\textgreater{} \texttt{list}
\end{itemize}

\subsubsection{Result:}\label{result-470}

Returns a list, containing the elements of the left-hand operand sorted
in ascending order by the value of the right-hand operand when it is
evaluated on them.

\subsubsection{Comment:}\label{comment-89}

the left-hand operand is casted to a list before applying the operator.
In the right-hand operand, the keyword each can be used to represent, in
turn, each of the elements.

\subsubsection{Special cases:}\label{special-cases-124}

\begin{itemize}
\tightlist
\item
  if the left-hand operand is nil, sort\_by throws an error
\end{itemize}

\subsubsection{Examples:}\label{examples-339}

\begin{verbatim}
 
list var0 <- [1,2,4,3,5,7,6,8] sort_by (each); // var0 equals [1,2,3,4,5,6,7,8] 
list var2 <- g2 sort_by (length(g2 out_edges_of each) ); // var2 equals [node9, node7, node10, node8, node11, node6, node5, node4] 
list var3 <- (list(node) sort_by (round(node(each).location.x)); // var3 equals [node5, node1, node0, node2, node3] 
list var4 <- [1::2, 5::6, 3::4] sort_by (each); // var4 equals [2, 4, 6]
\end{verbatim}

\subsubsection{See also:}\label{see-also-188}

\href{OperatorsDH\#group_by}{group\_by},

\begin{center}\rule{0.5\linewidth}{\linethickness}\end{center}

\subsection{\texorpdfstring{\texttt{source\_of}}{source\_of}}\label{source_of}

\subsubsection{Possible use:}\label{possible-use-487}

\begin{itemize}
\tightlist
\item
  \texttt{graph} \textbf{\texttt{source\_of}} \texttt{unknown}
  ---\textgreater{} \texttt{unknown}
\item
  \textbf{\texttt{source\_of}} (\texttt{graph} , \texttt{unknown})
  ---\textgreater{} \texttt{unknown}
\end{itemize}

\subsubsection{Result:}\label{result-471}

returns the source of the edge (right-hand operand) contained in the
graph given in left-hand operand.

\subsubsection{Special cases:}\label{special-cases-125}

\begin{itemize}
\tightlist
\item
  if the lef-hand operand (the graph) is nil, throws an Exception
\end{itemize}

\subsubsection{Examples:}\label{examples-340}

\begin{verbatim}
graph graphEpidemio <- generate_barabasi_albert( ["edges_species"::edge,"vertices_specy"::node,"size"::3,"m"::5] );  
unknown var1 <- graphEpidemio source_of(edge(3)); // var1 equals node1graph graphFromMap <-  as_edge_graph([{1,5}::{12,45},{12,45}::{34,56}]);  
point var3 <- graphFromMap source_of(link({1,5},{12,45})); // var3 equals {1,5}
\end{verbatim}

\subsubsection{See also:}\label{see-also-189}

\href{OperatorsSZ\#target_of}{target\_of},

\begin{center}\rule{0.5\linewidth}{\linethickness}\end{center}

\subsection{\texorpdfstring{\texttt{spatial\_graph}}{spatial\_graph}}\label{spatial_graph}

\subsubsection{Possible use:}\label{possible-use-488}

\begin{itemize}
\tightlist
\item
  \textbf{\texttt{spatial\_graph}} (\texttt{container})
  ---\textgreater{} \texttt{graph}
\end{itemize}

\subsubsection{Result:}\label{result-472}

allows to create a spatial graph from a container of vertices, without
trying to wire them. The container can be empty. Emits an error if the
contents of the container are not geometries, points or agents

\subsubsection{See also:}\label{see-also-190}

\href{OperatorsDH\#graph}{graph},

\begin{center}\rule{0.5\linewidth}{\linethickness}\end{center}

\subsection{\texorpdfstring{\texttt{species}}{species}}\label{species}

\subsubsection{Possible use:}\label{possible-use-489}

\begin{itemize}
\tightlist
\item
  \textbf{\texttt{species}} (\texttt{unknown}) ---\textgreater{}
  \texttt{species}
\end{itemize}

\subsubsection{Result:}\label{result-473}

casting of the operand to a species.

\subsubsection{Special cases:}\label{special-cases-126}

\begin{itemize}
\tightlist
\item
  if the operand is nil, returns nil;\\
\item
  if the operand is an agent, returns its species;\\
\item
  if the operand is a string, returns the species with this name (nil if
  not found);\\
\item
  otherwise, returns nil
\end{itemize}

\subsubsection{Examples:}\label{examples-341}

\begin{verbatim}
 
species var0 <- species(self); // var0 equals the species of the current agent 
species var1 <- species('node'); // var1 equals node 
species var2 <- species([1,5,9,3]); // var2 equals nil 
species var3 <- species(node1); // var3 equals node
\end{verbatim}

\begin{center}\rule{0.5\linewidth}{\linethickness}\end{center}

\subsection{\texorpdfstring{\texttt{species\_of}}{species\_of}}\label{species_of}

Same signification as \href{OperatorsSZ\#species}{species}

\begin{center}\rule{0.5\linewidth}{\linethickness}\end{center}

\subsection{\texorpdfstring{\texttt{sphere}}{sphere}}\label{sphere}

\subsubsection{Possible use:}\label{possible-use-490}

\begin{itemize}
\tightlist
\item
  \textbf{\texttt{sphere}} (\texttt{float}) ---\textgreater{}
  \texttt{geometry}
\end{itemize}

\subsubsection{Result:}\label{result-474}

A sphere geometry which radius is equal to the operand.

\subsubsection{Comment:}\label{comment-90}

the centre of the sphere is by default the location of the current agent
in which has been called this operator.

\subsubsection{Special cases:}\label{special-cases-127}

\begin{itemize}
\tightlist
\item
  returns a point if the operand is lower or equal to 0.
\end{itemize}

\subsubsection{Examples:}\label{examples-342}

\begin{verbatim}
 
geometry var0 <- sphere(10); // var0 equals a geometry as a circle of radius 10 but displays a sphere.
\end{verbatim}

\subsubsection{See also:}\label{see-also-191}

\href{OperatorsAA\#around}{around}, \href{OperatorsBC\#cone}{cone},
\href{OperatorsIM\#line}{line}, \href{OperatorsIM\#link}{link},
\href{OperatorsNR\#norm}{norm}, \href{OperatorsNR\#point}{point},
\href{OperatorsNR\#polygon}{polygon},
\href{OperatorsNR\#polyline}{polyline},
\href{OperatorsNR\#rectangle}{rectangle},
\href{OperatorsSZ\#square}{square},
\href{OperatorsSZ\#triangle}{triangle},

\begin{center}\rule{0.5\linewidth}{\linethickness}\end{center}

\subsection{\texorpdfstring{\texttt{split}}{split}}\label{split}

\subsubsection{Possible use:}\label{possible-use-491}

\begin{itemize}
\tightlist
\item
  \textbf{\texttt{split}} (\texttt{list}) ---\textgreater{}
  \texttt{list\textless{}list\textgreater{}}
\end{itemize}

\subsubsection{Result:}\label{result-475}

Splits a list of numbers into n=(1+3.3*log10(elements)) bins. The
splitting is strict (i.e.~elements are in the ith bin if they are
strictly smaller than the ith bound

\subsubsection{See also:}\label{see-also-192}

\href{OperatorsSZ\#split_in}{split\_in},
\href{OperatorsSZ\#split_using}{split\_using},

\begin{center}\rule{0.5\linewidth}{\linethickness}\end{center}

\subsection{\texorpdfstring{\texttt{split\_at}}{split\_at}}\label{split_at}

\subsubsection{Possible use:}\label{possible-use-492}

\begin{itemize}
\tightlist
\item
  \texttt{geometry} \textbf{\texttt{split\_at}} \texttt{point}
  ---\textgreater{} \texttt{list\textless{}geometry\textgreater{}}
\item
  \textbf{\texttt{split\_at}} (\texttt{geometry} , \texttt{point})
  ---\textgreater{} \texttt{list\textless{}geometry\textgreater{}}
\end{itemize}

\subsubsection{Result:}\label{result-476}

The two part of the left-operand lines split at the given right-operand
point

\subsubsection{Special cases:}\label{special-cases-128}

\begin{itemize}
\tightlist
\item
  if the left-operand is a point or a polygon, returns an empty list
\end{itemize}

\subsubsection{Examples:}\label{examples-343}

\begin{verbatim}
 
list<geometry> var0 <- polyline([{1,2},{4,6}]) split_at {7,6}; // var0 equals [polyline([{1.0,2.0},{7.0,6.0}]), polyline([{7.0,6.0},{4.0,6.0}])]
\end{verbatim}

\begin{center}\rule{0.5\linewidth}{\linethickness}\end{center}

\subsection{\texorpdfstring{\texttt{split\_geometry}}{split\_geometry}}\label{split_geometry}

\subsubsection{Possible use:}\label{possible-use-493}

\begin{itemize}
\tightlist
\item
  \texttt{geometry} \textbf{\texttt{split\_geometry}} \texttt{float}
  ---\textgreater{} \texttt{list\textless{}geometry\textgreater{}}
\item
  \textbf{\texttt{split\_geometry}} (\texttt{geometry} , \texttt{float})
  ---\textgreater{} \texttt{list\textless{}geometry\textgreater{}}
\item
  \texttt{geometry} \textbf{\texttt{split\_geometry}} \texttt{point}
  ---\textgreater{} \texttt{list\textless{}geometry\textgreater{}}
\item
  \textbf{\texttt{split\_geometry}} (\texttt{geometry} , \texttt{point})
  ---\textgreater{} \texttt{list\textless{}geometry\textgreater{}}
\item
  \textbf{\texttt{split\_geometry}} (\texttt{geometry}, \texttt{int},
  \texttt{int}) ---\textgreater{}
  \texttt{list\textless{}geometry\textgreater{}}
\end{itemize}

\subsubsection{Result:}\label{result-477}

A list of geometries that result from the decomposition of the geometry
by square cells of the given side size (geometry, size) A list of
geometries that result from the decomposition of the geometry according
to a grid with the given number of rows and columns (geometry, nb\_cols,
nb\_rows) A list of geometries that result from the decomposition of the
geometry by rectangle cells of the given dimension (geometry, \{size\_x,
size\_y\})

\subsubsection{Examples:}\label{examples-344}

\begin{verbatim}
 
list<geometry> var0 <- to_squares(self, 10.0); // var0 equals the list of the geometries corresponding to the decomposition of the geometry by squares of side size 10.0 
list<geometry> var1 <- to_rectangles(self, 10,20); // var1 equals the list of the geometries corresponding to the decomposition of the geometry of the agent applying the operator 
list<geometry> var2 <- to_rectangles(self, {10.0, 15.0}); // var2 equals the list of the geometries corresponding to the decomposition of the geometry by rectangles of size 10.0, 15.0
\end{verbatim}

\begin{center}\rule{0.5\linewidth}{\linethickness}\end{center}

\subsection{\texorpdfstring{\texttt{split\_in}}{split\_in}}\label{split_in}

\subsubsection{Possible use:}\label{possible-use-494}

\begin{itemize}
\tightlist
\item
  \texttt{list} \textbf{\texttt{split\_in}} \texttt{int}
  ---\textgreater{} \texttt{list\textless{}list\textgreater{}}
\item
  \textbf{\texttt{split\_in}} (\texttt{list} , \texttt{int})
  ---\textgreater{} \texttt{list\textless{}list\textgreater{}}
\item
  \textbf{\texttt{split\_in}} (\texttt{list}, \texttt{int},
  \texttt{bool}) ---\textgreater{}
  \texttt{list\textless{}list\textgreater{}}
\end{itemize}

\subsubsection{Result:}\label{result-478}

Splits a list of numbers into n bins defined by n-1 bounds between the
minimum and maximum values found in the first argument. The boolean
argument controls whether or not the splitting is strict (if true,
elements are in the ith bin if they are strictly smaller than the ith
bound Splits a list of numbers into n bins defined by n-1 bounds between
the minimum and maximum values found in the first argument. The
splitting is strict (i.e.~elements are in the ith bin if they are
strictly smaller than the ith bound

\subsubsection{See also:}\label{see-also-193}

\href{OperatorsSZ\#split}{split},
\href{OperatorsSZ\#split_using}{split\_using},

\begin{center}\rule{0.5\linewidth}{\linethickness}\end{center}

\subsection{\texorpdfstring{\texttt{split\_lines}}{split\_lines}}\label{split_lines}

\subsubsection{Possible use:}\label{possible-use-495}

\begin{itemize}
\tightlist
\item
  \textbf{\texttt{split\_lines}}
  (\texttt{container\textless{}geometry\textgreater{}})
  ---\textgreater{} \texttt{list\textless{}geometry\textgreater{}}
\item
  \texttt{container\textless{}geometry\textgreater{}}
  \textbf{\texttt{split\_lines}} \texttt{bool} ---\textgreater{}
  \texttt{list\textless{}geometry\textgreater{}}
\item
  \textbf{\texttt{split\_lines}}
  (\texttt{container\textless{}geometry\textgreater{}} , \texttt{bool})
  ---\textgreater{} \texttt{list\textless{}geometry\textgreater{}}
\end{itemize}

\subsubsection{Result:}\label{result-479}

A list of geometries resulting after cutting the lines at their
intersections. A list of geometries resulting after cutting the lines at
their intersections. if the last boolean operand is set to true, the
split lines will import the attributes of the initial lines

\subsubsection{Examples:}\label{examples-345}

\begin{verbatim}
 
list<geometry> var0 <- split_lines([line([{0,10}, {20,10}]), line([{0,10}, {20,10}])]); // var0 equals a list of four polylines: line([{0,10}, {10,10}]), line([{10,10}, {20,10}]), line([{10,0}, {10,10}]) and line([{10,10}, {10,20}]) 
list<geometry> var1 <- split_lines([line([{0,10}, {20,10}]), line([{0,10}, {20,10}])]); // var1 equals a list of four polylines: line([{0,10}, {10,10}]), line([{10,10}, {20,10}]), line([{10,0}, {10,10}]) and line([{10,10}, {10,20}])
\end{verbatim}

\begin{center}\rule{0.5\linewidth}{\linethickness}\end{center}

\subsection{\texorpdfstring{\texttt{split\_using}}{split\_using}}\label{split_using}

\subsubsection{Possible use:}\label{possible-use-496}

\begin{itemize}
\tightlist
\item
  \texttt{list} \textbf{\texttt{split\_using}}
  \texttt{msi.gama.util.IList\textless{}?\ extends\ java.lang.Comparable\textgreater{}}
  ---\textgreater{} \texttt{list\textless{}list\textgreater{}}
\item
  \textbf{\texttt{split\_using}} (\texttt{list} ,
  \texttt{msi.gama.util.IList\textless{}?\ extends\ java.lang.Comparable\textgreater{}})
  ---\textgreater{} \texttt{list\textless{}list\textgreater{}}
\item
  \textbf{\texttt{split\_using}} (\texttt{list},
  \texttt{msi.gama.util.IList\textless{}?\ extends\ java.lang.Comparable\textgreater{}},
  \texttt{bool}) ---\textgreater{}
  \texttt{list\textless{}list\textgreater{}}
\end{itemize}

\subsubsection{Result:}\label{result-480}

Splits a list of numbers into n+1 bins using a set of n bounds passed as
the second argument. The splitting is strict (i.e.~elements are in the
ith bin if they are strictly smaller than the ith bound Splits a list of
numbers into n+1 bins using a set of n bounds passed as the second
argument. The boolean argument controls whether or not the splitting is
strict (if true, elements are in the ith bin if they are strictly
smaller than the ith bound

\subsubsection{See also:}\label{see-also-194}

\href{OperatorsSZ\#split}{split},
\href{OperatorsSZ\#split_in}{split\_in},

\begin{center}\rule{0.5\linewidth}{\linethickness}\end{center}

\subsection{\texorpdfstring{\texttt{split\_with}}{split\_with}}\label{split_with}

\subsubsection{Possible use:}\label{possible-use-497}

\begin{itemize}
\tightlist
\item
  \texttt{string} \textbf{\texttt{split\_with}} \texttt{string}
  ---\textgreater{} \texttt{list}
\item
  \textbf{\texttt{split\_with}} (\texttt{string} , \texttt{string})
  ---\textgreater{} \texttt{list}
\item
  \textbf{\texttt{split\_with}} (\texttt{string}, \texttt{string},
  \texttt{bool}) ---\textgreater{} \texttt{list}
\end{itemize}

\subsubsection{Result:}\label{result-481}

Returns a list containing the sub-strings (tokens) of the left-hand
operand delimited by each of the characters of the right-hand operand.
Returns a list containing the sub-strings (tokens) of the left-hand
operand delimited either by each of the characters of the right-hand
operand (false) or by the whole right-hand operand (true).

\subsubsection{Comment:}\label{comment-91}

Delimiters themselves are excluded from the resulting list.Delimiters
themselves are excluded from the resulting list.

\subsubsection{Examples:}\label{examples-346}

\begin{verbatim}
 
list var0 <- 'to be or not to be,that is the question' split_with ' ,'; // var0 equals ['to','be','or','not','to','be','that','is','the','question'] 
list var1 <- 'aa::bb:cc' split_with ('::', true); // var1 equals ['aa','bb:cc']
\end{verbatim}

\begin{center}\rule{0.5\linewidth}{\linethickness}\end{center}

\subsection{\texorpdfstring{\texttt{sqrt}}{sqrt}}\label{sqrt}

\subsubsection{Possible use:}\label{possible-use-498}

\begin{itemize}
\tightlist
\item
  \textbf{\texttt{sqrt}} (\texttt{int}) ---\textgreater{} \texttt{float}
\item
  \textbf{\texttt{sqrt}} (\texttt{float}) ---\textgreater{}
  \texttt{float}
\end{itemize}

\subsubsection{Result:}\label{result-482}

Returns the square root of the operand.

\subsubsection{Special cases:}\label{special-cases-129}

\begin{itemize}
\tightlist
\item
  if the operand is negative, an exception is raised
\end{itemize}

\subsubsection{Examples:}\label{examples-347}

\begin{verbatim}
 
float var0 <- sqrt(4); // var0 equals 2.0 
float var1 <- sqrt(4); // var1 equals 2.0
\end{verbatim}

\begin{center}\rule{0.5\linewidth}{\linethickness}\end{center}

\subsection{\texorpdfstring{\texttt{square}}{square}}\label{square}

\subsubsection{Possible use:}\label{possible-use-499}

\begin{itemize}
\tightlist
\item
  \textbf{\texttt{square}} (\texttt{float}) ---\textgreater{}
  \texttt{geometry}
\end{itemize}

\subsubsection{Result:}\label{result-483}

A square geometry which side size is equal to the operand.

\subsubsection{Comment:}\label{comment-92}

the centre of the square is by default the location of the current agent
in which has been called this operator.

\subsubsection{Special cases:}\label{special-cases-130}

\begin{itemize}
\tightlist
\item
  returns nil if the operand is nil.
\end{itemize}

\subsubsection{Examples:}\label{examples-348}

\begin{verbatim}
 
geometry var0 <- square(10); // var0 equals a geometry as a square of side size 10.
\end{verbatim}

\subsubsection{See also:}\label{see-also-195}

\href{OperatorsAA\#around}{around}, \href{OperatorsBC\#circle}{circle},
\href{OperatorsBC\#cone}{cone}, \href{OperatorsIM\#line}{line},
\href{OperatorsIM\#link}{link}, \href{OperatorsNR\#norm}{norm},
\href{OperatorsNR\#point}{point}, \href{OperatorsNR\#polygon}{polygon},
\href{OperatorsNR\#polyline}{polyline},
\href{OperatorsNR\#rectangle}{rectangle},
\href{OperatorsSZ\#triangle}{triangle},

\begin{center}\rule{0.5\linewidth}{\linethickness}\end{center}

\subsection{\texorpdfstring{\texttt{squircle}}{squircle}}\label{squircle}

\subsubsection{Possible use:}\label{possible-use-500}

\begin{itemize}
\tightlist
\item
  \texttt{float} \textbf{\texttt{squircle}} \texttt{float}
  ---\textgreater{} \texttt{geometry}
\item
  \textbf{\texttt{squircle}} (\texttt{float} , \texttt{float})
  ---\textgreater{} \texttt{geometry}
\end{itemize}

\subsubsection{Result:}\label{result-484}

A mix of square and circle geometry (see :
\url{http://en.wikipedia.org/wiki/Squircle}), which side size is equal
to the first operand and power is equal to the second operand

\subsubsection{Comment:}\label{comment-93}

the center of the ellipse is by default the location of the current
agent in which has been called this operator.

\subsubsection{Special cases:}\label{special-cases-131}

\begin{itemize}
\tightlist
\item
  returns a point if the side operand is lower or equal to 0.
\end{itemize}

\subsubsection{Examples:}\label{examples-349}

\begin{verbatim}
 
geometry var0 <- squircle(4,4); // var0 equals a geometry as a squircle of side 4 with a power of 4.
\end{verbatim}

\subsubsection{See also:}\label{see-also-196}

\href{OperatorsAA\#around}{around}, \href{OperatorsBC\#cone}{cone},
\href{OperatorsIM\#line}{line}, \href{OperatorsIM\#link}{link},
\href{OperatorsNR\#norm}{norm}, \href{OperatorsNR\#point}{point},
\href{OperatorsNR\#polygon}{polygon},
\href{OperatorsNR\#polyline}{polyline},
\href{OperatorsSZ\#super_ellipse}{super\_ellipse},
\href{OperatorsNR\#rectangle}{rectangle},
\href{OperatorsSZ\#square}{square}, \href{OperatorsBC\#circle}{circle},
\href{OperatorsDH\#ellipse}{ellipse},
\href{OperatorsSZ\#triangle}{triangle},

\begin{center}\rule{0.5\linewidth}{\linethickness}\end{center}

\subsection{\texorpdfstring{\texttt{stack}}{stack}}\label{stack}

\subsubsection{Possible use:}\label{possible-use-501}

\begin{itemize}
\tightlist
\item
  \textbf{\texttt{stack}}
  (\texttt{msi.gama.util.IList\textless{}java.lang.Integer\textgreater{}})
  ---\textgreater{}
  \texttt{msi.gama.util.tree.GamaNode\textless{}java.lang.String\textgreater{}}
\end{itemize}

\begin{center}\rule{0.5\linewidth}{\linethickness}\end{center}

\subsection{\texorpdfstring{\texttt{standard\_deviation}}{standard\_deviation}}\label{standard_deviation}

\subsubsection{Possible use:}\label{possible-use-502}

\begin{itemize}
\tightlist
\item
  \textbf{\texttt{standard\_deviation}} (\texttt{container})
  ---\textgreater{} \texttt{float}
\end{itemize}

\subsubsection{Result:}\label{result-485}

the standard deviation on the elements of the operand. See
Standard\_deviation for more details.

\subsubsection{Comment:}\label{comment-94}

The operator casts all the numerical element of the list into float. The
elements that are not numerical are discarded.

\subsubsection{Examples:}\label{examples-350}

\begin{verbatim}
 
float var0 <- standard_deviation ([4.5, 3.5, 5.5, 7.0]); // var0 equals 1.2930100540985752
\end{verbatim}

\subsubsection{See also:}\label{see-also-197}

\href{OperatorsIM\#mean}{mean},
\href{OperatorsIM\#mean_deviation}{mean\_deviation},

\begin{center}\rule{0.5\linewidth}{\linethickness}\end{center}

\subsection{\texorpdfstring{\texttt{step\_sub\_model}}{step\_sub\_model}}\label{step_sub_model}

\subsubsection{Possible use:}\label{possible-use-503}

\begin{itemize}
\tightlist
\item
  \textbf{\texttt{step\_sub\_model}}
  (\texttt{msi.gama.kernel.experiment.IExperimentAgent})
  ---\textgreater{} \texttt{int}
\end{itemize}

\subsubsection{Result:}\label{result-486}

Load a submodel

\subsubsection{Comment:}\label{comment-95}

loaded submodel

\begin{center}\rule{0.5\linewidth}{\linethickness}\end{center}

\subsection{\texorpdfstring{\texttt{strahler}}{strahler}}\label{strahler}

\subsubsection{Possible use:}\label{possible-use-504}

\begin{itemize}
\tightlist
\item
  \textbf{\texttt{strahler}} (\texttt{graph}) ---\textgreater{}
  \texttt{map}
\end{itemize}

\subsubsection{Result:}\label{result-487}

retur for each edge, its strahler number

\begin{center}\rule{0.5\linewidth}{\linethickness}\end{center}

\subsection{\texorpdfstring{\texttt{string}}{string}}\label{string}

\subsubsection{Possible use:}\label{possible-use-505}

\begin{itemize}
\tightlist
\item
  \texttt{date} \textbf{\texttt{string}} \texttt{string}
  ---\textgreater{} \texttt{string}
\item
  \textbf{\texttt{string}} (\texttt{date} , \texttt{string})
  ---\textgreater{} \texttt{string}
\item
  \textbf{\texttt{string}} (\texttt{date}, \texttt{string},
  \texttt{string}) ---\textgreater{} \texttt{string}
\end{itemize}

\subsubsection{Result:}\label{result-488}

converts a date to astring following a custom pattern and using a
specific locale (e.g.: `fr', `en', etc.). The pattern can use ``\%Y \%M
\%N \%D \%E \%h \%m \%s \%z'' for outputting years, months, name of
month, days, name of days, hours, minutes, seconds and the time-zone. A
null or empty pattern will return the complete date as defined by the
ISO date \& time format. The pattern can also follow the pattern
definition found here, which gives much more control over the format of
the date:
\url{https://docs.oracle.com/javase/8/docs/api/java/time/format/DateTimeFormatter.html\#patterns}.
Different patterns are available by default as constants: \#iso\_local,
\#iso\_simple, \#iso\_offset, \#iso\_zoned and \#custom, which can be
changed in the preferences converts a date to astring following a custom
pattern. The pattern can use ``\%Y \%M \%N \%D \%E \%h \%m \%s \%z'' for
outputting years, months, name of month, days, name of days, hours,
minutes, seconds and the time-zone. A null or empty pattern will return
the complete date as defined by the ISO date \& time format. The pattern
can also follow the pattern definition found here, which gives much more
control over the format of the date:
\url{https://docs.oracle.com/javase/8/docs/api/java/time/format/DateTimeFormatter.html\#patterns}.
Different patterns are available by default as constants: \#iso\_local,
\#iso\_simple, \#iso\_offset, \#iso\_zoned and \#custom, which can be
changed in the preferences

\subsubsection{Examples:}\label{examples-351}

\begin{verbatim}
format(#now, 'yyyy-MM-dd') format(#now, 'yyyy-MM-dd') 
\end{verbatim}

\begin{center}\rule{0.5\linewidth}{\linethickness}\end{center}

\subsection{\texorpdfstring{\texttt{student\_area}}{student\_area}}\label{student_area}

\subsubsection{Possible use:}\label{possible-use-506}

\begin{itemize}
\tightlist
\item
  \texttt{float} \textbf{\texttt{student\_area}} \texttt{int}
  ---\textgreater{} \texttt{float}
\item
  \textbf{\texttt{student\_area}} (\texttt{float} , \texttt{int})
  ---\textgreater{} \texttt{float}
\end{itemize}

\subsubsection{Result:}\label{result-489}

Returns the area to the left of x in the Student T distribution with the
given degrees of freedom.

\begin{center}\rule{0.5\linewidth}{\linethickness}\end{center}

\subsection{\texorpdfstring{\texttt{student\_t\_inverse}}{student\_t\_inverse}}\label{student_t_inverse}

\subsubsection{Possible use:}\label{possible-use-507}

\begin{itemize}
\tightlist
\item
  \texttt{float} \textbf{\texttt{student\_t\_inverse}} \texttt{int}
  ---\textgreater{} \texttt{float}
\item
  \textbf{\texttt{student\_t\_inverse}} (\texttt{float} , \texttt{int})
  ---\textgreater{} \texttt{float}
\end{itemize}

\subsubsection{Result:}\label{result-490}

Returns the value, t, for which the area under the Student-t probability
density function (integrated from minus infinity to t) is equal to x.

\begin{center}\rule{0.5\linewidth}{\linethickness}\end{center}

\subsection{\texorpdfstring{\texttt{subtract\_days}}{subtract\_days}}\label{subtract_days}

Same signification as \href{OperatorsIM\#minus_days}{minus\_days}

\begin{center}\rule{0.5\linewidth}{\linethickness}\end{center}

\subsection{\texorpdfstring{\texttt{subtract\_hours}}{subtract\_hours}}\label{subtract_hours}

Same signification as \href{OperatorsIM\#minus_hours}{minus\_hours}

\begin{center}\rule{0.5\linewidth}{\linethickness}\end{center}

\subsection{\texorpdfstring{\texttt{subtract\_minutes}}{subtract\_minutes}}\label{subtract_minutes}

Same signification as \href{OperatorsIM\#minus_minutes}{minus\_minutes}

\begin{center}\rule{0.5\linewidth}{\linethickness}\end{center}

\subsection{\texorpdfstring{\texttt{subtract\_months}}{subtract\_months}}\label{subtract_months}

Same signification as \href{OperatorsIM\#minus_months}{minus\_months}

\begin{center}\rule{0.5\linewidth}{\linethickness}\end{center}

\subsection{\texorpdfstring{\texttt{subtract\_ms}}{subtract\_ms}}\label{subtract_ms}

Same signification as \href{OperatorsIM\#minus_ms}{minus\_ms}

\begin{center}\rule{0.5\linewidth}{\linethickness}\end{center}

\subsection{\texorpdfstring{\texttt{subtract\_seconds}}{subtract\_seconds}}\label{subtract_seconds}

Same signification as \href{OperatorsAA\#-}{-}

\begin{center}\rule{0.5\linewidth}{\linethickness}\end{center}

\subsection{\texorpdfstring{\texttt{subtract\_weeks}}{subtract\_weeks}}\label{subtract_weeks}

Same signification as \href{OperatorsIM\#minus_weeks}{minus\_weeks}

\begin{center}\rule{0.5\linewidth}{\linethickness}\end{center}

\subsection{\texorpdfstring{\texttt{subtract\_years}}{subtract\_years}}\label{subtract_years}

Same signification as \href{OperatorsIM\#minus_years}{minus\_years}

\begin{center}\rule{0.5\linewidth}{\linethickness}\end{center}

\subsection{\texorpdfstring{\texttt{successors\_of}}{successors\_of}}\label{successors_of}

\subsubsection{Possible use:}\label{possible-use-508}

\begin{itemize}
\tightlist
\item
  \texttt{graph} \textbf{\texttt{successors\_of}} \texttt{unknown}
  ---\textgreater{} \texttt{list}
\item
  \textbf{\texttt{successors\_of}} (\texttt{graph} , \texttt{unknown})
  ---\textgreater{} \texttt{list}
\end{itemize}

\subsubsection{Result:}\label{result-491}

returns the list of successors (i.e.~targets of out edges) of the given
vertex (right-hand operand) in the given graph (left-hand operand)

\subsubsection{Examples:}\label{examples-352}

\begin{verbatim}
 
list var1 <- graphEpidemio successors_of ({1,5}); // var1 equals [{12,45}] 
list var2 <- graphEpidemio successors_of node({34,56}); // var2 equals []
\end{verbatim}

\subsubsection{See also:}\label{see-also-198}

\href{OperatorsNR\#predecessors_of}{predecessors\_of},
\href{OperatorsNR\#neighbors_of}{neighbors\_of},

\begin{center}\rule{0.5\linewidth}{\linethickness}\end{center}

\subsection{\texorpdfstring{\texttt{sum}}{sum}}\label{sum}

\subsubsection{Possible use:}\label{possible-use-509}

\begin{itemize}
\tightlist
\item
  \textbf{\texttt{sum}} (\texttt{container}) ---\textgreater{}
  \texttt{unknown}
\item
  \textbf{\texttt{sum}} (\texttt{graph}) ---\textgreater{}
  \texttt{float}
\end{itemize}

\subsubsection{Result:}\label{result-492}

the sum of all the elements of the operand

\subsubsection{Comment:}\label{comment-96}

the behavior depends on the nature of the operand

\subsubsection{Special cases:}\label{special-cases-132}

\begin{itemize}
\tightlist
\item
  if it is a population or a list of other types: sum transforms all
  elements into float and sums them\\
\item
  if it is a map, sum returns the sum of the value of all elements\\
\item
  if it is a file, sum returns the sum of the content of the file (that
  is also a container)\\
\item
  if it is a graph, sum returns the total weight of the graph\\
\item
  if it is a matrix of int, float or object, sum returns the sum of all
  the numerical elements (i.e.~all elements for integer and float
  matrices)\\
\item
  if it is a matrix of other types: sum transforms all elements into
  float and sums them\\
\item
  if it is a list of colors: sum will sum them and return the blended
  resulting color\\
\item
  if it is a list of int or float: sum returns the sum of all the
  elements
\end{itemize}

\begin{verbatim}
 
int var0 <- sum ([12,10,3]); // var0 equals 25
\end{verbatim}

\begin{itemize}
\tightlist
\item
  if it is a list of points: sum returns the sum of all points as a
  point (each coordinate is the sum of the corresponding coordinate of
  each element)
\end{itemize}

\begin{verbatim}
 
unknown var1 <- sum([{1.0,3.0},{3.0,5.0},{9.0,1.0},{7.0,8.0}]); // var1 equals {20.0,17.0}
\end{verbatim}

\subsubsection{See also:}\label{see-also-199}

\href{OperatorsIM\#mul}{mul},

\begin{center}\rule{0.5\linewidth}{\linethickness}\end{center}

\subsection{\texorpdfstring{\texttt{sum\_of}}{sum\_of}}\label{sum_of}

\subsubsection{Possible use:}\label{possible-use-510}

\begin{itemize}
\tightlist
\item
  \texttt{container} \textbf{\texttt{sum\_of}} \texttt{any\ expression}
  ---\textgreater{} \texttt{unknown}
\item
  \textbf{\texttt{sum\_of}} (\texttt{container} ,
  \texttt{any\ expression}) ---\textgreater{} \texttt{unknown}
\end{itemize}

\subsubsection{Result:}\label{result-493}

the sum of the right-hand expression evaluated on each of the elements
of the left-hand operand

\subsubsection{Comment:}\label{comment-97}

in the right-hand operand, the keyword each can be used to represent, in
turn, each of the right-hand operand elements.

\subsubsection{Special cases:}\label{special-cases-133}

\begin{itemize}
\tightlist
\item
  if the left-operand is a map, the keyword each will contain each value
\end{itemize}

\begin{verbatim}
 
unknown var1 <- [1::2, 3::4, 5::6] sum_of (each + 3); // var1 equals 21
\end{verbatim}

\subsubsection{Examples:}\label{examples-353}

\begin{verbatim}
 
unknown var0 <- [1,2] sum_of (each * 100 ); // var0 equals 300
\end{verbatim}

\subsubsection{See also:}\label{see-also-200}

\href{OperatorsIM\#min_of}{min\_of},
\href{OperatorsIM\#max_of}{max\_of},
\href{OperatorsNR\#product_of}{product\_of},
\href{OperatorsIM\#mean_of}{mean\_of},

\begin{center}\rule{0.5\linewidth}{\linethickness}\end{center}

\subsection{\texorpdfstring{\texttt{svg\_file}}{svg\_file}}\label{svg_file}

\subsubsection{Possible use:}\label{possible-use-511}

\begin{itemize}
\tightlist
\item
  \textbf{\texttt{svg\_file}} (\texttt{string}) ---\textgreater{}
  \texttt{file}
\end{itemize}

\subsubsection{Result:}\label{result-494}

Constructs a file of type svg. Allowed extensions are limited to svg

\begin{center}\rule{0.5\linewidth}{\linethickness}\end{center}

\subsection{\texorpdfstring{\texttt{tan}}{tan}}\label{tan}

\subsubsection{Possible use:}\label{possible-use-512}

\begin{itemize}
\tightlist
\item
  \textbf{\texttt{tan}} (\texttt{float}) ---\textgreater{}
  \texttt{float}
\item
  \textbf{\texttt{tan}} (\texttt{int}) ---\textgreater{} \texttt{float}
\end{itemize}

\subsubsection{Result:}\label{result-495}

Returns the value (in {[}-1,1{]}) of the trigonometric tangent of the
operand (in decimal degrees).

\subsubsection{Special cases:}\label{special-cases-134}

\begin{itemize}
\tightlist
\item
  Operand values out of the range {[}0-359{]} are normalized. Notice
  that tan(360) does not return 0.0 but -2.4492935982947064E-16\\
\item
  The tangent is only defined for any real number except 90 + k
  \texttt{*} 180 (k an positive or negative integer). Nevertheless
  notice that tan(90) returns 1.633123935319537E16 (whereas we could
  except infinity).
\end{itemize}

\subsubsection{Examples:}\label{examples-354}

\begin{verbatim}
 
float var0 <- tan (0); // var0 equals 0.0 
float var1 <- tan(90); // var1 equals 1.633123935319537E16
\end{verbatim}

\subsubsection{See also:}\label{see-also-201}

\href{OperatorsBC\#cos}{cos}, \href{OperatorsSZ\#sin}{sin},

\begin{center}\rule{0.5\linewidth}{\linethickness}\end{center}

\subsection{\texorpdfstring{\texttt{tan\_rad}}{tan\_rad}}\label{tan_rad}

\subsubsection{Possible use:}\label{possible-use-513}

\begin{itemize}
\tightlist
\item
  \textbf{\texttt{tan\_rad}} (\texttt{float}) ---\textgreater{}
  \texttt{float}
\end{itemize}

\subsubsection{Result:}\label{result-496}

Returns the value (in {[}-1,1{]}) of the trigonometric tangent of the
operand (in radians).

\subsubsection{See also:}\label{see-also-202}

\href{OperatorsBC\#cos_rad}{cos\_rad},
\href{OperatorsSZ\#sin_rad}{sin\_rad},

\begin{center}\rule{0.5\linewidth}{\linethickness}\end{center}

\subsection{\texorpdfstring{\texttt{tanh}}{tanh}}\label{tanh}

\subsubsection{Possible use:}\label{possible-use-514}

\begin{itemize}
\tightlist
\item
  \textbf{\texttt{tanh}} (\texttt{int}) ---\textgreater{} \texttt{float}
\item
  \textbf{\texttt{tanh}} (\texttt{float}) ---\textgreater{}
  \texttt{float}
\end{itemize}

\subsubsection{Result:}\label{result-497}

Returns the value (in the interval {[}-1,1{]}) of the hyperbolic tangent
of the operand (which can be any real number, expressed in decimal
degrees).

\subsubsection{Examples:}\label{examples-355}

\begin{verbatim}
 
float var0 <- tanh(0); // var0 equals 0.0 
float var1 <- tanh(100); // var1 equals 1.0
\end{verbatim}

\begin{center}\rule{0.5\linewidth}{\linethickness}\end{center}

\subsection{\texorpdfstring{\texttt{target\_of}}{target\_of}}\label{target_of}

\subsubsection{Possible use:}\label{possible-use-515}

\begin{itemize}
\tightlist
\item
  \texttt{graph} \textbf{\texttt{target\_of}} \texttt{unknown}
  ---\textgreater{} \texttt{unknown}
\item
  \textbf{\texttt{target\_of}} (\texttt{graph} , \texttt{unknown})
  ---\textgreater{} \texttt{unknown}
\end{itemize}

\subsubsection{Result:}\label{result-498}

returns the target of the edge (right-hand operand) contained in the
graph given in left-hand operand.

\subsubsection{Special cases:}\label{special-cases-135}

\begin{itemize}
\tightlist
\item
  if the lef-hand operand (the graph) is nil, returns nil
\end{itemize}

\subsubsection{Examples:}\label{examples-356}

\begin{verbatim}
graph graphEpidemio <- generate_barabasi_albert( ["edges_species"::edge,"vertices_specy"::node,"size"::3,"m"::5] );  
unknown var1 <- graphEpidemio source_of(edge(3)); // var1 equals node1graph graphFromMap <-  as_edge_graph([{1,5}::{12,45},{12,45}::{34,56}]);  
unknown var3 <- graphFromMap target_of(link({1,5},{12,45})); // var3 equals {12,45}
\end{verbatim}

\subsubsection{See also:}\label{see-also-203}

\href{OperatorsSZ\#source_of}{source\_of},

\begin{center}\rule{0.5\linewidth}{\linethickness}\end{center}

\subsection{\texorpdfstring{\texttt{teapot}}{teapot}}\label{teapot}

\subsubsection{Possible use:}\label{possible-use-516}

\begin{itemize}
\tightlist
\item
  \textbf{\texttt{teapot}} (\texttt{float}) ---\textgreater{}
  \texttt{geometry}
\end{itemize}

\subsubsection{Result:}\label{result-499}

A teapot geometry which radius is equal to the operand.

\subsubsection{Comment:}\label{comment-98}

the centre of the teapot is by default the location of the current agent
in which has been called this operator.

\subsubsection{Special cases:}\label{special-cases-136}

\begin{itemize}
\tightlist
\item
  returns a point if the operand is lower or equal to 0.
\end{itemize}

\subsubsection{Examples:}\label{examples-357}

\begin{verbatim}
 
geometry var0 <- teapot(10); // var0 equals a geometry as a circle of radius 10 but displays a teapot.
\end{verbatim}

\subsubsection{See also:}\label{see-also-204}

\href{OperatorsAA\#around}{around}, \href{OperatorsBC\#cone}{cone},
\href{OperatorsIM\#line}{line}, \href{OperatorsIM\#link}{link},
\href{OperatorsNR\#norm}{norm}, \href{OperatorsNR\#point}{point},
\href{OperatorsNR\#polygon}{polygon},
\href{OperatorsNR\#polyline}{polyline},
\href{OperatorsNR\#rectangle}{rectangle},
\href{OperatorsSZ\#square}{square},
\href{OperatorsSZ\#triangle}{triangle},

\begin{center}\rule{0.5\linewidth}{\linethickness}\end{center}

\subsection{\texorpdfstring{\texttt{text\_file}}{text\_file}}\label{text_file}

\subsubsection{Possible use:}\label{possible-use-517}

\begin{itemize}
\tightlist
\item
  \textbf{\texttt{text\_file}} (\texttt{string}) ---\textgreater{}
  \texttt{file}
\end{itemize}

\subsubsection{Result:}\label{result-500}

Constructs a file of type text. Allowed extensions are limited to txt,
data, text

\begin{center}\rule{0.5\linewidth}{\linethickness}\end{center}

\subsection{\texorpdfstring{\texttt{TGauss}}{TGauss}}\label{tgauss}

Same signification as
\href{OperatorsSZ\#truncated_gauss}{truncated\_gauss}

\begin{center}\rule{0.5\linewidth}{\linethickness}\end{center}

\subsection{\texorpdfstring{\texttt{threeds\_file}}{threeds\_file}}\label{threeds_file}

\subsubsection{Possible use:}\label{possible-use-518}

\begin{itemize}
\tightlist
\item
  \textbf{\texttt{threeds\_file}} (\texttt{string}) ---\textgreater{}
  \texttt{file}
\end{itemize}

\subsubsection{Result:}\label{result-501}

Constructs a file of type threeds. Allowed extensions are limited to
3ds, max

\begin{center}\rule{0.5\linewidth}{\linethickness}\end{center}

\subsection{\texorpdfstring{\texttt{to}}{to}}\label{to}

Same signification as \href{OperatorsSZ\#until}{until}

\subsubsection{Possible use:}\label{possible-use-519}

\begin{itemize}
\tightlist
\item
  \texttt{date} \textbf{\texttt{to}} \texttt{date} ---\textgreater{}
  \texttt{msi.gama.util.IList\textless{}msi.gama.util.GamaDate\textgreater{}}
\item
  \textbf{\texttt{to}} (\texttt{date} , \texttt{date}) ---\textgreater{}
  \texttt{msi.gama.util.IList\textless{}msi.gama.util.GamaDate\textgreater{}}
\end{itemize}

\subsubsection{Result:}\label{result-502}

builds an interval between two dates (the first inclusive and the second
exclusive, which behaves like a read-only list of dates. The default
step between two dates is the step of the model

\subsubsection{Comment:}\label{comment-99}

The default step can be overruled by using the every operator applied to
this interval

\subsubsection{Examples:}\label{examples-358}

\begin{verbatim}
date('2000-01-01') to date('2010-01-01') // builds an interval between these two dates (date('2000-01-01') to date('2010-01-01')) every (#month) // builds an interval between these two dates which contains all the monthly dates starting from the beginning of the interval 
\end{verbatim}

\subsubsection{See also:}\label{see-also-205}

\href{OperatorsDH\#every}{every},

\begin{center}\rule{0.5\linewidth}{\linethickness}\end{center}

\subsection{\texorpdfstring{\texttt{to\_GAMA\_CRS}}{to\_GAMA\_CRS}}\label{to_gama_crs}

\subsubsection{Possible use:}\label{possible-use-520}

\begin{itemize}
\tightlist
\item
  \textbf{\texttt{to\_GAMA\_CRS}} (\texttt{geometry}) ---\textgreater{}
  \texttt{geometry}
\item
  \texttt{geometry} \textbf{\texttt{to\_GAMA\_CRS}} \texttt{string}
  ---\textgreater{} \texttt{geometry}
\item
  \textbf{\texttt{to\_GAMA\_CRS}} (\texttt{geometry} , \texttt{string})
  ---\textgreater{} \texttt{geometry}
\end{itemize}

\subsubsection{Special cases:}\label{special-cases-137}

\begin{itemize}
\tightlist
\item
  returns the geometry corresponding to the transformation of the given
  geometry to the GAMA CRS (Coordinate Reference System) assuming the
  given geometry is referenced by the current CRS, the one corresponding
  to the world's agent one
\end{itemize}

\begin{verbatim}
 
geometry var0 <- to_GAMA_CRS({121,14}); // var0 equals a geometry corresponding to the agent geometry transformed into the GAMA CRS
\end{verbatim}

\begin{itemize}
\tightlist
\item
  returns the geometry corresponding to the transformation of the given
  geometry to the GAMA CRS (Coordinate Reference System) assuming the
  given geometry is referenced by given CRS
\end{itemize}

\begin{verbatim}
 
geometry var1 <- to_GAMA_CRS({121,14}, "EPSG:4326"); // var1 equals a geometry corresponding to the agent geometry transformed into the GAMA CRS
\end{verbatim}

\begin{center}\rule{0.5\linewidth}{\linethickness}\end{center}

\subsection{\texorpdfstring{\texttt{to\_gaml}}{to\_gaml}}\label{to_gaml}

\subsubsection{Possible use:}\label{possible-use-521}

\begin{itemize}
\tightlist
\item
  \textbf{\texttt{to\_gaml}} (\texttt{unknown}) ---\textgreater{}
  \texttt{string}
\end{itemize}

\subsubsection{Result:}\label{result-503}

returns the literal description of an expression or description --
action, behavior, species, aspect, even model -- in gaml

\subsubsection{Examples:}\label{examples-359}

\begin{verbatim}
 
string var0 <- to_gaml(0); // var0 equals '0' 
string var1 <- to_gaml(3.78); // var1 equals '3.78' 
string var2 <- to_gaml(true); // var2 equals 'true' 
string var3 <- to_gaml({23, 4.0}); // var3 equals '{23.0,4.0,0.0}' 
string var4 <- to_gaml(5::34); // var4 equals '5::34' 
string var5 <- to_gaml(rgb(255,0,125)); // var5 equals 'rgb (255, 0, 125,255)' 
string var6 <- to_gaml('hello'); // var6 equals "'hello'" 
string var7 <- to_gaml([1,5,9,3]); // var7 equals '[1,5,9,3]' 
string var8 <- to_gaml(['a'::345, 'b'::13, 'c'::12]); // var8 equals "map(['a'::345,'b'::13,'c'::12])" 
string var9 <- to_gaml([[3,5,7,9],[2,4,6,8]]); // var9 equals '[[3,5,7,9],[2,4,6,8]]' 
string var10 <- to_gaml(a_graph); // var10 equals ([((1 as node)::(3 as node))::(5 as edge),((0 as node)::(3 as node))::(3 as edge),((1 as node)::(2 as node))::(1 as edge),((0 as node)::(2 as node))::(2 as edge),((0 as node)::(1 as node))::(0 as edge),((2 as node)::(3 as node))::(4 as edge)] as map ) as graph 
string var11 <- to_gaml(node1); // var11 equals  1 as node
\end{verbatim}

\begin{center}\rule{0.5\linewidth}{\linethickness}\end{center}

\subsection{\texorpdfstring{\texttt{to\_rectangles}}{to\_rectangles}}\label{to_rectangles}

\subsubsection{Possible use:}\label{possible-use-522}

\begin{itemize}
\tightlist
\item
  \textbf{\texttt{to\_rectangles}} (\texttt{geometry}, \texttt{point},
  \texttt{bool}) ---\textgreater{}
  \texttt{list\textless{}geometry\textgreater{}}
\item
  \textbf{\texttt{to\_rectangles}} (\texttt{geometry}, \texttt{int},
  \texttt{int}, \texttt{bool}) ---\textgreater{}
  \texttt{list\textless{}geometry\textgreater{}}
\end{itemize}

\subsubsection{Result:}\label{result-504}

A list of rectangles of the size corresponding to the given dimension
that result from the decomposition of the geometry into rectangles
(geometry, dimension, overlaps), if overlaps = true, add the rectangles
that overlap the border of the geometry A list of rectangles
corresponding to the given dimension that result from the decomposition
of the geometry into rectangles (geometry, nb\_cols, nb\_rows, overlaps)
by a grid composed of the given number of columns and rows, if overlaps
= true, add the rectangles that overlap the border of the geometry

\subsubsection{Examples:}\label{examples-360}

\begin{verbatim}
 
list<geometry> var0 <- to_rectangles(self, {10.0, 15.0}, true); // var0 equals the list of rectangles of size {10.0, 15.0} corresponding to the discretization into rectangles of the geometry of the agent applying the operator. The rectangles overlapping the border of the geometry are kept 
list<geometry> var1 <- to_rectangles(self, 5, 20, true); // var1 equals the list of rectangles corresponding to the discretization by a grid of 5 columns and 20 rows into rectangles of the geometry of the agent applying the operator. The rectangles overlapping the border of the geometry are kept
\end{verbatim}

\begin{center}\rule{0.5\linewidth}{\linethickness}\end{center}

\subsection{\texorpdfstring{\texttt{to\_squares}}{to\_squares}}\label{to_squares}

\subsubsection{Possible use:}\label{possible-use-523}

\begin{itemize}
\tightlist
\item
  \textbf{\texttt{to\_squares}} (\texttt{geometry}, \texttt{int},
  \texttt{bool}) ---\textgreater{}
  \texttt{list\textless{}geometry\textgreater{}}
\item
  \textbf{\texttt{to\_squares}} (\texttt{geometry}, \texttt{float},
  \texttt{bool}) ---\textgreater{}
  \texttt{list\textless{}geometry\textgreater{}}
\item
  \textbf{\texttt{to\_squares}} (\texttt{geometry}, \texttt{int},
  \texttt{bool}, \texttt{float}) ---\textgreater{}
  \texttt{list\textless{}geometry\textgreater{}}
\end{itemize}

\subsubsection{Result:}\label{result-505}

A list of a given number of squares from the decomposition of the
geometry into squares (geometry, nb\_square, overlaps,
precision\_coefficient), if overlaps = true, add the squares that
overlap the border of the geometry, coefficient\_precision should be
close to 1.0 A list of a given number of squares from the decomposition
of the geometry into squares (geometry, nb\_square, overlaps), if
overlaps = true, add the squares that overlap the border of the geometry
A list of squares of the size corresponding to the given size that
result from the decomposition of the geometry into squares (geometry,
size, overlaps), if overlaps = true, add the squares that overlap the
border of the geometry

\subsubsection{Examples:}\label{examples-361}

\begin{verbatim}
 
list<geometry> var0 <- to_squares(self, 10, true, 0.99); // var0 equals the list of 10 squares corresponding to the discretization into squares of the geometry of the agent applying the operator. The squares overlapping the border of the geometry are kept 
list<geometry> var1 <- to_squares(self, 10, true); // var1 equals the list of 10 squares corresponding to the discretization into squares of the geometry of the agent applying the operator. The squares overlapping the border of the geometry are kept 
list<geometry> var2 <- to_squares(self, 10.0, true); // var2 equals the list of squares of side size 10.0 corresponding to the discretization into squares of the geometry of the agent applying the operator. The squares overlapping the border of the geometry are kept
\end{verbatim}

\begin{center}\rule{0.5\linewidth}{\linethickness}\end{center}

\subsection{\texorpdfstring{\texttt{to\_sub\_geometries}}{to\_sub\_geometries}}\label{to_sub_geometries}

\subsubsection{Possible use:}\label{possible-use-524}

\begin{itemize}
\tightlist
\item
  \texttt{geometry} \textbf{\texttt{to\_sub\_geometries}}
  \texttt{list\textless{}float\textgreater{}} ---\textgreater{}
  \texttt{list\textless{}geometry\textgreater{}}
\item
  \textbf{\texttt{to\_sub\_geometries}} (\texttt{geometry} ,
  \texttt{list\textless{}float\textgreater{}}) ---\textgreater{}
  \texttt{list\textless{}geometry\textgreater{}}
\item
  \textbf{\texttt{to\_sub\_geometries}} (\texttt{geometry},
  \texttt{list\textless{}float\textgreater{}}, \texttt{float})
  ---\textgreater{} \texttt{list\textless{}geometry\textgreater{}}
\end{itemize}

\subsubsection{Result:}\label{result-506}

A list of geometries resulting after spliting the geometry into
sub-geometries. A list of geometries resulting after spliting the
geometry into sub-geometries.

\subsubsection{Examples:}\label{examples-362}

\begin{verbatim}
 
list<geometry> var0 <- to_sub_geometries(rectangle(10, 50), [0.1, 0.5, 0.4], 1.0); // var0 equals a list of three geometries corresponding to 3 sub-geometries using cubes of 1m size 
list<geometry> var1 <- to_sub_geometries(rectangle(10, 50), [0.1, 0.5, 0.4]); // var1 equals a list of three geometries corresponding to 3 sub-geometries
\end{verbatim}

\begin{center}\rule{0.5\linewidth}{\linethickness}\end{center}

\subsection{\texorpdfstring{\texttt{to\_triangles}}{to\_triangles}}\label{to_triangles}

Same signification as \href{OperatorsSZ\#triangulate}{triangulate}

\begin{center}\rule{0.5\linewidth}{\linethickness}\end{center}

\subsection{\texorpdfstring{\texttt{tokenize}}{tokenize}}\label{tokenize}

Same signification as \href{OperatorsSZ\#split_with}{split\_with}

\begin{center}\rule{0.5\linewidth}{\linethickness}\end{center}

\subsection{\texorpdfstring{\texttt{topology}}{topology}}\label{topology}

\subsubsection{Possible use:}\label{possible-use-525}

\begin{itemize}
\tightlist
\item
  \textbf{\texttt{topology}} (\texttt{unknown}) ---\textgreater{}
  \texttt{topology}
\end{itemize}

\subsubsection{Result:}\label{result-507}

casting of the operand to a topology.

\subsubsection{Special cases:}\label{special-cases-138}

\begin{itemize}
\tightlist
\item
  if the operand is a topology, returns the topology itself;\\
\item
  if the operand is a spatial graph, returns the graph topology
  associated;\\
\item
  if the operand is a population, returns the topology of the
  population;\\
\item
  if the operand is a shape or a geometry, returns the continuous
  topology bounded by the geometry;\\
\item
  if the operand is a matrix, returns the grid topology associated\\
\item
  if the operand is another kind of container, returns the multiple
  topology associated to the container\\
\item
  otherwise, casts the operand to a geometry and build a topology from
  it.
\end{itemize}

\subsubsection{Examples:}\label{examples-363}

\begin{verbatim}
 
topology var0 <- topology(0); // var0 equals niltopology(a_graph)   --: Multiple topology in POLYGON ((24.712119771887785 7.867357373616512, 24.712119771887785 61.283226839310565, 82.4013676510046  7.867357373616512)) at location[53.556743711446195;34.57529210646354] 
\end{verbatim}

\subsubsection{See also:}\label{see-also-206}

\href{OperatorsDH\#geometry}{geometry},

\begin{center}\rule{0.5\linewidth}{\linethickness}\end{center}

\subsection{\texorpdfstring{\texttt{topology}}{topology}}\label{topology-1}

\subsubsection{Possible use:}\label{possible-use-526}

\begin{itemize}
\tightlist
\item
  \textbf{\texttt{topology}} (\texttt{any}) ---\textgreater{}
  \texttt{topology}
\end{itemize}

\subsubsection{Result:}\label{result-508}

Casts the operand into the type topology

\begin{center}\rule{0.5\linewidth}{\linethickness}\end{center}

\subsection{\texorpdfstring{\texttt{touches}}{touches}}\label{touches}

\subsubsection{Possible use:}\label{possible-use-527}

\begin{itemize}
\tightlist
\item
  \texttt{geometry} \textbf{\texttt{touches}} \texttt{geometry}
  ---\textgreater{} \texttt{bool}
\item
  \textbf{\texttt{touches}} (\texttt{geometry} , \texttt{geometry})
  ---\textgreater{} \texttt{bool}
\end{itemize}

\subsubsection{Result:}\label{result-509}

A boolean, equal to true if the left-geometry (or agent/point) touches
the right-geometry (or agent/point).

\subsubsection{Comment:}\label{comment-100}

returns true when the left-operand only touches the right-operand. When
one geometry covers partially (or fully) the other one, it returns
false.

\subsubsection{Special cases:}\label{special-cases-139}

\begin{itemize}
\tightlist
\item
  if one of the operand is null, returns false.
\end{itemize}

\subsubsection{Examples:}\label{examples-364}

\begin{verbatim}
 
bool var0 <- polyline([{10,10},{20,20}]) touches {15,15}; // var0 equals false 
bool var1 <- polyline([{10,10},{20,20}]) touches {10,10}; // var1 equals true 
bool var2 <- {15,15} touches {15,15}; // var2 equals false 
bool var3 <- polyline([{10,10},{20,20}]) touches polyline([{10,10},{5,5}]); // var3 equals true 
bool var4 <- polyline([{10,10},{20,20}]) touches polyline([{5,5},{15,15}]); // var4 equals false 
bool var5 <- polyline([{10,10},{20,20}]) touches polyline([{15,15},{25,25}]); // var5 equals false 
bool var6 <- polygon([{10,10},{10,20},{20,20},{20,10}]) touches polygon([{15,15},{15,25},{25,25},{25,15}]); // var6 equals false 
bool var7 <- polygon([{10,10},{10,20},{20,20},{20,10}]) touches polygon([{10,20},{20,20},{20,30},{10,30}]); // var7 equals true 
bool var8 <- polygon([{10,10},{10,20},{20,20},{20,10}]) touches polygon([{10,10},{0,10},{0,0},{10,0}]); // var8 equals true 
bool var9 <- polygon([{10,10},{10,20},{20,20},{20,10}]) touches {15,15}; // var9 equals false 
bool var10 <- polygon([{10,10},{10,20},{20,20},{20,10}]) touches {10,15}; // var10 equals true
\end{verbatim}

\subsubsection{See also:}\label{see-also-207}

\href{OperatorsDH\#disjoint_from}{disjoint\_from},
\href{OperatorsBC\#crosses}{crosses},
\href{OperatorsNR\#overlaps}{overlaps},
\href{OperatorsNR\#partially_overlaps}{partially\_overlaps},
\href{OperatorsIM\#intersects}{intersects},

\begin{center}\rule{0.5\linewidth}{\linethickness}\end{center}

\subsection{\texorpdfstring{\texttt{towards}}{towards}}\label{towards}

\subsubsection{Possible use:}\label{possible-use-528}

\begin{itemize}
\tightlist
\item
  \texttt{geometry} \textbf{\texttt{towards}} \texttt{geometry}
  ---\textgreater{} \texttt{float}
\item
  \textbf{\texttt{towards}} (\texttt{geometry} , \texttt{geometry})
  ---\textgreater{} \texttt{float}
\end{itemize}

\subsubsection{Result:}\label{result-510}

The direction (in degree) between the two geometries (geometries,
agents, points) considering the topology of the agent applying the
operator.

\subsubsection{Examples:}\label{examples-365}

\begin{verbatim}
 
float var0 <- ag1 towards ag2; // var0 equals the direction between ag1 and ag2 and ag3 considering the topology of the agent applying the operator
\end{verbatim}

\subsubsection{See also:}\label{see-also-208}

\href{OperatorsDH\#distance_between}{distance\_between},
\href{OperatorsDH\#distance_to}{distance\_to},
\href{OperatorsDH\#direction_between}{direction\_between},
\href{OperatorsNR\#path_between}{path\_between},
\href{OperatorsNR\#path_to}{path\_to},

\begin{center}\rule{0.5\linewidth}{\linethickness}\end{center}

\subsection{\texorpdfstring{\texttt{trace}}{trace}}\label{trace}

\subsubsection{Possible use:}\label{possible-use-529}

\begin{itemize}
\tightlist
\item
  \textbf{\texttt{trace}} (\texttt{matrix}) ---\textgreater{}
  \texttt{float}
\end{itemize}

\subsubsection{Result:}\label{result-511}

The trace of the given matrix (the sum of the elements on the main
diagonal).

\subsubsection{Examples:}\label{examples-366}

\begin{verbatim}
 
float var0 <- trace(matrix([[1,2],[3,4]])); // var0 equals 5
\end{verbatim}

\begin{center}\rule{0.5\linewidth}{\linethickness}\end{center}

\subsection{\texorpdfstring{\texttt{transformed\_by}}{transformed\_by}}\label{transformed_by}

\subsubsection{Possible use:}\label{possible-use-530}

\begin{itemize}
\tightlist
\item
  \texttt{geometry} \textbf{\texttt{transformed\_by}} \texttt{point}
  ---\textgreater{} \texttt{geometry}
\item
  \textbf{\texttt{transformed\_by}} (\texttt{geometry} , \texttt{point})
  ---\textgreater{} \texttt{geometry}
\end{itemize}

\subsubsection{Result:}\label{result-512}

A geometry resulting from the application of a rotation and a scaling
(right-operand : point \{angle(degree), scale factor\} of the left-hand
operand (geometry, agent, point)

\subsubsection{Examples:}\label{examples-367}

\begin{verbatim}
 
geometry var0 <- self transformed_by {45, 0.5}; // var0 equals the geometry resulting from 45 degrees rotation and 50% scaling of the geometry of the agent applying the operator.
\end{verbatim}

\subsubsection{See also:}\label{see-also-209}

\href{OperatorsNR\#rotated_by}{rotated\_by},
\href{OperatorsSZ\#translated_by}{translated\_by},

\begin{center}\rule{0.5\linewidth}{\linethickness}\end{center}

\subsection{\texorpdfstring{\texttt{translated\_by}}{translated\_by}}\label{translated_by}

\subsubsection{Possible use:}\label{possible-use-531}

\begin{itemize}
\tightlist
\item
  \texttt{geometry} \textbf{\texttt{translated\_by}} \texttt{point}
  ---\textgreater{} \texttt{geometry}
\item
  \textbf{\texttt{translated\_by}} (\texttt{geometry} , \texttt{point})
  ---\textgreater{} \texttt{geometry}
\end{itemize}

\subsubsection{Result:}\label{result-513}

A geometry resulting from the application of a translation by the
right-hand operand distance to the left-hand operand (geometry, agent,
point)

\subsubsection{Examples:}\label{examples-368}

\begin{verbatim}
 
geometry var0 <- self translated_by {10,10,10}; // var0 equals the geometry resulting from applying the translation to the left-hand geometry (or agent).
\end{verbatim}

\subsubsection{See also:}\label{see-also-210}

\href{OperatorsNR\#rotated_by}{rotated\_by},
\href{OperatorsSZ\#transformed_by}{transformed\_by},

\begin{center}\rule{0.5\linewidth}{\linethickness}\end{center}

\subsection{\texorpdfstring{\texttt{translated\_to}}{translated\_to}}\label{translated_to}

Same signification as \href{OperatorsAA\#at_location}{at\_location}

\begin{center}\rule{0.5\linewidth}{\linethickness}\end{center}

\subsection{\texorpdfstring{\texttt{transpose}}{transpose}}\label{transpose}

\subsubsection{Possible use:}\label{possible-use-532}

\begin{itemize}
\tightlist
\item
  \textbf{\texttt{transpose}} (\texttt{matrix}) ---\textgreater{}
  \texttt{matrix}
\end{itemize}

\subsubsection{Result:}\label{result-514}

The transposition of the given matrix

\subsubsection{Examples:}\label{examples-369}

\begin{verbatim}
 
matrix var0 <- transpose(matrix([[5,-3],[6,-4]])); // var0 equals matrix([[5,6],[-3,-4]])
\end{verbatim}

\begin{center}\rule{0.5\linewidth}{\linethickness}\end{center}

\subsection{\texorpdfstring{\texttt{triangle}}{triangle}}\label{triangle}

\subsubsection{Possible use:}\label{possible-use-533}

\begin{itemize}
\tightlist
\item
  \textbf{\texttt{triangle}} (\texttt{float}) ---\textgreater{}
  \texttt{geometry}
\end{itemize}

\subsubsection{Result:}\label{result-515}

A triangle geometry which side size is given by the operand.

\subsubsection{Comment:}\label{comment-101}

the center of the triangle is by default the location of the current
agent in which has been called this operator.

\subsubsection{Special cases:}\label{special-cases-140}

\begin{itemize}
\tightlist
\item
  returns nil if the operand is nil.
\end{itemize}

\subsubsection{Examples:}\label{examples-370}

\begin{verbatim}
 
geometry var0 <- triangle(5); // var0 equals a geometry as a triangle with side_size = 5.
\end{verbatim}

\subsubsection{See also:}\label{see-also-211}

\href{OperatorsAA\#around}{around}, \href{OperatorsBC\#circle}{circle},
\href{OperatorsBC\#cone}{cone}, \href{OperatorsIM\#line}{line},
\href{OperatorsIM\#link}{link}, \href{OperatorsNR\#norm}{norm},
\href{OperatorsNR\#point}{point}, \href{OperatorsNR\#polygon}{polygon},
\href{OperatorsNR\#polyline}{polyline},
\href{OperatorsNR\#rectangle}{rectangle},
\href{OperatorsSZ\#square}{square},

\begin{center}\rule{0.5\linewidth}{\linethickness}\end{center}

\subsection{\texorpdfstring{\texttt{triangulate}}{triangulate}}\label{triangulate}

\subsubsection{Possible use:}\label{possible-use-534}

\begin{itemize}
\tightlist
\item
  \textbf{\texttt{triangulate}}
  (\texttt{list\textless{}geometry\textgreater{}}) ---\textgreater{}
  \texttt{list\textless{}geometry\textgreater{}}
\item
  \textbf{\texttt{triangulate}} (\texttt{geometry}) ---\textgreater{}
  \texttt{list\textless{}geometry\textgreater{}}
\item
  \texttt{geometry} \textbf{\texttt{triangulate}} \texttt{float}
  ---\textgreater{} \texttt{list\textless{}geometry\textgreater{}}
\item
  \textbf{\texttt{triangulate}} (\texttt{geometry} , \texttt{float})
  ---\textgreater{} \texttt{list\textless{}geometry\textgreater{}}
\item
  \textbf{\texttt{triangulate}} (\texttt{geometry}, \texttt{float},
  \texttt{float}) ---\textgreater{}
  \texttt{list\textless{}geometry\textgreater{}}
\end{itemize}

\subsubsection{Result:}\label{result-516}

A list of geometries (triangles) corresponding to the Delaunay
triangulation computed from the list of polylines A list of geometries
(triangles) corresponding to the Delaunay triangulation of the operand
geometry (geometry, agent, point) with the given tolerance for the
clipping A list of geometries (triangles) corresponding to the Delaunay
triangulation of the operand geometry (geometry, agent, point) with the
given tolerance for the clipping and for the triangulation A list of
geometries (triangles) corresponding to the Delaunay triangulation of
the operand geometry (geometry, agent, point)

\subsubsection{Examples:}\label{examples-371}

\begin{verbatim}
 
list<geometry> var0 <- triangulate([line([{0,50},{100,50}]), line([{50,0},{50,100}])); // var0 equals the list of geometries (triangles) corresponding to the Delaunay triangulation of the geometry of the agent applying the operator. 
list<geometry> var1 <- triangulate(self, 0.1); // var1 equals the list of geometries (triangles) corresponding to the Delaunay triangulation of the geometry of the agent applying the operator. 
list<geometry> var2 <- triangulate(self,0.1, 1.0); // var2 equals the list of geometries (triangles) corresponding to the Delaunay triangulation of the geometry of the agent applying the operator. 
list<geometry> var3 <- triangulate(self); // var3 equals the list of geometries (triangles) corresponding to the Delaunay triangulation of the geometry of the agent applying the operator.
\end{verbatim}

\begin{center}\rule{0.5\linewidth}{\linethickness}\end{center}

\subsection{\texorpdfstring{\texttt{truncated\_gauss}}{truncated\_gauss}}\label{truncated_gauss}

\subsubsection{Possible use:}\label{possible-use-535}

\begin{itemize}
\tightlist
\item
  \textbf{\texttt{truncated\_gauss}} (\texttt{point}) ---\textgreater{}
  \texttt{float}
\item
  \textbf{\texttt{truncated\_gauss}} (\texttt{list}) ---\textgreater{}
  \texttt{float}
\end{itemize}

\subsubsection{Result:}\label{result-517}

A random value from a normally distributed random variable in the
interval {]}mean - standardDeviation; mean + standardDeviation{[}.

\subsubsection{Special cases:}\label{special-cases-141}

\begin{itemize}
\tightlist
\item
  when the operand is a point, it is read as \{mean,
  standardDeviation\}\\
\item
  if the operand is a list, only the two first elements are taken into
  account as {[}mean, standardDeviation{]}\\
\item
  when truncated\_gauss is called with a list of only one element mean,
  it will always return 0.0
\end{itemize}

\subsubsection{Examples:}\label{examples-372}

\begin{verbatim}
 
float var0 <- truncated_gauss ({0, 0.3}); // var0 equals a float between -0.3 and 0.3 
float var1 <- truncated_gauss ([0.5, 0.0]); // var1 equals 0.5
\end{verbatim}

\subsubsection{See also:}\label{see-also-212}

\href{OperatorsDH\#gauss}{gauss},

\begin{center}\rule{0.5\linewidth}{\linethickness}\end{center}

\subsection{\texorpdfstring{\texttt{type\_of}}{type\_of}}\label{type_of}

\subsubsection{Possible use:}\label{possible-use-536}

\begin{itemize}
\tightlist
\item
  \textbf{\texttt{type\_of}} (\texttt{unknown}) ---\textgreater{}
  \texttt{msi.gaml.types.IType\textless{}?\textgreater{}}
\end{itemize}

\begin{center}\rule{0.5\linewidth}{\linethickness}\end{center}

\subsection{\texorpdfstring{\texttt{undirected}}{undirected}}\label{undirected}

\subsubsection{Possible use:}\label{possible-use-537}

\begin{itemize}
\tightlist
\item
  \textbf{\texttt{undirected}} (\texttt{graph}) ---\textgreater{}
  \texttt{graph}
\end{itemize}

\subsubsection{Result:}\label{result-518}

the operand graph becomes an undirected graph.

\subsubsection{Comment:}\label{comment-102}

the operator alters the operand graph, it does not create a new one.

\subsubsection{See also:}\label{see-also-213}

\href{OperatorsDH\#directed}{directed},

\begin{center}\rule{0.5\linewidth}{\linethickness}\end{center}

\subsection{\texorpdfstring{\texttt{union}}{union}}\label{union}

\subsubsection{Possible use:}\label{possible-use-538}

\begin{itemize}
\tightlist
\item
  \textbf{\texttt{union}}
  (\texttt{container\textless{}geometry\textgreater{}})
  ---\textgreater{} \texttt{geometry}
\item
  \texttt{container} \textbf{\texttt{union}} \texttt{container}
  ---\textgreater{} \texttt{list}
\item
  \textbf{\texttt{union}} (\texttt{container} , \texttt{container})
  ---\textgreater{} \texttt{list}
\end{itemize}

\subsubsection{Result:}\label{result-519}

returns a new list containing all the elements of both containers
without duplicated elements.

\subsubsection{Special cases:}\label{special-cases-142}

\begin{itemize}
\tightlist
\item
  if the right-operand is a container of points, geometries or agents,
  returns the geometry resulting from the union all the geometries\\
\item
  if the left or right operand is nil, union throws an error
\end{itemize}

\subsubsection{Examples:}\label{examples-373}

\begin{verbatim}
 
geometry var0 <- union([geom1, geom2, geom3]); // var0 equals a geometry corresponding to union between geom1, geom2 and geom3 
list var1 <- [1,2,3,4,5,6] union [2,4,9]; // var1 equals [1,2,3,4,5,6,9] 
list var2 <- [1,2,3,4,5,6] union [0,8]; // var2 equals [1,2,3,4,5,6,0,8] 
list var3 <- [1,3,2,4,5,6,8,5,6] union [0,8]; // var3 equals [1,3,2,4,5,6,8,0]
\end{verbatim}

\subsubsection{See also:}\label{see-also-214}

\href{OperatorsIM\#inter}{inter}, \href{OperatorsAA\#+}{+},

\begin{center}\rule{0.5\linewidth}{\linethickness}\end{center}

\subsection{\texorpdfstring{\texttt{unknown}}{unknown}}\label{unknown}

\subsubsection{Possible use:}\label{possible-use-539}

\begin{itemize}
\tightlist
\item
  \textbf{\texttt{unknown}} (\texttt{any}) ---\textgreater{}
  \texttt{unknown}
\end{itemize}

\subsubsection{Result:}\label{result-520}

Casts the operand into the type unknown

\begin{center}\rule{0.5\linewidth}{\linethickness}\end{center}

\subsection{\texorpdfstring{\texttt{unSerializeSimulation}}{unSerializeSimulation}}\label{unserializesimulation}

\subsubsection{Possible use:}\label{possible-use-540}

\begin{itemize}
\tightlist
\item
  \textbf{\texttt{unSerializeSimulation}} (\texttt{string})
  ---\textgreater{} \texttt{int}
\end{itemize}

\subsubsection{Result:}\label{result-521}

unSerializeSimulation

\begin{center}\rule{0.5\linewidth}{\linethickness}\end{center}

\subsection{\texorpdfstring{\texttt{unSerializeSimulationFromFile}}{unSerializeSimulationFromFile}}\label{unserializesimulationfromfile}

\subsubsection{Possible use:}\label{possible-use-541}

\begin{itemize}
\tightlist
\item
  \textbf{\texttt{unSerializeSimulationFromFile}} (\texttt{string})
  ---\textgreater{} \texttt{int}
\end{itemize}

\begin{center}\rule{0.5\linewidth}{\linethickness}\end{center}

\subsection{\texorpdfstring{\texttt{until}}{until}}\label{until}

\subsubsection{Possible use:}\label{possible-use-542}

\begin{itemize}
\tightlist
\item
  \textbf{\texttt{until}} (\texttt{date}) ---\textgreater{}
  \texttt{bool}
\item
  \texttt{any\ expression} \textbf{\texttt{until}} \texttt{date}
  ---\textgreater{} \texttt{bool}
\item
  \textbf{\texttt{until}} (\texttt{any\ expression} , \texttt{date})
  ---\textgreater{} \texttt{bool}
\end{itemize}

\subsubsection{Result:}\label{result-522}

Returns true if the current\_date of the model is before (or equel to)
the date passed in argument. Synonym of `current\_date \textless{}=
argument'

\subsubsection{Examples:}\label{examples-374}

\begin{verbatim}
reflex when: until(starting_date) {}    // This reflex will be run only once at the beginning of the simulation 
\end{verbatim}

\begin{center}\rule{0.5\linewidth}{\linethickness}\end{center}

\subsection{\texorpdfstring{\texttt{upper\_case}}{upper\_case}}\label{upper_case}

\subsubsection{Possible use:}\label{possible-use-543}

\begin{itemize}
\tightlist
\item
  \textbf{\texttt{upper\_case}} (\texttt{string}) ---\textgreater{}
  \texttt{string}
\end{itemize}

\subsubsection{Result:}\label{result-523}

Converts all of the characters in the string operand to upper case

\subsubsection{Examples:}\label{examples-375}

\begin{verbatim}
 
string var0 <- upper_case("Abc"); // var0 equals 'ABC'
\end{verbatim}

\subsubsection{See also:}\label{see-also-215}

\href{OperatorsIM\#lower_case}{lower\_case},

\begin{center}\rule{0.5\linewidth}{\linethickness}\end{center}

\subsection{\texorpdfstring{\texttt{URL\_file}}{URL\_file}}\label{url_file}

\subsubsection{Possible use:}\label{possible-use-544}

\begin{itemize}
\tightlist
\item
  \textbf{\texttt{URL\_file}} (\texttt{string}) ---\textgreater{}
  \texttt{file}
\end{itemize}

\subsubsection{Result:}\label{result-524}

Constructs a file of type URL. Allowed extensions are limited to url

\begin{center}\rule{0.5\linewidth}{\linethickness}\end{center}

\subsection{\texorpdfstring{\texttt{use\_cache}}{use\_cache}}\label{use_cache}

\subsubsection{Possible use:}\label{possible-use-545}

\begin{itemize}
\tightlist
\item
  \texttt{graph} \textbf{\texttt{use\_cache}} \texttt{bool}
  ---\textgreater{} \texttt{graph}
\item
  \textbf{\texttt{use\_cache}} (\texttt{graph} , \texttt{bool})
  ---\textgreater{} \texttt{graph}
\end{itemize}

\subsubsection{Result:}\label{result-525}

if the second operand is true, the operand graph will store in a cache
all the previously computed shortest path (the cache be cleared if the
graph is modified).

\subsubsection{Comment:}\label{comment-103}

the operator alters the operand graph, it does not create a new one.

\subsubsection{See also:}\label{see-also-216}

\href{OperatorsNR\#path_between}{path\_between},

\begin{center}\rule{0.5\linewidth}{\linethickness}\end{center}

\subsection{\texorpdfstring{\texttt{user\_input}}{user\_input}}\label{user_input}

\subsubsection{Possible use:}\label{possible-use-546}

\begin{itemize}
\tightlist
\item
  \textbf{\texttt{user\_input}} (\texttt{any\ expression})
  ---\textgreater{} \texttt{map\textless{}string,unknown\textgreater{}}
\item
  \texttt{string} \textbf{\texttt{user\_input}} \texttt{any\ expression}
  ---\textgreater{} \texttt{map\textless{}string,unknown\textgreater{}}
\item
  \textbf{\texttt{user\_input}} (\texttt{string} ,
  \texttt{any\ expression}) ---\textgreater{}
  \texttt{map\textless{}string,unknown\textgreater{}}
\end{itemize}

\subsubsection{Result:}\label{result-526}

asks the user for some values (not defined as parameters). Takes a
string (optional) and a map as arguments. The string is used to specify
the message of the dialog box. The map is to specify the parameters you
want the user to change before the simulation starts, with the name of
the parameter in string key, and the default value as value.

\subsubsection{Comment:}\label{comment-104}

This operator takes a map {[}string::value{]} as argument, displays a
dialog asking the user for these values, and returns the same map with
the modified values (if any). The dialog is modal and will interrupt the
execution of the simulation until the user has either dismissed or
accepted it. It can be used, for instance, in an init section to force
the user to input new values instead of relying on the initial values of
parameters :

\subsubsection{Examples:}\label{examples-376}

\begin{verbatim}
map<string,unknown> values <- user_input(["Number" :: 100, "Location" :: {10, 10}]); create bug number: int(values at "Number") with: [location:: (point(values at "Location"))]; map<string,unknown> values2 <- user_input("Enter numer of agents and locations",["Number" :: 100, "Location" :: {10, 10}]); create bug number: int(values2 at "Number") with: [location:: (point(values2 at "Location"))]; 
\end{verbatim}

\begin{center}\rule{0.5\linewidth}{\linethickness}\end{center}

\subsection{\texorpdfstring{\texttt{using}}{using}}\label{using}

\subsubsection{Possible use:}\label{possible-use-547}

\begin{itemize}
\tightlist
\item
  \texttt{any\ expression} \textbf{\texttt{using}} \texttt{topology}
  ---\textgreater{} \texttt{unknown}
\item
  \textbf{\texttt{using}} (\texttt{any\ expression} , \texttt{topology})
  ---\textgreater{} \texttt{unknown}
\end{itemize}

\subsubsection{Result:}\label{result-527}

Allows to specify in which topology a spatial computation should take
place.

\subsubsection{Special cases:}\label{special-cases-143}

\begin{itemize}
\tightlist
\item
  has no effect if the topology passed as a parameter is nil
\end{itemize}

\subsubsection{Examples:}\label{examples-377}

\begin{verbatim}
 
unknown var0 <- (agents closest_to self) using topology(world); // var0 equals the closest agent to self (the caller) in the continuous topology of the world
\end{verbatim}

\begin{center}\rule{0.5\linewidth}{\linethickness}\end{center}

\subsection{\texorpdfstring{\texttt{variance}}{variance}}\label{variance}

\subsubsection{Possible use:}\label{possible-use-548}

\begin{itemize}
\tightlist
\item
  \textbf{\texttt{variance}} (\texttt{container}) ---\textgreater{}
  \texttt{float}
\end{itemize}

\subsubsection{Result:}\label{result-528}

the variance of the elements of the operand. See Variance for more
details.

\subsubsection{Comment:}\label{comment-105}

The operator casts all the numerical element of the list into float. The
elements that are not numerical are discarded.

\subsubsection{Examples:}\label{examples-378}

\begin{verbatim}
 
float var0 <- variance ([4.5, 3.5, 5.5, 7.0]); // var0 equals 1.671875
\end{verbatim}

\subsubsection{See also:}\label{see-also-217}

\href{OperatorsIM\#mean}{mean}, \href{OperatorsIM\#median}{median},

\begin{center}\rule{0.5\linewidth}{\linethickness}\end{center}

\subsection{\texorpdfstring{\texttt{variance}}{variance}}\label{variance-1}

\subsubsection{Possible use:}\label{possible-use-549}

\begin{itemize}
\tightlist
\item
  \textbf{\texttt{variance}} (\texttt{float}) ---\textgreater{}
  \texttt{float}
\item
  \textbf{\texttt{variance}} (\texttt{int}, \texttt{float},
  \texttt{float}) ---\textgreater{} \texttt{float}
\end{itemize}

\subsubsection{Result:}\label{result-529}

Returns the variance of a data sequence. That is (sumOfSquares -
mean*sum) / size with mean = sum/size. Returns the variance from a
standard deviation.

\begin{center}\rule{0.5\linewidth}{\linethickness}\end{center}

\subsection{\texorpdfstring{\texttt{variance\_of}}{variance\_of}}\label{variance_of}

\subsubsection{Possible use:}\label{possible-use-550}

\begin{itemize}
\tightlist
\item
  \texttt{container} \textbf{\texttt{variance\_of}}
  \texttt{any\ expression} ---\textgreater{} \texttt{unknown}
\item
  \textbf{\texttt{variance\_of}} (\texttt{container} ,
  \texttt{any\ expression}) ---\textgreater{} \texttt{unknown}
\end{itemize}

\subsubsection{Result:}\label{result-530}

the variance of the right-hand expression evaluated on each of the
elements of the left-hand operand

\subsubsection{Comment:}\label{comment-106}

in the right-hand operand, the keyword each can be used to represent, in
turn, each of the right-hand operand elements.

\subsubsection{See also:}\label{see-also-218}

\href{OperatorsIM\#min_of}{min\_of},
\href{OperatorsIM\#max_of}{max\_of},
\href{OperatorsSZ\#sum_of}{sum\_of},
\href{OperatorsNR\#product_of}{product\_of},

\begin{center}\rule{0.5\linewidth}{\linethickness}\end{center}

\subsection{\texorpdfstring{\texttt{vertical}}{vertical}}\label{vertical}

\subsubsection{Possible use:}\label{possible-use-551}

\begin{itemize}
\tightlist
\item
  \textbf{\texttt{vertical}}
  (\texttt{msi.gama.util.GamaMap\textless{}java.lang.Object,java.lang.Integer\textgreater{}})
  ---\textgreater{}
  \texttt{msi.gama.util.tree.GamaNode\textless{}java.lang.String\textgreater{}}
\end{itemize}

\begin{center}\rule{0.5\linewidth}{\linethickness}\end{center}

\subsection{\texorpdfstring{\texttt{voronoi}}{voronoi}}\label{voronoi}

\subsubsection{Possible use:}\label{possible-use-552}

\begin{itemize}
\tightlist
\item
  \textbf{\texttt{voronoi}}
  (\texttt{list\textless{}point\textgreater{}}) ---\textgreater{}
  \texttt{list\textless{}geometry\textgreater{}}
\item
  \texttt{list\textless{}point\textgreater{}} \textbf{\texttt{voronoi}}
  \texttt{geometry} ---\textgreater{}
  \texttt{list\textless{}geometry\textgreater{}}
\item
  \textbf{\texttt{voronoi}} (\texttt{list\textless{}point\textgreater{}}
  , \texttt{geometry}) ---\textgreater{}
  \texttt{list\textless{}geometry\textgreater{}}
\end{itemize}

\subsubsection{Result:}\label{result-531}

A list of geometries corresponding to the Voronoi diagram built from the
list of points A list of geometries corresponding to the Voronoi diagram
built from the list of points according to the given clip

\subsubsection{Examples:}\label{examples-379}

\begin{verbatim}
 
list<geometry> var0 <- voronoi([{10,10},{50,50},{90,90},{10,90},{90,10}]); // var0 equals the list of geometries corresponding to the Voronoi Diagram built from the list of points. 
list<geometry> var1 <- voronoi([{10,10},{50,50},{90,90},{10,90},{90,10}], square(300)); // var1 equals the list of geometries corresponding to the Voronoi Diagram built from the list of points with a square of 300m side size as clip.
\end{verbatim}

\begin{center}\rule{0.5\linewidth}{\linethickness}\end{center}

\subsection{\texorpdfstring{\texttt{weight\_of}}{weight\_of}}\label{weight_of}

\subsubsection{Possible use:}\label{possible-use-553}

\begin{itemize}
\tightlist
\item
  \texttt{graph} \textbf{\texttt{weight\_of}} \texttt{unknown}
  ---\textgreater{} \texttt{float}
\item
  \textbf{\texttt{weight\_of}} (\texttt{graph} , \texttt{unknown})
  ---\textgreater{} \texttt{float}
\end{itemize}

\subsubsection{Result:}\label{result-532}

returns the weight of the given edge (right-hand operand) contained in
the graph given in right-hand operand.

\subsubsection{Comment:}\label{comment-107}

In a localized graph, an edge has a weight by default (the distance
between both vertices).

\subsubsection{Special cases:}\label{special-cases-144}

\begin{itemize}
\tightlist
\item
  if the left-operand (the graph) is nil, returns nil\\
\item
  if the right-hand operand is not an edge of the given graph,
  weight\_of checks whether it is a node of the graph and tries to
  return its weight\\
\item
  if the right-hand operand is neither a node, nor an edge, returns 1.
\end{itemize}

\subsubsection{Examples:}\label{examples-380}

\begin{verbatim}
graph graphFromMap <-  as_edge_graph([{1,5}::{12,45},{12,45}::{34,56}]);  
float var1 <- graphFromMap weight_of(link({1,5},{12,45})); // var1 equals 1.0
\end{verbatim}

\begin{center}\rule{0.5\linewidth}{\linethickness}\end{center}

\subsection{\texorpdfstring{\texttt{weighted\_means\_DM}}{weighted\_means\_DM}}\label{weighted_means_dm}

\subsubsection{Possible use:}\label{possible-use-554}

\begin{itemize}
\tightlist
\item
  \texttt{msi.gama.util.IList\textless{}java.util.List\textgreater{}}
  \textbf{\texttt{weighted\_means\_DM}}
  \texttt{msi.gama.util.IList\textless{}java.util.Map\textless{}java.lang.String,java.lang.Object\textgreater{}\textgreater{}}
  ---\textgreater{} \texttt{int}
\item
  \textbf{\texttt{weighted\_means\_DM}}
  (\texttt{msi.gama.util.IList\textless{}java.util.List\textgreater{}} ,
  \texttt{msi.gama.util.IList\textless{}java.util.Map\textless{}java.lang.String,java.lang.Object\textgreater{}\textgreater{}})
  ---\textgreater{} \texttt{int}
\end{itemize}

\subsubsection{Result:}\label{result-533}

The index of the candidate that maximizes the weighted mean of its
criterion values. The first operand is the list of candidates (a
candidate is a list of criterion values); the second operand the list of
criterion (list of map)

\subsubsection{Special cases:}\label{special-cases-145}

\begin{itemize}
\tightlist
\item
  returns -1 is the list of candidates is nil or empty
\end{itemize}

\subsubsection{Examples:}\label{examples-381}

\begin{verbatim}
 
int var0 <- weighted_means_DM([[1.0, 7.0],[4.0,2.0],[3.0, 3.0]], [["name"::"utility", "weight" :: 2.0],["name"::"price", "weight" :: 1.0]]); // var0 equals 1
\end{verbatim}

\subsubsection{See also:}\label{see-also-219}

\href{OperatorsNR\#promethee_dm}{promethee\_DM},
\href{OperatorsDH\#electre_dm}{electre\_DM},
\href{OperatorsDH\#evidence_theory_dm}{evidence\_theory\_DM},

\begin{center}\rule{0.5\linewidth}{\linethickness}\end{center}

\subsection{\texorpdfstring{\texttt{where}}{where}}\label{where}

\subsubsection{Possible use:}\label{possible-use-555}

\begin{itemize}
\tightlist
\item
  \texttt{container} \textbf{\texttt{where}} \texttt{any\ expression}
  ---\textgreater{} \texttt{list}
\item
  \textbf{\texttt{where}} (\texttt{container} ,
  \texttt{any\ expression}) ---\textgreater{} \texttt{list}
\end{itemize}

\subsubsection{Result:}\label{result-534}

a list containing all the elements of the left-hand operand that make
the right-hand operand evaluate to true.

\subsubsection{Comment:}\label{comment-108}

in the right-hand operand, the keyword each can be used to represent, in
turn, each of the right-hand operand elements.

\subsubsection{Special cases:}\label{special-cases-146}

\begin{itemize}
\tightlist
\item
  if the left-hand operand is a list nil, where returns a new empty
  list\\
\item
  if the left-operand is a map, the keyword each will contain each value
\end{itemize}

\begin{verbatim}
 
list var4 <- [1::2, 3::4, 5::6] where (each >= 4); // var4 equals [4, 6]
\end{verbatim}

\subsubsection{Examples:}\label{examples-382}

\begin{verbatim}
 
list var0 <- [1,2,3,4,5,6,7,8] where (each > 3); // var0 equals [4, 5, 6, 7, 8]  
list var2 <- g2 where (length(g2 out_edges_of each) = 0 ); // var2 equals [node9, node7, node10, node8, node11] 
list var3 <- (list(node) where (round(node(each).location.x) > 32); // var3 equals [node2, node3]
\end{verbatim}

\subsubsection{See also:}\label{see-also-220}

\href{OperatorsDH\#first_with}{first\_with},
\href{OperatorsIM\#last_with}{last\_with},
\href{OperatorsSZ\#where}{where},

\begin{center}\rule{0.5\linewidth}{\linethickness}\end{center}

\subsection{\texorpdfstring{\texttt{with\_lifetime}}{with\_lifetime}}\label{with_lifetime}

\subsubsection{Possible use:}\label{possible-use-556}

\begin{itemize}
\tightlist
\item
  \texttt{predicate} \textbf{\texttt{with\_lifetime}} \texttt{int}
  ---\textgreater{} \texttt{predicate}
\item
  \textbf{\texttt{with\_lifetime}} (\texttt{predicate} , \texttt{int})
  ---\textgreater{} \texttt{predicate}
\end{itemize}

\subsubsection{Result:}\label{result-535}

change the parameters of the given predicate

\subsubsection{Examples:}\label{examples-383}

\begin{verbatim}
predicate with_lifetime 10 
\end{verbatim}

\begin{center}\rule{0.5\linewidth}{\linethickness}\end{center}

\subsection{\texorpdfstring{\texttt{with\_max\_of}}{with\_max\_of}}\label{with_max_of}

\subsubsection{Possible use:}\label{possible-use-557}

\begin{itemize}
\tightlist
\item
  \texttt{container} \textbf{\texttt{with\_max\_of}}
  \texttt{any\ expression} ---\textgreater{} \texttt{unknown}
\item
  \textbf{\texttt{with\_max\_of}} (\texttt{container} ,
  \texttt{any\ expression}) ---\textgreater{} \texttt{unknown}
\end{itemize}

\subsubsection{Result:}\label{result-536}

one of elements of the left-hand operand that maximizes the value of the
right-hand operand

\subsubsection{Comment:}\label{comment-109}

in the right-hand operand, the keyword each can be used to represent, in
turn, each of the right-hand operand elements.

\subsubsection{Special cases:}\label{special-cases-147}

\begin{itemize}
\tightlist
\item
  if the left-hand operand is nil, with\_max\_of returns the default
  value of the right-hand operand
\end{itemize}

\subsubsection{Examples:}\label{examples-384}

\begin{verbatim}
 
unknown var0 <- [1,2,3,4,5,6,7,8] with_max_of (each ); // var0 equals 8 
unknown var2 <- g2 with_max_of (length(g2 out_edges_of each)  ) ; // var2 equals node4 
unknown var3 <- (list(node) with_max_of (round(node(each).location.x)); // var3 equals node3 
unknown var4 <- [1::2, 3::4, 5::6] with_max_of (each); // var4 equals 6
\end{verbatim}

\subsubsection{See also:}\label{see-also-221}

\href{OperatorsSZ\#where}{where},
\href{OperatorsSZ\#with_min_of}{with\_min\_of},

\begin{center}\rule{0.5\linewidth}{\linethickness}\end{center}

\subsection{\texorpdfstring{\texttt{with\_min\_of}}{with\_min\_of}}\label{with_min_of}

\subsubsection{Possible use:}\label{possible-use-558}

\begin{itemize}
\tightlist
\item
  \texttt{container} \textbf{\texttt{with\_min\_of}}
  \texttt{any\ expression} ---\textgreater{} \texttt{unknown}
\item
  \textbf{\texttt{with\_min\_of}} (\texttt{container} ,
  \texttt{any\ expression}) ---\textgreater{} \texttt{unknown}
\end{itemize}

\subsubsection{Result:}\label{result-537}

one of elements of the left-hand operand that minimizes the value of the
right-hand operand

\subsubsection{Comment:}\label{comment-110}

in the right-hand operand, the keyword each can be used to represent, in
turn, each of the right-hand operand elements.

\subsubsection{Special cases:}\label{special-cases-148}

\begin{itemize}
\tightlist
\item
  if the left-hand operand is nil, with\_max\_of returns the default
  value of the right-hand operand
\end{itemize}

\subsubsection{Examples:}\label{examples-385}

\begin{verbatim}
 
unknown var0 <- [1,2,3,4,5,6,7,8] with_min_of (each ); // var0 equals 1 
unknown var2 <- g2 with_min_of (length(g2 out_edges_of each)  ); // var2 equals node11 
unknown var3 <- (list(node) with_min_of (round(node(each).location.x)); // var3 equals node0 
unknown var4 <- [1::2, 3::4, 5::6] with_min_of (each); // var4 equals 2
\end{verbatim}

\subsubsection{See also:}\label{see-also-222}

\href{OperatorsSZ\#where}{where},
\href{OperatorsSZ\#with_max_of}{with\_max\_of},

\begin{center}\rule{0.5\linewidth}{\linethickness}\end{center}

\subsection{\texorpdfstring{\texttt{with\_optimizer\_type}}{with\_optimizer\_type}}\label{with_optimizer_type}

\subsubsection{Possible use:}\label{possible-use-559}

\begin{itemize}
\tightlist
\item
  \texttt{graph} \textbf{\texttt{with\_optimizer\_type}} \texttt{string}
  ---\textgreater{} \texttt{graph}
\item
  \textbf{\texttt{with\_optimizer\_type}} (\texttt{graph} ,
  \texttt{string}) ---\textgreater{} \texttt{graph}
\end{itemize}

\subsubsection{Result:}\label{result-538}

changes the shortest path computation method of the given graph

\subsubsection{Comment:}\label{comment-111}

the right-hand operand can be ``Djikstra'', ``Bellmann'', ``Astar'' to
use the associated algorithm. Note that these methods are dynamic: the
path is computed when needed. In contrarily, if the operand is another
string, a static method will be used, i.e.~all the shortest are
previously computed.

\subsubsection{Examples:}\label{examples-386}

\begin{verbatim}
graphEpidemio <- graphEpidemio with_optimizer_type "static"; 
\end{verbatim}

\subsubsection{See also:}\label{see-also-223}

\href{OperatorsSZ\#set_verbose}{set\_verbose},

\begin{center}\rule{0.5\linewidth}{\linethickness}\end{center}

\subsection{\texorpdfstring{\texttt{with\_precision}}{with\_precision}}\label{with_precision}

\subsubsection{Possible use:}\label{possible-use-560}

\begin{itemize}
\tightlist
\item
  \texttt{float} \textbf{\texttt{with\_precision}} \texttt{int}
  ---\textgreater{} \texttt{float}
\item
  \textbf{\texttt{with\_precision}} (\texttt{float} , \texttt{int})
  ---\textgreater{} \texttt{float}
\item
  \texttt{point} \textbf{\texttt{with\_precision}} \texttt{int}
  ---\textgreater{} \texttt{point}
\item
  \textbf{\texttt{with\_precision}} (\texttt{point} , \texttt{int})
  ---\textgreater{} \texttt{point}
\item
  \texttt{geometry} \textbf{\texttt{with\_precision}} \texttt{int}
  ---\textgreater{} \texttt{geometry}
\item
  \textbf{\texttt{with\_precision}} (\texttt{geometry} , \texttt{int})
  ---\textgreater{} \texttt{geometry}
\end{itemize}

\subsubsection{Result:}\label{result-539}

Rounds off the value of left-hand operand to the precision given by the
value of right-hand operand Rounds off the ordinates of the left-hand
point to the precision given by the value of right-hand operand A
geometry corresponding to the rounding of points of the operand
considering a given precison.

\subsubsection{Examples:}\label{examples-387}

\begin{verbatim}
 
float var0 <- 12345.78943 with_precision 2; // var0 equals 12345.79 
float var1 <- 123 with_precision 2; // var1 equals 123.00 
point var2 <- {12345.78943, 12345.78943, 12345.78943} with_precision 2 ; // var2 equals {12345.79, 12345.79, 12345.79} 
geometry var3 <- self with_precision 2; // var3 equals the geometry resulting from the rounding of points of the geometry with a precision of 0.1.
\end{verbatim}

\subsubsection{See also:}\label{see-also-224}

\href{OperatorsNR\#round}{round},

\begin{center}\rule{0.5\linewidth}{\linethickness}\end{center}

\subsection{\texorpdfstring{\texttt{with\_values}}{with\_values}}\label{with_values}

\subsubsection{Possible use:}\label{possible-use-561}

\begin{itemize}
\tightlist
\item
  \texttt{predicate} \textbf{\texttt{with\_values}} \texttt{map}
  ---\textgreater{} \texttt{predicate}
\item
  \textbf{\texttt{with\_values}} (\texttt{predicate} , \texttt{map})
  ---\textgreater{} \texttt{predicate}
\end{itemize}

\subsubsection{Result:}\label{result-540}

change the parameters of the given predicate

\subsubsection{Examples:}\label{examples-388}

\begin{verbatim}
predicate with_values ["time"::10] 
\end{verbatim}

\begin{center}\rule{0.5\linewidth}{\linethickness}\end{center}

\subsection{\texorpdfstring{\texttt{with\_weights}}{with\_weights}}\label{with_weights}

\subsubsection{Possible use:}\label{possible-use-562}

\begin{itemize}
\tightlist
\item
  \texttt{graph} \textbf{\texttt{with\_weights}} \texttt{list}
  ---\textgreater{} \texttt{graph}
\item
  \textbf{\texttt{with\_weights}} (\texttt{graph} , \texttt{list})
  ---\textgreater{} \texttt{graph}
\item
  \texttt{graph} \textbf{\texttt{with\_weights}} \texttt{map}
  ---\textgreater{} \texttt{graph}
\item
  \textbf{\texttt{with\_weights}} (\texttt{graph} , \texttt{map})
  ---\textgreater{} \texttt{graph}
\end{itemize}

\subsubsection{Result:}\label{result-541}

returns the graph (left-hand operand) with weight given in the map
(right-hand operand).

\subsubsection{Comment:}\label{comment-112}

this operand re-initializes the path finder

\subsubsection{Special cases:}\label{special-cases-149}

\begin{itemize}
\tightlist
\item
  if the right-hand operand is a list, affects the n elements of the
  list to the n first edges. Note that the ordering of edges may change
  overtime, which can create some problems\ldots{}\\
\item
  if the left-hand operand is a map, the map should contains pairs such
  as: vertex/edge::double
\end{itemize}

\begin{verbatim}
graph_from_edges (list(ant) as_map each::one_of (list(ant))) with_weights (list(ant) as_map each::each.food) 
\end{verbatim}

\begin{center}\rule{0.5\linewidth}{\linethickness}\end{center}

\subsection{\texorpdfstring{\texttt{without\_holes}}{without\_holes}}\label{without_holes}

\subsubsection{Possible use:}\label{possible-use-563}

\begin{itemize}
\tightlist
\item
  \textbf{\texttt{without\_holes}} (\texttt{geometry}) ---\textgreater{}
  \texttt{geometry}
\end{itemize}

\subsubsection{Result:}\label{result-542}

A geometry corresponding to the operand geometry (geometry, agent,
point) without its holes

\subsubsection{Examples:}\label{examples-389}

\begin{verbatim}
 
geometry var0 <- solid(self); // var0 equals the geometry corresponding to the geometry of the agent applying the operator without its holes.
\end{verbatim}

\begin{center}\rule{0.5\linewidth}{\linethickness}\end{center}

\subsection{\texorpdfstring{\texttt{writable}}{writable}}\label{writable}

\subsubsection{Possible use:}\label{possible-use-564}

\begin{itemize}
\tightlist
\item
  \texttt{file} \textbf{\texttt{writable}} \texttt{bool}
  ---\textgreater{} \texttt{file}
\item
  \textbf{\texttt{writable}} (\texttt{file} , \texttt{bool})
  ---\textgreater{} \texttt{file}
\end{itemize}

\subsubsection{Result:}\label{result-543}

Marks the file as read-only or not, depending on the second boolean
argument, and returns the first argument

\subsubsection{Comment:}\label{comment-113}

A file is created using its native flags. This operator can change them.
Beware that this change is system-wide (and not only restrained to
GAMA): changing a file to read-only mode (e.g. ``writable(f, false)'')

\subsubsection{Examples:}\label{examples-390}

\begin{verbatim}
 
file var0 <- shape_file("../images/point_eau.shp") writable false; // var0 equals returns a file in read-only mode
\end{verbatim}

\subsubsection{See also:}\label{see-also-225}

\href{OperatorsDH\#file}{file},

\begin{center}\rule{0.5\linewidth}{\linethickness}\end{center}

\subsection{\texorpdfstring{\texttt{xml\_file}}{xml\_file}}\label{xml_file}

\subsubsection{Possible use:}\label{possible-use-565}

\begin{itemize}
\tightlist
\item
  \textbf{\texttt{xml\_file}} (\texttt{string}) ---\textgreater{}
  \texttt{file}
\end{itemize}

\subsubsection{Result:}\label{result-544}

Constructs a file of type xml. Allowed extensions are limited to xml

\begin{center}\rule{0.5\linewidth}{\linethickness}\end{center}

\subsection{\texorpdfstring{\texttt{xor}}{xor}}\label{xor}

\subsubsection{Possible use:}\label{possible-use-566}

\begin{itemize}
\tightlist
\item
  \texttt{bool} \textbf{\texttt{xor}} \texttt{bool} ---\textgreater{}
  \texttt{bool}
\item
  \textbf{\texttt{xor}} (\texttt{bool} , \texttt{bool})
  ---\textgreater{} \texttt{bool}
\end{itemize}

\subsubsection{Result:}\label{result-545}

a bool value, equal to the logical xor between the left-hand operand and
the right-hand operand. False when they are equal

\subsubsection{Comment:}\label{comment-114}

both operands are always casted to bool before applying the operator.
Thus, an expression like 1 xor 0 is accepted and returns true.

\subsubsection{See also:}\label{see-also-226}

\href{OperatorsNR\#or}{or}, \href{OperatorsAA\#and}{and},
\href{OperatorsAA\#!}{!},

\begin{center}\rule{0.5\linewidth}{\linethickness}\end{center}

\subsection{\texorpdfstring{\texttt{years\_between}}{years\_between}}\label{years_between}

\subsubsection{Possible use:}\label{possible-use-567}

\begin{itemize}
\tightlist
\item
  \texttt{date} \textbf{\texttt{years\_between}} \texttt{date}
  ---\textgreater{} \texttt{int}
\item
  \textbf{\texttt{years\_between}} (\texttt{date} , \texttt{date})
  ---\textgreater{} \texttt{int}
\end{itemize}

\subsubsection{Result:}\label{result-546}

Provide the exact number of years between two dates. This number can be
positive or negative (if the second operand is smaller than the first
one)

\subsubsection{Examples:}\label{examples-391}

\begin{verbatim}
 
int var0 <- years_between(date('2000-01-01'), date('2010-01-01')); // var0 equals 10
\end{verbatim}

\printindex


\end{document}
